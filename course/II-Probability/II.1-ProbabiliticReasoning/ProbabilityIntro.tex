\documentclass{beamer}

\input{../../spec_files/course_preamble.tex}
\subtitle{Foundations of Neuro-Symbolic AI}
\date{Summer Term 2026}
\author[FONS]{Alex Goessmann}
\institute[]{
    University of Applied Science Würzburg-Schweinfurt
%    Weierstrass Institute for Applied Analysis and Stochastic
}

%\newcommand{\techwstitle}{
%\small
%%Workshop \\
%Logik für Erklärbare KI:
%Technische Einführung in das ENEXA Projekt}
%\newcommand{\smalltechwstitle}{ENEXA Workshop}

%\newcommand{\techwsdate}{15.+16. July, 2024}

%\newcommand{\techwsauthors}{
%Alex Goessmann
%}

%\newcommand{\techwsinclude}{
%	\usepackage{../../spec/beamercolorthemeclaw}
%	\usepackage{/Users/alexgoessmann/Documents/ENEXA/latex_macros/beamer_template/beamerfontthemeclaw}
%	\usepackage{/Users/alexgoessmann/Documents/ENEXA/latex_macros/beamer_template/beamerinnerthemeclaw}
%	\usepackage{/Users/alexgoessmann/Documents/ENEXA/latex_macros/beamer_template/beamerouterthemeclaw}
%
%	\input{/Users/alexgoessmann/Documents/ENEXA/latex_macros/packages.tex}
%	\input{/Users/alexgoessmann/Documents/ENEXA/latex_macros/macros.tex}
%	\input{/Users/alexgoessmann/Documents/ENEXA/latex_macros/macros_tc.tex}
%	\input{/Users/alexgoessmann/Documents/ENEXA/latex_macros/tikz_blocks.tex}
%
%	\subtitle{\techwstitle}
%	\date[\techwsdate]{\techwsdate}
%	\author[\smalltechwstitle]{\techwsauthors}
%	\institute[]{\eupic}
%}

\newcommand{\techwschapterone}{I-Tensors}
\newcommand{\techwschaptertwo}{II-Probabilities}
\newcommand{\techwschapterthree}{III-Logics}
\newcommand{\techwschapterfour}{IV-Applications}

\newcommand{\eupic}{
\begin{center}
	%\includegraphics[width=4cm]{/Users/alexgoessmann/Documents/ENEXA/latex_macros/images/fundedEU.png}
\end{center}
}

\newcommand{\enexadateveublock}{
\begin{center}\begin{tikzpicture}
  	%\node [anchor=center] at (0,0) {\includegraphics[width = 1.5cm]{/Users/alexgoessmann/Documents/ENEXA/latex_macros/images/DATEV.png}};
	%\node [anchor=center] at (2.5,0.5) {\includegraphics[width = 3.5cm]{/Users/alexgoessmann/Documents/ENEXA/latex_macros/images/enexa.png}};
	%\node [anchor=center] at (2.55,-0.5) {\includegraphics[width = 3cm]{/Users/alexgoessmann/Documents/ENEXA/latex_macros/images/fundedEU.png}};
\end{tikzpicture}\end{center}
}


%% OLD
\newcommand{\aselectionvariable}{L}
\newcommand{\vselectionvariable}{L}
\newcommand{\fselectionvariable}{L}
\newcommand{\cselectionvariable}{L}
\newcommand{\individualorder}{n}
\newcommand{\variableof}[1]{\indvariableof{#1}}
\newcommand{\sindex}{s}
\newcommand{\pindex}{p}
\newcommand{\oindex}{o}
\newcommand{\exquery}{q}
%\newcommand{\datapointof}[1]{x^{#1}}
\newcommand{\atomicqueryof}[1]{g_{#1}}
\newcommand{\facsystem}{\shortcatvariables}
\newcommand{\margprobof}[1]{\probat{#1}}
\newcommand{\mlnprobabilityof}[1]{\expdistof{#1}}
%\newcommand{\oldenexadateveublock}{
%	\begin{center}
%	\begin{minipage}{0.2\textwidth}
%		\begin{center}
%			\includegraphics[width = 2.5cm]{images/DATEV.png}
%		\end{center}
%	\end{minipage}
%	\begin{minipage}{0.55\textwidth}
%		\begin{center}
%			\includegraphics[width=5.5cm]{images/enexa.png} \\
%			\includegraphics[width=5.5cm]{images/fundedEU.png} \\
%		\end{center}
%	\end{minipage}
%	\end{center}
%}

\title[Probabilistic Reasoning]{
	\techwschaptertwo \\
	{\huge Probabilistic Reasoning}
}
\begin{document}

{\frame[plain]{\titlepage}}



%\begin{frame}{Average of the Observation}
%
%The \emph{empirical distribution} is the average of the one-hot encoded observed states:
%\begin{align*}
%	\empdistribution \coloneqq \frac{1}{\datanum}\sum_{\datindexin} \onehotmapof{\catindicesof{\datindex}}
%\end{align*}
%
%In a contraction diagram we denote the average by % Already the decomposition!
%\begin{center}
%	\begin{tikzpicture}[scale=0.35, thick] % , baseline = -3.5pt


    \drawatomindices{0}{2}
    \draw (-1,1) rectangle (5,-1);
    \node[anchor=center] (text) at (2,0) {$\empdistribution$};


    \node[anchor=center] (text) at (7,0) {${=}$};

    \node[anchor=center] (text) at (22,-1) {${\cdot}$};

    \begin{scope}
        [shift={(10,2)}]

        \newcommand{\conposseldec}{4.5,-5.5}

        \draw[fill] (\conposseldec) circle (\dotsize);
        \draw[-<-] (\conposseldec) -- (4.5,-7.5) node[midway, right] {\colorlabelsize $\datvariable$};
        \draw (3.5,-7.5) rectangle (5.5, -9.5);
        \node[anchor=center] (text) at (4.5,-8.5) {\corelabelsize $\frac{1}{\datanum} \ones$};

        \draw[-<-] (0,1) -- (0,-1) node[midway,left] {\colorlabelsize $\catvariableof{0}$};
        \draw (-1,-1) rectangle (1, -3);
        \node[anchor=center] (text) at (0,-2) {\corelabelsize $\datacoreof{0}$};
        \draw[-<-] (0,-3) to[bend right=20] (\conposseldec);


        \draw[-<-] (3,1) -- (3,-1) node[midway,left] {\colorlabelsize $\catvariableof{1}$};
        \draw (2,-1) rectangle (4, -3);
        \node[anchor=center] (text) at (3,-2) {\corelabelsize $\datacoreof{1}$};
        \draw[-<-] (3,-3) to[bend right=20]  (\conposseldec);

        \node[anchor=center] (text) at (6,-2) {$\cdots$};

        \draw[-<-] (9,1) -- (9,-1) node[midway,left] {\colorlabelsize $\catvariableof{\atomorder-1}$};
        \draw (7.75,-1) rectangle (10.25, -3);
        \node[anchor=center] (text) at (9,-2) {\corelabelsize $\datacoreof{\atomorder-1}$};
        \draw[-<-] (9,-3) to[bend left=20]  (\conposseldec);


    \end{scope}


\end{tikzpicture}
%\end{center}
%
%\end{frame}




\begin{frame}{Example: Being at a dentist}

We are reasoning about a factored system with three variables:
\begin{itemize}
	\item \emph{Toothache} $i\in[2]$, whether your tooth hurts
	\item \emph{Cavity} $j\in[2]$, whether there is a cavity in your tooth
	\item \emph{Catch} $k\in[2]$, whether the dentist catches in your tooth 
\end{itemize}

\begin{center}
	\includegraphics[width=12cm]{images/toothache_prob} 
\end{center}

\end{frame}




\begin{frame}{Formal Definition of Probability Tensors}

\begin{definition}[Probability Tensor]
	Let there be a factored system $\facsystem$ defined by variables $\catvariableof{\atomenumerator}$ taking values in $[\catdimof{\atomenumerator}]$. 
	A probability distribution over the states of $\facsystem$ is a map
		\[ \probtensor : \facstates \rightarrow [0,\infty) \]
	such that $ \sum_{\catindices} \probtensor(\catindices)= 1$.
\end{definition}

We depict 
\begin{center}
	\begin{tikzpicture}[scale=0.35,thick] % , baseline = -3.5pt

    \node[anchor=center] (text) at (-2,0) {$a)$};

    \node [circle, draw, thick, fill=\nodegrayscale, minimum size = \nodeminsize] (P1) at (0,-3) {\colorlabelsize $\catvariableof{0}$};
    \node [circle, draw, thick, fill=\nodegrayscale, minimum size = \nodeminsize] (P2) at (3,-3) {\colorlabelsize $\catvariableof{1}$};

    \node[anchor=center] (text) at (6,-3) {$\cdots$};

    \node [circle, draw, thick, fill=\nodegrayscale, minimum size = \nodeminsize] (P3) at (9,-3) {};

    \node[anchor=center] (text) at (9,-3) {\colorlabelsize $\catvariableof{\atomorder-1}$};


    \draw[->-]
    (4.5,0) to[bend right=25] (P1);
    \draw[->-]
    (4.5,0) to[bend right=10] (P2);
    \draw[->-]
    (4.5,0) to[bend right=-25] (P3);

    \node[anchor=center] (text) at (4.5,0.5) {$\edge$};


    \begin{scope}
        [shift={(20,0)}]

        \node[anchor=center] (text) at (-2,0) {$b)$};

        \draw (-1,-1) rectangle (5,-3);
        \node[anchor=center] (text) at (2,-2) {\corelabelsize $\probtensor$};
%\draw[->-] (0,-3)--(0,-5) node[midway,left] {\colorlabelsize $\catvariableof{0}$};
%\draw[->-] (1.5,-3)--(1.5,-5) node[midway,left] {\colorlabelsize $\catvariableof{1}$};
        \node[anchor=center] (text) at (3,-4) {$\cdots$};
%\draw[->-] (4,-3)--(4,-5) node[midway,right] {\colorlabelsize $\catvariableof{\atomorder-1}$};


        \draw[midarrow]  (0,-3) -- (0,-5) node[midway,left] {\colorlabelsize $\catvariableof{0}$};
        \draw[midarrow]
        (1.5,-3)--(1.5,-5) node[midway,left] {\colorlabelsize $\catvariableof{1}$};
        \draw[midarrow]
        (4,-3)--(4,-5) node[midway,right] {\colorlabelsize $\catvariableof{\atomorder-1}$};
    \end{scope}


\end{tikzpicture}
\end{center}

\end{frame}



\begin{frame}{Directed Tensors}

We depict the condition, that coordinate sums are one, by directions on the legs.

\begin{definition}[Directed Tensor]
	A tensor 
		\[ \hypercore \in \bigotimes_{\nodein}\rr^{\catdimof{\node}} \]
	is said to be directed with incoming variables $\innodes$ and outgoing variables $\outnodes$, where $\nodes=\innodes\dot{\cup}\outnodes$, when
		\[ \contractionof{\{\hypercore\}}{\innodes} =  \onesof{\innodes} \]
	where $\onesof{\innodes}$ denoted the trivial tensor in  $\bigotimes_{\node\in\innodes}\rr^{\catdimof{\node}}$ which coordinates are all $1$.
\end{definition}

\end{frame}


\begin{frame}{Example: Marginal Distributions}

What is the probability that there is a cavity?

\begin{align}
	\probtensor^{\mathrm{Dentist},\mathrm{Cavity}}[\mathrm{Cavity}] 
	= \sum_{j\in[2]} \sum_{k\in[2]} \probtensor^{\mathrm{Dentist}}(:,j,k)
\end{align}

This is called a \emph{marginal} distribution.

\begin{block}{Exercise}
	Calculate the marginal probability of \emph{Cavity} given the probability tensor $\probtensor^{\mathrm{Dentist}}$.
\end{block}
\begin{center}
	\includegraphics[width=12cm]{images/toothache_prob} 
\end{center}

\end{frame}


\begin{frame}{Formal Definition of Marginal Distributions}

\begin{definition}[Marginal Probability]\label{def:marginalProbability}
	Given a distribution $\probtensor$ of the categorical variables $\exrandom$ and $\secexrandom$ the marginal distribution of the categorical variable $\exrandom$ is defined for each $\catindexof{\exrandom}$ as
		\[ 
		\margprobof{\exrandom=\catindexof{\exrandom}}{\exrandom}
		 = \sum_{\catindexof{\secexrandom}\in[\catdimof{\secexrandom}]} \probof{\exrandom=\catindexof{\exrandom},\secexrandom=\catindexof{\secexrandom}} \, . \]
\end{definition}

Marginal probabilities are contractions
\begin{align*}
	\probof{\exrandom} = \contractionof{\probtensor}{\exrandom}
\end{align*}
depicted by
\begin{center}
	\begin{tikzpicture}[scale=0.3,thick] % , baseline = -3.5pt

\draw (-19,-1) rectangle (-15,-3);
\node[anchor=center] (text) at (-17,-2) {\corelabelsize $\margprobat{\exrandom}$};
\draw[midarrow]  (-17,-3)--(-17,-5) node[midway,left] {\colorlabelsize $\exrandom$};

\node[anchor=center] (text) at (-13,-2) {${=}$};

\draw (-11,-1) rectangle (-5,-3);
\node[anchor=center] (text) at (-8,-2) {\corelabelsize $\probat{\exrandom,\secexrandom}$};
\draw[midarrow]  (-10,-3)--(-10,-5) node[midway,left] {\colorlabelsize $\exrandom$};
\draw[midarrow]  (-6,-3)--(-6,-5) node[midway,left] {\colorlabelsize $\secexrandom$};
\draw (-7,-5) rectangle (-5,-7); 
\node[anchor=center] (text) at (-6,-6) {$\ones$};

\end{tikzpicture}
\end{center}

\end{frame}






\begin{frame}{Marginal Distributions in Contraction Formalism}

Contractions are useful to
\begin{itemize}
	\item Specify the Probability Tensor, or a decomposition of it
	\item Specify the variables to marginalize over as the ones left open
\end{itemize}

\begin{center}
	\begin{tikzpicture}[scale=0.3,thick] % , baseline = -3.5pt

\draw (-19,-1) rectangle (-15,-3);
\node[anchor=center] (text) at (-17,-2) {\corelabelsize $\margprobat{\exrandom}$};
\draw[midarrow]  (-17,-3)--(-17,-5) node[midway,left] {\colorlabelsize $\exrandom$};

\node[anchor=center] (text) at (-13,-2) {${=}$};

\draw (-11,-1) rectangle (-5,-3);
\node[anchor=center] (text) at (-8,-2) {\corelabelsize $\probat{\exrandom,\secexrandom}$};
\draw[midarrow]  (-10,-3)--(-10,-5) node[midway,left] {\colorlabelsize $\exrandom$};
\draw[midarrow]  (-6,-3)--(-6,-5) node[midway,left] {\colorlabelsize $\secexrandom$};
\draw (-7,-5) rectangle (-5,-7); 
\node[anchor=center] (text) at (-6,-6) {$\ones$};

\end{tikzpicture}
\end{center}

\medskip 

Directed notation preserved: Marginal probabilities are again probability distributions, since
\begin{align}
	\sum_{i\in[2]}\probtensor^{\mathrm{Dentist},\mathrm{Cavity}}[i] = \sum_{i\in[2]} \sum_{j\in[2]} \sum_{k\in[2]} \probtensor^{\mathrm{Dentist}}_{i,j,k} = 1 \, . 
\end{align}


\end{frame}


\begin{frame}{Example: Conditional Distributions}

What is the probability of having a cavity when having a toothache?

\begin{align}
	\probtensor^{\mathrm{Dentist}}[\mathrm{Cavity}|\mathrm{Toothache}] 
	=\frac{ \sum_{j\in[2]} \probtensor^{\mathrm{Dentist}}_{:,j,:} }
	{ \sum_{j,k\in[2]} \probtensor^{\mathrm{Dentist}}_{:,j,k} }
\end{align}

\medskip

This is called a \emph{conditional} distribution.

\medskip

\begin{block}{Exercise}
	Calculate the probability of \emph{Cavity} conditioned on \emph{Toothache}. %given the probability tensor $\probtensor^{\mathrm{Dentist}}$.
\end{block}

\begin{center}
	\includegraphics[width=12cm]{images/toothache_prob} 
\end{center}

\end{frame}


\begin{frame}{Formal Definition of Conditional Distributions}

\begin{definition}[Conditional Probability]
	Given a distribution $\probtensor$ of the categorical variables $\catvariable$ and $Y$, the conditioned distribution of $\catvariable$ is defined by
		\[ \condprobof{\catvariable=\catindexof{\catvariable}}{Y=\catindexof{Y}}
		= \frac{\probof{\catvariable = \catindexof{\catvariable},Y=\catindexof{Y}}}{\probof{Y=\catindexof{Y}}} \, . \]
\end{definition}

\begin{center}
	\begin{tikzpicture}[scale=0.3, thick] % , baseline = -3.5pt

\draw (-21,-1) rectangle (-15,-3);
\node[anchor=center] (text) at (-18,-2) {\small $\condprobof{\exrandom}{\secexrandom}$};
\draw[->]  (-20,-3)--(-20,-5) node[midway,left] {\tiny $\exrandom$}; 

\draw[<-]  (-16,-3)--(-16,-5) node[midway,left] {\tiny $\secexrandom$}; 
\draw[dashed] (-15,-5) rectangle (-17,-7); 
\node[anchor=center] (text) at (-16,-6) {\small $\onehotmapof{\catindexof{\secexrandom}}$};

\node[anchor=center] (text) at (-13,-2) {${=}$};


\begin{scope}[shift={(0,6)}]

\draw (-11,-1) rectangle (-5,-3);
\node[anchor=center] (text) at (-8,-2) {\small $\probof{\exrandom,\secexrandom}$};
\draw[->]  (-10,-3)--(-10,-5) node[midway,left] {\tiny $\exrandom$}; 
\draw[->]  (-6,-3)--(-6,-5) node[midway,left] {\tiny $\secexrandom$};
\draw[dashed] (-7,-5) rectangle (-5,-7); 
\node[anchor=center] (text) at (-6,-6) {\small $\onehotmapof{\catindexof{\secexrandom}}$};

\end{scope}

\draw (-12,-2) -- (-4,-2);

\begin{scope}[shift={(0,-2)}]

\draw (-11,-1) rectangle (-5,-3);
\node[anchor=center] (text) at (-8,-2) {\small $\probof{\exrandom,\secexrandom}$};
\draw[->]  (-10,-3)--(-10,-5) node[midway,left] {\tiny $\exrandom$}; 
\draw (-11,-5) rectangle (-9,-7); 
\node[anchor=center] (text) at (-10,-6) {$\ones$};
\draw[->]  (-6,-3)--(-6,-5) node[midway,left] {\tiny $\secexrandom$};
\draw[dashed] (-7,-5) rectangle (-5,-7); 
\node[anchor=center] (text) at (-6,-6) {\small $\onehotmapof{\catindexof{\secexrandom}}$};

\end{scope}

\end{tikzpicture}
\end{center}

\end{frame}



\begin{frame}{Normations}

\begin{definition}[Normation of Tensor Networks]
	A tensor network $\extnet$ on variables $\nodes$ can be normed on $\secnodes$, if the coordinates of no slice with respect to $\secnodes$ sum to $0$.
	Then we define the normed tensor
		\[ \normalizationofwrt{\extnet}{\outnodes}{\innodes}
		\in \left( \bigotimes_{\node\in\innodes} \rr^{\catdimof{\node}} \right) \otimes \left( \bigotimes_{\node\in\outnodes} \rr^{\catdimof{\node}} \right) \]
	by
	 \begin{align*}
	 	\normalizationofwrt{\extnet}{\outnodes}{\innodes}
		= \sum_{\atomlegindexof{\innodes}\in\bigtimes_{\node\in\innodes}\catdimof{\node}} 
		\onehotmapof{\atomlegindexof{\innodes}} \otimes \frac{
		\contractionof{\extnet\cup\{\onehotmapof{\atomlegindexof{\innodes}}\}}{\outnodes}
		}{
		\contractionof{\extnet\cup\{\onehotmapof{\atomlegindexof{\innodes}}\}}{\varnothing}
		} \, . 
	 \end{align*}
\end{definition}

\end{frame}



\begin{frame}{Conditional Distributions by Normations}

Conditioning is the normation
	\[ \condprobof{\exrandom}{\secexrandom} = \normalizationofwrt{\probtensor}{\exrandom}{\secexrandom} \]
depicted by
\begin{center}
	\begin{tikzpicture}[scale=0.3, thick] % , baseline = -3.5pt

\draw (-21,-1) rectangle (-15,-3);
\node[anchor=center] (text) at (-18,-2) {\small $\condprobof{\exrandom}{\secexrandom}$};
\draw[->]  (-20,-3)--(-20,-5) node[midway,left] {\tiny $\exrandom$}; 

\draw[<-]  (-16,-3)--(-16,-5) node[midway,left] {\tiny $\secexrandom$}; 
\draw[dashed] (-15,-5) rectangle (-17,-7); 
\node[anchor=center] (text) at (-16,-6) {\small $\onehotmapof{\catindexof{\secexrandom}}$};

\node[anchor=center] (text) at (-13,-2) {${=}$};


\begin{scope}[shift={(0,6)}]

\draw (-11,-1) rectangle (-5,-3);
\node[anchor=center] (text) at (-8,-2) {\small $\probof{\exrandom,\secexrandom}$};
\draw[->]  (-10,-3)--(-10,-5) node[midway,left] {\tiny $\exrandom$}; 
\draw[->]  (-6,-3)--(-6,-5) node[midway,left] {\tiny $\secexrandom$};
\draw[dashed] (-7,-5) rectangle (-5,-7); 
\node[anchor=center] (text) at (-6,-6) {\small $\onehotmapof{\catindexof{\secexrandom}}$};

\end{scope}

\draw (-12,-2) -- (-4,-2);

\begin{scope}[shift={(0,-2)}]

\draw (-11,-1) rectangle (-5,-3);
\node[anchor=center] (text) at (-8,-2) {\small $\probof{\exrandom,\secexrandom}$};
\draw[->]  (-10,-3)--(-10,-5) node[midway,left] {\tiny $\exrandom$}; 
\draw (-11,-5) rectangle (-9,-7); 
\node[anchor=center] (text) at (-10,-6) {$\ones$};
\draw[->]  (-6,-3)--(-6,-5) node[midway,left] {\tiny $\secexrandom$};
\draw[dashed] (-7,-5) rectangle (-5,-7); 
\node[anchor=center] (text) at (-6,-6) {\small $\onehotmapof{\catindexof{\secexrandom}}$};

\end{scope}

\end{tikzpicture}
\end{center}

The directed notation highlights \emph{Conditions} by incoming legs and \emph{Distributions} by outgoing legs.


\end{frame}


\begin{frame}{The Bayes Theorem}

The Bayes Theorem relates conditional probabilities:

\begin{theorem}[Bayes Theorem]
	For any joint distribution of two categorical variables $\exrandom$ and $\secexrandom$ it holds that
	\[ \condprobof{\exrandom}{\secexrandom} = \frac{\probof{\exrandom,\secexrandom}}{\probof{\secexrandom}} 
	= \frac{\condprobof{\secexrandom}{\exrandom}\probof{\exrandom}}{\probof{\secexrandom}}  \, . \]
\end{theorem}	

\end{frame}


\begin{frame}{Bayes Theorem in the Dentist Example}

Directions of Reasoning
\begin{itemize}
	\item \emph{Causal direction}: Toothache is caused by cavity
	\item \emph{Diagnostic direction}: Cavity is probable because of toothache
\end{itemize}

\medskip

Bayes Theorem allows us to reason in diagnostic direction, given an underlying causal influence:
\begin{align*}
	& \probtensor^{\mathrm{Dentist}}[\mathrm{Cavity}|\mathrm{Toothache}] \\
	 & \quad\quad\quad =  
	\probtensor^{\mathrm{Dentist}}[\mathrm{Toothache}|\mathrm{Cavity}] 
	\frac{\probtensor^{\mathrm{Dentist}}[\mathrm{Cavity}] }{\probtensor^{\mathrm{Dentist}}[\mathrm{Toothache}]}
\end{align*}

\end{frame}





\begin{frame}{Contraction Calculus for Probability Tensors}

Probabilistic queries can be answered by contractions
\begin{itemize}
	\item Marginal probabilities 
		\[ \probof{\exrandom} = \contractionof{\probtensor}{\exrandom} \]
	\item Conditional probabilities
		\[ \condprobof{\exrandom}{\secexrandom} = \normalizationofwrt{\probtensor}{\exrandom}{\secexrandom} \]
\end{itemize}

\medskip 

Outlook: \emph{Tensor network decompositions of $\probtensor$} increase the execution efficiency!

\end{frame}


\end{document}


