\documentclass{beamer}

\input{../../spec_files/course_preamble.tex}
\subtitle{Foundations of Neuro-Symbolic AI}
\date{Summer Term 2026}
\author[FONS]{Alex Goessmann}
\institute[]{
    University of Applied Science Würzburg-Schweinfurt
%    Weierstrass Institute for Applied Analysis and Stochastic
}

%\newcommand{\techwstitle}{
%\small
%%Workshop \\
%Logik für Erklärbare KI:
%Technische Einführung in das ENEXA Projekt}
%\newcommand{\smalltechwstitle}{ENEXA Workshop}

%\newcommand{\techwsdate}{15.+16. July, 2024}

%\newcommand{\techwsauthors}{
%Alex Goessmann
%}

%\newcommand{\techwsinclude}{
%	\usepackage{../../spec/beamercolorthemeclaw}
%	\usepackage{/Users/alexgoessmann/Documents/ENEXA/latex_macros/beamer_template/beamerfontthemeclaw}
%	\usepackage{/Users/alexgoessmann/Documents/ENEXA/latex_macros/beamer_template/beamerinnerthemeclaw}
%	\usepackage{/Users/alexgoessmann/Documents/ENEXA/latex_macros/beamer_template/beamerouterthemeclaw}
%
%	\input{/Users/alexgoessmann/Documents/ENEXA/latex_macros/packages.tex}
%	\input{/Users/alexgoessmann/Documents/ENEXA/latex_macros/macros.tex}
%	\input{/Users/alexgoessmann/Documents/ENEXA/latex_macros/macros_tc.tex}
%	\input{/Users/alexgoessmann/Documents/ENEXA/latex_macros/tikz_blocks.tex}
%
%	\subtitle{\techwstitle}
%	\date[\techwsdate]{\techwsdate}
%	\author[\smalltechwstitle]{\techwsauthors}
%	\institute[]{\eupic}
%}

\newcommand{\techwschapterone}{I-Tensors}
\newcommand{\techwschaptertwo}{II-Probabilities}
\newcommand{\techwschapterthree}{III-Logics}
\newcommand{\techwschapterfour}{IV-Applications}

\newcommand{\eupic}{
\begin{center}
	%\includegraphics[width=4cm]{/Users/alexgoessmann/Documents/ENEXA/latex_macros/images/fundedEU.png}
\end{center}
}

\newcommand{\enexadateveublock}{
\begin{center}\begin{tikzpicture}
  	%\node [anchor=center] at (0,0) {\includegraphics[width = 1.5cm]{/Users/alexgoessmann/Documents/ENEXA/latex_macros/images/DATEV.png}};
	%\node [anchor=center] at (2.5,0.5) {\includegraphics[width = 3.5cm]{/Users/alexgoessmann/Documents/ENEXA/latex_macros/images/enexa.png}};
	%\node [anchor=center] at (2.55,-0.5) {\includegraphics[width = 3cm]{/Users/alexgoessmann/Documents/ENEXA/latex_macros/images/fundedEU.png}};
\end{tikzpicture}\end{center}
}


%% OLD
\newcommand{\aselectionvariable}{L}
\newcommand{\vselectionvariable}{L}
\newcommand{\fselectionvariable}{L}
\newcommand{\cselectionvariable}{L}
\newcommand{\individualorder}{n}
\newcommand{\variableof}[1]{\indvariableof{#1}}
\newcommand{\sindex}{s}
\newcommand{\pindex}{p}
\newcommand{\oindex}{o}
\newcommand{\exquery}{q}
%\newcommand{\datapointof}[1]{x^{#1}}
\newcommand{\atomicqueryof}[1]{g_{#1}}
\newcommand{\facsystem}{\shortcatvariables}
\newcommand{\margprobof}[1]{\probat{#1}}
\newcommand{\mlnprobabilityof}[1]{\expdistof{#1}}
%\newcommand{\oldenexadateveublock}{
%	\begin{center}
%	\begin{minipage}{0.2\textwidth}
%		\begin{center}
%			\includegraphics[width = 2.5cm]{images/DATEV.png}
%		\end{center}
%	\end{minipage}
%	\begin{minipage}{0.55\textwidth}
%		\begin{center}
%			\includegraphics[width=5.5cm]{images/enexa.png} \\
%			\includegraphics[width=5.5cm]{images/fundedEU.png} \\
%		\end{center}
%	\end{minipage}
%	\end{center}
%}

\title[Gibbs Sampling]{
	\techwschaptertwo \\
	{\huge Gibbs Sampling}
}
\begin{document}

{\frame[plain]{\titlepage}}



\begin{frame}{How to draw random samples?}

\begin{block}{Sampling a distribution $\probtensor$}
	Sampling is a random procedure to get index pairs $\catindices\in\facstates$ such that the probability of $\catindices$ is $\probtensor(\catindices)$.
\end{block}

{A naive approach would not succeed:} Typically, the tensors $\probtensor$ is too large to be created.

\medskip

\begin{block}{Approach}
	Use the representations like Markov Networks
		\[ \probtensor = \normalizationofwrt{\{\hypercoreof{\edge} : \edge \in \edges\}}{\nodes}{\varnothing}  \]
	to draw samples.
\end{block}

\end{frame}






\begin{frame}{Gibbs Sampling}


Draw State for atom $\atomenumerator$ a state $\atomlegindexof{\atomenumerator}$ from marginalized distribution 
	\[ \probof{\catvariableof{\atomenumerator}} = \frac{\contractionof{\probtensor}{\catvariableof{\atomenumerator}}
		}{
		\contractionof{\{\probtensor\}}{\varnothing}
		} \]

\medskip

Iterate over $\atomenumeratorin$ and draw $\atomlegindexof{\atomenumerator}$ from 
		\[ \condprobof{\catvariableof{\atomenumerator}}{\{\onehotmapof{\atomlegindexof{\secatomenumerator}} \, : \secatomenumerator \neq \atomenumerator\}} = 
		\frac{
		\contractionof{\{\probtensor\} \cup \{\onehotmapof{\atomlegindexof{\secatomenumerator}} \, : \secatomenumerator \neq \atomenumerator \}}{\catvariableof{\atomenumerator}}
		}{
		\contractionof{\{\probtensor\} \cup \{\onehotmapof{\atomlegindexof{\secatomenumerator}} \, : \secatomenumerator \neq \atomenumerator \}}{\varnothing}
		} \]

\begin{block}{Intuition}
	While we initialize with an independent sample, the resampling iterations implement the dependencies of the variables with respect to each other.
\end{block}

\end{frame}


\begin{frame}{Implementation in \tnreason: \\
Subpackage \spreasoning{}}

The subpackage \spreasoning{} implements algorithms such as Gibbs Sampling.

\begin{center}
\begin{tikzpicture}[scale=0.27]

\draw[dashed] (-30,15) -- (12,15) -- (12,-3) -- (-30,-3) -- (-30,15);

\draw (-10,10) rectangle (10,14); 
\node [anchor=center] at (0,12) {\spapplication{}};

\node [anchor=center] at (-20,12) {\layerthreespec};
\draw[dashed] (-30,9) -- (12,9);
\node [anchor=center,blue] at (-20,6) {\layertwospec};

\draw[->,blue] (6,10) -- (6,8);
\draw[blue] (2,4) rectangle (10,8); 
\node [anchor=center,blue] at (6,6) {\spreasoning{}};
\draw[->,blue] (6,4) -- (6,2);

\draw[->] (-6,10) -- (-6,8);
\draw[] (-10,4) rectangle (-2,8); 
\node [anchor=center] at (-6,6) {\sprepresentation{}};
\draw[->] (-6,4) -- (-6,2);

\draw[dashed] (-30,3) -- (12,3);
\node [anchor=center] at (-20,0) {\layeronespec};

\draw (-10,-2) rectangle (10,2); 
\node [anchor=center] at (0,0) {\spengine};
\end{tikzpicture}
\end{center}

\end{frame}

\end{document}