\documentclass{beamer}

\input{../../spec_files/course_preamble.tex}
\subtitle{Foundations of Neuro-Symbolic AI}
\date{Summer Term 2026}
\author[FONS]{Alex Goessmann}
\institute[]{
    University of Applied Science Würzburg-Schweinfurt
%    Weierstrass Institute for Applied Analysis and Stochastic
}

%\newcommand{\techwstitle}{
%\small
%%Workshop \\
%Logik für Erklärbare KI:
%Technische Einführung in das ENEXA Projekt}
%\newcommand{\smalltechwstitle}{ENEXA Workshop}

%\newcommand{\techwsdate}{15.+16. July, 2024}

%\newcommand{\techwsauthors}{
%Alex Goessmann
%}

%\newcommand{\techwsinclude}{
%	\usepackage{../../spec/beamercolorthemeclaw}
%	\usepackage{/Users/alexgoessmann/Documents/ENEXA/latex_macros/beamer_template/beamerfontthemeclaw}
%	\usepackage{/Users/alexgoessmann/Documents/ENEXA/latex_macros/beamer_template/beamerinnerthemeclaw}
%	\usepackage{/Users/alexgoessmann/Documents/ENEXA/latex_macros/beamer_template/beamerouterthemeclaw}
%
%	\input{/Users/alexgoessmann/Documents/ENEXA/latex_macros/packages.tex}
%	\input{/Users/alexgoessmann/Documents/ENEXA/latex_macros/macros.tex}
%	\input{/Users/alexgoessmann/Documents/ENEXA/latex_macros/macros_tc.tex}
%	\input{/Users/alexgoessmann/Documents/ENEXA/latex_macros/tikz_blocks.tex}
%
%	\subtitle{\techwstitle}
%	\date[\techwsdate]{\techwsdate}
%	\author[\smalltechwstitle]{\techwsauthors}
%	\institute[]{\eupic}
%}

\newcommand{\techwschapterone}{I-Tensors}
\newcommand{\techwschaptertwo}{II-Probabilities}
\newcommand{\techwschapterthree}{III-Logics}
\newcommand{\techwschapterfour}{IV-Applications}

\newcommand{\eupic}{
\begin{center}
	%\includegraphics[width=4cm]{/Users/alexgoessmann/Documents/ENEXA/latex_macros/images/fundedEU.png}
\end{center}
}

\newcommand{\enexadateveublock}{
\begin{center}\begin{tikzpicture}
  	%\node [anchor=center] at (0,0) {\includegraphics[width = 1.5cm]{/Users/alexgoessmann/Documents/ENEXA/latex_macros/images/DATEV.png}};
	%\node [anchor=center] at (2.5,0.5) {\includegraphics[width = 3.5cm]{/Users/alexgoessmann/Documents/ENEXA/latex_macros/images/enexa.png}};
	%\node [anchor=center] at (2.55,-0.5) {\includegraphics[width = 3cm]{/Users/alexgoessmann/Documents/ENEXA/latex_macros/images/fundedEU.png}};
\end{tikzpicture}\end{center}
}


%% OLD
\newcommand{\aselectionvariable}{L}
\newcommand{\vselectionvariable}{L}
\newcommand{\fselectionvariable}{L}
\newcommand{\cselectionvariable}{L}
\newcommand{\individualorder}{n}
\newcommand{\variableof}[1]{\indvariableof{#1}}
\newcommand{\sindex}{s}
\newcommand{\pindex}{p}
\newcommand{\oindex}{o}
\newcommand{\exquery}{q}
%\newcommand{\datapointof}[1]{x^{#1}}
\newcommand{\atomicqueryof}[1]{g_{#1}}
\newcommand{\facsystem}{\shortcatvariables}
\newcommand{\margprobof}[1]{\probat{#1}}
\newcommand{\mlnprobabilityof}[1]{\expdistof{#1}}
%\newcommand{\oldenexadateveublock}{
%	\begin{center}
%	\begin{minipage}{0.2\textwidth}
%		\begin{center}
%			\includegraphics[width = 2.5cm]{images/DATEV.png}
%		\end{center}
%	\end{minipage}
%	\begin{minipage}{0.55\textwidth}
%		\begin{center}
%			\includegraphics[width=5.5cm]{images/enexa.png} \\
%			\includegraphics[width=5.5cm]{images/fundedEU.png} \\
%		\end{center}
%	\end{minipage}
%	\end{center}
%}

\title[One-hot encodings]{
	\techwschapterone \\
	{\huge One-hot encoding of Factored Systems}
}

\begin{document}

{\frame[plain]{\titlepage}}


\begin{frame}{Systems of Categorical Variables}

\begin{block}{Categorical Variables}
	Categorical variables $\catvariable$ are assigned by elements of finite sets, which we enumerate by
		\[\catvariable \in \{0,\ldots,\catdim-1 \} = [\catdim]\]
\end{block}

In artificial intelligence we distinguish between 
\begin{itemize}
	\item \emph{Atomic} Systems: Described by a single categorical variable
	\item \emph{Factored} Systems: Described by a fixed set of categorical variables
	\item \emph{Structured} Systems: Described by a varying set of categorical variables
\end{itemize}

\end{frame}




\begin{frame}{Example: Atomic System of a Student}

\begin{block}{Example: Semester Variable}
	We describe a student $s$ by the semester $\catvariableof{s}$ the student is in.
	We can choose (given a four semester program)
		\[ \catvariableof{s} \in \{0,1,2,3\} = [4] \, . \]
\end{block}

A single student is an \emph{atomic} system, described by the semester variable.

\medskip

When knowing that $\catvariableof{s}=2$, our knowledge is stored by the real number
	\[ 2 \in \rr \, .\]
But how to store the knowledge, that a student
\begin{itemize}
	\item is either in the semester $0$ or $2$ (since we have a summer semester)?
	\item is with $80\%$ in semester $2$ and with $20\%$ in semester $0$ (since electives are typically attended in semester $2$)?
\end{itemize}

\end{frame}


\begin{frame}{Single numbers are not enough!}

But how to store the knowledge, that a student
\begin{itemize}
	\item is either in the semester $0$ or $2$ (since we have a summer semester)? \\
		\emph{Just summing both representations would result in
			\[ 0 + 2 = 2 \]
		and certainly does not help us.}
	\item is with $80\%$ in semester $2$ and with $20\%$ in semester $0$ (since electives are typically attended in semester $2$)? \\
		\emph{When just representing our knowledge by the average
			\[ 2\cdot 80\% + 0 \cdot 20\% = 1.6 \]
		we do not capture everything.}
\end{itemize}

\end{frame}


\begin{frame}{One-hot encodings: \\
 Mapping Categorical Variables to Vectors}
	Arranged collections of $\catdim$ numbers are richer: \emph{Vectors}, which are elements of the space
		\[ \rr^{\catdim} \, . \]
	\begin{definition}[One-hot Encoding]
		The one-hot encoding of a categorical variable is a map
			\[ \onehotmap : \{0,1,\ldots,\catdim-1\} \rightarrow \rr^{\catdim}\]
		defined by mapping $\catindexofin{}$ to basis vectors
		\begin{align*}
			\onehotmap \left( \catindex \right) = 
			\onehotmapof{\catindex} = \begin{bmatrix}
			0 & \cdots & 0 & 1 &  0 & \cdots & 0
			\end{bmatrix} \, ,
		\end{align*}
		where the $1$ is at the $\catindex$th coordinate of the vector.
	\end{definition}
\end{frame}



\begin{frame}{Example: System of Students}

\begin{block}{Example: Semester Variable}
	We describe each student $s$ by the semester $\catvariableof{s}$ they are in.
	We can choose
		\[ \catvariableof{s} \in \{0,1,2,3\} = [4] \, . \]
\end{block}

The one-hot encoding of the semester variable is one of the four basis vectors 
\begin{align}
	\onehotmapof{\catindexof{s}} \in \left\{
	\begin{bmatrix}
		1 \\ 0 \\ 0 \\ 0
	\end{bmatrix},
	\begin{bmatrix}
		0 \\ 1 \\ 0 \\ 0
	\end{bmatrix},
	\begin{bmatrix}
		0 \\ 0 \\ 1 \\ 0
	\end{bmatrix},
	\begin{bmatrix}
		0 \\ 0 \\ 0 \\ 1
	\end{bmatrix}
\right\} \, . 
\end{align}

%A single student is interpreted as an \emph{atomic system}, described by a single variable.

\end{frame}



\begin{frame}{Calculus of one-hot encodings}

\begin{block}{Advantage of one-hot encodings}
	Knowledge about the system can be stored in the coordinates.
\end{block}

\only<2-3>{
\textbf{Back to the example:} Store the knowledge, that a student is either in the semester $0$ or $2$! \\
}
\only<3>{
\emph{Just sum up the one-hot encodings of the states
\begin{align*}
	\onehotmapof{0} + \onehotmapof{2} = &  
	\begin{bmatrix}
		1 & 0 & 0 & 0 
	\end{bmatrix} 
	+
	\begin{bmatrix}
		0 & 0 & 1 & 0 
	\end{bmatrix}  \\
	= & 
	\begin{bmatrix}
		1 & 0 & 1 & 0 
	\end{bmatrix} \, .
\end{align*}
To retrieve the possible semesters, we simply look at the $1$ coordinates.
}
}


\only<4-5>{
\textbf{Back to the example:} Store the knowledge, that a student is with $80\%$ probability in semester $2$ and with $20\%$ probability in semester $0$! \\
}
\only<5>{
\emph{Build the weighted sum of one-hot encodings 
\begin{align*}
	20\% \cdot \onehotmapof{0} + 80\% \cdot \onehotmapof{2} = &  
	\begin{bmatrix}
		0.2 & 0 & 0 & 0 
	\end{bmatrix} 
	+
	\begin{bmatrix}
		0 & 0 & 0.8 & 0 
	\end{bmatrix}  \\
	= & 
	\begin{bmatrix}
		0.2 & 0 & 0.8 & 0 
	\end{bmatrix} \, .
\end{align*}
To retrieve the probability of the semester, we simply look at the respective coordinates.
}
}

\end{frame}



\begin{frame}{More general: Representation of any Probability Distribution}

	\begin{example}
		Let $\probtensor_\catindex \in [0,1]$ express for $\catindexofin{}$ the probability that the variable $\catvariable$ is in state $\catindex$.
		In order to describe a probability distribution we need 
			\[ \sum_{\catindex\in[\catdim]}\probtensor_\catindex=1 \, .  \]
	\end{example}
	%for $\catindex\in[\catdim]$ with $\sum_{\catindex\in[\catdim]}\probtensor_\catindex=1$ with the interpretation that $\probtensor_\catindex$ is the probability of a system being in state $\catindex$. 
	We can represent our knowledge of the system by storing $\probtensor_\catindex$ in the $\catindex$th coordinate of a vector $\probtensor\in\rr^{\catdim}$ defined as
	\[ \probtensor = \sum_{\catindex\in[\catdim]} \probtensor_\catindex \cdot \onehotmapof{\catindex} =
		\begin{bmatrix}
		\probtensor_0 & \probtensor_1 & \cdots & \probtensor_{\catdim-1}
		\end{bmatrix} \, . 
	\]
\end{frame}




\begin{frame}{Example of a Factored System: Collection of Students}

\begin{block}{Example: Students in a class}
	We now have a collection of $\atomorder$ students attending a class and want to represent the semester of each.
	Each student enumerated by $\atomenumeratorin$ is represented by 
		\[ \catvariableof{\atomenumerator} \in \{0,1,2,3\} = [4] \, . \]
\end{block}

We assume  
\begin{itemize}
	\item a fixed number of students 
	\item knowledge about all students
\end{itemize}

All students together are an example of a \emph{factored system}, described by a fixed collection of semester variables.

\end{frame}



\begin{frame}{How to represent a Factored Systems}

\begin{block}{Exponential growth of the number of states}
	The number of possible states grows exponentially: 
	Adding a student to $\atomorder$ students results in increase of the number of states from $4^{\atomorder}$ to 
		\[ 4^{\atomorder}\cdot 4 = 4^{\atomorder+1} \, . \]
\end{block}

We can always just enumerate all states by the tuples $(\catindices)$ and use the atomic encoding
	\[ \onehotmap : [4^\atomorder] \rightarrow \rr^{\left(4^{\atomorder}\right)} \]
but we will end up with large amounts of numbers which are hard to store and process!

\begin{center}
	\bf How can we avoid dealing with these large amounts of numbers?
\end{center}


\end{frame}






\begin{frame}{One-hot encoding of two students: Matrices}

\textbf{Example:} Student $0$ is in semester $\catvariableof{0}=2$ and student $2$ in semester $\catvariableof{2}=0$.
Then we have
\begin{align}
	\onehotmapof{(0,2)} 
	=
	\begin{bmatrix}
		0 & 0 & 0 & 0 \\
		0 & 0 & 0 & 0 \\
		1 & 0 & 0 & 0 \\
		0 & 0 & 0 & 0 
	\end{bmatrix} \in \rr^{16}\, . 
\end{align}

\begin{block}{Observation: Matrix Structure}
	When enumerating states by tuples, there is a matrix structure to exploit!
\end{block}

\end{frame}




%\begin{frame}{Tensor Product of one-hot encodings}
%
%	\begin{definition}
%		The tensor product of two vectors $V\in\rr^{m_1}$ and $W\in\rr^{m_2}$ is the matrix
%			\[ (V \otimes W) \in \rr^{m_1} \otimes \rr^{m_2} \cong \rr^{m_1\times m_2} \]
%		which is coordinatewise defined by the product of real numbers
%			\[ (V \otimes W)_{ij} = V_i \cdot W_j \, . \]
%	\end{definition}
%
%	\begin{definition}
%		The one-hot encoding of the state $(\catindexof{1},\catindexof{2})$ of two categorical variables $\catvariableof{1},\catvariableof{2}$ is the \emph{tensor product}
%		\begin{align}
%			\onehotmapof{\catindexof{1},\catindexof{2}} = \onehotmapof{\catindexof{1}} \otimes \onehotmapof{\catindexof{2}} \, .
%		\end{align}
%	\end{definition}
%\end{frame}



\begin{frame}{One-hot encoding of two students: Matrices}

\textbf{Example:} Student $0$ is in semester $\catvariableof{0}=2$ and student $2$ in semester $\catvariableof{2}=0$.
Then we have
\begin{align}
	\onehotmapof{(0,2)} 
	= \onehotmapof{0} \otimes \onehotmapof{2} =
	\begin{bmatrix}
		0 \\ 0 \\ 1 \\ 0
	\end{bmatrix}
	\otimes 	
	\begin{bmatrix}
		1 & 0 & 0 & 0
	\end{bmatrix}
	=
	\begin{bmatrix}
		0 & 0 & 0 & 0 \\
		0 & 0 & 0 & 0 \\
		1 & 0 & 0 & 0 \\
		0 & 0 & 0 & 0 
	\end{bmatrix} \, . 
\end{align}

\begin{block}{Observation: Storage efficiency}
	To store the matrix $\onehotmapof{0,2}$ we need $16$ numbers, to store $\onehotmapof{0}$ and $\onehotmapof{2}$ just $8$ numbers.
	This is a simple example of a \emph{tensor network decomposition}.
\end{block}

\end{frame}



%\begin{frame}{One-hot encoding of multiple students: Tensors}
%
%	\begin{definition}[Tensor Products]
%		Let there be vectors $V^{(\atomenumerator)}\in\rr^{\catdimof{\atomenumerator}}$ for $\atomenumerator$.
%		We define their tensor product 
%			\[ \bigotimes_{\atomenumeratorin} V^{(\atomenumerator)} \in \bigotimes_{\atomenumeratorin} \rr^{\catdimof{\atomenumerator}} \cong  \rr^{\prod_{\atomenumeratorin} \catdimof{\atomenumerator}} \]
%		coordinatewise by
%			\[ \left( \bigotimes_{\atomenumeratorin} V^{(\atomenumerator)} \right)_{\catindices}  = \prod_{\atomenumeratorin} V^{(\atomenumerator)}_{\catindexof{\atomenumerator}} \, . \]
%	\end{definition}
%
%\end{frame}




\begin{frame}{One-hot encoding of multiple students: Tensors}

	\begin{definition}[One-hot encoding of Factored Systems]
		The one-hot encoding of a factored systems with variables $\{\catvariableof{\atomenumerator} \, : \, \atomenumeratorin \}$ in state $\catindices$ is the tensor product 
 		\begin{align}
			\onehotmapof{(\catindices)} = \bigotimes_{\atomenumeratorin} \onehotmapof{\catindexof{\atomenumerator}} 
		\end{align}
		of the one-hot encodings of the variables.
	\end{definition}
	
	
\end{frame}


\begin{frame}{Visualization of Tensors}

We think of tensors as an alignment of real numbers, where each axis is one direction of the alignment.
\begin{itemize}
	\item \emph{Vectors} (Order-1 Tensors): Alignment in one direction
	\item \emph{Matrices} (Order-2 Tensors): Alignment in two directions
	\item Order-3 Tensors: Alignment in three directions, for example:

\begin{align*}
	\onehotmapof{(0,2,1)} 
	= \onehotmapof{0} \otimes \onehotmapof{2} \otimes \onehotmapof{1}=
	\begin{bmatrix}
		0 \\ 0 \\ 1 \\ 0
	\end{bmatrix}
	\otimes 	
	\begin{bmatrix}
		1 & 0 & 0 & 0
	\end{bmatrix}
	\otimes 
	\begin{matrix}
	& & & 0 ] \\ 
	& & 0 & \\
	& 1 & & \\
	[0 & & &  
	\end{matrix}
\end{align*}
\end{itemize}

\begin{center}
	\bf When encoding more than three variables we need a more abstract graphical notation!
\end{center}

\end{frame}


\begin{frame}{Outlook}

Let us think of more general situations
\begin{itemize}
	\item number of students is differing over time
	\item students have specific properties, semester being just one
\end{itemize}

\medskip

Students are seen as objects of the system, which number is differing from state to state.
This is an example of a \emph{structured system}.

\medskip

\begin{block}{Focus of this course}
	We will focus on \emph{factored systems} in this course, since they can be represented by tensors of fixed orders.
\end{block}

%\medskip
%Similar to object-oriented programming!

\end{frame}





\end{document}