\documentclass{beamer}

\input{../../spec_files/course_preamble.tex}
\subtitle{Foundations of Neuro-Symbolic AI}
\date{Summer Term 2026}
\author[FONS]{Alex Goessmann}
\institute[]{
    University of Applied Science Würzburg-Schweinfurt
%    Weierstrass Institute for Applied Analysis and Stochastic
}

%\newcommand{\techwstitle}{
%\small
%%Workshop \\
%Logik für Erklärbare KI:
%Technische Einführung in das ENEXA Projekt}
%\newcommand{\smalltechwstitle}{ENEXA Workshop}

%\newcommand{\techwsdate}{15.+16. July, 2024}

%\newcommand{\techwsauthors}{
%Alex Goessmann
%}

%\newcommand{\techwsinclude}{
%	\usepackage{../../spec/beamercolorthemeclaw}
%	\usepackage{/Users/alexgoessmann/Documents/ENEXA/latex_macros/beamer_template/beamerfontthemeclaw}
%	\usepackage{/Users/alexgoessmann/Documents/ENEXA/latex_macros/beamer_template/beamerinnerthemeclaw}
%	\usepackage{/Users/alexgoessmann/Documents/ENEXA/latex_macros/beamer_template/beamerouterthemeclaw}
%
%	\input{/Users/alexgoessmann/Documents/ENEXA/latex_macros/packages.tex}
%	\input{/Users/alexgoessmann/Documents/ENEXA/latex_macros/macros.tex}
%	\input{/Users/alexgoessmann/Documents/ENEXA/latex_macros/macros_tc.tex}
%	\input{/Users/alexgoessmann/Documents/ENEXA/latex_macros/tikz_blocks.tex}
%
%	\subtitle{\techwstitle}
%	\date[\techwsdate]{\techwsdate}
%	\author[\smalltechwstitle]{\techwsauthors}
%	\institute[]{\eupic}
%}

\newcommand{\techwschapterone}{I-Tensors}
\newcommand{\techwschaptertwo}{II-Probabilities}
\newcommand{\techwschapterthree}{III-Logics}
\newcommand{\techwschapterfour}{IV-Applications}

\newcommand{\eupic}{
\begin{center}
	%\includegraphics[width=4cm]{/Users/alexgoessmann/Documents/ENEXA/latex_macros/images/fundedEU.png}
\end{center}
}

\newcommand{\enexadateveublock}{
\begin{center}\begin{tikzpicture}
  	%\node [anchor=center] at (0,0) {\includegraphics[width = 1.5cm]{/Users/alexgoessmann/Documents/ENEXA/latex_macros/images/DATEV.png}};
	%\node [anchor=center] at (2.5,0.5) {\includegraphics[width = 3.5cm]{/Users/alexgoessmann/Documents/ENEXA/latex_macros/images/enexa.png}};
	%\node [anchor=center] at (2.55,-0.5) {\includegraphics[width = 3cm]{/Users/alexgoessmann/Documents/ENEXA/latex_macros/images/fundedEU.png}};
\end{tikzpicture}\end{center}
}


%% OLD
\newcommand{\aselectionvariable}{L}
\newcommand{\vselectionvariable}{L}
\newcommand{\fselectionvariable}{L}
\newcommand{\cselectionvariable}{L}
\newcommand{\individualorder}{n}
\newcommand{\variableof}[1]{\indvariableof{#1}}
\newcommand{\sindex}{s}
\newcommand{\pindex}{p}
\newcommand{\oindex}{o}
\newcommand{\exquery}{q}
%\newcommand{\datapointof}[1]{x^{#1}}
\newcommand{\atomicqueryof}[1]{g_{#1}}
\newcommand{\facsystem}{\shortcatvariables}
\newcommand{\margprobof}[1]{\probat{#1}}
\newcommand{\mlnprobabilityof}[1]{\expdistof{#1}}
%\newcommand{\oldenexadateveublock}{
%	\begin{center}
%	\begin{minipage}{0.2\textwidth}
%		\begin{center}
%			\includegraphics[width = 2.5cm]{images/DATEV.png}
%		\end{center}
%	\end{minipage}
%	\begin{minipage}{0.55\textwidth}
%		\begin{center}
%			\includegraphics[width=5.5cm]{images/enexa.png} \\
%			\includegraphics[width=5.5cm]{images/fundedEU.png} \\
%		\end{center}
%	\end{minipage}
%	\end{center}
%}

\title[\tnreason]{
	\techwschapterone \\
	{\huge Implementations in \tnreason{}}
}

\begin{document}

{\frame[plain]{\titlepage}}


\begin{frame}{Demonstration package in python: \tnreason{}}

\begin{center}
	\emph{tnreason} = \emph{T}ensor-\emph{N}etwork \emph{Reason}ing
\end{center}

Functionality:
\begin{itemize}
	\item Inference of Factored Systems based on logical and probabilistic concepts
	\item Neuro-Symbolic Reasoning Methods based on parametrized logics
\end{itemize}

%Submodules:
%\begin{itemize}
%	\item \textbf{engine}: Representation and contraction of Tensor Networks
%	\item \textbf{encoding}: Creation of Tensor Networks based on logics
%	\item \textbf{algorithms}: Tensor Network algorithms applied in reasoning
%	\item \textbf{knowledge}: Deductive and inductive reasoning given knowledge and data
%\end{itemize}

\tnreason is structured in four subpackages and three layers
\begin{itemize}
	\item Layer 1: Storage and numerical manipulations, by subpackage \spengine, "Tensor Networks" -> building "tn" of \tnreason
	\item Layer 2: Specification of workload, subpackage \sprepresentation{} specific for storage, subpackage \spreasoning{} specific for manipulations
	\item Layer 3: Applications in reasoning, by subpackage \spapplication{}, "Reasoning" -> building "reason" of \tnreason
\end{itemize}


%Demonstrations in Colab:
%\href{https://drive.google.com/drive/folders/1CpeWP2TTFKjjcvwDgxOrq-baZ7nyCpI1}{TN Reason Demos}

\end{frame}



\begin{frame}{Implementation in \tnreason: \\
Subpackage \spengine}

The subpackage \spengine is dedicated to tensor networks and contractions.

\begin{center}
\begin{tikzpicture}[scale=0.27]

\draw[dashed] (-30,15) -- (12,15) -- (12,-3) -- (-30,-3) -- (-30,15);

\draw (-10,10) rectangle (10,14); 
\node [anchor=center] at (0,12) {\spapplication{}};

\node [anchor=center] at (-20,12) {\layerthreespec};
\draw[dashed] (-30,9) -- (12,9);
\node [anchor=center] at (-20,6) {\layertwospec};

\draw[->] (6,10) -- (6,8);
\draw (2,4) rectangle (10,8); 
\node [anchor=center] at (6,6) {\spreasoning{}};
\draw[->] (6,4) -- (6,2);

\draw[->] (-6,10) -- (-6,8);
\draw (-10,4) rectangle (-2,8); 
\node [anchor=center] at (-6,6) {\sprepresentation{}};
\draw[->] (-6,4) -- (-6,2);

\draw[dashed] (-30,3) -- (12,3);
\node [anchor=center,blue] at (-20,0) {\layeronespec};

\draw[blue] (-10,-2) rectangle (10,2); 
\node [anchor=center,blue] at (0,0) {\spengine};
\end{tikzpicture}
\end{center}


\end{frame}



\begin{frame}{Subpackage \spengine : Cores}

Each Tensor core has attributes
\begin{itemize}
	\item \emph{values} (array-like): storing the value of the coordinates
	\item \emph{colors} (list of str): specifying the name of the variables represented by its axes
	\item \emph{name} (str): to distinguish from other cores
\end{itemize} 
The implemented core types differ in the values argument.
Cores are instantiated by
\begin{centeredscript}
	\emph{engine.getCore(coreType)(coreValues, coreColors, coreName)}
\end{centeredscript}

\end{frame}


\begin{frame}{Subpackage \spengine : Contractions}

Reflected in the notation
	\[ \contractionof{\tnetof{\graph}}{\nodes} \]
a contraction is defined by
\begin{itemize}
	\item Tensor Network $\tnetof{\graph}$, i.e. a dictionary of tensor cores
	\item Open Variables $\nodes$
\end{itemize}
Contraction calls are done by
\begin{centeredscript}
	\emph{engine.contract(contractionMethod, coreDict, openColors)}
\end{centeredscript}
Where
\begin{itemize}
	\item \emph{contractionMethod:} str, chooses one of the contraction providers
	\item \emph{coreDict}: Dictionary of TensorCores (of the above formats), representing the Tensor Network $\tnetof{\graph}$ 
	\item \emph{openColors}: List $\nodes$ of str, each str identifying a color, that is a variable to be left open in the contraction
\end{itemize}
\end{frame}






\begin{frame}{Installation}

\tnreason is maintained on github:
\begin{center}
	\href{https://github.com/EnexaProject/enexa-tensor-reasoning}{https://github.com/EnexaProject/enexa-tensor-reasoning}
\end{center}

Installation using pip:
\begin{centeredscript}
	!pip install tnreason
\end{centeredscript}





\end{frame}

\end{document}


