\section{Analysis in the tree-based implementation}

We denote for each pair $(\sedge,\redge)$ the subset $\preedgeset\subset\edges$ as the subset of edges $\edgein$, for which each path to $\redge$ passes through $\sedge$.
Note, that by construction $\sedge\in\preedgeset$.

\begin{theorem}
    \label{the:mpGuaranteeTree}
    For any tensor network on a tree hypergraph, \algoref{alg:contractionPropagation} terminates in the tree-based implementation and returns final messages
    \begin{align*}
        \messagewith
        = \contractionof{\{\edgehypercorewith\wcols\edge\in\preedgeset\}}{\catvariableof{\sedge\cap\redge}}
    \end{align*}
\end{theorem}
\begin{proof}
    We show this property by induction over the edge sets $\preedgeset$ to pairs $(\sedge,\redge)\in\dirovedges$, such that $\cardof{\preedgeset}\leq n$.
    Notice, that since always $\sedge\in\preedgeset$ we have $n\geq1$.

    $n=1$: In this case we have $\preedgeset=\{\sedge\}$ and $\sedge$ is a leaf of the tree-hypergraph $\graph$.
    The claimed message property holds thus by definition.

    $n\rightarrow n+1$: Let us assume, that the message obeys the claimed property for edge sets with cardinality up to $n$.
    If there is no edge set with cardinality $n+1$, the property holds also for those with cardinality up to $n+1$.
    If there is an edge set $\preedgeset$ with size $n+1$, we have
    \begin{align*}
        \preedgeset = \{\sedge\} \cup \left(\bigcup_{\secsedge \in\dirovedges} \preedgesetwrt{\secsedge}{\sedge}\right) \, .
    \end{align*}
    The message $\mesfromto{\sedge}{\redge}$ is sent, once all messages $\mesfromto{\secsedge}{\sedge}$ to $(\secsedge,\sedge)\in\dirovedges\{(\redge,\sedge)\}$ arrived.
    By definition we have
    \begin{align*}
        \mesfromtowith{\sedge}{\redge}%{\catvariableof{\sedge\cap \redge}}
        = \contractionof{\{\hypercoreofat{\sedge}{\catvariableof{\sedge}}\}
            \cup \{\mesfromtowith{\secsedge}{\sedge} \wcols (\secsedge,\sedge) \in\dirovedges \ncond \secsedge \neq \redge \}
        }{\catvariableof{\sedge\cap \redge}}
    \end{align*}
    Now we use the induction assumption on $\preedgesetwrt{\secsedge}{\sedge}$ (since its cardinality is at most $n$) and get
    \begin{align*}
        \mesfromtoat{\sedge}{\redge}{\catvariableof{\sedge\cap \redge}}
        &= \contractionof{
            \{\hypercoreofat{\sedge}{\catvariableof{\sedge}}\} \cup
            \left(\bigcup_{(\secsedge,\sedge)\in\dirovedges \ncond \secsedge\neq \redge}
                \contractionof{\{\hypercoreofat{\thirdsedge}{\catvariableof{\thirdsedge}} \wcols  \thirdsedge \in \preedgesetwrt{\secsedge}{\sedge}\}}{\catvariableof{\secsedge\cap \sedge}} \right)
        }{\catvariableof{\sedge\cap \redge}} \\
        &= \contractionof{
            \{\hypercoreofat{\sedge}{\catvariableof{\sedge}}\} \cup
            \left(\bigcup_{(\secsedge,\sedge)\in\dirovedges \ncond \secsedge\neq \redge} \{\hypercoreofat{\thirdsedge}{\catvariableof{\thirdsedge}} \wcols  \thirdsedge \in \preedgesetwrt{\secsedge}{\sedge}\} \right)
        }{\catvariableof{\sedge\cap \redge}} \\
        &= \contractionof{\{\edgehypercorewith\wcols\edge\in\preedgeset\}}{\catvariableof{\sedge\cap\redge}}
    \end{align*}
    Here we used the commutation of contraction property in the second equation, which is justified by the assumed tree property of the hypergraph.
    Thus, the message property holds also for any edge sets of size $n+1$.

    By induction, the claimed message property therefore holds for all final messages.
\end{proof}