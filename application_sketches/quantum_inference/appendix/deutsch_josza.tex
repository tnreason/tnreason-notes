\section{Deutsch-Josza Algorithm}

We so far worked mainly with basis encoding states, which are prepared by a Hadamard transform and the \computationCircuit{} on statistic qubits in the ground state $\fbasisat{\headvariable}$.
This ensured that the deterministic function relation is a hard constraint in the measurement distribution.
If we apply a \computationCircuit{} on a different initial state of the statistic qubits, this constraint is in general not satisfied any more.

\subsection{Walsh-Hadamard transform}

The application of Hadamard gates on the distributed variables afterwards results in a Walsh-Hadamard transform of the prepared state.
If and only if the function is constant, then the transformed state is the ground state.
The transform is performed by applying Hadamard gates on each variable
\begin{align*}
    \bigotimes_{\catenumeratorin} \hgateat{\catvariableof{\catenumerator},\catvariableof{\catenumerator}} \, .
\end{align*}

\subsubsection{Constant check}

In the Deutsch-Josza algorithm \cite{deutsch_rapid_1992}, the Walsh-Hadamard transform of a sign encoding of a boolean function is measured to determine whether the function is constant.
If $\exformula$ is constant, then the Walsh-Hadamard transform of the sign encoding results in the tensor product of the ground state $\onehotmapofat{0\ldots0}{\shortcatvariables}$ with a head qubit state.