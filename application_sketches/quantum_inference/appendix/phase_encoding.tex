\section{Phase Encoding}

We so far applied the \computationCircuit{} on statistic qubits in the ground state $\fbasisat{\headvariable}$.
This ensured that the deterministic function relation is a hard constraint in the measurement distribution.
If we apply a \computationCircuit{} on a different initial state of the statistic qubits, this constraint is in general not satisfied any more.

\subsection{Deutsch-Josza preparation}

In the Deutsch-Josza algorithm \cite{deutsch_rapid_1992}, the \computationCircuit{} is applied on intial statistic qubit states
\begin{align*}
    \frac{1}{\sqrt{2}} \left( \tbasisat{\headvariable} - \fbasisat{\headvariable} \right) = H \fbasisat{\headvariable} \, ,
\end{align*}
which can be prepared by applying a Hadamard gate $H$ on the state $\fbasisat{\headvariable}$.

Limitations:
\begin{itemize}
    \item The prepared state $\psi^{\exformula}$ is distinguished from the state $\psi^{\lnot\exformula}$ only by a non-observable phase factor
    \begin{align*}
        \psi^{\exformula} = (-1) \cdot \psi^{\lnot\exformula}
    \end{align*}
    \item With this construction the statistic qubit is disentangled with the distributed qubits.
Further manipulation of the statistic qubits alone (as we do with \activationCircuits{}) will not retrieve any information about the encoded function.
\end{itemize}


\subsection{Walsh-Hadamard transform}

The application of Hadamard gates on the distributed variables afterwards results in a Walsh-Hadamard transform of the prepared state.
If and only if the function is constant, then the transformed state is the ground state.
