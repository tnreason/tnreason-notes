\section{Quantum Computation Basics}

\subsection{State Encoding Schemes}

Basis Encoding in Quantum Computation refers to the representation of classical $n$ bit strings by $n$ qubit basis states, and is called one-hot encoding in \tnreason{}.
The Basis Encoding scheme in \tnreason{} goes beyond this scheme and also encodes subsets by sums of one-hot encodings to the members of the set.
In this way, relations and functions are represented by boolean tensors and contraction of them is refered as \BasisCalculus{}.

Amplitude Encoding in Quantum Computation refers to the storage of complex numbers in the amplitudes of quantum states.
The pendant in \tnreason{} is the Coordinate Encoding scheme, where real numbers are stored in the coordinates of real-valued tensors.
Compared to Amplitude Encoding, Coordinate Encoding does not have the normalization constraint of quantum states.
The Amplitude Encoding of the square root of a probability distribution is sometimes called q-sample.

\subsection{Controlled Single Qubit Gates}

We define the rotation gate around the Y-axis by an angle $\alpha$ as
\begin{align*}
    \yrotationofat{\alpha}{\avariableof{\insymbol},\avariableof{\outsymbol}} \coloneqq
    \begin{bmatrix}
        \cosof{\frac{\alpha}{2}} & -\sinof{\frac{\alpha}{2}} \\
        \sinof{\frac{\alpha}{2}} & \cosof{\frac{\alpha}{2}}
    \end{bmatrix}
\end{align*}

Further we define the Pauli-X:
\begin{align*}
    \paulixat{\avariableof{\insymbol},\avariableof{\outsymbol}} \coloneqq
    \begin{bmatrix}
        0 & 1 \\
        1 & 0
    \end{bmatrix}
\end{align*}

%% Control notation
Controlled single qubit gates are defined using control qubits, where the gate is applied to the target qubit if the control qubits are in a specific state and the identity is applied otherwise.
In the tensor network diagrams, we do not distinguish between incoming and outgoing control qubit variables, since the control acts as a Dirac tensor.
Thus, controlled unitary with target qubit $\catvariableof{t}$ and control qubits $\catvariableof{c}$ are represented by tensors
\begin{align*}
    \contunitaryat{\catvariableof{t,\insymbol},\catvariableof{t,\outsymbol},\catvariableof{c}}
\end{align*}
where for each state $\catindexof{c}$ to the control variables we have that
\begin{align*}
    \contunitaryat{\catvariableof{t,\insymbol},\catvariableof{t,\outsymbol},\indexedcatvariableof{c}}
\end{align*}
is a unitary matrix acting on the leg space of the target variable.

\subsection{Measurement and Phases}

The computational basis measurement of the qubits $\catvariableof{\variableset}$ of a Quantum State $\qstatewith$ is equal to drawing samples from a distribution
\begin{align*}
    \probat{\catvariableof{\variableset}} = \contractionof{
        \qstatewith, \comconqstatewith
    }{\catvariableof{\variableset}} \, .
\end{align*}
Here $\comconqstatewith$ is the complex conjugate of $\qstatewith$.
When $\qstate$ is prepared by a quantum circuit acting on a initial state, the complex conjugate is the hermitean conjugate of the circuit acting on the complex conjgate of the initial state.

%% Add drawing!
We abbreviate these contractions by extending the contraction diagrams with measurement symbols (see \figref{fig:measurementSketch}).
\begin{figure}
    \begin{center}
        \begin{tikzpicture}[scale=0.35,thick]

    \draw (0,0) rectangle (2,10);
    \node[anchor=center] (text) at (1,5) {$\qstate$};

    \draw (2,1) -- (4,1);

    \draw (2,9) -- (4,9) node[midway,above] {\colorlabelsize $\catvariableof{0}$};
    \drawqcmeasuresymbol{5}{9}
    \node[anchor=center] (text) at (3,8.25) {$\vdots$};
    \draw (2,6) -- (4,6) node[midway,above] {\colorlabelsize $\catvariableof{\atomenumerator}$};
    \drawqcmeasuresymbol{5}{6}

    \draw (2,4) -- (4,4) node[midway,above] {\colorlabelsize $\catvariableof{\atomenumerator+1}$};
    \node[anchor=center] (text) at (3,3.25) {$\vdots$};
    \draw (2,1) -- (4,1) node[midway,above] {\colorlabelsize $\catvariableof{\atomorder-1}$};

    \node[anchor=center] (text) at (8,5) {${=}$};

    \begin{scope}
        [shift={(10,0)}]
        \draw (0,0) rectangle (2,10);
        \node[anchor=center] (text) at (1,5) {$\qstate$};

        \draw (2,1) -- (4,1);

        \draw (2,9) -- (4,9) node[midway,above] {\colorlabelsize $\catvariableof{0}$};
        \node[anchor=center] (text) at (3,8.25) {$\vdots$};
        \draw (2,6) -- (4,6) node[midway,above] {\colorlabelsize $\catvariableof{\atomenumerator}$};

        \drawvariabledot{4}{9}
        \draw (4,9) -- (4,11);

        \drawvariabledot{6}{6}
        \draw (6,6) -- (6,11);

        \node[anchor=center] (text) at (5,10.25) {$\cdots$};

        \draw (4,9) -- (6,9);
        \draw (4,6) -- (6,6);
        \draw (4,4) -- (6,4);
        \draw (4,1) -- (6,1);

        \draw (2,4) -- (4,4) node[midway,above] {\colorlabelsize $\catvariableof{\atomenumerator+1}$};
        \node[anchor=center] (text) at (3,3.25) {$\vdots$};
        \draw (2,1) -- (4,1) node[midway,above] {\colorlabelsize $\catvariableof{\atomorder-1}$};


        \begin{scope}
            [xscale = -1, shift={(-10,0)}]
            \draw (0,0) rectangle (2,10);
            \node[anchor=center] (text) at (1,5) {$\comconqstate$};

            \draw (2,1) -- (4,1);

            \draw (2,9) -- (4,9) node[midway,above] {\colorlabelsize $\catvariableof{0}$};

            \node[anchor=center] (text) at (3,8.25) {$\vdots$};
            \draw (2,6) -- (4,6) node[midway,above] {\colorlabelsize $\catvariableof{\atomenumerator}$};


            \draw (2,4) -- (4,4) node[midway,above] {\colorlabelsize $\catvariableof{\atomenumerator+1}$};
            \node[anchor=center] (text) at (3,3.25) {$\vdots$};
            \draw (2,1) -- (4,1) node[midway,above] {\colorlabelsize $\catvariableof{\atomorder-1}$};
        \end{scope}

    \end{scope}

    \node[anchor=center] (text) at (23,5) {${=}$};

    \begin{scope}
        [shift={(25,0)}]
        \draw (0,0) rectangle (2,10);
        \node[anchor=center] (text) at (1,5) {$\absof{\qstate}^2$};

        \draw (2,1) -- (4,1);

        \draw (2,9) -- (4,9) node[midway,above] {\colorlabelsize $\catvariableof{0}$};
        \node[anchor=center] (text) at (3,8.25) {$\vdots$};
        \draw (2,6) -- (4,6) node[midway,above] {\colorlabelsize $\catvariableof{\atomenumerator}$};

        \draw (2,4) -- (4,4) node[midway,above] {\colorlabelsize $\catvariableof{\atomenumerator+1}$};
        \draw (4,3) rectangle (6,5);
        \node[anchor=center] (text) at (5,4) {$\ones$};
        \node[anchor=center] (text) at (3,3.25) {$\vdots$};
        \draw (2,1) -- (4,1) node[midway,above] {\colorlabelsize $\catvariableof{\atomorder-1}$};
        \draw (4,0) rectangle (6,2);
        \node[anchor=center] (text) at (5,1) {$\ones$};
    \end{scope}

\end{tikzpicture}
    \end{center}
    \caption{Computational Basis Measurement of a quantum state $\qstate$.
    The measurement symbols on the left side indicate the measured qubits and the first equation is understood as a definition.
    In the second equation we sketch, that the measurement distribution is equal to the contraction of the square absolute transform of $\qstate$ to the measured variables.
    }\label{fig:measurementSketch}
\end{figure}

%% Phase-Absolut decomposition
Each complex-valued tensor $\qstatewith$ has a decomposition into a phase tensor $\phasecorewith$ and an absolute tensor $\absof{\qstate}[\shortcatvariables]$ defined by
\begin{align*}
    \qstatewith = \contractionof{\expof{i\cdot\phasecorewith}, \absof{\qstate}[\shortcatvariables]}{\shortcatvariables}\, .
\end{align*}

The measurement distribution is depends only on $\absof{\phi}$, that is
\begin{align*}
    \probwith = \absof{\qstate}^2[\shortcatvariables] \, .
\end{align*}

Note, that when only a subset of variables is measured, the distribution is the contraction of the absolute square transform (these operations do not commute)
\begin{align*}
    \probat{\catvariableof{\variableset}} = \contractionof{\absof{\qstate}^2[\shortcatvariables]}{\catvariableof{\variableset}} \, .
\end{align*}

%% Phasecore vanishing as gauging
When we are interested in the preparation of quantum states with a specific computational basis measurement distribution, we can restrict to states with vanishing phase cores, that is
\begin{align*}
    \qstatewith
    = \contractionof{\expof{i\cdot\zerosat{\shortcatvariables}}, \absof{\qstate}[\shortcatvariables]}{\shortcatvariables}
    = \absof{\qstate}[\shortcatvariables]  \, .
\end{align*}


\subsection{Graph-Controlled circuits}



\begin{definition}[Graph-Controlled Circuit]
    Let $\graph=(\nodes,\edges)$ be a directed acyclic hypergraph, where each hyperedge has exactly one outgoing node and all nodes appear exactly once as outgoing nodes of an hyperedge.
    Then a by $\graph$ controlled circuit is a decoration of the edges $\edge=(\innodes,\{\node\})\in\edges$ by controlled unitaries
    \begin{align*}
        \contunitaryofat{\edge}{\catvariableof{\node,\insymbol},\catvariableof{\node,\outsymbol},\catvariableof{\innodes,\outsymbol}} \, .
    \end{align*}
\end{definition}

\begin{theorem}\label{the:graphControlledPreparesBN}
    Let $\graph$ be a directed acyclic graph.
    The measurement distributions of the by $\graph$ controlled circuits acting on disentangled initial states are equal to the Bayesian Networks on $\graph$.
\end{theorem}

\begin{lemma}
    Any Bayesian network on a directed acyclic graph $\graph$ can be prepared by a $\graph$-controlled circuit with activation circuits of the conditional probability tensors.
\end{lemma}
\begin{proof}
    Let $\probwith$ be a Bayesian network on the graph $\graph$.
    Enumerate the nodes $\nodes$ of the $\graph$ by $[\atomorder]$, such that for each $\catenumeratorin$ we have $\parentsof{\catenumerator}\subset[\catenumerator]$.
    Then define a $\graph$-controlled circuit, by choosing for each $\catenumeratorin$ controlled unitaries which satisfy
    \begin{align*}
        \contunitaryofat{\catenumerator}{\catvariableof{\catenumerator,\insymbol}=0,\catvariableof{\node,\outsymbol},\catvariableof{\parentsof{\catenumerator},\outsymbol}}
        =\sqrt{\condprobof{\catvariableof{\catenumerator}}{\catvariableof{\parentsof{\catenumerator}}}} \, .
    \end{align*}
    Here we specified only the action of the controlled unitary on the basis vector $\onehotmapofat{0}{\catvariableof{\catenumerator}}$, the action on $\onehotmapofat{1}{\catvariableof{\catenumerator}}$ can be chosen by an arbitrary orthogonal unit vector. % Maybe link here the activation circuit scheme, which is already used?
    Any such defined $\graph$-controlled circuit acting on the initial state $\bigotimes_{\catenumeratorin}\onehotmapofat{0}{\catvariableof{\catenumerator}}$ prepares a quantum state $\qstatewith$ with measurement distribution
    \begin{align*}
        \absof{\qstate}^2\left[\shortcatvariables\right] \, .
    \end{align*}
    Given arbitrary $\shortcatindicesin$ we have
        \begin{align*}
        \absof{\qstate}^2\left[\indexedshortcatvariables\right]
            = \prod_{\catenumeratorin} \condprobof{\indexedcatvariableof{\catenumerator}}{\indexedcatvariableof{\parentsof{\catenumerator}}}
            = \probat{\indexedshortcatvariables} \, .
    \end{align*}
    Here we used in the last equation, that $\probwith$ is a Bayesian network.
    Since the equivalence holds for any coordinate, this establishes the equivalence of the measurement distribution of $\qstatewith$ and $\probwith$.
\end{proof}


\begin{lemma}
    Let $(\graph,\contunitary)$ be a $\graph$-controlled circuit acting on a disentangled initial state and $\probat{\nodevariables}$ the corresponding measurement distribution.
    Then we have for each $\node\in\nodes$ the conditional independence
    \begin{align*}
        \condindependent{\catvariableof{\node}}{\catvariableof{\nondescendantsof{\node}}}{\catvariableof{\parentsof{\node}}} \, .
    \end{align*}
\end{lemma}
\begin{proof}
    We choose to a given $\node\in\nodes$ an enumeration $[\catorder]$ of the nodes, such that for each $\catenumeratorin$ we have $\parentsof{\catenumerator}\subset[\catenumerator]$ and for the enumerator $\seccatenumerator$ of $\node$ we further have $\nondescendantsof{\seccatenumerator}\subset[\seccatenumerator]$.
    Let $\probwith$ be the measurement distribution of the $\graph$-controlled circuit acting on a disentangled initial state $\bigotimes_{\catenumeratorin}\qstateofat{\catenumerator}{\catvariableof{\catenumerator}}$ and choose arbitrary $\catindexof{[\catenumerator]}$.
    We then have
    \begin{align*}
        \probat{\catvariableof{\seccatenumerator},\indexedcatvariableof{[\catenumerator]}}
        &= \contractionof{\left(\bigcup_{\catenumeratorin}\{\contunitaryof{\catenumerator},\contunitaryof{\catenumerator,\dagger},\qstateof{\catenumerator},\qstateof{\catenumerator,*}\right)\}
        \cup \left(\bigcup_{\catenumerator\in[\seccatenumerator]} \onehotmapofat{\catindexof{\catenumerator}}{\catvariableof{\catenumerator,\outsymbol}}\right)
        }{
            \catvariableof{\seccatenumerator}
        } \\
        &= \contractionof{\bigcup_{\catenumerator\in[\seccatenumerator]}\{\contunitaryof{\catenumerator},\contunitaryof{\catenumerator,\dagger},\qstateof{\catenumerator},\qstateof{\catenumerator,*},\onehotmapofat{\catindexof{\catenumerator}}{\catvariableof{\catenumerator,\outsymbol}}\}}{
            \catvariableof{\seccatenumerator}
        } \\
        & = \absof{\contractionof{\contunitaryofat{\seccatenumerator}{\catvariableof{\seccatenumerator,\insymbol},\catvariableof{\seccatenumerator,\outsymbol}},\indexedcatvariableof{\parentsof{\seccatenumerator},\outsymbol}}{\catvariableof{\seccatenumerator,\outsymbol}}}^2 \\
        & \quad \quad \cdot \prod_{\catenumerator\in[\seccatenumerator]}
        \left(\contraction{\qstateofat{\catenumerator}{\catvariableof{\catenumerator,\insymbol}},\contunitaryofat{\catenumerator}{\catvariableof{\catenumerator,\insymbol},\catvariableof{\catenumerator,\outsymbol},\indexedcatvariableof{\parentsof{\catenumerator},\outsymbol}}}\right)^2
        %\contraction{\qstateof{\catenumerator,*},\onehotmapofat{\catindexof{\catenumerator}}{\catvariableof{\catenumerator}}}{} \, .
    \end{align*}
    Here we used in the second equation the unitarity of the controlled unitaries to $\catenumerator\notin[\seccatenumerator]$.
    Since the indices $\catindexof{[\seccatenumerator]/\parentsof{\seccatenumerator}} = \catindexof{\nondescendantsof{\seccatenumerator}}$ appear only in the constant term, we conclude
    \begin{align*}
        \condprobat{\catvariableof{\seccatenumerator}}{\catvariableof{[\catenumerator]}} = \condprobat{\catvariableof{\seccatenumerator}}{\catvariableof{\parentsof{\seccatenumerator}}} \otimes \onesat{\catvariableof{\nondescendantsof{\seccatenumerator}}}\, ,
    \end{align*}
    which establishes the conditional independence $\condindependent{\catvariableof{\node}}{\catvariableof{\nondescendantsof{\node}}}{\catvariableof{\parentsof{\node}}}$.
\end{proof}

\begin{proof}[Proof of \theref{the:graphControlledPreparesBN}]
    The theorem follows directly from the two lemmas, using that Bayesian Networks are characterized by the conditional independence of each variable to its non-descendants given its parents.
\end{proof}


Another question is, whether each quantum state, which measurement distribution is a Bayesian Network can be prepared by a $\graph$-controlled circuit.
This is not the case, since the phase tensor of a by $\graph$-controlled circuit has a decomposition
\begin{align*}
    \phasecoreat{\shortcatvariables}
    = \sum_{\catenumeratorin} \phasecoreofat{\catenumerator}{\catvariableof{\catenumerator},\catvariableof{\parentsof{\catenumerator}}} \otimes \onesat{\catvariableof{[\catorder]/\{\{\catenumerator\}\cup\parentsof{\catenumerator}\}}} \, ,
\end{align*}
where the phase cores $\phasecoreof{\catenumerator}$ can be read of the controlled unitaries.