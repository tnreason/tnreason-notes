\section{Moment Amplification}

We here apply \computationCircuits{} on anti-symmetric ancilla states (as for sign encoding)
\begin{itemize}
    \item A single \computationCircuit{} effectively prepares a reflection across the models of the computed formula
    \item Building the reflections based on ancilla ground states (as for basis encoding), we would need an additional Pauli-Z and two \computationCircuits{}
    \item Further ancilla qubits used in circuit decompositions need to be uncomputed (i.e. by applying the same partial \computationCircuit{} again) for further usage
\end{itemize}


\subsection{Effective reflection by single \computationCircuits{}}

\red{Note, that the \computationCircuit{} is not a reflection, but acts as one when preparing the ancilla in the real anti-symmetric state.}

To start let
\begin{itemize}
    \item $\qstateofat{0}{\shortcatvariables}$ be a q-sample of a probability distribution $\probofat{0}{\shortcatvariables}$
    \item $\unitarysymbol$ be a unitary preparing the state
    \item $\qcbencodingof{\exformula}$ be a \computationCircuit{} to a formula $\exformula$
\end{itemize}

\subsubsection{Decomposition of the q-sample}

The we split the initial state into vectors supported at the models of $\exformula$ and the complement as
\begin{align*}
    \qstateofat{0}{\shortcatvariables}
    =\qstateofat{\goodstatesymbol}{\shortcatvariables}+\qstateofat{\badstatesymbol}{\shortcatvariables}
\end{align*}
where
\begin{align*}
    \qstateofat{\goodstatesymbol}{\shortcatvariables}
    &= \sum_{\shortcatindicesin\wcols\exformulaat{\indexedshortcatvariables}=1} \sqrt{\probat{\indexedshortcatvariables}}
    \cdot \onehotmapofat{\shortcatindices}{\shortcatvariables} \\
    \qstateofat{\badstatesymbol}{\shortcatvariables}
    &= \sum_{\shortcatindicesin\wcols\exformulaat{\indexedshortcatvariables}=0} \sqrt{\probat{\indexedshortcatvariables}}
    \cdot \onehotmapofat{\shortcatindices}{\shortcatvariables} \, .
\end{align*}

We further define for a boolean formula
\begin{align*}
    \qstateofat{\probof{0},\exformula}{\shortcatvariables}
    = \frac{\qstateofat{\goodstatesymbol}{\shortcatvariables}}{\normof{\qstateofat{\goodstatesymbol}{\shortcatvariables}}}
\end{align*}
and have consistently for its negation
\begin{align*}
    \qstateofat{\probof{0},\lnot\exformula}{\shortcatvariables}
    = \frac{\qstateofat{\badstatesymbol}{\shortcatvariables}}{\normof{\qstateofat{\badstatesymbol}{\shortcatvariables}}}
\end{align*}

We define an angle $\rotanglesymbol$ by
\begin{align*}
    \sinof{\frac{\rotanglesymbol}{2}} \coloneqq \normof{\qstateofat{\goodstatesymbol}{\shortcatvariables}}
\end{align*}
and have
\begin{align*}
    \qstateofat{0}{\shortcatvariables}
    = \sinof{\frac{\rotanglesymbol}{2}} \qstateofat{\probof{0},\exformula}{\shortcatvariables} +
    \cosof{\frac{\rotanglesymbol}{2}} \qstateofat{\probof{0},\exformula}{\shortcatvariables} \, .
\end{align*}
Further we have that
\begin{align*}
    \meanparam^{0}
    &= \contraction{\probofat{0}{\shortcatvariables},\formulawith} \\
    &= \sum_{\shortcatindicesin} \absof{\qstateofat{0}{\indexedshortcatvariables}\formulaat{\indexedshortcatvariables}}^2 \\
    &= \left(\|\qstateofat{\goodstatesymbol}{\shortcatvariables}\|_2\right)^2 \\
    &= \left(\sinof{\frac{\rotanglesymbol}{2}}\right)^2 \, .
\end{align*}

We prepare an ancilla qubit in the anti-symmetric state and consider as initial state
\begin{align*}
    \qstateofat{0}{\shortcatvariables} \otimes
    \sqrt{\frac{1}{2}}\cdot\coloredmatrixof{1\\-1}{\avariable} \,.
\end{align*}

The \computationCircuit{} $\qcbencodingof{\exformula}$ acting on this state prepares a state
\begin{align*}
    \left(\qstateofat{\badstatesymbol}{\shortcatvariables}-\qstateofat{\goodstatesymbol}{\shortcatvariables}\right) \otimes
    \sqrt{\frac{1}{2}}\cdot\coloredmatrixof{1\\-1}{\avariable}  \,.
\end{align*}
This is the reflection on the subspace spanned by the one-hot encodings of the models of $\exformula$.

\subsubsection{Decomposition into auxiliary statistic qubits}

% Decomposition Sparsity requires uncomputation of the ancilla qubits
We can exploit \decompositionSparsity{} to find a sparse representation of a \computationCircuit{} with further auxiliary statistic qubits.
Any auxiliary statistic qubits $\headvariable$ is prepared in the initial state $\fbasisat{\headvariable}$, only the head statistic qubit is prepared in the anti-symmetric state.
To prepare a sign encoding such that the auxiliary statistic qubits are disentangled, one has to uncompute all auxiliary statistic qubits, by applying the respective \computationCircuits{} with auxiliary target qubits again.


To be more precise, let $\headvariables$ be auxiliary statistic qubits representing the subformulas $\hlnstat$ in a syntactic decomposition of $\exformula$, applying a decomposed \computationCircuit{} on the initial state then gives
\begin{align*}
    \left(\sum_{\shortcatindicesin} (-1)^{\exformula(\shortcatindices)} \qstateofat{0}{\indexedshortcatvariables} \cdot \onehotmapofat{\shortcatindices}{\shortcatvariables} \otimes \onehotmapofat{\hlnstat(\shortcatindices)}{\headvariables}\right)
\end{align*}
Notice, that in this case the auxiliary qubits are entangled with the distributed qubits.
This can be resolved by applying the adjoint of the \computationCircuit{} except for those where the control is on the ancilla qubit.
We then have
\begin{align*}
    &\left(\sum_{\shortcatindicesin} (-1)^{\exformula(\shortcatindices)} \qstateofat{0}{\indexedshortcatvariables} \cdot \onehotmapofat{\shortcatindices}{\shortcatvariables} \right) \otimes \onehotmapofat{\hlnstat(\shortcatindices)\oplus \hlnstat(\shortcatindices)}{\headvariables} \\
    &\quad =\left(\sum_{\shortcatindicesin} (-1)^{\exformula(\shortcatindices)} \qstateofat{0}{\indexedshortcatvariables} \cdot \onehotmapofat{\shortcatindices}{\shortcatvariables} \right) \otimes \onehotmapofat{0_{[\seldim]}}{\headvariables}
\end{align*}

Alternatively, one can include the auxiliary statistic qubits into the distributed qubits, and understand the parts of the \computationCircuit{} not affecting the ancilla qubit as part of the initial state preparing unitary $\unitarysymbol$.
Note that this perspective has been applied in section \secref{sec:sampling}, where we took $\unitarysymbol$ as the \ComputationActivationCircuit{} preparing the q-sample of the ancilla augmentation.


\subsection{Rotation along a formula}

Rotating $\repnum$ times across the initial state then gives the tensor product of the antisymmetric ancilla state with
\begin{align*}
    \qstateofat{\repnum}{\shortcatvariables}
    = \sinof{\left(\frac{1}{2}+\repnum\right)\rotanglesymbol} \qstateofat{\probof{0},\exformula}{\shortcatvariables} +
    \cosof{\left(\frac{1}{2}+\repnum\right)\rotanglesymbol} \qstateofat{\probof{0},\lnot\exformula}{\shortcatvariables} \, .
\end{align*}
This reflection can be performed by the operator on the distributed qubits
\begin{align*}
    U^t \left(
            \identityat{\shortcatvariables^{\insymbol},\shortcatvariables^{\outsymbol}}-2\onehotmapofat{0}{\shortcatvariables^{\insymbol}} \otimes \onehotmapofat{0}{\shortcatvariables^{\outsymbol}}
    \right) U \, .
\end{align*}


\begin{figure}[t]
    \begin{center}
        \begin{tikzpicture}[>=stealth, scale=4]

    % Coordinates
    \coordinate (O) at (0,0);
    \coordinate (W) at (0,1);     % Target state |w>
    \coordinate (Sprime) at (1,0); % All other states |s'>

    % Draw Axes
    \draw[->, thick] (-0.1,0) -- (1.1,0) node[right] {\corelabelsize $\qstateof{\probof{0},\lnot\exformula}$};
    \draw[->, thick] (0,-0.1) -- (0,1.1) node[above] {\corelabelsize $\qstateof{\probof{0},\exformula}$};

    \def\sthalfangle{15}
    \def\nextangle{3*\sthalfangle} % 3 * theta
    \draw[->, thick] (O) -- (\sthalfangle:0.9) node[right] {\corelabelsize $\qstateof{0}=\sinof{\frac{\rotanglesymbol}{2}}\qstateof{\probof{0},\exformula}+\cosof{\frac{\rotanglesymbol}{2}}\qstateof{\probof{0},\lnot\exformula}$};
    \draw[dashed, thick] (0.4,0) arc (0:\sthalfangle:0.4) node[midway, right, xshift=2pt] {\corelabelsize $\frac{\rotanglesymbol}{2}$};

    \draw[<-, thick, blue] (-1*\sthalfangle:0.6) arc (-\sthalfangle:\sthalfangle:0.6) node[midway, right, yshift=-5pt, xshift=2pt] {\corelabelsize $\rotanglesymbol$};

    % Draw the Oracle Reflection (reflected across s' axis)
    \draw[->, thick, blue] (O) -- (-\sthalfangle:0.9) node[right] {\corelabelsize $-\sinof{\frac{\rotanglesymbol}{2}}\qstateof{\probof{0},\exformula}+\cosof{\frac{\rotanglesymbol}{2}}\qstateof{\probof{0},\lnot\exformula}$};


    \draw[->, thick, red] (-1*\sthalfangle:0.8) arc (-\sthalfangle:3*\sthalfangle:0.8) node[midway, right, yshift=15pt, xshift=-5pt] {\corelabelsize $2\rotanglesymbol$};

    % Draw the first Grover Iteration (rotated by 2*theta from original |s>)

    \draw[->, red, thick] (O) -- (\nextangle:0.9) node[above right] {\corelabelsize $\qstateof{1}=\sinof{\left(\frac{1}{2}+1\right)\rotanglesymbol}\qstateof{\probof{0},\exformula}+\cosof{\left(\frac{1}{2}+1\right)\rotanglesymbol}\qstateof{\probof{0},\lnot\exformula}$};

    \node[below left] at (O) {\corelabelsize $0$};

\end{tikzpicture}
    \end{center}
    \caption{Geometric visualization of the states $\qstateof{1}$ resulting from a Grover rotation on $\qstateof{0}$, in the subspace spanned by $\qstateof{\probof{0},\exformula}$ and $\qstateof{\probof{0},\lnot\exformula}$.
    The effective reflection by the \computationCircuit{} results in the gray state $\qstateof{\badstatesymbol}-\qstateof{\goodstatesymbol}$.
    Reflecting again on the initial state $\qstateof{0}$ results in the amplified state $\qstateof{1}$.
    Each such Grover iteration is therefore a rotation by the angle $\rotanglesymbol$ in the.}
\end{figure}


\subsubsection{Maximum Entropy}

% Q-sample
When $\left(\frac{1}{2}+\repnum\right)\rotanglesymbol\leq \frac{\pi}{2}$ then all amplitudes remain positive, and $\qstateofat{\repnum}{\shortcatvariables}$ is a q-sample.
What is more, we can show that the property of maximum entropy with respect to a statistic containing the amplified formula.

\begin{theorem}
    \label{the:rotationKeepsMaxEntropy}
    Let $\qstateof{0}$ be the q-sample of a maximum entropy distribution with respect to the uniform base measure and the statistic $\formulaset$ containing the amplified formula, and let $\repnum\in\nn$ be a rotation number such that $\left(\frac{1}{2}+\repnum\right)\rotanglesymbol\leq\frac{\pi}{2}$.
    Then also $\qstateof{\repnum}$ is the q-sample of a maximum entropy distribution.
\end{theorem}

We show \theref{the:rotationKeepsMaxEntropy} later based on a more technical description of the rotated state, which we show as the next lemma.

\begin{lemma}
    \label{lem:rotationCanParamIncrease}
    Let $\qstateof{0}$ be the q-sample of the elementary \ComputationActivationNetwork{} with canonical parameters $\hybridparam$, and let us amplify the formula $\formulaof{\selindex}$.
    Then $\qstateof{\repnum}$ is the q-sample of the elementary \ComputationActivationNetwork{} with canonical parameters
    \begin{align*}
        \hardlegset^{\repnum} =
        \begin{cases}
            \hardlegset \cup \{\selindex\}, \headindexof{\hardlegset}^{\selindex} = (\headindexof{\hardlegset},1) & \ifspace \sinof{(1+2\cdot\repnum)\sin^{-1}\left(\sqrt{\meanparamofat{0}{\indexedselvariable}}\right)} = 1 \\
            \hardlegset, \headindexof{\hardlegset} & \elsetext
        \end{cases}
    \end{align*}
    and
    \begin{align*}
        \canparamofat{\repnum}{\indexedselvariable} =
        \begin{cases}
            0 & \ifspace  \sinof{(1+2\cdot\repnum)\sin^{-1}\left(\sqrt{\meanparamofat{0}{\indexedselvariable}}\right)} = 1 \\
            \canparamat{\indexedselvariable} + \frac{
                \cos^2\left(\frac{\rotanglesymbol}{2}\right) \left(1-
                                                                 \cos^2\left(\left(\frac{1}{2}+\repnum\right) \rotanglesymbol \right)
                \right)
            }{
                \sin^2\left(\frac{\rotanglesymbol}{2}\right)
                \cdot \cos^2\left(\left(\frac{1}{2}+\repnum\right)\rotanglesymbol\right)
            } & \elsetext
        \end{cases} \, .
    \end{align*}
\end{lemma}
\begin{proof}
    Notice, that
    \begin{align*}
        \contractionof{\probofat{0}{\shortcatvariables},\bencodingofat{\formulaof{\selindex}}{\headvariableof{\selindex},\shortcatvariables}}{\headvariableof{\selindex}}
        = \coloredmatrixof{
            \left(\cosof{\frac{\rotanglesymbol}{2}}\right)^2 \\
            \left(\sinof{\frac{\rotanglesymbol}{2}}\right)^2
        }{\headvariableof{\selindex}}
        = \coloredmatrixof{
            1-\meanparamofat{0}{\indexedselvariable} \\
            \meanparamofat{0}{\indexedselvariable}
        }{\headvariableof{\selindex}}
    \end{align*}
    and
    \begin{align*}
        \contractionof{\probofat{\repnum}{\shortcatvariables},\bencodingofat{\formulaof{\selindex}}{\headvariableof{\selindex},\shortcatvariables}}{\headvariableof{\selindex}}
        =
        \coloredmatrixof{
            \left(\cosof{\left(\frac{1}{2}+\repnum\right)\rotanglesymbol}\right)^2 \\
            \left(\sinof{\left(\frac{1}{2}+\repnum\right)\rotanglesymbol}\right)^2
        }{\headvariableof{\selindex}}
        \coloredmatrixof{
            1-\meanparamofat{\repnum}{\indexedselvariable} \\
            \meanparamofat{\repnum}{\indexedselvariable}
        }{\headvariableof{\selindex}}
    \end{align*}
    If $\meanparamofat{\repnum}{\indexedselvariable}=1$ (that is $\sinof{\left(\frac{1}{2}+\repnum\right)\rotanglesymbol}=1$), the amplitude amplification amounts to adding $\formulaof{\selindex}$ as a hard constraint.
    This is equal to $\hardlegset^{\repnum}=\hardlegset\cup\{\selindex\}$ and $\headindexof{\hardlegset}^{\repnum}=(\headindexof{\hardlegset},1)$.

    If $\meanparamofat{\repnum}{\indexedselvariable}\neq 1$ we choose $\canparamofat{\Delta}{\indexedselvariable}\geq 0$ as in the claim and have after some algebra
    \begin{align*}
        \frac{
            \left(\cosof{\frac{\rotanglesymbol}{2}}\right)^2
        }{
            \left(\cosof{\frac{\rotanglesymbol}{2}}\right)^2 + \canparamofat{\Delta}{\indexedselvariable} \cdot \left(\sinof{\frac{\rotanglesymbol}{2}}\right)^2
        } = \left(\cosof{\left(\frac{1}{2}+\repnum\right)\rotanglesymbol}\right)^2
    \end{align*}
    and therefore
    \begin{align*}
        \contractionof{\probofat{\repnum}{\shortcatvariables},\bencodingofat{\formulaof{\selindex}}{\headvariableof{\selindex},\shortcatvariables}}{\headvariableof{\selindex}}
        = \normalizationof{
            \expof{\canparamofat{\Delta}{\indexedselvariable}\cdot\formulaofat{\selindex}{\canparam}}
            \probofat{0}{\shortcatvariables},\bencodingofat{\formulaof{\selindex}}{\headvariableof{\selindex},\shortcatvariables}
        }{\headvariableof{\selindex}} \, .
    \end{align*}
    Thus, the rotation of the q-sample corresponds with an increase of the canonical parameter $\canparamofat{0}{\indexedselvariable}$ by $\canparamofat{\Delta}{\indexedselvariable}$.
\end{proof}

\begin{proof}[Proof of \theref{the:rotationKeepsMaxEntropy}]
    By \lemref{lem:rotationCanParamIncrease} the amplitude amplified state is an elementary \ComputationActivationNetwork{} and thus a maximum entropy distribution.
        %   Since the measurement distribution of any quantum state in the plane spanned by $\qstateof{\goodstatesymbol}$ and $\qstateof{\badstatesymbol}$ is an elementary \ComputationActivationNetwork{}.
        %   When the assumption on $\repnum$ is met, we further have positive real amplitudes and therefore a q-sample.
        %  Since each amplitude amplification can be understood as an change of mean parameters.
        %  To be more precise, we characterize the change of canonical parameters in the lemma below.
\end{proof}

%
Notice, that we can only increase the canonical parameter by amplitude amplification.
Decreasing could be done by amplitude amplification on $\lnot\formulaof{\selindex}$ instead.
