\section{Moment Matching} %based preparation of \CompActNets{}

We now investigate which \ComputationActivationNetworks{} can be prepared directly with amplitude amplification.
While we already have studied sparse representations of the computation network parts, we here apply amplitude amplification by Grover rotations as an activation mechanism.

We orient on the iterative proportional fitting algorithm, which is used to estimate the canonical parameters in exponential families  \cite{wainwright_graphical_2008} and more generally of \ComputationActivationNetworks{} \cite{goessmann_tensor-network_2025}.

The algorithm loops over the moment estimation and moment amplification subroutines, which have been introduced in the sections above.
The idea is to first compute the current moments of a distribution and second amplify one.


% As exponential family parameters
The mean parameter of $\exformula$ with respect to the measurement distribution is
\begin{align*}
    \meanparam^{\repnum}
    = \contraction{\probofat{\repnum}{\shortcatvariables},\formulawith}
    = \sinof{\left(\frac{1}{2}+\repnum\right)\rotanglesymbol}^2
    = \sinof{\left(1+2\repnum\right)\sin^{-1}\left(\sqrt{\meanparam^{0}}\right)}^2
    %= \left(\frac{
    %    2\cdot \sinof{\left(\frac{1}{2}+\repnum\right)\rotanglesymbol} \cdot \sin^{-1}\left(\sqrt{\meanparam^{0}}\right)
    %}{\meanparam^{0}}\right)^2
\end{align*}
Here we used that $\rotanglesymbol = 2\cdot\sin^{-1}\left(\sqrt{\meanparam^{0}}\right)$.

%This change of mean parameters can be understood by increasing the canonical parameter to $\exformula$ by
%\begin{align*}
%    \canparamof{\repnum} - \canparamof{0}
%    = \frac{
%        \cos^2\left(\frac{\rotanglesymbol}{2}\right) \left(1-
%                                                         \cos^2\left(\left(\frac{1}{2}+\repnum\right) \rotanglesymbol \right)
%        \right)
%    }{
%        \sin^2\left(\frac{\rotanglesymbol}{2}\right)
%        \cdot \cos^2\left(\left(\frac{1}{2}+\repnum\right)\rotanglesymbol\right)
%    }
%\end{align*}
%we here allow for $\canparam=\infty$, which corresponds to the case of a hard activation to $\exformula$.

We understand this rotation as a moment matching operation to the formula $\exformula$, see \algoref{alg:QCMM}.

\begin{algorithm}
    \caption{Quantum Circuit Moment Matcher}\label{alg:QCMM}
    \begin{algorithmic}
        \Require Formulas $\formulaofat{\seldim}{\shortcatvariables}$ for $\selindexin$, vector $\datameanat{\selvariable}$ of mean parameters
        \Ensure Circuit $\unitarysymbol$ preparing a state (when applied on the ground state) which measurement distribution matched $\datameanat{\selvariable}$\\
        \hrule
        \State $U\algdefsymbol (\paulixsymbol\circ\hgate)\left[\avariableof{\insymbol},\avariableof{\outsymbol}\right] \otimes
        \left(\bigotimes_{\catenumeratorin}\hgateat{\catvariableof{\catenumerator,\insymbol},\catvariableof{\catenumerator,\outsymbol}}\right)
        $
        \While{Convergence criterion not met}
%        \For{$\selindexin$}
            \State Estimate $\meanparamat{\selvariable}$,
            \begin{itemize}
                \item either by particle bases inference
                \item or by quantum counting (measuring the eigenvalue of the Grover operator), see \secref{sec:amplitudeEstimation}
            \end{itemize}
            \State Select an $\selindexin$, which moment has to be updated (e.g. by comparison of $\meanparamat{\selvariable}$ with $\datameanat{\selvariable}$)
            \State Choose whether to amplify $\formulaof{\selindex}$ or $\lnot\formulaof{\selindex}$ based on whether the estimated mean is smaller than the target mean
            \State Compute optimal $\repnum$ to match $\datameanat{\selvariable}$
            \State Extend Circuit $\unitarysymbol$ by $\repnum$ alternations of
            \begin{itemize}
                \item $\qcbencodingof{\formulaof{\selindex}}$ (effectively an reflection across the subspace spanned by one-hot encoded models)
                \item reflections across the current state (by $U(I-2e_0e_0)U$)
            \end{itemize}
        \EndWhile
    \end{algorithmic}
\end{algorithm}


% Outook
We can use \algoref{alg:QCMM} as a preparation algorithm of a q-sample before doing ancilla augmentation as in the main section.
Note that is important to keep track of the canonical parameters in the preparation.
When activating the statistic qubits, only the difference of the already prepared canonical parameters and the target canonical parameters has to be used (importance sampling interpretation).