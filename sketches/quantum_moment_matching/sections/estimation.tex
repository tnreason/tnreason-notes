\section{Moment Estimation}\label{sec:amplitudeEstimation}

Given a distribution $\probwith$ and a propositional formula $\formulaat{\shortcatvariables}$, we now investigate how to use Quantum Phase Estimation to estimate
\begin{align*}
    \meanparam
    = \contraction{\probat{\shortcatvariables},\formulaat{\shortcatvariables}} \, .
\end{align*}

We assume that
\begin{itemize}
    \item a circuit $\unitarysymbol$ preparing a q-sample of $\probwith$
    \item a \computationCircuit{} for $\formulawith$
\end{itemize}

\subsection{Eigenvalues the formula-amplifying Grover operator}

The Grover operator $\groverofat{\probtensor,\formula}{\shortcatvariables^{\insymbol},\shortcatvariables^{\outsymbol},\avariable^{\insymbol},\avariable^{\outsymbol}}$ amplifying $\formula$
\begin{align*}
    (\unitaryof{\textdagger}\circ\reflectionof{0}\circ \unitarysymbol) \otimes \qcbencodingof{\exformula}
\end{align*}
acting on the initial state
\begin{align*}
    \onehotmapofat{0}{\shortcatvariables} \otimes \sqrt{\frac{1}{2}}\cdot\coloredmatrixof{1\\-1}{\avariable} \, .
\end{align*}

% Plane projection
We describe the Grover operator by its action on the projection onto the plane
\begin{align*}
    \spanof{\qstateof{\probtensor,\formula},\qstateof{\probtensor,\lnot\formula}}
\end{align*}
as
\begin{align*}
    \coloredmatrixof{\cosof{\rotanglesymbol} & \sinof{\rotanglesymbol} \\
    -\sinof{\rotanglesymbol} & \cosof{\rotanglesymbol}}{
        \indvariableof{\insymbol},\indvariableof{\outsymbol}
    } \, .
\end{align*}
Here we introduce a boolean variables $\indvariable$ to select the two axis on the plane, and copy it to $\indvariableof{\insymbol},\indvariableof{\outsymbol}$.
The rotation angle is related to $\meanparam$ as
\begin{align*}
    \rotanglesymbol = 2 \cdot \sin^{-1}\left(\sqrt{\meanparam}\right)
\end{align*}

Its eigenvalues are thus
\begin{align*}
    \expof{\pm i\rotanglesymbol}
\end{align*}
to the eigenstates
\begin{align*}
    \frac{1}{2}
    \coloredmatrixof{1\\\pm i}{\indvariable} \, .
\end{align*}

\subsection{Quantum Phase Estimation}

We prepare auxiliary qubits $\avariables$ and perform for each $\selindexin$ $2^{\selindex}$ repetitions of $\groverof{\probtensor,\exformula}$ controlled on $\avariableof{\selindex}$.
The inverse Fourier transform on the auxiliary qubits is supported on the (2-adic representation of the) phases on the eigenvalues, i.e. $\pm\rotanglesymbol$, and can be detected by a computational basis measure of the ancilla variables.

