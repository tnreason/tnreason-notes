\section{Rung 2: Intervention}

So far, there are several hypergraphs to model a probability distributions by a Bayesian Network.
We now want to understand the directed hyperedges as causal relationships between the variables, beyond encoding of conditional independencies (see \exaref{exa:patientTreatmentOutcome}).
To this end, we model the effects of interventions, which will distinguish between multiple Bayesian Networks modeling the same joint distribution.

\begin{example}[Patient treatment outcome]\label{exa:patientTreatmentOutcome}
    We model a relationship about the treatment of a sick patient and the outcome of the treatment.
    Whether a treatment is conducted is modelled by the binary variable $\catvariableof{T}$ and whether the patient improves after the treatment by the variable $\catvariableof{0}$.

    There are in general two possibilities of representing the joint distribution of $\catvariableof{T}$ and $\catvariableof{O}$ by a Bayesian Network:
    \begin{itemize}
        \item Treatment directs to outcome:
        \begin{align*}
            \graph = \left(\{\catvariableof{T},\catvariableof{O}\},\{(\varnothing,\catvariableof{T}),(\{\catvariableof{T}\},\{\catvariableof{O}\})\}\right)
        \end{align*}
        \item Outcome directs to treatment:
        \begin{align*}
            \graph = \left(\{\catvariableof{T},\catvariableof{O}\},\{(\varnothing,\catvariableof{T}),(\{\catvariableof{O}\},\{\catvariableof{T}\})\}\right)
        \end{align*}
    \end{itemize}
    In the typical intuition, we expect the treatment to have a causal effect on the outcome and not the other way round.
    From a Bayesian Network perspective, both models are however equivalent and represent the same joint distribution.
\end{example}

Interventions on Bayesian Networks are modeled by the do-calculus \cite{pearl_causality_2009}, which we capture here based on do-variables.

\subsection{Do calculus}

We introduce the do-variables $\dovariableof{\catenumerator}$ taking values in $[\catdimof{\catenumerator}+1]$ for each variable $\catvariableof{\catenumerator}$ of dimension $\catdimof{\catenumerator}$, which model the effects of interventions.
The states are interpreted as
\begin{itemize}
    \item $\doindexof{\catenumerator}\in[\catdimof{\catenumerator}]$: An intervention setting $\catvariableof{\catenumerator}$ to the value $\doindexof{\catenumerator}$
    \item $\doindexof{\catenumerator}=\catdimof{\catenumerator}$: No intervention on $\catvariableof{\catenumerator}$
\end{itemize}

\begin{definition}
    The causally augmented conditional probability core is defined as
    \begin{align*}
        \condprobwrtof{\causalsymbol}{\catvariableof{\atomenumerator}}{\dovariableof{\catenumerator},\catvariableof{\parentsof{\catenumerator}}}
        =& \condprobof{\catvariableof{\catenumerator}}{\catvariableof{\parentsof{\catenumerator}}} \otimes \onehotmapofat{\catdimof{\catenumerator}}{\dovariableof{\catenumerator}} \\
        &+ \sum_{\doindexof{\catenumerator}\in[\catdimof{\catenumerator}]} \onehotmapofat{\doindexof{\catenumerator}}{\catvariableof{\catenumerator}} \otimes \onehotmapofat{\doindexof{\catenumerator}}{\dovariableof{\catenumerator}} \otimes \onesat{\catvariableof{\parentsof{\catenumerator}}}\, .
    \end{align*}
    The causally augmented Bayesian Network is the tensor network
    \begin{align*}
        \contractionof{
            \{\condprobwrtof{\causalsymbol}{\catvariableof{\catenumerator}}{\dovariableof{\catenumerator},\catvariableof{\parentsof{\catenumerator}}} \wcols \catenumeratorin\}
        }{\shortcatvariables,\dovariableof{[\catorder]}} \, .
    \end{align*}
\end{definition}

The Bayesian Networks investigated so far are retrieved by contractions with $\onehotmapofat{\catdimof{\catenumerator}}{\dovariableof{\catenumerator}}$.
This corresponds with the situation, where no interventions are performed on any variable.

\begin{lemma}
    For any Bayesian Network $\probwith$ on a directed acyclic hypergraph $\graph=([\catorder],\edges)$ we have
    \begin{align*}
        \probwith =
        \contractionof{
            \{\condprobwrtof{\causalsymbol}{\catvariableof{\catenumerator}}{\dovariableof{\catenumerator},\catvariableof{\parentsof{\catenumerator}}} \wcols \catenumeratorin\}
            \cup \{\onehotmapofat{\catdimof{\catenumerator}}{\dovariableof{\catenumerator}} \wcols \catenumeratorin \}
        }{\shortcatvariables} \, .
    \end{align*}
\end{lemma}
\begin{proof}
    By construction we have
    \begin{align*}
        \condprobwrtof{\causalsymbol}{\catvariableof{\atomenumerator}}{\dovariableof{\catenumerator}=\catdimof{\catenumerator},\catvariableof{\parentsof{\catenumerator}}}
        =& \condprobof{\catvariableof{\catenumerator}}{\catvariableof{\parentsof{\catenumerator}}}  \, .
    \end{align*}
    Performing this on each causally augmented conditional distribution retrieves thus the original Bayesian Network.
\end{proof}

\subsection{Intervention Queries}

In a Bayesian Network, when intervening on a variable $\catvariableof{\node}$, the corresponding conditional probability gets trivialized.
The probability tensor is then captured by
\begin{align*}
    \condprobat{\catvariableof{\secnodes}}{\doof{\indexedcatvariableof{\thirdnodes}}}
    %  \condprobat{\catvariableof{\secnodes}}{\doof{X}}
    = \contractionof{
        \{\condprobof{\catvariableof{\node}}{\catvariableof{\parentsof{\node}}} \,:\, \node\notin\thirdnodes \}
        \cup \{\frac{1}{\catdimof{\node}} \onesat{\catvariableof{\node},\catvariableof{\parentsof{\node}}} \,:\, \node\in\thirdnodes\}
        \cup \{\onehotmapofat{\catindexof{\node}}{\catvariableof{\node}} \,:\, \node\in\thirdnodes\}
    }{\catvariableof{\secnodes}} \, .
\end{align*}

Since $\frac{1}{\catdimof{\node}} \onesat{\catvariableof{\node},\catvariableof{\parentsof{\node}}}$ is directed with $\catvariableof{\parentsof{\node}}$ incoming and $\catvariableof{\node}$ outgoing, the partition function of the network stays $1$ and the normalization is captured by the contraction.

%    \subsection{Intervention Variables}
%
%    We modify conditional probability cores in Bayesian Networks to capture interventions, by introducing intervention variable
%
%    \begin{align*}
%        \hypercoreat{\catvariableof{\node},\catvariableof{\parentsof{\node}},\dovariableof{\node}}
%        = \condprobat{\catvariableof{\node}}{\catvariableof{\parentsof{\node}}} \otimes \onehotmapofat{0}{\dovariableof{\node}}
%        + \onesat{\catvariableof{\node},\catvariableof{\parentsof{\node}}} \otimes \tbasisat{\dovariableof{\node}}
%    \end{align*}

\subsection{Simplifications}

The back-door and front-door criterion provide conditions for intervention queries being equal to conditional queries.
We can proof them in the tensor network formalism based on network separations, which contribution to contractions are scalar multiplications, which therefore are dropped in normalizations.

An example for a simplification is given by the randomized controlled trial.

\begin{example}[Randomized controlled trial]
    The randomized controlled trial is a joint distribution of variables
    \begin{itemize}
        \item Confounding variable $\catvariableof{C}$ summarizing other factors affecting the outcome.
        \item Treatment assignment $\catvariableof{T}$ with values in $\{0,1\}$.
        Here, $1$ indicates assignment to the treatment group and $0$ to the control group.
        The experiment is conducted such that the treatment variable is independent of the confounding variable $\catvariableof{C}$.
        \item Outcome $\catvariableof{O}$ .
    \end{itemize}
    We want to estimate the causal effect of the treatment on the outcome, namely calculate the intervention query
    \begin{align*}
        \condprobat{\catvariableof{O}}{\dovariableof{T}=1} \, .
    \end{align*}
    By design of the randomized control, the confounding variable has no causal effect on $\catvariableof{T}$.
    Thus, we have
    \begin{align*}
        \condprobat{\catvariableof{O}}{\dovariableof{T}=1}
        = \condprobat{\catvariableof{O}}{\catvariableof{T}=1}
    \end{align*}
    that is, the intervention query is equal to the conditional probability.
\end{example}