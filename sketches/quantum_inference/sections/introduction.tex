\section{Introduction}

By its central axioms, quantum mechanics of multiple qubits is formulated by tensors capturing states and discrete time evolutions.
Motivated by the structural similarity, we investigate how quantum circuits can be utilized for the tensor-network based approach towards efficient and explainable AI in the \tnreason{} framework.

We follow two main ideas:
\begin{itemize}
    \item \textbf{Sampling of \ComputationActivationNetworks{}:} Prepare quantum states, which measurement statistics can be utilized to prepare samples from \ComputationActivationNetworks{}.
    \item \textbf{Quantum Circuits as Contraction providers:} Quantum circuits are contractions of multiple tensors and therefore tensor networks, and measurement probabilities are given by contractions.
    Here we investigate how we can exploit these as contraction provider for \tnreason{}.
\end{itemize}

Main approach: Quantum Inference scheme on Bayesian networks \cite{low_quantum_2014}
\begin{itemize}
    \item Extend to more general tensor networks than Bayesian networks: \ComputationActivationNetworks{}
\end{itemize}

Further literature:
\begin{itemize}
    \item \cite{schuld_machine_2021}: Sect 7.3.2 Reviewing the paper \cite{low_quantum_2014} as an application of fault-tolerant quantum computing
    \item \cite{wittek_quantum_2017}: Review of Markov Logic Network sampling (which are a special case of \ComputationActivationNetworks{})
\end{itemize}

Potential Advantage: \emph{Quantum Parallelism} (see \cite[Section 3.2.5]{schuld_machine_2021}).
\begin{itemize}
    \item Evaluation of multiple function values by single circuit evaluation.
    \item Contraction perspective: Loop-tolerant efficient contractions.
    \item However: Need a generic encoding scheme to exploit this advantage, which is not known yet.
    \item We here only provide a scheme based on post-selection, which provides a quantum advantage only through amplitude amplification.
    Without amplitude amplification and post-selection of samples, the encoded distribution is always uniform.
\end{itemize}
Comparison with classical side, which we can call \emph{Tensor Parallelism}:
Message-passing schemes for efficient contractions, but exact in limited cases.


\subsection{Related Works}

Circuit preparing schemes based on approximation:
\begin{itemize}
    \item Q-Alchemy \cite{araujo_low-rank_2023}, Q-Tucker CITE
    \item Tensor-Network Optimization based (alternating schemes) \cite{rudolph_synergistic_2023,rudolph_decomposition_2023}
\end{itemize}

Exact circuit preparing schemes for distributions:
\begin{itemize}
    \item Grover-Rudolph \cite{grover_creating_2002}, but no quantum advantage \cite{herbert_no_2021}
\end{itemize}

Circuit simulation:
Since \qcreason{} can prepare quantum circuits to arbitrary tensor networks, it can also be used to simulate quantum circuits (with an overhead!).
\begin{itemize}
    \item \cite{sander_large-scale_2025,sander_quantum_2025}
\end{itemize}

Quantum algorithms for linear algebra:
\begin{itemize}
    \item Contraction by SWAP and Hadamard test (see Appendix)
    \item HHL algorithm to solve linear equations \cite{harrow_quantum_2009}
\end{itemize}