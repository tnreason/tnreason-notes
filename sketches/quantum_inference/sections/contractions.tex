\section{Computation of Contractions}

We here study, how contractions of boolean formulas can be performed by circuits and read off by measurements.
We develop a theory of \emph{Quantum Parallelism} based on these contractions.


\subsection{Deutsch-Josza Algorithm}

We so far worked mainly with basis encoding states, which are prepared by a Hadamard transform and the \computationCircuit{} on statistic qubits in the ground state $\fbasisat{\headvariable}$.
This ensured that the deterministic function relation is a hard constraint in the measurement distribution.
If we apply a \computationCircuit{} on a different initial state of the statistic qubits, this constraint is in general not satisfied any more.

In the Deutsch-Josza algorithm \cite{deutsch_rapid_1992}, the Walsh-Hadamard transform of a sign encoding of a boolean function is measured to determine whether the function is constant.
If $\exformula$ is constant, then the Walsh-Hadamard transform of the sign encoding results in the tensor product of the ground state $\onehotmapofat{0\ldots0}{\shortcatvariables}$ with a head qubit state.

%% Application: Check equivalence of two formulas
Concatenating the \computationCircuit{} to formulas $\formula$ and $\secformula$ prepares the \computationCircuit{} of the formula $\formula\oplus\secformula$.
Both are equal or negations of each other, if and only if the distributed qubits are disentangled from the head qubit.
Doing the Deutsch-Josza test on $\formula\oplus\secformula$ thus decides whether two formulas are equal or negations of each other.

\red{While the Deutsch-Josza Algorithm, }

\subsection{Inversion Test}

%% Inversion test
\red{The inversion test is checking whether $\basqstateof{\exformula\oplus\secexformula}$ is (parallel to) uniform, using a Walsh-Hadamard transform.
This is similar to the Deutsch-Josza algorithm, with the difference that the basis encoding instead of the sign encoding is used.
}

For two formulas $\exformula,\secexformula$ we have (see \lemref{lem:basQstateOverlap})
\begin{align*}
    \contraction{\basqstateof{\exformula},\basqstateof{\secexformula}}
    = \frac{1}{2^{\atomorder}} \cdot \cardof{\left\{\shortcatindices\in\atomstates \wcols \exformulaat{\indexedshortcatvariables}=\secexformula\left[\indexedshortcatvariables\right]\right\}}
    = \frac{1}{2^{\atomorder}} \cdot \contraction{\exformulaat{\shortcatvariables},\secexformula\left[\shortcatvariables\right]}\, .
\end{align*}

Since we have by the \computationCircuit{} a circuit preparing the basis encoding states $\basqstateof{\exformula}$ and $\basqstateof{\secexformula}$ we can apply the inversion test to investigate their overlap.


The inversion test is a quantum circuit, being a composition of the basis encoding circuits of two formulas $\exformula,\secexformula$ and their inverses, accompanied by Hadamard gates on all qubits before and after the circuit application.
The probability of measuring the ground state $0,\ldots,0,0$ in a computational basis measurement is then
\begin{align*}
    \frac{1}{2^{2\atomorder}} \cdot \contraction{\exformulaat{\shortcatvariables},\secexformulaat{\shortcatvariables}} \, .
\end{align*}
Further, the probability of measuring the state $0,\ldots,0,1$ is
\begin{align*}
    \frac{1}{2^{2\atomorder}} \cdot \contraction{\exformulaat{\shortcatvariables},\lnot\secexformulaat{\shortcatvariables}} \, .
\end{align*}
In such way, the contraction of two boolean formulas can be estimated by the rate of the ground state in a computational basis measurement.

\subsection{Further Overlap-measuring Circuits}

Quantum Circuits such as the SWAP test and the Hadamard test (\cite[Section 3.6.1]{schuld_machine_2021}) can be used to measure overlaps of quantum states, which are the squared absolutes of contractions of two state tensors.
\begin{itemize}
    \item When we have preparation schemes for two tensors, we can control them with a common ancilla qubit and measure their contraction by a Hadamard test (alternatively, using the SWAP test and state preparation in two registers).
    \item Can we extend these schemes to contractions of more general tensor networks?
\end{itemize}