\section{Sampling from \ComputationActivationNetworks{}}

We investigate, how the above circuit encoding schemes can be applied in the preparation of states, which computational basis measurements are samples from specific distributions.

In particular, we build graph-controlled circuits being compositions of \computationCircuits{} and \activationCircuits{}, for which specific conditional distributions coincide with \ComputationActivationNetworks{}.

\subsection{Preparing ancilla augmented distributions for \ComputationActivationNetworks{}}

The computation network consists already of directed cores and therefore is already a Bayesian network.
The activation network however needs ancilla augmentation.
Therefore we have:
\begin{itemize}
    \item Pendant for Coordinate Encoding in \tnreason{}: Amplitude Encoding, storing the function value in the amplitude of an ancilla qubit.
    This is realized by an \textbf{\ActivationCircuit{}} acting on an ancilla qubit in the ground state.
    \item Pendant for Basis Encoding in \tnreason{}: \textbf{\ComputationCircuit{}}, with composition by contraction property.
    Applied on the ground state, the \computationCircuit{} generates the basis encoding quantum state, which is parallel to the basis encoding.
\end{itemize}
Both are defined using controlled single qubit gates (see Sections 4.2-3 in \cite{nielsen_quantum_2010}) with ancilla qubits being the target qubits. % where the incoming qubit variable is $\avariableof{\insymbol}$ and the outgoing $\avariableof{\outsymbol}$.

\begin{figure}[t]
    \begin{center}
        \begin{tikzpicture}[scale=0.35,thick] % , baseline = -3.5pt

    \node[anchor=center] (text) at (-2,4) {${a)}$};

    \draw (0,3) rectangle (2,5);
    \node[anchor=center] at (1,4) {$\acttensorof{0}$};

    \draw (8,3) rectangle (10,5);
    \node[anchor=center] at (9,4) {$\acttensorof{\seldim\shortminus1}$};

    \drawvariabledot{1}{2.5}
    \draw[->-] (1,1) -- (1,3) node[midway,left] {\colorlabelsize$\headvariableof{0}$};
    \node[anchor=center] at (5,2) {$\cdots$};
    \drawvariabledot{9}{2.5}
    \draw[->-] (9,1) -- (9,3) node[midway,right] {\colorlabelsize$\headvariableof{\seldim\shortminus1}$};
    \draw (0,-1) rectangle (10,1);
    \node[anchor=center] at (5,0) {\corelabelsize $\bencodingof{\sstat}$};

    \draw[->-] (1,-3) -- (1,-1) node[midway,left] {\colorlabelsize$\catvariableof{0}$};
    \node[anchor=center] at (5,-2) {$\cdots$};
    \draw[->-] (9,-3) -- (9,-1) node[midway,right] {\colorlabelsize$\catvariableof{\catorder\shortminus1}$};


    \begin{scope}[shift={(20,0)}]
        \node[anchor=center] (text) at (-2,4) {${b)}$};

        \draw[->-] (1,5) -- (1,7) node[midway,left] {\colorlabelsize$\avariableof{0}$};
        \draw (0,3) rectangle (2,5);
        \node[anchor=center] at (1,4) {$\anaugmentationof{{0}}$};

        \draw[->-] (9,5) -- (9,7) node[midway,right] {\colorlabelsize$\avariableof{\seldim\shortminus1}$};
        \draw (8,3) rectangle (10,5);
        \node[anchor=center] at (9,4) {$\anaugmentationof{{\seldim\shortminus1}}$};

        \draw[->-] (1,1) -- (1,3) node[midway,left] {\colorlabelsize$\headvariableof{0}$};
        \node[anchor=center] at (5,2) {$\cdots$};
        \draw[->-] (9,1) -- (9,3) node[midway,right] {\colorlabelsize$\headvariableof{\seldim\shortminus1}$};
        \draw (0,-1) rectangle (10,1);
        \node[anchor=center] at (5,0) {\corelabelsize $\bencodingof{\sstat}$};

        \draw[->-] (1,-3) -- (1,-1) node[midway,left] {\colorlabelsize$\catvariableof{0}$};
        \node[anchor=center] at (5,-2) {$\cdots$};
        \draw[->-] (9,-3) -- (9,-1) node[midway,right] {\colorlabelsize$\catvariableof{\catorder\shortminus1}$};
    \end{scope}

\end{tikzpicture}
    \end{center}
    \caption{Ancilla augmentation of an elementary \ComputationActivationNetwork{} a) by replacing each leg vector of the elementary activation tensor by an directed ancilla augmentation.}
    \label{fig:canElementaryAugmentation}
\end{figure}

%% Elementary CAN Augmentation
Elementary \ComputationActivationNetworks{} can be augmented by ancilla augmentation of each leg vector in the activation tensor network (see \figref{fig:canElementaryAugmentation}).
We understand ancilla variables as variables in a Bayesian Network having single parents, corresponding with the ancilla augmentation of an elementary tensor.
For an example, see \figref{fig:toyAccounting}.

%% Hidden variables in activation graph
To a generic activation hypergraph $\graph$, we would do ancilla augmentation for the activation tensor network, where each hidden variable is treated as a distributed qubit.
To treat it as a distributed qubit, each hidden activation qubit is prepared in uniform state (i.e. a Hadamard gate acting on a ground state) and will be omitted in the measurement scheme (either not measured, or its measurement ignored).
This can be understood from a tensor network perspective, where marginalization means contracting, which is exactly how hidden activation variables are used.
For an example in the $\ttformat$ format, see \figref{fig:canTTAugmentation}.

\begin{figure}[t]
    \begin{center}
        \begin{tikzpicture}[scale=0.35,thick] % , baseline = -3.5pt

    \node[anchor=center] (text) at (-2,4) {${a)}$};

    \draw (0,4) rectangle (2,6);
    \node[anchor=center] at (1,5) {$\acttensorof{0}$};

    \drawvariabledot{2.5}{3}
    \draw (2.5,1.5) -- (2.5,3) node[midway,left] {\colorlabelsize$\thirdcatvariableof{0}$};

    \draw (1.5,4) to[bend right = 50] (2.5,3);
    \draw (3.5,4) to[bend right = -50] (2.5,3);

    \draw (3,4) rectangle (5,6);
    \node[anchor=center] at (4,5) {$\acttensorof{1}$};

    \draw (4.5,4) to[bend right = 50] (5.5,3);
    \drawvariabledot{5.5}{3}
    \draw (5.5,1.5) -- (5.5,3) node[midway,left] {\colorlabelsize$\thirdcatvariableof{1}$};

    \draw (8.5,4) to[bend right = -50] (7.5,3);
    \drawvariabledot{7.5}{3}
    \draw (7.5,1.5) -- (7.5,3) node[midway,right] {\colorlabelsize $\thirdcatvariableof{\seldim\shortminus2}$};

    \draw (8,4) rectangle (10,6);
    \node[anchor=center] at (9,5) {$\acttensorof{\seldim\shortminus1}$};

    \draw (0.5,3) -- (0.5,4);
    \drawvariabledot{0.5}{2.5}
    \draw[->-] (0.5,1) -- (0.5,3) node[midway,left] {\colorlabelsize$\headvariableof{0}$};

    \draw (4,3) -- (4,4);
    \drawvariabledot{4}{2.5}
    \draw[->-] (4,1) -- (4,3) node[midway,left] {\colorlabelsize$\headvariableof{1}$};

    \node[anchor=center] at (6.5,3) {\colorlabelsize $\cdots$};
    \draw (9.5,3) -- (9.5,4);
    \drawvariabledot{9.5}{2.5}
    \draw[->-] (9.5,1) -- (9.5,3) node[midway,right] {\colorlabelsize$\headvariableof{\seldim\shortminus1}$};
    \draw (0,-1) rectangle (10,1);
    \node[anchor=center] at (5,0) {\corelabelsize $\bencodingof{\sstat}$};

    \draw[->-] (1,-3) -- (1,-1) node[midway,left] {\colorlabelsize$\catvariableof{0}$};
    \node[anchor=center] at (5,-2) {$\cdots$};
    \draw[->-] (9,-3) -- (9,-1) node[midway,right] {\colorlabelsize$\catvariableof{\catorder\shortminus1}$};


    \begin{scope}[shift={(20,0)}]
        \node[anchor=center] (text) at (-2,4) {${b)}$};

        \draw[->-] (1,6) -- (1,8) node[midway,left] {\colorlabelsize$\avariableof{0}$};
        \draw (0,4) rectangle (2,6);
        \node[anchor=center] at (1,5) {$\anaugmentationof{0}$};

        \drawvariabledot{2.5}{3}
        \draw (2.5,1.5) -- (2.5,3) node[midway,left] {\colorlabelsize$\thirdcatvariableof{0}$};

        \draw[-<-] (1.5,4) to[bend right = 50] (2.5,3);
        \draw[-<-] (3.5,4) to[bend right = -50] (2.5,3);

        \draw[->-] (4,6) -- (4,8) node[midway,left] {\colorlabelsize$\avariableof{1}$};
        \draw (3,4) rectangle (5,6);
        \node[anchor=center] at (4,5) {$\anaugmentationof{1}$};

        \draw[-<-] (4.5,4) to[bend right = 50] (5.5,3);
        \drawvariabledot{5.5}{3}
        \draw (5.5,1.5) -- (5.5,3) node[midway,left] {\colorlabelsize$\thirdcatvariableof{1}$};

        \draw[-<-] (8.5,4) to[bend right = -50] (7.5,3);
        \drawvariabledot{7.5}{3}
        \draw (7.5,1.5) -- (7.5,3) node[midway,right] {\colorlabelsize $\thirdcatvariableof{\seldim\shortminus2}$};

        \draw[->-] (9,6) -- (9,8) node[midway,left] {\colorlabelsize$\avariableof{\seldim\shortminus1}$};
        \draw (8,4) rectangle (10,6);
        \node[anchor=center] at (9,5) {$\anaugmentationof{\seldim\shortminus1}$};

        \draw[->-] (0.5,3) -- (0.5,4);
        \drawvariabledot{0.5}{2.5}
        \draw[->-] (0.5,1) -- (0.5,3) node[midway,left] {\colorlabelsize$\headvariableof{0}$};

        \draw (4,3) -- (4,4);
        \drawvariabledot{4}{2.5}
        \draw[->-] (4,1) -- (4,3) node[midway,left] {\colorlabelsize$\headvariableof{1}$};

        \node[anchor=center] at (6.5,3) {\colorlabelsize $\cdots$};
        \draw[->-] (9.5,3) -- (9.5,4);
        \drawvariabledot{9.5}{2.5}
        \draw[->-] (9.5,1) -- (9.5,3) node[midway,right] {\colorlabelsize$\headvariableof{\seldim\shortminus1}$};
        \draw (0,-1) rectangle (10,1);
        \node[anchor=center] at (5,0) {\corelabelsize $\bencodingof{\sstat}$};

        \draw[->-] (1,-3) -- (1,-1) node[midway,left] {\colorlabelsize$\catvariableof{0}$};
        \node[anchor=center] at (5,-2) {$\cdots$};
        \draw[->-] (9,-3) -- (9,-1) node[midway,right] {\colorlabelsize$\catvariableof{\catorder\shortminus1}$};

    \end{scope}

\end{tikzpicture}
    \end{center}
    \caption{Ancilla augmentation of an $\ttformat$ \ComputationActivationNetwork{} a) by a Bayesian Network b).
        The hidden variables in the $\ttformat$ activation tensor are treated as distributed variables.
        Marginalization over the hidden variables and conditioning on the activation variables in state $1$, we reproduce the \ComputationActivationNetwork{}.}
    \label{fig:canTTAugmentation}
\end{figure}

\subsection{Amplitude Amplification}

Literature:
\begin{itemize}
    \item \cite{grover_fast_1996} Grover algorithm (search in unstructured database)
    \item \cite{ozols_quantum_2013} introduced quantum rejection sampling (using amplitude amplification)
    \item \cite{low_quantum_2014} used quantum rejection sampling for Bayesian network sampling (which is NP-hard when conditioned on evidence, see e.g. \cite{koller_probabilistic_2009})
\end{itemize}

Note, that the variable qubits are uniformly distributed when only the computation circuit is applied.
When sampling the probability distribution, we need the ancilla qubits to be in state $1$ in order for the sample to be valid.
Any other states will have to be rejected.

Classically, this can be simulated in the same way:
Just draw the variables from uniform, calculate the value qubit by a logical circuit inference and accept with probability by the computed value.

For this procedure to be more effective (and in particular not having an efficient classical pendant), we need amplitude amplification on the value qubit.
This can provide a square root speedup in the complexity compared with classical rejection sampling.

\textbf{Open Question:} Is there a way to avoid amplitude amplification and use a more direct circuit implementation of the activation network?
- Cannot be the case, when the encoding is determined by the activation tensor alone: Needs to use the computated statistic as well.

\subsection{Sampling from \ComputationActivationNetworks{} as Quantum Circuits}

\red{So far: Sample from \HybridLogicNetworks{}, would need qudits for more general \ComputationActivationNetworks{}.
Can do non-elementary \ComputationActivationNetworks{}, when activating whole activation core.}

\tnreason{} provides tensor network representations of knowledge bases and exponential families following a Computation Activation architecture.
Here are some ideas to utilize quantum circuits for sampling from \ComputationActivationNetworks{}.
We can produce Q-samples for ancilla augmented \ComputationActivationNetworks{}  using \computationCircuits{} and \activationCircuits{}:
\begin{itemize}
    \item For each (sub-) statistic, prepare a qubit by \ComputationCircuits{}
    \item Based on the computed qubits, prepare ancilla qubits by \ActivationCircuits{} to the activation cores.
\end{itemize}

\begin{figure}
    \begin{center}
        \begin{tikzpicture}[scale=0.35, thick]
    %% Ancilla Qubits
    \draw (-1,16) rectangle (1,18);
    \drawtextnode{0}{17}{\corelabelsize $\fbasis$}
    \draw (1,17) -- (13.5,17);
    \draw (13.5,16) rectangle (16.5,18);
    \drawtextnode{15}{17}{$\qcaencodingof{\hypercoreof{\seldim-1}}$};
    \draw (16.5,17) -- (20,17);

    \draw (15,10) -- (15,16);
    \drawvariabledot{15}{10}

    \node[rotate=45] at (13.5,15) {$\cdots$};
    \node[rotate=45] at (13.5,8) {$\cdots$};

    \drawtextnode{0}{15.25}{$\vdots$};
    \drawtextnode{-3.5}{15}{Ancilla\\Qubits};

    \draw (-1,12) rectangle (1,14);
    \drawtextnode{0}{13}{\corelabelsize $\fbasis$};
    \draw (1,13) -- (11,13);
    \draw (11,12) rectangle (13,14);
    \drawtextnode{12}{13}{$\qcaencodingof{\hypercoreof{0}}$};
    \draw (13,13) -- (20,13);

    \draw (12,6) -- (12,12);
    \drawvariabledot{12}{6}

    %% Statistic Qubits
    \draw[dashed] (-3,11.5) -- (26.5,11.5);
    \draw (-1,9) rectangle (1,11);
    \drawtextnode{0}{10}{\corelabelsize $\fbasis$};
    \draw (1,10) -- (7.5,10);
    \draw (7.5,9) rectangle (10.5,11);
    \drawtextnode{9}{10}{$\qcbencodingof{f_{\seldim-1}}$};
    \draw (10.5,10) -- (20,10);

    \draw (8.5,3) -- (8.5,9);
    \drawvariabledot{8.5}{3}
    \draw (9.5,-1) -- (9.5,9);
    \drawvariabledot{9.5}{-1}

    \node[rotate=45] at (7.5,8) {$\cdots$};
    \node[rotate=45] at (7.5,1) {$\cdots$};

    \drawtextnode{0}{8.25}{$\vdots$};
    \drawtextnode{-3.5}{8}{Statistic\\Qubits};

    \draw (-1,5) rectangle (1,7);
    \drawtextnode{0}{6}{\corelabelsize $\fbasis$};
    \draw (1,6) -- (5,6);
    \draw (5,5) rectangle (7,7);
    \drawtextnode{6}{6}{$\qcbencodingof{f_{0}}$};
    \draw (7,6) -- (20,6);

    \draw (5.5,3) -- (5.5,5);
    \drawvariabledot{5.5}{3}
    \draw (6.5,-1) -- (6.5,5);
    \drawvariabledot{6.5}{-1}

    %% Distributed Qubits
    \draw[dashed] (-3,4.5) -- (26.5,4.5);

    \draw (-1,2) rectangle (1,4);
    \drawtextnode{0}{3}{\corelabelsize $\fbasis$};
    \draw (1,3) -- (2,3);
    \draw (2,2) rectangle (4,4);
    \drawtextnode{3}{3}{\corelabelsize $\hgate$};

    \draw (4,3) -- (20,3);

    \drawtextnode{0}{1.25}{$\vdots$};
    \drawtextnode{-3.5}{1}{Distributed\\Qubits};

    \draw (-1,-2) rectangle (1,0);
    \drawtextnode{0}{-1}{\corelabelsize $\fbasis$};
    \draw (1,-1) -- (2,-1);
    \draw (2,-2) rectangle (4,0);
    \drawtextnode{3}{-1}{\corelabelsize $\hgate$};

    \draw (4,-1) -- (20,-1);

    %% Measurements
    \drawqcmeasuresymbol{21}{17}
    \draw (22,17) -- (24,17) node[midway,above]{\colorlabelsize $\avariableof{\seldim\shortminus1}$};
    \draw (24,16) rectangle (26,18);
    \node[anchor=center] at (25,17) {\corelabelsize $\tbasis$};

    \drawtextnode{23}{15.5}{$\vdots$};
    \drawqcmeasuresymbol{21}{13}
    \draw (22,13) -- (24,13) node[midway,above]{\colorlabelsize $\avariableof{0}$};
    \draw (24,12) rectangle (26,14);
    \node[anchor=center] at (25,13) {\corelabelsize $\tbasis$};

    \drawqcmeasuresymbol{21}{10}
    \draw (22,10) -- (24,10) node[midway,above]{\colorlabelsize $\headvariableof{\seldim\shortminus1}$};
    \drawvariabledot{24}{10}
    \drawtextnode{23}{8.5}{$\vdots$};
    \drawqcmeasuresymbol{21}{6}
    \draw (22,6) -- (24,6) node[midway,above]{\colorlabelsize $\headvariableof{0}$};
    \drawvariabledot{24}{6}

    \drawqcmeasuresymbol{21}{3}
    \draw (22,3) -- (24,3) node[midway,above]{\colorlabelsize $\catvariableof{\catorder\shortminus1}$};
    \drawtextnode{23}{1.5}{$\vdots$};
    \drawqcmeasuresymbol{21}{-1}
    \draw (22,-1) -- (24,-1) node[midway,above]{\colorlabelsize $\catvariableof{0}$};

    \begin{scope}[shift={(-15,0)}]
        \draw (-1,4) rectangle (3,-2);
        \draw (3,3) -- (5,3) node[midway,above]{\colorlabelsize $\catvariableof{\catorder\shortminus1}$};
        \draw (3,-1) -- (5,-1) node[midway,above]{\colorlabelsize $\catvariableof{0}$};
        \drawtextnode{4}{1.5}{$\vdots$}
        \drawtextnode{1}{1}{$\probof{\hybridparam}$};
        \drawtextnode{-2}{1}{$\acceptanceprob\cdot$}
        \drawtextnode{6}{1}{${=}$}
    \end{scope}

\end{tikzpicture}
    \end{center}
    \caption{
        Quantum Circuit to reproduce a \ComputationActivationNetwork{} (with elementary activation) by rejection sampling.
        We measure the distributed qubits $\shortcatvariables$ and the ancilla qubits $\avariableof{[\seldim]}$ and reject all samples, where an ancilla qubit is measured as $0$.
    }\label{fig:caCircuit}
\end{figure}

\begin{example}[Toy Accounting Example]
    \label{exa:toyAccounting}
    For a more detailed example, let us consider a system of three variables $A1$ Account 1 is booked, $A2$ Account 2 is booked, $F$ a feature on an invoice.
    Assume the following two rules have to be respected:
    \begin{itemize}
        \item \textcolor{\concolor}{Exactly one account must be booked.}
        \item \textcolor{\probcolor}{If feature $\mathrm{F}$ is present on the invoice, the account $\mathrm{A1}$ is typically booked.}
    \end{itemize}
    We formalize this with the statistic
    \begin{align*}
        \hlnstat = (\catvariableof{A1} \oplus \catvariableof{A2}, \catvariableof{F}\Rightarrow \catvariableof{A1})\, .
    \end{align*}
    Any elementary \ComputationActivationNetwork{} with the statistic $\hlnstat$ can be realized by the circuit shown in \figref{fig:toyAccounting}.
    %In particular, the \computationCircuit{} consists of:
    For the preparation of the first statistic qubit $\headvariableof{0}$ to $\sstatcoordinateof{0}=\catvariableof{A1} \oplus \catvariableof{A2}$ by a \computationCircuit{} we exploit that
    \begin{align*}
        \catvariableof{A1}\oplus\catvariableof{A2}
        &= \onehotmapofat{1}{\catvariableof{A1}} \otimes \onehotmapofat{0}{\catvariableof{A2}}
        + \onehotmapofat{0}{\catvariableof{A1}} \otimes \onehotmapofat{1}{\catvariableof{A2}} \\
        &= \left(\onehotmapofat{1}{\catvariableof{A1}} \otimes \onehotmapofat{0}{\catvariableof{A2}}\right)
        \oplus \left(\onehotmapofat{0}{\catvariableof{A1}} \otimes \onehotmapofat{1}{\catvariableof{A2}}\right)
    \end{align*}
    where by $\oplus$ we denote coordinatewise summation mod 2.
    The statistic qubit $\headvariableof{0}$ is therefore prepared by two controlled NOT gates (see \figref{fig:toyAccounting}b).

    For the preparation of the second statistic qubit $\headvariableof{1}$ to $\sstatcoordinateof{1}=\catvariableof{F}\Rightarrow \catvariableof{A1}$ we exploit that the implication is $\truesymbol$ except for the case where the premise $\catvariableof{F}$ is $\truesymbol$ and the head $\catvariableof{A1}$ is $\falsesymbol$.
    In out mod 2 calculus, this amounts to
    \begin{align*}
        \catvariableof{F}\Rightarrow \catvariableof{A1}
        = \onesat{\catvariableof{F},\catvariableof{A1}} \oplus \onehotmapofat{1}{\catvariableof{F}} \otimes \onehotmapofat{0}{\catvariableof{A1}}\, .
    \end{align*}
    It follows that the statistic qubit $\headvariableof{1}$ is prepared by a NOT gate (that is a Pauli-X gate $\paulixsymbol$) and a second controlled NOT gate (see \figref{fig:toyAccounting}b).

    For both cases, the preparation of the statistic qubit can be done by two controlled NOT gates exploiting the \polynomialSparsity{} of the connectives.

    Any non-elementary \ComputationActivationNetwork{} with the statistic $\hlnstat$ can be prepared by the circuit, when choosing a single ancilla qubit, which is uniformly controlled by the qubits $\headvariableof{[2]}$.

    \begin{figure}[t]
    \begin{center}
        \begin{tikzpicture}[scale=0.35, thick] % , baseline = -3.5pt

            \node[anchor=center] at (-16,14) {$a)$};
            \node[anchor=center] at (-4,14) {$b)$};

            \begin{scope}[shift={(-12,2.25)}]

                \node[circle, draw, thick, fill=\nodegrayscale, minimum size = \nodeminsize] (A0) at (-1.5,10) {};
                \node[anchor=center] (A) at (-1.5,10) {\corelabelsize $\avariableof{0}$};
                \node[circle, draw, thick, fill=\nodegrayscale, minimum size = \nodeminsize] (A1) at (1.5,10) {};
                \node[anchor=center] (A) at (1.5,10) {\corelabelsize $\avariableof{1}$};

                \node[circle, draw, thick, fill=\nodegrayscale, minimum size = \nodeminsize] (H0) at (-1.5,5) {};
                \node[anchor=center] (A) at (-1.5,5) {\corelabelsize $\headvariableof{0}$};
                \node[circle, draw, thick, fill=\nodegrayscale, minimum size = \nodeminsize] (H1) at (1.5,5) {};
                \node[anchor=center] (A) at (1.5,5) {\corelabelsize $\headvariableof{1}$};

                \node[circle, draw, thick, fill=\nodegrayscale, minimum size = \nodeminsize] (F) at (3,0) {};
                \node[anchor=center] (A) at (3,0) {\corelabelsize $\catvariableof{F}$};

                \node[circle, draw, thick, fill=\nodegrayscale, minimum size = \nodeminsize] (XA1) at (0,0) {};
                \node[anchor=center] (A) at (0,0) {\corelabelsize $\catvariableof{A1}$};

                \node[circle, draw, thick, fill=\nodegrayscale, minimum size = \nodeminsize] (XA2) at (-3,0) {};
                \node[anchor=center] (A) at (-3,0) {\corelabelsize $\catvariableof{A2}$};

                \draw[->-] (H0) -- (A0);
                \draw[->-] (H1) -- (A1);

                \coordinate (F0) at (-1.5,2.5);
                \draw[->-] (XA2) -- (F0);
                \draw[->-] (XA1) -- (F0);
                \draw[->-] (F0) -- (H0);

                \coordinate (F1) at (1.5,2.5);
                \draw[->-] (F) -- (F1);
                \draw[->-] (XA1) -- (F1);
                \draw[->-] (F1) -- (H1);
            \end{scope}

            %% Ancilla Qubits

            \draw (-1,15) rectangle (1,13);
            %\node[anchor=center] (text) at (-2,14) {\colorlabelsize $\avariableof{1}$};
            \node[anchor=center] (text) at (0,14) {$\onehotmapof{0}$};
            \draw (1,14) -- (13.5,14);
            \draw (13.5,15) rectangle (15.5,13);
            \node[anchor=center] (text) at (14.5,14) {$\qcaencodingof{\hypercoreof{1}}$};
            \draw (15.5,14) -- (18,14);
            \drawqcmeasuresymbol{19}{14}
            \draw (20,14) -- (21.5,14) node[above,midway] {\colorlabelsize $\avariableof{1}$};

            \draw (14.5,8.5) -- (14.5,13);
            \drawvariabledot{14.5}{8.5}

            \draw (-1,10.5) rectangle (1,12.5);
            %\node[anchor=center] (text) at (-2,11.5) {\colorlabelsize $\avariableof{0}$};
            \node[anchor=center] (text) at (0,11.5) {$\onehotmapof{0}$};
            \draw (1,11.5) -- (11.5,11.5);
            \draw (11.5,10.5) rectangle (13.5,12.5);
            \node[anchor=center] (text) at (12.5,11.5) {$\qcaencodingof{\hypercoreof{0}}$};
            \draw (13.5,11.5) -- (18,11.5);
            \drawqcmeasuresymbol{19}{11.5}
            \draw (20,11.5) -- (21.5,11.5) node[above,midway] {\colorlabelsize $\avariableof{0}$};

            \draw (12.5,6) -- (12.5,10.5);
            \drawvariabledot{12.5}{6}

            %% Statistic Qubits
            \draw[dashed] (-3,10) -- (21,10);

            \draw (-1,7.5) rectangle (1,9.5);
            %\node[anchor=center] (text) at (-2,8.5) {\colorlabelsize $\headvariableof{1}$};
            \node[anchor=center] (text) at (0,8.5) {$\onehotmapof{0}$};
            \draw (1,8.5) -- (4,8.5);
            \draw (4,7.5) rectangle (6,9.5);
            \node[anchor=center] (text) at (5,8.5) {$\paulixsymbol$};
            \draw (6,8.5) -- (18,8.5);
            \drawcnotsymbol{11}{8.5}
            \draw (11,8.5) -- (11,-2);
            \drawqcmeasuresymbol{19}{8.5}
            \draw (20,8.5) -- (21.5,8.5) node[above,midway] {\colorlabelsize $\headvariableof{1}$};
            \drawvariabledot{21.5}{8.5}

            \draw (-1,5) rectangle (1,7);
            %\node[anchor=center] (text) at (-2,6) {\colorlabelsize $\headvariableof{0}$};
            \node[anchor=center] (text) at (0,6) {$\onehotmapof{0}$};
            \draw (1,6) -- (18,6);
            \drawcnotsymbol{9}{6}
            \draw (9,6) -- (9,-2);
            \drawqcmeasuresymbol{19}{6}
            \draw (20,6) -- (21.5,6) node[above,midway] {\colorlabelsize $\headvariableof{0}$};
            \drawvariabledot{21.5}{6}

            \drawcnotsymbol{5}{6}
            \draw (5,6) -- (5,-2);

            %% Distributed Qubits
            \draw[dashed] (-3,4.5) -- (21,4.5);

            \draw (-1,2) rectangle (1,4);
            %\node[anchor=center] (text) at (-2,3) {\colorlabelsize $\catvariableof{F}$};
            \node[anchor=center] (text) at (0,3) {$\onehotmapof{0}$};
            \draw (1,3) -- (2,3);
            \draw (2,2) rectangle (4,4);
            \node[anchor=center] (text) at (3,3) {$\hgate$};
            \draw (4,3) -- (18,3);
            \drawqcmeasuresymbol{19}{3}
            \draw (20,3) -- (21.5,3) node[above,midway] {\colorlabelsize $\catvariableof{F}$};

            \draw (-1,-0.5) rectangle (1,1.5);
            %\node[anchor=center] (text) at (-2,0.5) {\colorlabelsize $\catvariableof{A2}$};
            \node[anchor=center] (text) at (0,0.5) {$\onehotmapof{0}$};
            \draw (1,0.5) -- (2,0.5);
            \draw (2,-0.5) rectangle (4,1.5);
            \node[anchor=center] (text) at (3,0.5) {$\hgate$};
            \draw (4,0.5) -- (6,0.5);
            \draw (6,-0.5) rectangle (8,1.5);
            \node[anchor=center] at (7,0.5) {$\paulixsymbol$};

            \draw (8,0.5) -- (18,0.5);
            \drawqcmeasuresymbol{19}{0.5}
            \draw (20,0.5) -- (21.5,0.5) node[above,midway] {\colorlabelsize $\catvariableof{A2}$};

            \draw (-1,-3) rectangle (1,-1);
            %\node[anchor=center] (text) at (-2,-2) {\colorlabelsize $\catvariableof{A1}$};
            \node[anchor=center] (text) at (0,-2) {$\onehotmapof{0}$};
            \draw (1,-2) -- (2,-2);
            \draw (2,-3) rectangle (4,-1);
            \node[anchor=center] at (3,-2) {$\hgate$};
            \draw (4,-2) -- (6,-2);
            \draw (6,-3) rectangle (8,-1);
            \node[anchor=center] at (7,-2) {$\paulixsymbol$};
            \draw (8,-2) -- (12,-2);
            \draw (12,-3) rectangle (14,-1);
            \node[anchor=center] at (13,-2) {$\paulixsymbol$};
            \draw (14,-2) -- (18,-2);
            \drawqcmeasuresymbol{19}{-2}
            \draw (20,-2) -- (21.5,-2) node[above,midway] {\colorlabelsize $\catvariableof{A1}$};

            \drawcontroldot{5}{0.5}
            \drawcontroldot{5}{-2}

            \drawcontroldot{9}{0.5}
            \drawcontroldot{9}{-2}

            \drawcontroldot{11}{3}
            \drawcontroldot{11}{-2}

        \end{tikzpicture}
    \end{center}
    \caption{Circuit to sample from the \ComputationActivationNetwork{} in the toy accounting \exaref{exa:toyAccounting}.
    %Note that a first Pauli-X gate $\paulixsymbol$ acting on $\catvariableof{A2}$ can be omitted, since the prepared stated is uniform.
    }
    \label{fig:toyAccounting}
\end{figure}

\end{example}



\subsection{Application}

%% Usage as forward inferer
When sampling from probability distributions, we can use these samples to estimate probabilistic queries.
Building on such particle-based inference schemes, we can perform various inference schemes for \ComputationActivationNetworks{}, such as backward inference and message passing schemes.
