\section{Graph-Controlled circuits}

Let us now introduce the most generic circuit construction schemes used in this work.
To this end, we exploit the definition of uniformly controlled unitaries (see \cite{mottonen_transformation_2005}), and study their alignment along directed acyclic hypergraphs.

\subsection{Definition}

We follow the definition of Bayesian networks as tensor networks on directed acyclic hypergraphs (see \cite{goessmann_tensor-network_2025}).

% Hypergraphs
We define graph-controlled circuits using hypergraphs, which nodes are associated with the qubits of the circuit.
The hyperedges are tuples of two subsets of the qubits, namely the incoming qubits representing control and the outgoing qubit representing the target of a unitary.
To each hyperedge we then associate a uniformly controlled unitary.

\begin{definition}[Graph-Controlled Circuit]
    Let $\graph=(\nodes,\edges)$ be a directed acyclic hypergraph, where each hyperedge has exactly one outgoing node and all nodes appear exactly once as outgoing nodes of an hyperedge.
    Then a by $\graph$ controlled circuit is a decoration of the edges $\edge=(\incomingnodes,\{\node\})\in\edges$ by uniformly controlled unitaries
    \begin{align*}
        \contunitaryofat{\edge}{\catvariableof{\node,\insymbol},\catvariableof{\node,\outsymbol},\catvariableof{\incomingnodes,\outsymbol}} \, .
    \end{align*}
\end{definition}

% Orders of the qubits respecting directionality
Since the hypergraph $\graph$ to a graph-controlled circuit is acyclic, we find an order of the qubits, such that to each qubit appearing in the outgoing qubits of a hyperedge is ordered after the incoming qubits.
In such way, we have a unique definition of the graph-controlled circuit as a concatenation of the controlled unitaries respecting a directionality order.
When there are multiple orders respecting the directionality, then the concatenation respecting the orders all lead to an equivalent circuit.
% Should we show this using TN?

%%
\begin{example}[Generic State Representation]
    \label{exa:genericControlCircuit}
    In \cite{mottonen_transformation_2005} a preparation scheme for arbitrary states with real and positive amplitudes by graph-controlled circuits is presented.
    Therein, the hypergraph $\graph=(\nodes,\edges)$ is constructed as follows:
    \begin{itemize}
        \item Nodes $\nodes={[\catorder]}$
        \item Edges $\edges=\left\{([\catenumerator],\{\catenumerator\})\wcols\catenumeratorin\right\}$
    \end{itemize}
    It is possible to realize each controlled unitary by a sequence of single qubit rotations and C-NOTs (see \cite{mottonen_transformation_2005}), where the angles are computed using gray codes.
\end{example}

\subsection{\ActivationCircuit{}}

%We now investigate quantum pendants to the function encoding schemes used in \tnreason{}{}.
A specific scheme to construct graph-controlled circuits with specific Bayesian Networks as measurement distributions are \activationCircuits{}.
Here we restrict during the preparation procedure to real and positive amplitudes (i.e. prepare q-samples).


\ActivationCircuits{} are uniformly controlled unitaries (see \cite{mottonen_transformation_2005}), where the rotation axis is chosen as the $y$-axes of the Bloch sphere, and the angles are computed by the function $h(\cdot)$ acting on a function value to be encoded.
%\red{Introduce rotation encodings based on activation circuit action, hint at a sampling scheme of probability distributions.}

%% Angle preparing
We define the angle preparing function on $p\in[0,1]$ by
\begin{align*}
    h(p) = 2 \cdot \mathrm{cos}^{-1}\left(\sqrt{1-p}\right) \, .
\end{align*}
For any $p\in[0,1]$ we then have
\begin{align*}
    \contractionof{\onehotmapofat{0}{\avariableof{\insymbol}},\yrotationofat{h(p)}{\avariableof{\insymbol},\avariableof{\outsymbol}}}{\avariableof{\outsymbol}}
    = \begin{bmatrix}
          \sqrt{1-p} \\
          \sqrt{p}
    \end{bmatrix} \, .
\end{align*}

\begin{definition}[\ActivationCircuit{}]
    Given a function
    \begin{align*}
        \exformula : \bigtimes_{\selindexin} [2] \rightarrow [0,1]
    \end{align*}
    its \activationCircuit{} is the uniformly controlled unitary
    $\qcaencodingofat{\hypercore}{\avariableof{\insymbol},\avariableof{\outsymbol},\headvariables}$ defined as
    \begin{align*} %% Could ease the definition by coordinates
        \qcaencodingofat{\hypercore}{\avariableof{\insymbol},\avariableof{\outsymbol},\headvariables}
        = \sum_{\shortheadindices} \onehotmapofat{\shortheadindices}{\headvariables}
        %\otimes \onehotmapofat{\shortheadindices}{\headvariableof{\insymbol,[\seldim]}}
        \otimes \yrotationofat{h(\exformulaat{\shortheadindices})}{\avariableof{\insymbol},\avariableof{\outsymbol}} \, .
    \end{align*}
\end{definition}

%% Drop the in and out
We will ease our notation by dropping the $\insymbol$ and $\outsymbol$ labels to the control variables.
This amounts to understanding the Dirac delta tensors in \activationCircuits{} as hyperedges.
Along that picture the quantum circuit is a tensor network on hyperedges instead of edges.

%% As a graph-controlled circuit
Each activation circuit is a graph-controlled circuit, where the corresponding hypergraph consists in $\nodes=\headvariables\cup\{\avariable\}$ and a single edge $\edges=(\headvariables,\{\avariable\})$.

When we have a probability tensor, its \activationCircuit{} be prepared, since all values are in $[0,1]$.
Note that for rejection sampling, only the quotients of the values are important, we can therefore scale the value by a scalar such that the mode is $1$.

%% Rescaling tensors with larger coordinates
Tensors $\hypercoreat{\headvariables}$ with non-negative coordinates can be encoded after dividing them by their maximum, that is the \activationCircuit{} of the function
\begin{align*}
    \exformulaat{\shortheadindices} = \frac{\hypercoreat{\headvariables=\shortheadindices}}{\max_{\secheadindexof{[\seldim]}} \hypercoreat{\headvariables=\secheadindexof{[\seldim]}}}
\end{align*}
When the maximum of the tensor is not known, it can be replaced by an upper bound (reducing the acceptance rate of the rejection sampling).

%% Generic Tensors with complex amplitudes: Require besides rescaling more general unitaries.

\subsubsection{Q-samples}

Towards providing additional insights onto the usage of \activationCircuits{} we now present them as a scheme to prepare q-samples (see \cite{low_quantum_2014}).

%\red{Merge with the above.
%}

In general, we define Q-samples to be quantum states, which measured in the computational basis reproduce a given probability distribution.

\begin{definition}[Q-sample]
    Given a probability distribution $\probtensor:\atomstates\rightarrow\rr$ (i.e. $\contraction{\probtensor}=1$ and $\zeros \prec \probtensor$) its q-sample is
    \begin{align*}
        \qstateofat{\probtensor}{\shortcatvariables}
        = \sum_{\shortcatindicesin} \sqrt{\probat{\indexedshortcatvariables}} \cdot \onehotmapofat{\shortcatindices}{\shortcatvariables} \, .
    \end{align*}
\end{definition}

Q-samples are more restrictive than arbitrary states having a distribution as a measurement distribution, since they demand real and positive amplitudes.

%% Low Paper
In \cite{low_quantum_2014} the Q-sample has been introduced.
It prepares a scheme to realize property 1 (purity) + 2 (q-sampling) of a qpdf, but fails to realize property 3 (q-stochasticity).
%The q-sample can be prepared for Bayesian Networks, where each child qubit is prepared densely by C-NOTs conditioning on parent qubits.

\begin{example}[Q-sample of the uniform distribution]
    \label{exa:qSampleUniform}
    The q-sample of the uniform distribution can be prepared by Hadamard gates acting on the ground state.
    This is an example of a Walsh-Hadamard transform (see \secref{sec:walshHadamardTransform}), which can be performed by unary Hadamard gates.
\end{example}

Q-samples can be prepared by \activationCircuit{}s acting on the uniform q-sample (see \exaref{exa:qSampleUniform}), as we show next. % (Hadamard gates acting on ground state).

%    \begin{lemma}
%        The \activationCircuit{} of $\probwith$ acting on the uniform state of $\shortcatvariables$ prepares a q-sample.
%    \end{lemma}
%    \begin{proof}
%        Follows directly from construction:
%        The uniform state is the action of Hadamard gates on the ground state $\onehotmapofat{0}{\shortcatvariables}$, which prepares the state $\normalizationof{\ones}{\shortcatvariables}$.
%        The diagonal \activationCircuit{} transforms
%    \end{proof}

Doing rejection sampling on the ancilla qubit corresponds with sampling from the normalized contraction with the activation tensor.

\begin{lemma}
    \label{lem:qSampleDistribution}
    Given a distribution $\probat{\shortcatvariables}$, we construct a circuit preparing its q-sample and add the ancilla encoding of a tensor $\hypercoreat{\shortcatvariables}$.
    The rejection sampling scheme, measuring the ancilla qubit and the $\shortcatvariables$ qubits, rejecting the ancilla qubit measured as $0$, prepares samples from the distribution
    \begin{align*}
        \normalizationof{\probat{\shortcatvariables},\hypercoreat{\shortcatvariables}}{\shortcatvariables} \, .
    \end{align*}
\end{lemma}

\subsubsection{Encoding of conditional distributions}

%\red{\lemref{lem:qSampleDistribution} states that any distribution can be }

Following the schemes in \cite{low_quantum_2014}, we can prepare the acyclic networks of directed and non-negative tensors by a sequence of controlled rotations.
Directed and non-negative tensors correspond with conditional probability distributions and their acyclic networks are Bayesian Networks.
We prepare them by \activationCircuit{}s of functions (see \figref{fig:cpdEncoding})
\begin{align*}
    \catindexof{\parentsof{\catenumerator}} \rightarrow \condprobat{\catvariableof{\catenumerator}=1}{\indexedcatvariableof{\parentsof{\catenumerator}}} \, .
\end{align*}

\subsubsection{Encoding of Bayesian Networks}

We revisit the correspondence of Bayesian Networks with graph-controlled circuits, by showing that any Bayesian Network can be prepared by \activationCircuits{} of the conditional probability distributions.
Bayesian Networks can be prepared as quantum circuits, where each conditional probability distribution is prepared by an \activationCircuit{}.

\begin{theorem}[Low et al.]
    Any Bayesian Network of variables $\shortcatvariables$, where the enumeration by $[\atomorder]$ respects the partial order by child-parent relations, can be prepared as a quantum circuit by concatenating the \activationCircuit{}s
    \begin{align*}
        \qcaencodingofat{
            \condprobat{\catvariableof{\catenumerator}=1}{\catvariableof{\parentsof{\catenumerator}}}
        }{\catvariableof{\catenumerator,\insymbol},\catvariableof{\catenumerator},\catvariableof{\parentsof{\catenumerator}}}
    \end{align*}
    for $\atomenumeratorin$ and acting on the initial state $\bigotimes_{\atomenumeratorin}\onehotmapofat{0}{\catvariableof{\catenumerator,\insymbol}}$ .
\end{theorem}


%To this end, one iterates over the states of the incoming variables, and performs a controlled rotation on the outgoing variable, where the angle is given by the value of the tensor at the incoming state.
%This generalizes the basis encoding scheme, which demands boolean tensors.

\begin{figure}
    \begin{center}
        \begin{tikzpicture}[scale=0.35,thick] % , baseline = -3.5pt

    \node[anchor=center] (text) at (16,1) {${b)}$};

    \begin{scope}
        [shift={(20,-7)}]

        \node[anchor=center] (text) at (-1,6.5) {$\catvariableof{\catenumerator,\insymbol}$};
        \node[anchor=center] (text) at (12,6.5) {$\catvariableof{\catenumerator,\outsymbol}$};

        \draw (5,4) -- (5,5.5);
        \drawvariabledot{5}{4}
        \draw (6,1) -- (6,5.5);
        \drawvariabledot{6}{1}

        \draw (1,6.5) -- (2,6.5);
        \draw (2,5.5) rectangle (9,7.5);
        \node[anchor=center] (text) at (5.65,6.5) {$\qcaencodingof{\condprobof{\catvariableof{\catenumerator}=1}{\catvariableof{\parentsof{\catenumerator}}} }$};
        \draw (9,6.5) -- (10,6.5);

        \draw[dashed] (0.5,5) -- (10.5,5);

        \draw (1,4) -- (10,4);

        \node[anchor=center] (text) at (3,2.75) {$\vdots$};
        \node[anchor=center] (text) at (-1,2.5) {$\catvariableof{\parentsof{\catenumerator}}$};

        \node[anchor=center] (text) at (8,2.75) {$\vdots$};
        \node[anchor=center] (text) at (12,2.5) {$\catvariableof{\parentsof{\catenumerator}}$};

        \draw (1,1) -- (10,1);

    \end{scope}


    \node[anchor=center] (text) at (0,1) {${a)}$};

    \draw[->-] (4.5,-1) -- (4.5,1) node[midway, right]{$\catvariableof{\catenumerator}$};
    \draw (0,-1) rectangle (9,-4);
    \node[anchor=center] (text) at (4.5,-2.5) { $\condprobof{\catvariableof{\catenumerator}}{\catvariableof{\parentsof{\catenumerator}}} $};
    \draw[->-] (1,-6) -- (1,-4);

    \node[anchor=center] (text) at (4.5,-4.75) {$\cdots$};
    \node[anchor=center] (text) at (4.5,-6) {$\catvariableof{\parentsof{\catenumerator}}$};

    \draw[->-] (8,-6) -- (8,-4);


\end{tikzpicture}
    \end{center}
    \caption{
        Representation of directed and positive tensor by a controlled rotation.
        a) Conditional probability tensor $\condprobat{\catvariableof{\catenumerator}}{\catvariableof{\parentsof{\catenumerator}}}$ being a tensor in a Bayesian Network.
        b) Circuit Encoding as a controlled rotation, which is the \ActivationCircuit{} of the tensor $\condprobat{\catvariableof{\catenumerator}=1}{\catvariableof{\parentsof{\catenumerator}}}$.
    }\label{fig:cpdEncoding}
\end{figure}

%\subsubsection{Encoding of independent distributions}
%
%An extremal case of Bayesian networks is those where all variables are mutually independent.

\subsection{Equivalence with Bayesian Networks}

\red{Above we have shown, how Bayesian Networks can be prepared by \activationCircuits{}, which are a class of graph-controlled circuits.
We now show, that the measurement distribution of any graph-controlled circuit is a Bayesian Network.
It then follows that the set of distributions prepared by graph-controlled circuits is exactly the set of Bayesian Networks.}



This is closely connected to the chain decomposition of probability distributions, and their construction.
In \cite{low_quantum_2014} Bayesian Networks are prepared by graph-controlled circuits.
When in the graph $\graph$ each node appears at most once as outgoing node, it is also the hypergraph to a family of Bayesian Networks.
The measurement distributions of a state prepared by a $\graph$-controlled Circuit acting on disentangled initial states are exactly the Bayesian Networks with respect to $\graph$.

\begin{theorem}
    \label{the:graphControlledPreparesBN}
    Let $\graph$ be a directed acyclic hypergraph, such that node appears at most once as an outgoing node.
    The measurement distributions of the by $\graph$ controlled circuits acting on disentangled initial states are equal to the Bayesian Networks on $\graph$.
\end{theorem}

\begin{lemma}
    \label{lem:BNtoGCC}
    Let $\graph$ be a directed acyclic hypergraph, such that node appears at most once as an outgoing node.
    Then any Bayesian network on $\graph$ can be prepared by a $\graph$-controlled circuit with activation circuits of the conditional probability tensors.
\end{lemma}
\begin{proof}
    Let $\probwith$ be a Bayesian network on the graph $\graph$.
    Enumerate the nodes $\nodes$ of the $\graph$ by $[\atomorder]$, such that for each $\catenumeratorin$ we have $\parentsof{\catenumerator}\subset[\catenumerator]$.
    Then define a $\graph$-controlled circuit, by choosing for each $\catenumeratorin$ controlled unitaries which satisfy
    \begin{align*}
        \contunitaryofat{\catenumerator}{\catvariableof{\catenumerator,\insymbol}=0,\catvariableof{\node,\outsymbol},\catvariableof{\parentsof{\catenumerator},\outsymbol}}
        =\sqrt{\condprobof{\catvariableof{\catenumerator}}{\catvariableof{\parentsof{\catenumerator}}}} \, .
    \end{align*}
    Here we specified only the action of the controlled unitary on the basis vector $\onehotmapofat{0}{\catvariableof{\catenumerator}}$, the action on $\onehotmapofat{1}{\catvariableof{\catenumerator}}$ can be chosen by an arbitrary orthogonal unit vector. % Maybe link here the activation circuit scheme, which is already used?
    For more explicit construction, see the \activationCircuits{}.
    Any such defined $\graph$-controlled circuit acting on the initial state $\bigotimes_{\catenumeratorin}\onehotmapofat{0}{\catvariableof{\catenumerator}}$ prepares a quantum state $\qstatewith$ with measurement distribution
    \begin{align*}
        \absof{\qstate}^2\left[\shortcatvariables\right] \, .
    \end{align*}
    Given arbitrary $\shortcatindicesin$ we have
    \begin{align*}
        \absof{\qstate}^2\left[\indexedshortcatvariables\right]
        = \prod_{\catenumeratorin} \condprobof{\indexedcatvariableof{\catenumerator}}{\indexedcatvariableof{\parentsof{\catenumerator}}}
        = \probat{\indexedshortcatvariables} \, .
    \end{align*}
    Here we used in the last equation, that $\probwith$ is a Bayesian network.
    Since the equivalence holds for any coordinate, this establishes the equivalence of the measurement distribution of $\qstatewith$ and $\probwith$.
\end{proof}

While we have already shown by \lemref{lem:BNtoGCC} that arbitrary Bayesian Networks can be prepared by graph-controlled circuits, we now show the converse, namely that any distribution prepared by graph-controlled circuits is a Bayesian Network.
To this end, we use the characterization of Bayesian Networks by the conditional independencies they encode (see \cite{koller_probabilistic_2009}).
To be more precise, we need to show that any node variable is conditionally independent of its non-descendants given its parents.

\begin{lemma}
    \label{lem:GCCtoBN}
    Let $(\graph,\contunitary)$ be a $\graph$-controlled circuit acting on a disentangled initial state and $\probat{\nodevariables}$ the corresponding measurement distribution.
    Then we have for each $\node\in\nodes$ the conditional independence
    \begin{align*}
        \condindependent{\catvariableof{\node}}{\catvariableof{\nondescendantsof{\node}}}{\catvariableof{\parentsof{\node}}} \, .
    \end{align*}
\end{lemma}
\begin{proof}
    We choose to a given $\node\in\nodes$ an enumeration $[\catorder]$ of the nodes, such that for each $\catenumeratorin$ we have $\parentsof{\catenumerator}\subset[\catenumerator]$ and for the enumerator $\seccatenumerator$ of $\node$ we further have $\nondescendantsof{\seccatenumerator}\subset[\seccatenumerator]$.
    Let $\probwith$ be the measurement distribution of the $\graph$-controlled circuit acting on a disentangled initial state $\bigotimes_{\catenumeratorin}\qstateofat{\catenumerator}{\catvariableof{\catenumerator}}$ and choose arbitrary $\catindexof{[\catenumerator]}$.
    We then have
    \begin{align*}
        \probat{\catvariableof{\seccatenumerator},\indexedcatvariableof{[\catenumerator]}}
        &= \contractionof{\left(\bigcup_{\catenumeratorin}\{\contunitaryof{\catenumerator},\contunitaryof{\catenumerator,\dagger},\qstateof{\catenumerator},\qstateof{\catenumerator,*}\right)\}
            \cup \left(\bigcup_{\catenumerator\in[\seccatenumerator]} \onehotmapofat{\catindexof{\catenumerator}}{\catvariableof{\catenumerator,\outsymbol}}\right)
        }{
            \catvariableof{\seccatenumerator}
        } \\
        &= \contractionof{\bigcup_{\catenumerator\in[\seccatenumerator]}\{\contunitaryof{\catenumerator},\contunitaryof{\catenumerator,\dagger},\qstateof{\catenumerator},\qstateof{\catenumerator,*},\onehotmapofat{\catindexof{\catenumerator}}{\catvariableof{\catenumerator,\outsymbol}}\}}{
            \catvariableof{\seccatenumerator}
        } \\
        & = \absof{\contractionof{\contunitaryofat{\seccatenumerator}{\catvariableof{\seccatenumerator,\insymbol},\catvariableof{\seccatenumerator,\outsymbol}},\indexedcatvariableof{\parentsof{\seccatenumerator},\outsymbol}}{\catvariableof{\seccatenumerator,\outsymbol}}}^2 \\
        & \quad \quad \cdot \prod_{\catenumerator\in[\seccatenumerator]}
        \left(\contraction{\qstateofat{\catenumerator}{\catvariableof{\catenumerator,\insymbol}},\contunitaryofat{\catenumerator}{\catvariableof{\catenumerator,\insymbol},\catvariableof{\catenumerator,\outsymbol},\indexedcatvariableof{\parentsof{\catenumerator},\outsymbol}}}\right)^2
        %\contraction{\qstateof{\catenumerator,*},\onehotmapofat{\catindexof{\catenumerator}}{\catvariableof{\catenumerator}}}{} \, .
    \end{align*}
    Here we used in the second equation the unitarity of the controlled unitaries to $\catenumerator\notin[\seccatenumerator]$.
    Since the indices $\catindexof{[\seccatenumerator]/\parentsof{\seccatenumerator}} = \catindexof{\nondescendantsof{\seccatenumerator}}$ appear only in the constant term, we conclude
    \begin{align*}
        \condprobat{\catvariableof{\seccatenumerator}}{\catvariableof{[\catenumerator]}} = \condprobat{\catvariableof{\seccatenumerator}}{\catvariableof{\parentsof{\seccatenumerator}}} \otimes \onesat{\catvariableof{\nondescendantsof{\seccatenumerator}}}\, ,
    \end{align*}
    which establishes the conditional independence $\condindependent{\catvariableof{\node}}{\catvariableof{\nondescendantsof{\node}}}{\catvariableof{\parentsof{\node}}}$.
\end{proof}

\begin{proof}[Proof of \theref{the:graphControlledPreparesBN}]
    The theorem follows directly from \lemref{lem:BNtoGCC} and \lemref{lem:GCCtoBN}, using that Bayesian Networks are characterized by the conditional independence of each variable to its non-descendants given its parents.
\end{proof}

\begin{example}[Chain decomposition]
    The hypergraph in \exaref{exa:genericControlCircuit} corresponds with a family of Bayesian Networks subsuming any probability distribution.
    This can be verified based on the chain decomposition of a generic distribution.
\end{example}

%% Phase representation
Another question is, whether each quantum state, which measurement distribution is a Bayesian Network can be prepared by a $\graph$-controlled circuit.
This is not always the case, since the phase tensor does not influence the measurement distribution.
Any phase tensor, of a by $\graph$-controlled circuit prepared state has however a decomposition
\begin{align*}
    \phasecoreat{\shortcatvariables}
    = \sum_{\catenumeratorin} \phasecoreofat{\catenumerator}{\catvariableof{\catenumerator},\catvariableof{\parentsof{\catenumerator}}} \otimes \onesat{\catvariableof{[\catorder]/\{\{\catenumerator\}\cup\parentsof{\catenumerator}\}}} \, ,
\end{align*}
where the phase cores $\phasecoreof{\catenumerator}$ can be read of the controlled unitaries.
When there are phase cores which do not have such a decomposition, the corresponding states are not representable.
\cite{mottonen_transformation_2005} shows, that when the graph is chosen as in \exaref{exa:genericControlCircuit}, then also the phases can be represented.


%\section{Function Encoding Schemes}\label{sec:functionEncoding}

\subsection{Ancilla Augmentation}

\red{
    Graph controlled circuits prepare Bayesian Networks, and do not apply to undirected graphical models such as Markov Networks and \ComputationActivationNetworks{}.
    We therefore introduce in the following ancilla augmentation, which is a method to represent Markov Networks as conditioned Bayesian Network.
    This enables usage of the previously used preparation schemes for Bayesian Networks.}

%% NEEDED ALREADY HERE?
%For more flexible sampling schemes of \ComputationActivationNetworks{} we need to introduce ancilla qubits.

\subsubsection{Post-selection by Ancilla Variables}

\begin{definition}[Ancilla Augmented Distribution]
    Let $\probwith$ be a probability distribution over variables $\shortcatvariables$.
    Another joint distribution $\secprobtensor$ of $\shortcatvariables$ and ancilla variables $\avariables$ is called an ancilla augmented distribution, if
    \begin{align*}
        \secprobat{\shortcatvariables|\avariables=\onesat{[\seldim]}}
        = \probwith \, .
    \end{align*}
\end{definition}

Sampling from the distribution can be done by rejection sampling on the ancilla augmented distribution, measuring all variables and rejecting all samples where an ancilla variable is $0$.

%% Usage through rejection sampling
Given an augmented Q-sample of a distribution, we can prepare samples from the distribution by rejection sampling, measuring all variables $\shortcatvariables$ and $\avariables$ and rejecting all samples where an ancilla qubit is measured as $0$.

%
Sampling can be done by sampling from an ancilla augmented distribution and post-selection on the ancilla variables being $1$.

\subsubsection{Construction}

%% Main point: Construction by activation circuits
Given a distribution $\probat{\shortcatvariables}$ we add an ancilla variable $\avariable$ and define the augmented distribution (see \figref{fig:ancillaAugmentation})
\begin{align*}
    \secprobat{\avariable,\shortcatvariables} =
    \frac{1}{\prod_{\catenumeratorin}\catdimof{\catenumerator}}
    \sum_{\shortcatindicesin} \onehotmapofat{\shortcatindices}{\shortcatvariables}
    \otimes \Big(
    \probat{\shortcatindices} \cdot \tbasisat{\avariable} + (1- \probat{\shortcatindices}) \cdot \fbasisat{\avariable}
    \Big) \, .
\end{align*}
Then we have
\begin{align*}
    \secprobat{\shortcatvariables|\avariable=1} = \probat{\shortcatvariables} \, .
\end{align*}

\begin{figure}
    \begin{center}
        \begin{tikzpicture}[scale=0.35,thick] % , baseline = -3.5pt

    \node[anchor=center] (text) at (28,1) {${b)}$};

    \begin{scope}
        [shift={(31,-7)}]

        \node[anchor=center] (text) at (-1,6.5) {\colorlabelsize $\avariableof{\insymbol}$};
        \node[anchor=center] (text) at (12,6.5) {\colorlabelsize $\avariableof{\outsymbol}$};

        \draw (5,4) -- (5,5.5);
        \drawvariabledot{5}{4}
        \draw (6,1) -- (6,5.5);
        \drawvariabledot{6}{1}

        \draw (1,6.5) -- (2,6.5);
        \draw (2,5.5) rectangle (9,7.5);
        \node[anchor=center] (text) at (5.65,6.5) {\corelabelsize $\qcaencodingof{\probtensor}$};
        \draw (9,6.5) -- (10,6.5);

        \draw[dashed] (0.5,5) -- (10.5,5);

        \draw (1,4) -- (10,4);

        \node[anchor=center] (text) at (3,2.75) {$\vdots$};
        \node[anchor=center] (text) at (-1,2.5) {\colorlabelsize $\shortcatvariables$};

        \node[anchor=center] (text) at (8,2.75) {$\vdots$};
        \node[anchor=center] (text) at (12,2.5) {\colorlabelsize $\shortcatvariables$};

        \draw (1,1) -- (10,1);

    \end{scope}


    \node[anchor=center] (text) at (1,1) {${a)}$};

    \draw[->-] (4.5,-1) -- (4.5,1) node[midway, right]{\colorlabelsize $\avariable$};
    \draw (3,-1) rectangle (6,-4);
    \node[anchor=center] (text) at (4.5,-2.5) {\corelabelsize $\secprobtensor$};

    \draw[->-] (3.5,-6) -- (3.5,-4);
    \node[anchor=center] (text) at (4.5,-4.75) {$\cdots$};
    \node[anchor=center] (text) at (4.5,-6) {\colorlabelsize $\shortcatvariables$};
    \draw[->-] (5.5,-6) -- (5.5,-4);

    \node[anchor=center] (text) at (8,-2.5) {${=}$};

    \begin{scope}
        [shift={(3,2)}]

        \node[anchor=center] (text) at (8,-4.5) {$\sum_{\shortcatindices}$};

        \draw[->-] (15,-1) -- (15,1) node[midway, right]{\colorlabelsize $\avariable$};
        \draw (10,-4) rectangle (20,-1);
        \node[anchor=center] (text) at (15,-2.5) {\corelabelsize $\begin{bmatrix}
                                                       1 - \probat{\indexedshortcatvariables} \\
                                                       \probat{\indexedshortcatvariables}
        \end{bmatrix}$};

        \draw (13,-5) rectangle (17,-7);
        \node[anchor=center] (text) at (15,-6) {\corelabelsize $\onehotmapof{\shortcatindices}$};
        \draw[->-] (13.5,-7) -- (13.5,-9);
        \node[anchor=center] (text) at (15,-7.4) {$\cdots$};
        \node[anchor=center] (text) at (15,-8.5) {\colorlabelsize $\shortcatvariables$};
        \draw[->-] (16.5,-7) -- (16.5,-9);

    \end{scope}

\end{tikzpicture}
    \end{center}
    \caption{
        Ancilla augmentation of a distribution $\probwith$.
        a) Augmented distribution $\secprobat{\avariable,\shortcatvariables}$ with the property that $\probwith = \secprobat{\shortcatvariables|\avariable=1}$.
        b) Preparation of the augmented distribution by the \activationCircuit{} of $\probwith$.
    }\label{fig:ancillaAugmentation}
\end{figure}

\begin{example}[Elementary Tensors]
    Let there be an elementary distribution
    \begin{align*}
        \probwith = \bigotimes_{\catenumeratorin} \probofat{\catenumerator}{\catvariableof{\catenumerator}} \, ,
    \end{align*}
    that is, all variables are pairwise independent.
    Then the ancilla augmentation of the distribution is given by the Bayesian Network with the hypergraph shown in \figref{fig:ancillaAugmentationElementary}a).
    Compared with the ancilla augmentation of a generic distribution this augmentation is sparser in the circuit depth (see \figref{fig:ancillaAugmentationElementary}b), but introduces multiple ancilla variables.
    The sparsity originates since each ancilla variable is controlled by a single qubit.
    The total amount of controlled rotations is thus $\sum_{\catenumeratorin}[\catdimof{\catenumerator}]$ instead of $\prod_{\catenumeratorin}[\catdimof{\catenumerator}]$ for the generic implementation.

    \begin{figure}
        \begin{center}
            \begin{tikzpicture}[scale=0.35, thick]

                \node[anchor=center] at (-2, 1) {$a)$};

                \node[circle, draw, thick, fill=\nodegrayscale, minimum size = \nodeminsize] (X0) at (0,-5) {};
                \node[anchor=center] (A) at (0,-5) {\corelabelsize $\catvariableof{0}$};
                \node[circle, draw, thick, fill=\nodegrayscale, minimum size = \nodeminsize] (A0) at (0,0) {};
                \node[anchor=center] (A) at (0,0) {\corelabelsize $\avariableof{0}$};
                \draw[->] (X0) -- (A0);

                \node[anchor=center] at (4, -2.5) {$\cdots$};

                \begin{scope}[shift={(8,0)}]
                    \node[circle, draw, thick, fill=\nodegrayscale, minimum size = \nodeminsize] (X0) at (0,-5) {};
                    \node[anchor=center] (A) at (0,-5) {\corelabelsize $\catvariableof{\catorder\shortminus1}$};
                    \node[circle, draw, thick, fill=\nodegrayscale, minimum size = \nodeminsize] (A0) at (0,0) {};
                    \node[anchor=center] (A) at (0,0) {\corelabelsize $\avariableof{\catorder\shortminus1}$};
                    \draw[->] (X0) -- (A0);
                \end{scope}

                \node[anchor=center] at (15, 1) {$b)$};

                \begin{scope}[shift={(20,-10)}]

                    \draw (-1,9) rectangle (1,11);
                    \node[anchor=center] (text) at (0,10) {$\onehotmapof{0}$};
                    \draw (1,10) -- (7,10);
                    \draw (7,9) rectangle (9,11);
                    \node[anchor=center] (text) at (8,10) {\corelabelsize $\qcaencodingof{\probof{\catorder\shortminus1}}$};
                    \draw (8,9) -- (8,3);
                    \drawvariabledot{8}{3}
                    \node[rotate=45] at (7,1) {$\cdots$};
                    \draw (9,10) -- (10,10);
                    \drawqcmeasuresymbol{11}{10}
                    \draw (12,10) -- (14,10) node[midway,above]{\colorlabelsize $\avariableof{\catorder\shortminus1}$};

                    \node[anchor=center] (text) at (0,8.25) {$\vdots$};

                    \draw (-1,5) rectangle (1,7);
                    \node[anchor=center] (text) at (0,6) {$\onehotmapof{0}$};
                    \draw (1,6) -- (4,6);
                    \draw (4,5) rectangle (6,7);
                    \node[anchor=center] (text) at (5,6) {\corelabelsize $\qcaencodingof{\probof{0}}$};
                    \draw (5,5) -- (5,-1);
                    \drawvariabledot{5}{-1}
                    \draw (6,6) -- (10,6);
                    \drawqcmeasuresymbol{11}{6}
                    \draw (12,6) -- (14,6) node[midway,above]{\colorlabelsize $\avariableof{0}$};


                    %% Distributed Qubits
                    \draw[dashed] (-3,4.5) -- (15,4.5);

                    \draw (-1,2) rectangle (1,4);
                    \node[anchor=center] (text) at (0,3) {$\onehotmapof{0}$};
                    \draw (1,3) -- (2,3);
                    \draw (2,2) rectangle (4,4);
                    \node[anchor=center] (text) at (3,3) {$\hgate$};

                    \draw (4,3) -- (10,3);
                    \drawqcmeasuresymbol{11}{3}
                    \draw (12,3) -- (14,3) node[midway,above]{\colorlabelsize $\catvariableof{\catorder\shortminus1}$};

                    \node[anchor=center] (text) at (0,1.25) {$\vdots$};
                    %\node[anchor=east] (text) at (-2,1) {Distributed Qubits};%{$\shortcatvariables$};

                    \draw (-1,-2) rectangle (1,0);
                    \node[anchor=center] (text) at (0,-1) {$\onehotmapof{0}$};
                    \draw (1,-1) -- (2,-1);
                    \draw (2,-2) rectangle (4,0);
                    \node[anchor=center] (text) at (3,-1) {$\hgate$};

                    \draw (4,-1) -- (10,-1);
                    \drawqcmeasuresymbol{11}{-1}
                    \draw (12,-1) -- (14,-1) node[midway,above]{\colorlabelsize $\catvariableof{0}$};

                \end{scope}
            \end{tikzpicture}
        \end{center}
        \caption{Ancilla augmentation of an elementary distribution (which variables are pairwise independent).
        a) Hypergraph to the Bayesian Network representing the ancilla augmentation.
        b) Circuit preparing the ancilla augmented distribution by \activationCircuits{} acting on the disentangled initial state.
        }\label{fig:ancillaAugmentationElementary}
    \end{figure}

\end{example}

