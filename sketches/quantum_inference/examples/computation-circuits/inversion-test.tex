\begin{example}[Inversion Test]
    Concatenating the \computationCircuits{} to two formulas $\exformula$ and $\secexformula$ prepares the state
    \begin{align*}
        \basqstateofat{\exformula\oplus\secexformula}{\shortcatvariables,\headvariable}
    \end{align*}
    The probability of measuring the ground state after a Walsh-Hadamard transform of the distributed qubits is then
    \begin{align*}
        \frac{1}{2^{\catorder}} \absof{\contractionof{\basqstateofat{\exformula\oplus\secexformula}{\shortcatvariables,\headvariable}}{\headvariable}}^2
        &= \frac{1}{2^{2\cdot\catorder}}
        \begin{bmatrix}
            \contraction{\exformula\oplus\secexformula}
            \contraction{\lnot(\exformula\oplus\secexformula)}
        \end{bmatrix} \\
        &= \frac{1}{2^{2\cdot\catorder}}
        \begin{bmatrix}
            \#{\left\{\shortcatindices\in\atomstates \wcols \exformulaat{\indexedshortcatvariables}=\secexformula\left[\indexedshortcatvariables\right]\right\}} \\
            \#{\left\{\shortcatindices\in\atomstates \wcols \exformulaat{\indexedshortcatvariables}\neq\secexformula\left[\indexedshortcatvariables\right]\right\}}
        \end{bmatrix}
    \end{align*}
    The second equation holds since $\lnot(\exformula\oplus\secexformula)$ is equivalent with the biconditional $\exformula\Leftrightarrow\secexformula$.

    % Inversion test
    This is in more generality the inversion test of measuring the overlap of $\basqstateof{\exformula}$ with $\basqstateof{\secexformula}$ (see \cite{schuld_machine_2021}).
    Here we used that the \computationCircuit{} is self-adjoint and understand the concatenation as the unitaries preparing $\basqstateof{\exformula}$ and the adjoint (together with the Walsh-Hadamard transform) of $\basqstateof{\secexformula}$.
\end{example}
