\section{Moment Matching by Amplitude Amplification}


We now investigate which \ComputationActivationNetworks{} can be prepared directly with amplitude amplification.

To start let
\begin{itemize}
    \item $\qstateofat{0}{\shortcatvariables}$ be a q-sample of a probability distribution $\probofat{0}{\shortcatvariables}$
    \item $U$ be a unitary preparing the state
    \item $\qcbencodingof{\exformula}$ be a \computationCircuit{} to a formula $\exformula$
\end{itemize}

The we split the initial state into vectors supported at the models of $\exformula$ and the complement as
\begin{align*}
    \qstateofat{0}{\shortcatvariables}=\qstateofat{+}{\shortcatvariables}+\qstateofat{-}{\shortcatvariables}
\end{align*}
where
\begin{align*}
    \qstateofat{+}{\shortcatvariables}
    &= \sum_{\shortcatindicesin\wcols\exformulaat{\indexedshortcatvariables}=1} \sqrt{\probat{\indexedshortcatvariables}} \onehotmapofat{\shortcatindices}{\shortcatvariables} \\
    \qstateofat{-}{\shortcatvariables}
    &= \sum_{\shortcatindicesin\wcols\exformulaat{\indexedshortcatvariables}=0} \sqrt{\probat{\indexedshortcatvariables}} \onehotmapofat{\shortcatindices}{\shortcatvariables} \, .
\end{align*}
We define an angle $\rotanglesymbol$ by
\begin{align*}
    \sinof{\frac{\rotanglesymbol}{2}} \coloneqq \|\qstateofat{+}{\shortcatvariables}\|_2
\end{align*}
and have
\begin{align*}
    \qstateofat{0}{\shortcatvariables}
    = \sinof{\frac{\rotanglesymbol}{2}} \frac{\qstateofat{+}{\shortcatvariables}}{\|\qstateofat{+}{\shortcatvariables}\|_2} +
    \cosof{\frac{\rotanglesymbol}{2}} \frac{\qstateofat{+}{\shortcatvariables}}{\|\qstateofat{+}{\shortcatvariables}\|_2} \, .
\end{align*}
Further we have that
\begin{align*}
    \meanparam^{0}
    = \contraction{\probwith,\formulawith}
    = \left(\|\qstateofat{+}{\shortcatvariables}\|_2\right)^2
    = \left(\sinof{\frac{\rotanglesymbol}{2}}\right)^2 \, .
\end{align*}

We prepare an ancilla qubit in the anti-symmetric state and consider as initial state
\begin{align*}
    \qstateofat{0}{\shortcatvariables} \otimes
    \sqrt{\frac{1}{2}}\cdot
    \begin{bmatrix}
        1 \\ -1
    \end{bmatrix} \,.
\end{align*}

The \computationCircuit{} $\qcbencodingof{\exformula}$ acts on this state as the
\begin{align*}
    \left(\qstateofat{-}{\shortcatvariables}-\qstateofat{+}{\shortcatvariables}\right) \otimes
    \sqrt{\frac{1}{2}}\cdot
    \begin{bmatrix}
        1 \\ -1
    \end{bmatrix}[\avariable] \,.
\end{align*}
This is the reflection on the subspace spanned by the one-hot encodings of the models of $\exformula$.

Rotating $\repnum$ times across the intial state then gives the tensor product of the antisymmetric ancilla state with
\begin{align*}
    \qstateofat{\repnum}{\shortcatvariables}
    = \sinof{\left(\frac{1}{2}+\repnum\right)\rotanglesymbol} \frac{\qstateofat{+}{\shortcatvariables}}{\|\qstateofat{+}{\shortcatvariables}\|_2} +
    \cosof{\left(\frac{1}{2}+\repnum\right)\rotanglesymbol} \frac{\qstateofat{+}{\shortcatvariables}}{\|\qstateofat{+}{\shortcatvariables}\|_2} \, .
\end{align*}
This reflection can be performed by the operator on the distributed qubits
\begin{align*}
    U^t \left(
            \identityat{\shortcatvariables^{\insymbol},\shortcatvariables^{\outsymbol}}-2\onehotmapofat{0}{\shortcatvariables^{\insymbol}} \otimes \onehotmapofat{0}{\shortcatvariables^{\outsymbol}}
    \right) U \, .
\end{align*}

When $\left(\frac{1}{2}+\repnum\right)\rotanglesymbol\leq \frac{\pi}{2}$ then all amplitudes remain positive, and $\qstateofat{\repnum}{\shortcatvariables}$ is a q-sample.


% As exponential family parameters
The mean parameter of $\exformula$ with respect to the measurement distribution is
\begin{align*}
    \meanparam^{\repnum}
    = \contraction{\probofat{\repnum}{\shortcatvariables},\formulawith}
    = \left(\frac{
        2\cdot \sinof{\left(\frac{1}{2}+\repnum\right)\rotanglesymbol} \cdot \sin^{-1}\left(\sqrt{\meanparam^{0}}\right)
    }{\meanparam^{0}}\right)^2
\end{align*}
This change of mean parameters can be understood by increasing the canonical parameter to $\exformula$ by
\begin{align*}
    \canparam
    = \frac{
        \cos^2\left(\frac{\rotanglesymbol}{2}\right) \left(1-
                                                         \cos^2\left(\left(\frac{1}{2}+\repnum\right) \rotanglesymbol \right)
        \right)
    }{
        \sin^2\left(\frac{\rotanglesymbol}{2}\right)
        \cdot \cos^2\left(\left(\frac{1}{2}+\repnum\right)\rotanglesymbol\right)
    }
\end{align*}
we here allow for $\canparam=\infty$, which corresponds to the case of a hard activation to $\exformula$.

We understand this rotation as a moment matching operation to the formula $\exformula$, see \algoref{alg:QCMM}.

\begin{algorithm}
    \caption{Quantum Circuit Moment Matcher}\label{alg:QCMM}
    \begin{algorithmic}
        \Require Formulas $\formulaofat{\seldim}{\shortcatvariables}$ for $\selindexin$, vector $\datameanat{\selvariable}$ of mean parameters
        \Ensure Circuit $U$ preparing a state (when applied on the ground state) which measurement distribution matched $\datameanat{\selvariable}$\\
        \hrule
        \State $U\algdefsymbol (\paulixsymbol\circ\hgate)\left[\avariableof{\insymbol},\avariableof{\outsymbol}\right] \otimes
        \left(\bigotimes_{\catenumeratorin}\hgateat{\catvariableof{\catenumerator,\insymbol},\catvariableof{\catenumerator,\outsymbol}}\right)
        $
        \While{Convergence criterion not met}
%        \For{$\selindexin$}
            \State Estimate $\meanparamat{\selvariable}$,
            \begin{itemize}
                \item either by particle bases inference
                \item or by quantum counting (measuring the eigenvalue of the Grover operator)
            \end{itemize}
            \State Select an $\selindexin$, which moment has to be updated (e.g. by comparison of $\meanparamat{\selvariable}$ with $\datameanat{\selvariable}$)
            \State Compute optimal $\repnum$ to match $\datameanat{\selvariable}$
            \State Extend Circuit $U$ by $\repnum$ alternations of
            \begin{itemize}
                \item $\qcbencodingof{\formulaof{\selindex}}$ (effectively an reflection across the subspace spanned by one-hot encoded models)
                \item reflections across the current state (by $U(I-2e_0e_0)U$)
            \end{itemize}
        \EndWhile
    \end{algorithmic}
\end{algorithm}
