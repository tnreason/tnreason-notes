\section{Moment Matching by Amplitude Amplification}

We here apply \computationCircuits{} on anti-symmetric ancilla states (as for sign encoding)
\begin{itemize}
    \item A single \computationCircuit{} effectively prepares a reflection across the models of the computed formula
    \item Building the reflections based on ancilla ground states (as for basis encoding), we would need an additional Pauli-Z and two \computationCircuits{}
    \item Further ancilla qubits used in circuit decompositions need to be uncomputed (i.e. by applying the same partial \computationCircuit{} again) for further usage
\end{itemize}

We now investigate which \ComputationActivationNetworks{} can be prepared directly with amplitude amplification.

\subsection{Reflection by single \computationCircuits{}}

To start let
\begin{itemize}
    \item $\qstateofat{0}{\shortcatvariables}$ be a q-sample of a probability distribution $\probofat{0}{\shortcatvariables}$
    \item $U$ be a unitary preparing the state
    \item $\qcbencodingof{\exformula}$ be a \computationCircuit{} to a formula $\exformula$
\end{itemize}

The we split the initial state into vectors supported at the models of $\exformula$ and the complement as
\begin{align*}
    \qstateofat{0}{\shortcatvariables}=\qstateofat{+}{\shortcatvariables}+\qstateofat{-}{\shortcatvariables}
\end{align*}
where
\begin{align*}
    \qstateofat{+}{\shortcatvariables}
    &= \sum_{\shortcatindicesin\wcols\exformulaat{\indexedshortcatvariables}=1} \sqrt{\probat{\indexedshortcatvariables}}
    \cdot \onehotmapofat{\shortcatindices}{\shortcatvariables} \\
    \qstateofat{-}{\shortcatvariables}
    &= \sum_{\shortcatindicesin\wcols\exformulaat{\indexedshortcatvariables}=0} \sqrt{\probat{\indexedshortcatvariables}}
    \cdot \onehotmapofat{\shortcatindices}{\shortcatvariables} \, .
\end{align*}
We define an angle $\rotanglesymbol$ by
\begin{align*}
    \sinof{\frac{\rotanglesymbol}{2}} \coloneqq \normof{\qstateofat{+}{\shortcatvariables}}
\end{align*}
and have
\begin{align*}
    \qstateofat{0}{\shortcatvariables}
    = \sinof{\frac{\rotanglesymbol}{2}} \frac{\qstateofat{+}{\shortcatvariables}}{\|\qstateofat{+}{\shortcatvariables}\|_2} +
    \cosof{\frac{\rotanglesymbol}{2}} \frac{\qstateofat{+}{\shortcatvariables}}{\|\qstateofat{+}{\shortcatvariables}\|_2} \, .
\end{align*}
Further we have that
\begin{align*}
    \meanparam^{0}
    &= \contraction{\probofat{0}{\shortcatvariables},\formulawith} \\
    &= \sum_{\shortcatindicesin} \absof{\qstateofat{0}{\indexedshortcatvariables}\formulaat{\indexedshortcatvariables}}^2 \\
    &= \left(\|\qstateofat{+}{\shortcatvariables}\|_2\right)^2 \\
    &= \left(\sinof{\frac{\rotanglesymbol}{2}}\right)^2 \, .
\end{align*}

We prepare an ancilla qubit in the anti-symmetric state and consider as initial state
\begin{align*}
    \qstateofat{0}{\shortcatvariables} \otimes
    \sqrt{\frac{1}{2}}\cdot
    \begin{bmatrix}
        1 \\ -1
    \end{bmatrix} \,.
\end{align*}

The \computationCircuit{} $\qcbencodingof{\exformula}$ acts on this state as the
\begin{align*}
    \left(\qstateofat{-}{\shortcatvariables}-\qstateofat{+}{\shortcatvariables}\right) \otimes
    \sqrt{\frac{1}{2}}\cdot
    \begin{bmatrix}
        1 \\ -1
    \end{bmatrix}[\avariable] \,.
\end{align*}
This is the reflection on the subspace spanned by the one-hot encodings of the models of $\exformula$.

Rotating $\repnum$ times across the intial state then gives the tensor product of the antisymmetric ancilla state with
\begin{align*}
    \qstateofat{\repnum}{\shortcatvariables}
    = \sinof{\left(\frac{1}{2}+\repnum\right)\rotanglesymbol} \frac{\qstateofat{+}{\shortcatvariables}}{\|\qstateofat{+}{\shortcatvariables}\|_2} +
    \cosof{\left(\frac{1}{2}+\repnum\right)\rotanglesymbol} \frac{\qstateofat{+}{\shortcatvariables}}{\|\qstateofat{+}{\shortcatvariables}\|_2} \, .
\end{align*}
This reflection can be performed by the operator on the distributed qubits
\begin{align*}
    U^t \left(
            \identityat{\shortcatvariables^{\insymbol},\shortcatvariables^{\outsymbol}}-2\onehotmapofat{0}{\shortcatvariables^{\insymbol}} \otimes \onehotmapofat{0}{\shortcatvariables^{\outsymbol}}
    \right) U \, .
\end{align*}

When $\left(\frac{1}{2}+\repnum\right)\rotanglesymbol\leq \frac{\pi}{2}$ then all amplitudes remain positive, and $\qstateofat{\repnum}{\shortcatvariables}$ is a q-sample.

\begin{theorem}
    Let $\qstateof{0}$ be the q-sample of a maximum entropy distribution with respect to the uniform base measure and the statistic $\formulaset$, and let $\repnum\in\nn$ be a rotation number such that $\left(\frac{1}{2}+\repnum\right)\rotanglesymbol\leq\frac{\pi}{2}$.
    Then also $\qstateof{\repnum}$ is the q-sample of a maximum entropy distribution .
\end{theorem}
\begin{proof}
    Since the measurement distribution of any quantum state in the plane spanned by $\qstateof{+}$ and $\qstateof{-}$ is an elementary \ComputationActivationNetwork{}.
    When the assumption on $\repnum$ is met, we further have positive real amplitudes and therefore a q-sample.
%    Since each amplitude amplification can be understood as an change of mean parameters.
    To be more precise, we characterize the change of canonical parameters in the lemma below.
\end{proof}



\begin{lemma}
    Let $\qstateof{0}$ be the q-sample of the elementary \ComputationActivationNetwork{} with canonical parameters $\hybridparam$, and let us amplify the formula $\formulaof{\selindex}$.
    Then $\qstateof{\repnum}$ is the q-sample of the elementary \ComputationActivationNetwork{} with canonical parameters
    \begin{align*}
        \hardlegset^{\repnum} =
        \begin{cases}
            \hardlegset \cup \{\selindex\}, \headindexof{\hardlegset}^{\selindex} = (\headindexof{\hardlegset},1) & \ifspace \sinof{(1+2\cdot\repnum)\sin^{-1}\left(\sqrt{\meanparamofat{0}{\indexedselvariable}}\right)} = 1 \\
            \hardlegset, \headindexof{\hardlegset} & \elsetext
        \end{cases}
    \end{align*}
    and
    \begin{align*}
        \canparamofat{\repnum}{\indexedselvariable} =
        \begin{cases}
            0 & \ifspace  \sinof{(1+2\cdot\repnum)\sin^{-1}\left(\sqrt{\meanparamofat{0}{\indexedselvariable}}\right)} = 1 \\
            \canparamat{\indexedselvariable} + \frac{
                \cos^2\left(\frac{\rotanglesymbol}{2}\right) \left(1-
                                                                 \cos^2\left(\left(\frac{1}{2}+\repnum\right) \rotanglesymbol \right)
                \right)
            }{
                \sin^2\left(\frac{\rotanglesymbol}{2}\right)
                \cdot \cos^2\left(\left(\frac{1}{2}+\repnum\right)\rotanglesymbol\right)
            } & \elsetext
        \end{cases} \, .
    \end{align*}
\end{lemma}
\begin{proof}
    Notice, that
    \begin{align*}
        \contractionof{\probofat{0}{\shortcatvariables},\bencodingofat{\formulaof{\selindex}}{\headvariableof{\selindex},\shortcatvariables}}{\headvariableof{\selindex}}
        = \coloredmatrixof{
              \left(\cosof{\frac{\rotanglesymbol}{2}}\right)^2 \\
              \left(\sinof{\frac{\rotanglesymbol}{2}}\right)^2
        }{\headvariableof{\selindex}}
        = \coloredmatrixof{
            1-\meanparamofat{0}{\indexedselvariable} \\
            \meanparamofat{0}{\indexedselvariable}
        }{\headvariableof{\selindex}}
    \end{align*}
    and
    \begin{align*}
        \contractionof{\probofat{\repnum}{\shortcatvariables},\bencodingofat{\formulaof{\selindex}}{\headvariableof{\selindex},\shortcatvariables}}{\headvariableof{\selindex}}
        = 
        \coloredmatrixof{
              \left(\cosof{\left(\frac{1}{2}+\repnum\right)\rotanglesymbol}\right)^2 \\
              \left(\sinof{\left(\frac{1}{2}+\repnum\right)\rotanglesymbol}\right)^2
        }{\headvariableof{\selindex}}
        \coloredmatrixof{
            1-\meanparamofat{\repnum}{\indexedselvariable} \\
            \meanparamofat{\repnum}{\indexedselvariable}
        }{\headvariableof{\selindex}}
    \end{align*}
    If $\meanparamofat{\repnum}{\indexedselvariable}=1$ (that is $\sinof{\left(\frac{1}{2}+\repnum\right)\rotanglesymbol}=1$), the amplitude amplification amounts to adding $\formulaof{\selindex}$ as a hard constraint.
    This is equal to $\hardlegset^{\repnum}=\hardlegset\cup\{\selindex\}$ and $\headindexof{\hardlegset}^{\repnum}=(\headindexof{\hardlegset},1)$.

    If $\meanparamofat{\repnum}{\indexedselvariable}\neq 1$ we choose $\canparamofat{\Delta}{\indexedselvariable}\geq 0$ as in the claim and have
    \begin{align*}
        \frac{
        \left(\cosof{\left(\frac{1}{2}+\repnum\right)\rotanglesymbol}\right)^2 
        }{
        \left(\cosof{\left(\frac{1}{2}+\repnum\right)\rotanglesymbol}\right)^2 + \canparamofat{\Delta}{\indexedselvariable} \cdot \left(\sinof{\left(\frac{1}{2}+\repnum\right)\rotanglesymbol}\right)^2
        } = \left(\cosof{\left(\frac{1}{2}+\repnum\right)\rotanglesymbol}\right)^2
    \end{align*}
    and therefore
    \begin{align*}
         \contractionof{\probofat{\repnum}{\shortcatvariables},\bencodingofat{\formulaof{\selindex}}{\headvariableof{\selindex},\shortcatvariables}}{\headvariableof{\selindex}}
        = \normalizationof{
         \expof{\canparamofat{\Delta}{\indexedselvariable}\cdot\formulaofat{\selindex}{\canparam}}
         \probofat{0}{\shortcatvariables},\bencodingofat{\formulaof{\selindex}}{\headvariableof{\selindex},\shortcatvariables}
         }{\headvariableof{\selindex}} \, .
    \end{align*}
    Thus, the manipulation of the q-sample corresponds with an increase of the canonical parameter $\canparamofat{0}{\indexedselvariable}$ by $\canparamofat{\Delta}{\indexedselvariable}$.
\end{proof}

%
Notice, that we can only increase the canonical parameter by amplitude amplification.
Decreasing could be done by amplitude amplification on $\lnot\formulaof{\selindex}$ instead.


\subsection{Moment Matching based preparation of \CompActNets{}}

% As exponential family parameters
The mean parameter of $\exformula$ with respect to the measurement distribution is
\begin{align*}
    \meanparam^{\repnum}
    = \contraction{\probofat{\repnum}{\shortcatvariables},\formulawith}
    = \sinof{\left(\frac{1}{2}+\repnum\right)\rotanglesymbol}^2
    = \sinof{\left(1+2\repnum\right)\sin^{-1}\left(\sqrt{\meanparam^{0}}\right)}^2
    %= \left(\frac{
    %    2\cdot \sinof{\left(\frac{1}{2}+\repnum\right)\rotanglesymbol} \cdot \sin^{-1}\left(\sqrt{\meanparam^{0}}\right)
    %}{\meanparam^{0}}\right)^2
\end{align*}
Here we used that $\rotanglesymbol = 2\cdot\sin^{-1}\left(\sqrt{\meanparam^{0}}\right)$.

%This change of mean parameters can be understood by increasing the canonical parameter to $\exformula$ by
%\begin{align*}
%    \canparamof{\repnum} - \canparamof{0}
%    = \frac{
%        \cos^2\left(\frac{\rotanglesymbol}{2}\right) \left(1-
%                                                         \cos^2\left(\left(\frac{1}{2}+\repnum\right) \rotanglesymbol \right)
%        \right)
%    }{
%        \sin^2\left(\frac{\rotanglesymbol}{2}\right)
%        \cdot \cos^2\left(\left(\frac{1}{2}+\repnum\right)\rotanglesymbol\right)
%    }
%\end{align*}
%we here allow for $\canparam=\infty$, which corresponds to the case of a hard activation to $\exformula$.

We understand this rotation as a moment matching operation to the formula $\exformula$, see \algoref{alg:QCMM}.

\begin{algorithm}
    \caption{Quantum Circuit Moment Matcher}\label{alg:QCMM}
    \begin{algorithmic}
        \Require Formulas $\formulaofat{\seldim}{\shortcatvariables}$ for $\selindexin$, vector $\datameanat{\selvariable}$ of mean parameters
        \Ensure Circuit $U$ preparing a state (when applied on the ground state) which measurement distribution matched $\datameanat{\selvariable}$\\
        \hrule
        \State $U\algdefsymbol (\paulixsymbol\circ\hgate)\left[\avariableof{\insymbol},\avariableof{\outsymbol}\right] \otimes
        \left(\bigotimes_{\catenumeratorin}\hgateat{\catvariableof{\catenumerator,\insymbol},\catvariableof{\catenumerator,\outsymbol}}\right)
        $
        \While{Convergence criterion not met}
%        \For{$\selindexin$}
            \State Estimate $\meanparamat{\selvariable}$,
            \begin{itemize}
                \item either by particle bases inference
                \item or by quantum counting (measuring the eigenvalue of the Grover operator)
            \end{itemize}
            \State Select an $\selindexin$, which moment has to be updated (e.g. by comparison of $\meanparamat{\selvariable}$ with $\datameanat{\selvariable}$)
            \State Compute optimal $\repnum$ to match $\datameanat{\selvariable}$
            \State Extend Circuit $U$ by $\repnum$ alternations of
            \begin{itemize}
                \item $\qcbencodingof{\formulaof{\selindex}}$ (effectively an reflection across the subspace spanned by one-hot encoded models)
                \item reflections across the current state (by $U(I-2e_0e_0)U$)
            \end{itemize}
        \EndWhile
    \end{algorithmic}
\end{algorithm}


% Outook
We can use \algoref{alg:QCMM} as a preparation algorithm of a q-sample before doing ancilla augmentation as in the main section.
Note that is important to keep track of the canonical parameters in the preparation.
When activating the statistic qubits, only the difference of the already prepared canonical parameters and the target canonical parameters has to be used (importance sampling interpretation).