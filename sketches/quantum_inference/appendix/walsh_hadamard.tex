\section{Walsh-Hadamard transform}

The application of Hadamard gates on the distributed variables afterwards results in a Walsh-Hadamard transform of the prepared state.

\subsection{Transformation tensor}

The transform is performed by applying Hadamard gates on each variable.
This is equivalent with contraction of the tensor
\begin{align*}
    \hgate^{\catorder}\left[\catvariableof{[\catorder],\insymbol},\catvariableof{[\catorder],\outsymbol}\right] =
    \bigotimes_{\catenumeratorin} \hgateat{\catvariableof{\catenumerator,\insymbol},\catvariableof{\catenumerator,\outsymbol}} \, ,
\end{align*}
which coordinates are
\begin{align*}
    \hgate^{\catorder}\left[\indexedcatvariableof{[\catorder],\insymbol},\indexedcatvariableof{[\catorder],\outsymbol}\right] =
    \sqrt{\frac{1}{2^{\catorder}}} \left(-1\right)^{\sum_{\catenumeratorin}\catindexof{\catenumerator,\insymbol} \cdot \catindexof{\catenumerator,\outsymbol}} \, .
\end{align*}

We have
\begin{align*}
    \contractionof{\onehotmapofat{0}{\catvariableof{\insymbol}},
        \hgateat{\catvariableof{\insymbol},\catvariableof{\outsymbol}}}{\catvariableof{\outsymbol}}
    &= \sqrt{\frac{1}{2}} \onesat{\catvariableof{\outsymbol}} \\
    \contractionof{\onehotmapofat{1}{\catvariableof{\insymbol}},
        \hgateat{\catvariableof{\insymbol},\catvariableof{\outsymbol}}}{\catvariableof{\outsymbol}}
    &= \sqrt{\frac{1}{2}}
    \begin{bmatrix}
        1 \\
        -1
    \end{bmatrix}\left[\catvariableof{\outsymbol}\right]
\end{align*}
and conversely
\begin{align*}
    \contractionof{\sqrt{\frac{1}{2}} \onesat{\catvariableof{\insymbol}},
        \hgateat{\catvariableof{\insymbol},\catvariableof{\outsymbol}}}{\catvariableof{\outsymbol}}
    &= \onehotmapofat{0}{\catvariableof{\outsymbol}} \\
    \contractionof{\sqrt{\frac{1}{2}}
        \begin{bmatrix}
            1 \\
            -1
        \end{bmatrix}\left[\catvariableof{\insymbol}\right],
        \hgateat{\catvariableof{\insymbol},\catvariableof{\outsymbol}}}{\catvariableof{\outsymbol}}
    &= \onehotmapofat{1}{\catvariableof{\outsymbol}} \, .
\end{align*}
If and only if the function is constant, then the transformed state is the ground state.

\subsection{Deutsch-Josza Algorithm}

We so far worked mainly with basis encoding states, which are prepared by a Hadamard transform and the \computationCircuit{} on statistic qubits in the ground state $\fbasisat{\headvariable}$.
This ensured that the deterministic function relation is a hard constraint in the measurement distribution.
If we apply a \computationCircuit{} on a different initial state of the statistic qubits, this constraint is in general not satisfied any more.

In the Deutsch-Josza algorithm \cite{deutsch_rapid_1992}, the Walsh-Hadamard transform of a sign encoding of a boolean function is measured to determine whether the function is constant.
If $\exformula$ is constant, then the Walsh-Hadamard transform of the sign encoding results in the tensor product of the ground state $\onehotmapofat{0\ldots0}{\shortcatvariables}$ with a head qubit state.

%% Application: Check equivalence of two formulas
Concatenating the \computationCircuit{} to formulas $\formula$ and $\secformula$ prepares the \computationCircuit{} of the formula $\formula\oplus\secformula$.
Both are equal or negations of each other, if and only if the distributed qubits are disentangled from the head qubit.
Doing the Deutsch-Josza test on $\formula\oplus\secformula$ thus decides whether two formulas are equal or negations of each other.