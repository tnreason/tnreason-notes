\section{Extension: Sampling from proposal distributions}

We can prepare basis circuit encodings to selection augmented formulas, in this way introducing formula selecting networks.

\textbf{Idea for an inductive reasoning scheme:} Prepare a q-sample from the empirical distribution and the current distribution.
Then prepare the basis circuit encodings, where the selection variables are shared and the distributed variables assigned to the prepared samples.
Now, the ancilla qubits can be designed to $\onehotmapof{1}$ and $\onehotmapof{0}$ accordingly.
The rejection sampling scheme on both ancillas being $1$ and the measurement of $\selvariable$ prepares then the distribution
\begin{align*}
    \normalizationof{
        \contractionof{\empdistributionwith, \sencmlnstatwith}{\selvariable}, \contractionof{\currentdistributionwith}{\selvariable}
    }{\selvariable}
\end{align*}
That is, the probability of selecting $\selindex$ is proportional to
\begin{align*}
    \datameanat{\indexedselvariable} \cdot (1-\currentmeanat{\indexedselvariable})
\end{align*}
and thus prefers formulas, which have a large empirical mean, but a small current mean.

\textbf{Open Question:} Since the distribution is "similar" to $\expof{\datameanat{\indexedselvariable}-\currentmeanat{\indexedselvariable})}$ (terms appear in Taylor of first order), can we tune the distribution with an inverse temperature parameter $\beta$?

