\section{Comparing tensor networks and quantum circuits}

First of all, we need to extend to complex tensors, which are maps
\begin{align*}
    \hypercore : \atomstates \rightarrow \mathbb{C} \,
\end{align*}
with image in $\mathbb{C}$ instead of $\mathbb{R}$ as in the report.

A coarse comparison of the nomenclature used for quantum circuits and tensor networks:

\begin{center}
    \begin{tabular}{l|l}
        \textbf{Quantum Circuit} & \textbf{Tensor Network}   \\
        \hline
        Qubit                    & Boolean Variable          \\
        Quantum Gate             & Unitary Tensor            \\
        Quantum Circuit          & Tensor Network on a graph
    \end{tabular}
\end{center}

Some constraints appear for a tensor network to be a quantum circuit
\begin{itemize}
    \item \textbf{Unitarity of each gate:} That is the variables of each tensor are bipartite into sets $\variablesetof{\insymbol}$ and $\variablesetof{\outsymbol}$ of same cardinality and the basis encoding with respect to this bipartition, that is
    \begin{align*}
        T_{\insymbol \rightarrow \outsymbol}[\catvariableof{\insymbol},\catvariableof{\outsymbol}] : \bigotimes_{\atomenumerator\in\variablesetof{\insymbol}} \mathbb{C}^2 \rightarrow \bigotimes_{\atomenumerator\in\variablesetof{\outsymbol}} \mathbb{C}^2  \, ,
    \end{align*}
    is a unitary map, that is
    \begin{align*}
        \left(T_{\insymbol \rightarrow \outsymbol}\right)^H \circ \left(T_{\insymbol \rightarrow \outsymbol}\right)
        = \contractionof{
            T_{\insymbol \rightarrow \outsymbol}[\catvariableof{\insymbol},\seccatvariable],
            \overline{T}_{\insymbol \rightarrow \outsymbol}[\seccatvariable,\catvariableof{\outsymbol}]
        }{\catvariableof{\outsymbol},\catvariableof{\insymbol}}
        = \identityat{\catvariableof{\outsymbol},\catvariableof{\insymbol}}.
    \end{align*}
    \item \textbf{Incoming-Outgoing structure:} Variable appear at most once as incoming and at most once as outgoing variables.
    Those not appearing as outgoing (respectively as incoming) are the input and the output variables of the whole circuit.
    \item \textbf{Acyclicity:} Incoming and outgoing variables of each tensor core provide a direction of each edge tensor. With respect to this directionality the graph underlying the tensor network has to be acyclic.
\end{itemize}

The unitary tensors can be aligned layerwise, if and only if the last two assumption hold, i.e. the directed graph is acyclic and each variable appears at most once as an incoming and at most once as an outgoing variable.



