\section{POVM measurements as contractions}

The main difficulty of using quantum circuits as contraction providers is that we can only extract information through measurements.
Therefore measurement is the only way to execute contractions of the circuit, which come with restrictions when interested in contraction with open variables.

The most general measurement formalism is through a POVM, a set $\{E_\seccatindex \, : \, \seccatindex\in[r]\}$ of positive operators with % Nielsen book notation
\begin{align*}
    \sum_{\seccatindex\in[r]} E_\seccatindex = I
\end{align*}

Measuring a pure state $\ket{\psi}$ We then get outcome $m$ with probability
\begin{align*}
    \braket{\psi|E_\seccatindex|\psi}
\end{align*}

We define a measurement variable $\headvariable$ taking indices $\seccatindex\in[r]$ and a measurement tensor
\begin{align*}
    E[\headvariable,\catvariableof{\insymbol},\catvariableof{\outsymbol}]
\end{align*}
with slices
\begin{align*}
    E[\headvariable=\seccatindex,\catvariableof{\insymbol},\catvariableof{\outsymbol}] = E_\seccatindex \, .
\end{align*}

Repeating the measurement asymptotically on a state $\ket{\psi}$ prepared by a quantum circuit $\tnetof{\graph}$ acting on the trivial start state $\ones$, we denote the measurement outcome by $\seccatindex^\datindex$.
In the limit $\datanum\rightarrow\infty$ we get almost surely
%    we thus get the contraction
\begin{align*}
    \frac{1}{\datanum} \sum_{\datindexin} \onehotmapofat{\seccatindex^\datindex}{\headvariable} \rightarrow
    \contractionof{\tnetof{\graph}[\catvariableof{\insymbol}],E[\headvariable,\catvariableof{\insymbol},\catvariableof{\outsymbol}],\tnetof{\secgraph}[\catvariableof{\outsymbol}]}{\headvariable} \, .
\end{align*}

% Case of computational basis measurements
POVMs to computational basis measurements of subsets of qubits are constructed as products with delta tensors on the non-measured qubits.