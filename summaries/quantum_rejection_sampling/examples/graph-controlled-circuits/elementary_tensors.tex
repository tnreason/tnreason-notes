\begin{example}[Elementary Tensors]
    Let there be an elementary distribution
    \begin{align*}
        \probwith = \bigotimes_{\catenumeratorin} \probofat{\catenumerator}{\catvariableof{\catenumerator}} \, ,
    \end{align*}
    that is, all variables are pairwise independent.
    Then the ancilla augmentation of the distribution is given by the Bayesian Network with the hypergraph shown in \figref{fig:ancillaAugmentationElementary}a).
    Compared with the ancilla augmentation of a generic distribution this augmentation is sparser in the circuit depth (see \figref{fig:ancillaAugmentationElementary}b), but introduces multiple ancilla variables.
    The sparsity originates since each ancilla variable is controlled by a single qubit.
    The total amount of controlled rotations is thus $\sum_{\catenumeratorin}[\catdimof{\catenumerator}]$ instead of $\prod_{\catenumeratorin}[\catdimof{\catenumerator}]$ for the generic implementation.

    \begin{figure}
        \begin{center}
            \begin{tikzpicture}[scale=0.35, thick]

                \node[anchor=center] at (-2, 1) {$a)$};

                \node[circle, draw, thick, fill=\nodegrayscale, minimum size = \nodeminsize] (X0) at (0,-5) {};
                \node[anchor=center] (A) at (0,-5) {\corelabelsize $\catvariableof{0}$};
                \node[circle, draw, thick, fill=\nodegrayscale, minimum size = \nodeminsize] (A0) at (0,0) {};
                \node[anchor=center] (A) at (0,0) {\corelabelsize $\avariableof{0}$};
                \draw[->] (X0) -- (A0);

                \node[anchor=center] at (4, -2.5) {$\cdots$};

                \begin{scope}[shift={(8,0)}]
                    \node[circle, draw, thick, fill=\nodegrayscale, minimum size = \nodeminsize] (X0) at (0,-5) {};
                    \node[anchor=center] (A) at (0,-5) {\corelabelsize $\catvariableof{\catorder\shortminus1}$};
                    \node[circle, draw, thick, fill=\nodegrayscale, minimum size = \nodeminsize] (A0) at (0,0) {};
                    \node[anchor=center] (A) at (0,0) {\corelabelsize $\avariableof{\catorder\shortminus1}$};
                    \draw[->] (X0) -- (A0);
                \end{scope}

                \node[anchor=center] at (15, 1) {$b)$};

                \begin{scope}[shift={(20,-10)}]

                    \draw (-1,9) rectangle (1,11);
                    \node[anchor=center] (text) at (0,10) {$\onehotmapof{0}$};
                    \draw (1,10) -- (7,10);
                    \draw (7,9) rectangle (9,11);
                    \node[anchor=center] (text) at (8,10) {\corelabelsize $\qcaencodingof{\probof{\catorder\shortminus1}}$};
                    \draw (8,9) -- (8,3);
                    \drawvariabledot{8}{3}
                    \node[rotate=45] at (7,1) {$\cdots$};
                    \draw (9,10) -- (10,10);
                    \drawqcmeasuresymbol{11}{10}
                    \draw (12,10) -- (14,10) node[midway,above]{\colorlabelsize $\avariableof{\catorder\shortminus1}$};

                    \node[anchor=center] (text) at (0,8.25) {$\vdots$};

                    \draw (-1,5) rectangle (1,7);
                    \node[anchor=center] (text) at (0,6) {$\onehotmapof{0}$};
                    \draw (1,6) -- (4,6);
                    \draw (4,5) rectangle (6,7);
                    \node[anchor=center] (text) at (5,6) {\corelabelsize $\qcaencodingof{\probof{0}}$};
                    \draw (5,5) -- (5,-1);
                    \drawvariabledot{5}{-1}
                    \draw (6,6) -- (10,6);
                    \drawqcmeasuresymbol{11}{6}
                    \draw (12,6) -- (14,6) node[midway,above]{\colorlabelsize $\avariableof{0}$};


                    %% Distributed Qubits
                    \draw[dashed] (-3,4.5) -- (15,4.5);

                    \draw (-1,2) rectangle (1,4);
                    \node[anchor=center] (text) at (0,3) {$\onehotmapof{0}$};
                    \draw (1,3) -- (2,3);
                    \draw (2,2) rectangle (4,4);
                    \node[anchor=center] (text) at (3,3) {$\hgate$};

                    \draw (4,3) -- (10,3);
                    \drawqcmeasuresymbol{11}{3}
                    \draw (12,3) -- (14,3) node[midway,above]{\colorlabelsize $\catvariableof{\catorder\shortminus1}$};

                    \node[anchor=center] (text) at (0,1.25) {$\vdots$};
                    %\node[anchor=east] (text) at (-2,1) {Distributed Qubits};%{$\shortcatvariables$};

                    \draw (-1,-2) rectangle (1,0);
                    \node[anchor=center] (text) at (0,-1) {$\onehotmapof{0}$};
                    \draw (1,-1) -- (2,-1);
                    \draw (2,-2) rectangle (4,0);
                    \node[anchor=center] (text) at (3,-1) {$\hgate$};

                    \draw (4,-1) -- (10,-1);
                    \drawqcmeasuresymbol{11}{-1}
                    \draw (12,-1) -- (14,-1) node[midway,above]{\colorlabelsize $\catvariableof{0}$};

                \end{scope}
            \end{tikzpicture}
        \end{center}
        \caption{Ancilla augmentation of an elementary distribution (which variables are pairwise independent).
        a) Hypergraph to the Bayesian Network representing the ancilla augmentation.
        b) Circuit preparing the ancilla augmented distribution by \activationCircuits{} acting on the disentangled initial state.
        }\label{fig:ancillaAugmentationElementary}
    \end{figure}

\end{example}