\begin{example}[Inversion Test]
    Concatenating the \computationCircuits{} to two formulas $\exformula$ and $\secexformula$ prepares the state
    \begin{align*}
        \basqstateofat{\exformula\oplus\secexformula}{\shortcatvariables,\headvariable}
    \end{align*}
    The probability of measuring the ground state after a Walsh-Hadamard transform of the distributed qubits (see \figref{fig:inversionTest}) is then
    \begin{align*}
        \frac{1}{2^{\catorder}} \absof{\contractionof{\basqstateofat{\exformula\oplus\secexformula}{\shortcatvariables,\headvariable}}{\headvariable}}^2
        &= \frac{1}{2^{2\cdot\catorder}}
        \begin{bmatrix}
            \contraction{\exformula\oplus\secexformula}
            \contraction{\lnot(\exformula\oplus\secexformula)}
        \end{bmatrix} \\
        &= \frac{1}{2^{2\cdot\catorder}}
        \begin{bmatrix}
            \#{\left\{\shortcatindices\in\atomstates \wcols \exformulaat{\indexedshortcatvariables}=\secexformula\left[\indexedshortcatvariables\right]\right\}} \\
            \#{\left\{\shortcatindices\in\atomstates \wcols \exformulaat{\indexedshortcatvariables}\neq\secexformula\left[\indexedshortcatvariables\right]\right\}}
        \end{bmatrix} \, .
    \end{align*}
    The second equation holds since $\lnot(\exformula\oplus\secexformula)$ is equivalent with the biconditional $\exformula\Leftrightarrow\secexformula$.

    % Inversion test
    This is in more generality the inversion test of measuring the overlap of $\basqstateof{\exformula}$ with $\basqstateof{\secexformula}$ (see \cite{schuld_machine_2021}).
    Here we used that the \computationCircuit{} is self-adjoint and understand the concatenation as the unitaries preparing $\basqstateof{\exformula}$ and the adjoint (together with the Walsh-Hadamard transform) of $\basqstateof{\secexformula}$.

    \begin{figure}
        \begin{center}

            \begin{tikzpicture}[scale=0.35,thick]

                \draw (0,0) rectangle (2,7.5);
                \node[anchor=center] (text) at (1,3.75) {\corelabelsize $\basqstateof{\exformula\oplus\secexformula}$};

                \draw (2,6.5) -- (4,6.5) node[midway,above] {\colorlabelsize $\headvariable$};
                \drawqcmeasuresymbol{5}{6.5}
                \draw (6,6.5) -- (8,6.5) node[midway,above] {\colorlabelsize $\headvariable$};

                \draw (2,4) -- (4,4) node[midway,above] {\colorlabelsize $\catvariableof{0}$};
                \draw (4,3) rectangle (6,5);
                \node[anchor=center] at (5,4) {$\hgate$};
                \draw (6,4) -- (7,4);
                \drawqcmeasuresymbol{8}{4}
                \draw (9,4) -- (10,4);
                \draw (10,3) rectangle (12,5);
                \node[anchor=center] at (11,4) {$\fbasis$};

                \node[anchor=center] (text) at (3,3.25) {$\vdots$};
                \draw (2,1) -- (4,1) node[midway,above] {\colorlabelsize $\catvariableof{\catorder\shortminus 1}$};
                \draw (4,0) rectangle (6,2);
                \node[anchor=center] at (5,1) {$\hgate$};
                \draw (6,1) -- (7,1);
                \drawqcmeasuresymbol{8}{1}
                \draw (9,1) -- (10,1);
                \draw (10,0) rectangle (12,2);
                \node[anchor=center] at (11,1) {$\fbasis$};

                \node[anchor=center] (text) at (14.5,3.75) {${=}\,\, \frac{1}{2^{2\cdot\catorder}}$};

                \begin{scope}[shift={(18,3.75)}]

                    \draw (3,-0.5) ellipse (4.5cm and 3cm);
                    \node[anchor=center] at (5,3.25) {$\absof{\cdot}^2$};

                    \draw[->-] (3,1) -- (3,3) node[midway,right] {\colorlabelsize $\headvariable$};
                    \draw (0,-1) rectangle (6,1);
                    \drawtextnode{3}{0}{\corelabelsize $\bencodingof{\exformula\oplus\secexformula}$}
                    \draw[->-] (1,-2) -- (1,-1) node[midway,left] {\colorlabelsize $\catvariableof{0}$};
                    \drawvariabledot{1}{-2}
                    \drawtextnode{3}{-1.5}{$\cdots$}
                    \draw[->-] (5,-2) -- (5,-1) node[midway,right] {\colorlabelsize $\catvariableof{\catorder\shortminus1}$};
                    \drawvariabledot{5}{-2}
                \end{scope}

            \end{tikzpicture}
        \end{center}
        \caption{Measurement setup for the inversion test on the formulas $\exformula$ and $\secexformula$.
        We measure the probability of the statistic qubit $\headvariable$, when the distributed qubits $\shortcatvariables$ are in the ground state after a Walsh-Hadamard transform.
        This is equal to a scaled and squared contraction of the basis encoding to the formula $\exformula\oplus\secexformula$.
        }\label{fig:inversionTest}
    \end{figure}
\end{example}
