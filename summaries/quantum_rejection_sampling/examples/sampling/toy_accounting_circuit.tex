\begin{example}[Toy Accounting Example]
    \label{exa:toyAccounting}
    For a more detailed example, let us consider a system of three variables $A1$ Account 1 is booked, $A2$ Account 2 is booked, $F$ a feature on an invoice.
    Assume the following two rules have to be respected:
    \begin{itemize}
        \item \textcolor{\concolor}{Exactly one account must be booked.}
        \item \textcolor{\probcolor}{If feature $\mathrm{F}$ is present on the invoice, the account $\mathrm{A1}$ is typically booked.}
    \end{itemize}
    We formalize this with the statistic
    \begin{align*}
        \hlnstat = (\catvariableof{A1} \oplus \catvariableof{A2}, \catvariableof{F}\Rightarrow \catvariableof{A1})\, .
    \end{align*}
    Any elementary \ComputationActivationNetwork{} with the statistic $\hlnstat$ can be realized by the circuit shown in \figref{fig:toyAccounting}.
    %In particular, the \computationCircuit{} consists of:
    For the preparation of the first statistic qubit $\headvariableof{0}$ to $\sstatcoordinateof{0}=\catvariableof{A1} \oplus \catvariableof{A2}$ by a \computationCircuit{} we exploit that
    \begin{align*}
        \catvariableof{A1}\oplus\catvariableof{A2}
        &= \onehotmapofat{1}{\catvariableof{A1}} \otimes \onehotmapofat{0}{\catvariableof{A2}}
        + \onehotmapofat{0}{\catvariableof{A1}} \otimes \onehotmapofat{1}{\catvariableof{A2}} \\
        &= \left(\onehotmapofat{1}{\catvariableof{A1}} \otimes \onehotmapofat{0}{\catvariableof{A2}}\right)
        \oplus \left(\onehotmapofat{0}{\catvariableof{A1}} \otimes \onehotmapofat{1}{\catvariableof{A2}}\right)
    \end{align*}
    where by $\oplus$ we denote coordinatewise summation mod 2.
    The statistic qubit $\headvariableof{0}$ is therefore prepared by two controlled NOT gates (see \figref{fig:toyAccounting}b).

    For the preparation of the second statistic qubit $\headvariableof{1}$ to $\sstatcoordinateof{1}=\catvariableof{F}\Rightarrow \catvariableof{A1}$ we exploit that the implication is $\truesymbol$ except for the case where the premise $\catvariableof{F}$ is $\truesymbol$ and the head $\catvariableof{A1}$ is $\falsesymbol$.
    In out mod 2 calculus, this amounts to
    \begin{align*}
        \catvariableof{F}\Rightarrow \catvariableof{A1}
        = \onesat{\catvariableof{F},\catvariableof{A1}} \oplus \onehotmapofat{1}{\catvariableof{F}} \otimes \onehotmapofat{0}{\catvariableof{A1}}\, .
    \end{align*}
    It follows that the statistic qubit $\headvariableof{1}$ is prepared by a NOT gate (that is a Pauli-X gate $\paulixsymbol$) and a second controlled NOT gate (see \figref{fig:toyAccounting}b).

    For both cases, the preparation of the statistic qubit can be done by two controlled NOT gates exploiting the \polynomialSparsity{} of the connectives.

    Any non-elementary \ComputationActivationNetwork{} with the statistic $\hlnstat$ can be prepared by the circuit, when choosing a single ancilla qubit, which is uniformly controlled by the qubits $\headvariableof{[2]}$.

    \begin{figure}[t]
    \begin{center}
        \begin{tikzpicture}[scale=0.35, thick] % , baseline = -3.5pt

            \node[anchor=center] at (-16,14) {$a)$};
            \node[anchor=center] at (-4,14) {$b)$};

            \begin{scope}[shift={(-12,2.25)}]

                \node[circle, draw, thick, fill=\nodegrayscale, minimum size = \nodeminsize] (A0) at (-1.5,10) {};
                \node[anchor=center] (A) at (-1.5,10) {\corelabelsize $\avariableof{0}$};
                \node[circle, draw, thick, fill=\nodegrayscale, minimum size = \nodeminsize] (A1) at (1.5,10) {};
                \node[anchor=center] (A) at (1.5,10) {\corelabelsize $\avariableof{1}$};

                \node[circle, draw, thick, fill=\nodegrayscale, minimum size = \nodeminsize] (H0) at (-1.5,5) {};
                \node[anchor=center] (A) at (-1.5,5) {\corelabelsize $\headvariableof{0}$};
                \node[circle, draw, thick, fill=\nodegrayscale, minimum size = \nodeminsize] (H1) at (1.5,5) {};
                \node[anchor=center] (A) at (1.5,5) {\corelabelsize $\headvariableof{1}$};

                \node[circle, draw, thick, fill=\nodegrayscale, minimum size = \nodeminsize] (F) at (3,0) {};
                \node[anchor=center] (A) at (3,0) {\corelabelsize $\catvariableof{F}$};

                \node[circle, draw, thick, fill=\nodegrayscale, minimum size = \nodeminsize] (XA1) at (0,0) {};
                \node[anchor=center] (A) at (0,0) {\corelabelsize $\catvariableof{A1}$};

                \node[circle, draw, thick, fill=\nodegrayscale, minimum size = \nodeminsize] (XA2) at (-3,0) {};
                \node[anchor=center] (A) at (-3,0) {\corelabelsize $\catvariableof{A2}$};

                \draw[->-] (H0) -- (A0);
                \draw[->-] (H1) -- (A1);

                \coordinate (F0) at (-1.5,2.5);
                \draw[->-] (XA2) -- (F0);
                \draw[->-] (XA1) -- (F0);
                \draw[->-] (F0) -- (H0);

                \coordinate (F1) at (1.5,2.5);
                \draw[->-] (F) -- (F1);
                \draw[->-] (XA1) -- (F1);
                \draw[->-] (F1) -- (H1);
            \end{scope}

            %% Ancilla Qubits

            \draw (-1,15) rectangle (1,13);
            %\node[anchor=center] (text) at (-2,14) {\colorlabelsize $\avariableof{1}$};
            \node[anchor=center] (text) at (0,14) {$\onehotmapof{0}$};
            \draw (1,14) -- (13.5,14);
            \draw (13.5,15) rectangle (15.5,13);
            \node[anchor=center] (text) at (14.5,14) {$\qcaencodingof{\hypercoreof{1}}$};
            \draw (15.5,14) -- (18,14);
            \drawqcmeasuresymbol{19}{14}
            \draw (20,14) -- (21.5,14) node[above,midway] {\colorlabelsize $\avariableof{1}$};

            \draw (14.5,8.5) -- (14.5,13);
            \drawvariabledot{14.5}{8.5}

            \draw (-1,10.5) rectangle (1,12.5);
            %\node[anchor=center] (text) at (-2,11.5) {\colorlabelsize $\avariableof{0}$};
            \node[anchor=center] (text) at (0,11.5) {$\onehotmapof{0}$};
            \draw (1,11.5) -- (11.5,11.5);
            \draw (11.5,10.5) rectangle (13.5,12.5);
            \node[anchor=center] (text) at (12.5,11.5) {$\qcaencodingof{\hypercoreof{0}}$};
            \draw (13.5,11.5) -- (18,11.5);
            \drawqcmeasuresymbol{19}{11.5}
            \draw (20,11.5) -- (21.5,11.5) node[above,midway] {\colorlabelsize $\avariableof{0}$};

            \draw (12.5,6) -- (12.5,10.5);
            \drawvariabledot{12.5}{6}

            %% Statistic Qubits
            \draw[dashed] (-3,10) -- (21,10);

            \draw (-1,7.5) rectangle (1,9.5);
            %\node[anchor=center] (text) at (-2,8.5) {\colorlabelsize $\headvariableof{1}$};
            \node[anchor=center] (text) at (0,8.5) {$\onehotmapof{0}$};
            \draw (1,8.5) -- (4,8.5);
            \draw (4,7.5) rectangle (6,9.5);
            \node[anchor=center] (text) at (5,8.5) {$\paulixsymbol$};
            \draw (6,8.5) -- (18,8.5);
            \drawcnotsymbol{11}{8.5}
            \draw (11,8.5) -- (11,-2);
            \drawqcmeasuresymbol{19}{8.5}
            \draw (20,8.5) -- (21.5,8.5) node[above,midway] {\colorlabelsize $\headvariableof{1}$};
            \drawvariabledot{21.5}{8.5}

            \draw (-1,5) rectangle (1,7);
            %\node[anchor=center] (text) at (-2,6) {\colorlabelsize $\headvariableof{0}$};
            \node[anchor=center] (text) at (0,6) {$\onehotmapof{0}$};
            \draw (1,6) -- (18,6);
            \drawcnotsymbol{9}{6}
            \draw (9,6) -- (9,-2);
            \drawqcmeasuresymbol{19}{6}
            \draw (20,6) -- (21.5,6) node[above,midway] {\colorlabelsize $\headvariableof{0}$};
            \drawvariabledot{21.5}{6}

            \drawcnotsymbol{5}{6}
            \draw (5,6) -- (5,-2);

            %% Distributed Qubits
            \draw[dashed] (-3,4.5) -- (21,4.5);

            \draw (-1,2) rectangle (1,4);
            %\node[anchor=center] (text) at (-2,3) {\colorlabelsize $\catvariableof{F}$};
            \node[anchor=center] (text) at (0,3) {$\onehotmapof{0}$};
            \draw (1,3) -- (2,3);
            \draw (2,2) rectangle (4,4);
            \node[anchor=center] (text) at (3,3) {$\hgate$};
            \draw (4,3) -- (18,3);
            \drawqcmeasuresymbol{19}{3}
            \draw (20,3) -- (21.5,3) node[above,midway] {\colorlabelsize $\catvariableof{F}$};

            \draw (-1,-0.5) rectangle (1,1.5);
            %\node[anchor=center] (text) at (-2,0.5) {\colorlabelsize $\catvariableof{A2}$};
            \node[anchor=center] (text) at (0,0.5) {$\onehotmapof{0}$};
            \draw (1,0.5) -- (2,0.5);
            \draw (2,-0.5) rectangle (4,1.5);
            \node[anchor=center] (text) at (3,0.5) {$\hgate$};
            \draw (4,0.5) -- (6,0.5);
            \draw (6,-0.5) rectangle (8,1.5);
            \node[anchor=center] at (7,0.5) {$\paulixsymbol$};

            \draw (8,0.5) -- (18,0.5);
            \drawqcmeasuresymbol{19}{0.5}
            \draw (20,0.5) -- (21.5,0.5) node[above,midway] {\colorlabelsize $\catvariableof{A2}$};

            \draw (-1,-3) rectangle (1,-1);
            %\node[anchor=center] (text) at (-2,-2) {\colorlabelsize $\catvariableof{A1}$};
            \node[anchor=center] (text) at (0,-2) {$\onehotmapof{0}$};
            \draw (1,-2) -- (2,-2);
            \draw (2,-3) rectangle (4,-1);
            \node[anchor=center] at (3,-2) {$\hgate$};
            \draw (4,-2) -- (6,-2);
            \draw (6,-3) rectangle (8,-1);
            \node[anchor=center] at (7,-2) {$\paulixsymbol$};
            \draw (8,-2) -- (12,-2);
            \draw (12,-3) rectangle (14,-1);
            \node[anchor=center] at (13,-2) {$\paulixsymbol$};
            \draw (14,-2) -- (18,-2);
            \drawqcmeasuresymbol{19}{-2}
            \draw (20,-2) -- (21.5,-2) node[above,midway] {\colorlabelsize $\catvariableof{A1}$};

            \drawcontroldot{5}{0.5}
            \drawcontroldot{5}{-2}

            \drawcontroldot{9}{0.5}
            \drawcontroldot{9}{-2}

            \drawcontroldot{11}{3}
            \drawcontroldot{11}{-2}

        \end{tikzpicture}
    \end{center}
    \caption{Circuit to sample from the \ComputationActivationNetwork{} in the toy accounting \exaref{exa:toyAccounting}.
    %Note that a first Pauli-X gate $\paulixsymbol$ acting on $\catvariableof{A2}$ can be omitted, since the prepared stated is uniform.
    }
    \label{fig:toyAccounting}
\end{figure}

\end{example}