\begin{example}[Sum of integers in $\catdim$-adic representation]
    \label{exa:madicRepresentation}
    Let us develop a tensor network representation of integer summations on the set $[\catdim^{\catorder}]=\{0,\ldots,\catdim^{\catorder}-1\}$, where $\catdim,\catorder\in\nn$,
    \begin{align*}
        + \defcols [\catdim^\catorder] \times [\catdim^\catorder] \rightarrow [\catdim^{\catorder+1}]
        \quad,\quad
        +(i,\tilde{i}) = i+\tilde{i} \,
    \end{align*}
    which have a $\catdim$-adic representation of length $\catorder$.
    We define an index interpretation map
    \begin{align*}
        \indexinterpretation \defcols \bigtimes_{\catenumeratorin}[\catdim] \rightarrow [\catdim^{\catorder}]
        \quad,\quad
        \indexinterpretationat{\shortcatindices}
        = \sum_{\catenumeratorin} \catindexof{\catenumerator} \cdot \catdim^{\catenumerator} \, ,
    \end{align*}
    which enables the parameterization of $[\catdim^{\catorder}]$ as the states of $\catorder$ categorical variables $\shortcatvariables$ of dimension $\catdim$.
    We analogously represent a second set $[\catdim^{\catorder}]$ by variables $\tildecatvariableof{[\catorder]}$ and the set $[\catdim^{\catorder+1}]$ of possible sums by $\headvariableof{[\catorder+1]}$.
    The basis encoding of the sum is then
    \begin{align*}
        \bencodingofat{+}{\headvariableof{[\catorder+1]},\shortcatvariables,\seccatvariableof{[\catorder]}}
        = \sum_{\shortcatindices,\tildecatindexof{[\catorder]}} \onehotmapofat{\invindexinterpretationat{
            \indexinterpretationat{\shortcatindices} + \indexinterpretationat{\tildecatindexof{[\catorder]}}
        }}{\headvariableof{[\catorder+1]}}
        \otimes \onehotmapofat{\shortcatindices}{\shortcatvariables}
        \otimes \onehotmapofat{\tildecatindexof{[\catorder]}}{\tildecatvariableof{[\catorder]}} \, .
    \end{align*}
    We notice that the tensor space of $\bencodingof{+}$ is of dimension $\catdim^{3\cdot\catorder+1}$ increasing exponentially in $\catorder$.
    Feasible representation of this tensor for large $\catorder$ therefore requires tensor network decompositions, which we now provide based on a decomposition hypergraph.
    The targeted function to be decomposed is the representation of the integer sum by
    \begin{align*}
        \marysumsymbol \defcols \left(\bigtimes_{\catenumeratorin}[\catdim]\right) \times \left(\bigtimes_{\catenumeratorin}[\catdim]\right) \rightarrow \bigtimes_{\catenumerator\in[\catorder+1]}[\catdim]
        \quad,\quad
        \marysumsymbol(\shortcatindices,\tildecatindexof{[\catorder]}) =
        \invindexinterpretationat{\indexinterpretationat{\shortcatindices}+\indexinterpretationat{\tildecatindexof{[\catorder]}}} \, .
    \end{align*}
    We build a decomposition hypergraph $\graph=(\nodes,\edges)$ (see \defref{def:decompositionHypergraph}) consistent in $4\cdot \catorder$ nodes (see \figref{fig:decompositionMarySum}a) .
    The nodes carry the $(3\cdot\catorder +1)$ variables $\shortcatvariables,\seccatvariableof{[\catorder]},\headvariableof{[\catorder+1]}$ of dimension $\catdim$ constructed above and $\catorder-1$ auxiliary variables $\thirdcatvariableof{[\catorder-1]}$ of dimension $2$ representing carry bits.
    The directed hyperedges of $\graph$ are
    \begin{align*}
        \edges
        =&\left\{(\{\catvariableof{0},\tildecatvariableof{0}\},\{\headvariableof{0},\thirdcatvariableof{0}\})\right\}
        \cup \left\{(\{\thirdcatvariableof{\catenumerator-1},\catvariableof{\catenumerator},\tildecatvariableof{\catenumerator}\},\{\headvariableof{\catenumerator},\thirdcatvariableof{\catenumerator}\})
                 \wcols \catenumerator\in\{1,\ldots,\catorder-2\}\right\} \\
        &\cup \left\{(\{\thirdcatvariableof{\catorder-2},\catvariableof{\catorder-1},\tildecatvariableof{\catorder-1}\},\{\headvariableof{\catorder-1},\headvariableof{\catorder}\})\right\}
    \end{align*}
    and are decorated by local summation functions
    \begin{align*}
        \modsumsymbol \defcols [2] \times [\catdim] \times [\catdim] \rightarrow [\catdim] \times [2]
        \quad,\quad
        \modsumsymbol(\thirdcatindex,\catindex,\tildecatindex)
        = \left((\thirdcatindex + \catindex + \tildecatindex) \modspace\catdim,
              \left\lfloor\frac{\thirdcatindex + \catindex + \tildecatindex}{\catdim}\right\rfloor\right) \, .
    \end{align*}
    Since to the first hyperedge we do not have a carry bit, the decorating function is the restriction of the first argument to $0$.

    It is known that the composition of the local summations $\modsumsymbol$ is the global summation $\marysumsymbol$ of integers in $\catdim$-adic representation.
    Thus, the composition function $\exfunctionof{\graph}$ is $\marysumsymbol$.
    By \theref{the:functionDecompositionRep} we have a decomposition of the basis encoding to $\exfunctionof{\graph}$ (see \figref{fig:decompositionMarySum}b) as
    \begin{align*}
        \bencodingofat{\marysumsymbol}{\headvariableof{[\catorder+1]},\shortcatvariables,\tildecatvariableof{[\catorder]}}
        = \breakablecontractionof{
            &\{\bencodingofat{\modsumsymbol,0}{\headvariableof{0},\thirdcatvariableof{0},\catvariableof{0},\tildecatvariableof{0}}\} \cup \\
            &\{\bencodingofat{\modsumsymbol,\catenumerator}{
            \headvariableof{\catenumerator},\thirdcatvariableof{\catenumerator},\catvariableof{\catenumerator},\tildecatvariableof{\catenumerator},\thirdcatvariableof{\catenumerator-1}
            }\wcols \catenumerator\in\{1,\ldots,\catorder-2\}\} \cup \\
            &\{\bencodingofat{\modsumsymbol,\catorder-2}{\headvariableof{\catorder-1},\headvariableof{\catorder},\catvariableof{\catorder-1},\tildecatvariableof{\catorder-1},\thirdcatvariableof{\catorder-2}}\}
        }{\headvariableof{[\catorder+1]},\shortcatvariables,\tildecatvariableof{[\catorder]}} \, .
    \end{align*}

    \begin{figure}[t]
        \begin{center}
            \begin{tikzpicture}[scale=0.35, thick]

                \begin{scope}[shift={(-0.5,10)}]

                    \node[anchor=east]  at (-9,4) {$a)$};

                    \node (of) at (0,1.35) {};

                    \draw[thick, dashed, rounded corners=10pt]  ($(-5,3)+(of)$) -- ($(-5,3)-(of)$)  -- ($(28.5,3)-(of)$) -- ($(28.5,3)+(of)$) -- cycle;
                    \node[anchor=center] (A) at (-3,3) {\corelabelsize $\node^{\outsymbol}$};

                    \draw[thick, dashed, rounded corners=10pt]  ($(-5,-3)+(of)$) -- ($(-5,-3)-(of)$)  -- ($(28.5,-3)-(of)$) -- ($(28.5,-3)+(of)$) -- cycle;
                    \node[anchor=center] (A) at (-3,-3) {\corelabelsize $\node^{\insymbol}$};

                    \node[circle, draw, thick, fill=\nodegrayscale, minimum size = \nodeminsize] (Y0) at (1.5,3) {};
                    \node[] (text) at (1.5,3) {\corelabelsize $\headvariableof{0}$};
                    \node[circle, draw, thick, fill=\nodegrayscale, minimum size = \nodeminsize] (X00) at (0,-3) {};
                    \node[] (text) at (0,-3) {\corelabelsize $\catvariableof{0}$};
                    \node[circle, draw, thick, fill=\nodegrayscale, minimum size = \nodeminsize] (X10) at (3,-3) {};
                    \node[] (text) at (3,-3) {\corelabelsize $\tildecatvariableof{0}$};
                    \node[circle, draw, thick, fill=\nodegrayscale, minimum size = \nodeminsize] (Z0) at (5,0) {};
                    \node[] (text) at (5,0) {\corelabelsize $\thirdcatvariableof{0}$};

                    \coordinate (m0) at (1.5,0);
                    \node[anchor=east]  at (1.5,0) {$\edgeof{0}$};
                    \draw[->-] (X00) -- (m0);
                    \draw[->-] (X10) -- (m0);
                    \draw[->-] (m0) -- (Y0);
                    \draw[->-] (m0) -- (Z0);

                    \node[circle, draw, thick, fill=\nodegrayscale, minimum size = \nodeminsize] (Y1) at (8.5,3) {};
                    \node[] (text) at (8.5,3) {\corelabelsize $\headvariableof{1}$};
                    \node[circle, draw, thick, fill=\nodegrayscale, minimum size = \nodeminsize] (X01) at (7,-3) {};
                    \node[] (text) at (7,-3) {\corelabelsize $\catvariableof{1}$};
                    \node[circle, draw, thick, fill=\nodegrayscale, minimum size = \nodeminsize] (X11) at (10,-3) {};
                    \node[] (text) at (10,-3) {\corelabelsize $\tildecatvariableof{1}$};
                    \node[circle, draw, thick, fill=\nodegrayscale, minimum size = \nodeminsize] (Z1) at (12,0) {};
                    \node[] (text) at (12,0) {\corelabelsize $\thirdcatvariableof{1}$};

                    \coordinate (m1) at (8.5,0);
                    \node[anchor=east]  at (8.5,0.75) {$\edgeof{1}$};
                    \draw[->-] (Z0) -- (m1);
                    \draw[->-] (X01) -- (m1);
                    \draw[->-] (X11) -- (m1);
                    \draw[->-] (m1) -- (Y1);
                    \draw[->-] (m1) -- (Z1);

                    \draw[->-] (Z1) -- (15,0) node[anchor=west]{$\cdots$};

                    \begin{scope}[shift={(15,0)}]
                        \node[circle, draw, thick, fill=\nodegrayscale, minimum size = \nodeminsize] (Z0) at (5,0) {};
                        \node[] (text) at (5,0) {\corelabelsize $\thirdcatvariableof{\catorder\shortminus2}$};

                        \node[circle, draw, thick, fill=\nodegrayscale, minimum size = \nodeminsize] (Y1) at (8.5,3) {};
                        \node[] (text) at (8.5,3) {\corelabelsize $\headvariableof{\catorder\shortminus 1}$};
                        \node[circle, draw, thick, fill=\nodegrayscale, minimum size = \nodeminsize] (X01) at (7,-3) {};
                        \node[] (text) at (7,-3) {\corelabelsize $\catvariableof{\catorder\shortminus1}$};
                        \node[circle, draw, thick, fill=\nodegrayscale, minimum size = \nodeminsize] (X11) at (10,-3) {};
                        \node[] (text) at (10,-3) {\corelabelsize $\tildecatvariableof{\catorder\shortminus1}$};
                        \node[circle, draw, thick, fill=\nodegrayscale, minimum size = \nodeminsize] (Z1) at (12,3) {};
                        \node[] (text) at (12,3) {\corelabelsize $\headvariableof{\catorder}$};

                        \coordinate (me) at (8.5,0);
                        \node[anchor=east]  at (8.5,1) {$\edgeof{\catorder\shortminus 1}$};
                        \draw[->-] (2,0)  -- (Z0);
                        \draw[->-] (Z0) -- (me);
                        \draw[->-] (X01) -- (me);
                        \draw[->-] (X11) -- (me);
                        \draw[->-] (me) -- (Y1);
                        \draw[->-] (me) -- (Z1);
                    \end{scope}

                \end{scope}

                \node[anchor=east]  at (-9,4) {$b)$};

                \begin{scope}[shift={(-8,0)}]
                    \draw (-2,-1) rectangle (4,1);
                    \draw[->-] (-1.25,1)--(-1.25,2.5) node[midway,left] {\colorlabelsize $\headvariableof{0}$};
                    \draw[->-] (3.25,1)--(3.25,2.5) node[midway,right] {\colorlabelsize $\headvariableof{\catorder\shortminus1}$};
                    \node[anchor=center] (A) at (1,2.5) {\corelabelsize $\cdots$};
                    \node[anchor=center] (A) at (1,0) {\corelabelsize $\bencodingof{\exfunctionof{\graph}}$};
                    \draw[-<-] (-1.5,-1)--(-1.5,-2.5) node[midway,left] {\colorlabelsize $\catvariableof{0}$};
                    \draw[-<-] (-1,-1)--(-1,-2.5) node[midway,right] {\colorlabelsize $\tildecatvariableof{0}$};
                    \draw[-<-] (3,-1)--(3,-2.5) node[midway,left] {\colorlabelsize $\catvariableof{\catorder\shortminus1}$};
                    \draw[-<-] (3.5,-1)--(3.5,-2.5) node[midway,right] {\colorlabelsize $\tildecatvariableof{\catorder\shortminus1}$};
                    \node[anchor=center] (A) at (1,-2.5) {\corelabelsize $\cdots$};
                \end{scope}

                \node[anchor=east]  at (-1.5,0) {$=$};

                \draw (-1,-1) rectangle (3,1);
                \node[anchor=center] (A) at (1,0) {\corelabelsize $\bencodingof{\modsumsymbol,0}$};
                \draw[->-] (1,1)--(1,3) node[midway,left] {\colorlabelsize $\headvariableof{0}$};
                \draw[-<-] (0,-1)--(0,-2.5) node[midway,left] {\colorlabelsize $\catvariableof{0}$};
                \draw[-<-] (2,-1)--(2,-2.5) node[midway,right] {\colorlabelsize $\tildecatvariableof{0}$};
                \draw[->-] (3,0)--(6,0) node[midway,above] {\colorlabelsize $\thirdcatvariableof{0}$};

                \begin{scope}[shift={(7,0)}]
                    \draw (-1,-1) rectangle (3,1);
                    \node[anchor=center] (A) at (1,0) {\corelabelsize $\bencodingof{\modsumsymbol,1}$};
                    \draw[->-] (1,1)--(1,3) node[midway,left] {\colorlabelsize $\headvariableof{1}$};
                    \draw[-<-] (0,-1)--(0,-2.5) node[midway,left] {\colorlabelsize $\catvariableof{1}$};
                    \draw[-<-] (2,-1)--(2,-2.5) node[midway,right] {\colorlabelsize $\tildecatvariableof{1}$};
                    \draw[->-] (3,0)--(6,0) node[midway,above] {\colorlabelsize $\thirdcatvariableof{1}$};
                \end{scope}

                \node[anchor=center] at (15.5,0) {$\cdots$};

                \begin{scope}[shift={(22,0)}]
                    \draw[->-] (-4,0)--(-1,0) node[midway,above] {\colorlabelsize $\thirdcatvariableof{\catorder\shortminus 2}$};
                    \draw (-1,-1) rectangle (3,1);
                    \node[anchor=center] (A) at (1,0) {\corelabelsize $\bencodingof{\modsumsymbol,\catorder-1}$};
                    \draw[->-] (1,1)--(1,3) node[midway,left] {\colorlabelsize $\headvariableof{\catorder\shortminus1}$};
                    \draw[-<-] (0,-1)--(0,-2.5) node[midway,left] {\colorlabelsize $\catvariableof{\catorder\shortminus1}$};
                    \draw[-<-] (2,-1)--(2,-2.5) node[midway,right] {\colorlabelsize $\tildecatvariableof{\catorder\shortminus1}$};
                    \draw (3,0)--(4.5,0) -- (4.5,1);
                    \draw[->-] (4.5,1)--(4.5,3) node[midway,left] {\colorlabelsize $\headvariableof{\catorder}$};
                \end{scope}


            \end{tikzpicture}
        \end{center}
        \caption{Example of a decomposition hypergraph to the sum of integers (see \exaref{exa:madicRepresentation}).
        a) Hypergraph of directed edges $\edgeof{\catenumerator}$ for $\catenumeratorin$, each decorated by an integer summation $+$ preparing an index $\headvariableof{\catenumerator}$ of the resulting sum.
        b) Corresponding tensor network decomposition of the basis encoded composition function, which is the sum of integers in $\catdim$-adic representation.}
        \label{fig:decompositionMarySum}
    \end{figure}


%
%
%    We parametrize numbers by bits in fixed point representations, which are understood as categorical variables in a factored system representation.
%
%
%%\section{Modular Calculus}
%
%    We have two basic functions calculating the mod
%    \begin{align*}
%        \exfunction : \facstates \rightarrow [2]
%        \andspace \exfunctionat{\shortcatindices} = \sum_{\atomenumeratorin} \catindexof{\atomenumerator} \,\, \mathrm{mod} \,\, 2
%    \end{align*}
%    and the integer division by two
%    \begin{align*}
%        \secexfunction : \facstates \rightarrow [2]
%        \andspace \secexfunctionat{\shortcatindices} = \left\lfloor \frac{\sum_{\atomenumeratorin} \catindexof{\atomenumerator}}{2}\right\rfloor
%    \end{align*}
%
%
%%\section{Sums}
%
%    Given the bit representations of summands, we want to calculate the bit representation of their sum.
%
%%\subsection{Binary Addition}
%
%    Basis calculus of binary additon is a TT architecture, where each core performs the addition of two bits and a carry bit, producing a sum bit and a carry bit.
%
%    Addition of two numbers with $d$ bits:
%    \begin{itemize}
%        \item Bit variables of the first number: $X_{[d]}$
%        \item Bit variables of the second number: $Z_{[d]}$
%        \item Output bit variables: $Y_{[d+1]}$
%        \item Carry bit variables: $C_{[d]}$, with $C_{1} = 0$
%    \end{itemize}
%
%    The sum of any two numbers is represented by the boolean tensor
%    \begin{align*}
%        &\hypercoreat{X_{[d]},Z_{[d]},Y_{[d+1]}}
%        \coloneqq \\
%        & \quad \contractionof{\{\onehotmapofat{0}{C_0},\identityat{C_{d-1},Y_{d}}\} \cup
%        \bigcup_{k\in[d]}\{
%            \bencodingofat{\exfunction}{Y_{k},X_{k},Z_{k},C_{k-1}},
%            \bencodingofat{\secexfunction}{C_{k},X_{k},Z_{k},C_{k-1}} \}
%        }{X_{[d]},Z_{[d]},Y_{[d+1]}} \, ,
%    \end{align*}
%    where $Y_k$ and $C_k$ are the head variables of the basis encodings to $\exfunction$ and $\secexfunction$.
%    If any only if for given indices $x_{[d]},z_{[d]},y_{[d+1]}$ we have $\hypercoreat{X_{[d]}=x_{[d]},Z_{[d]}=z_{[d]},Y_{[d+1]}=y_{[d+1]}}=1$, then the by the indices $y_{[d+1]}$ represented number is the sum of the by $x_{[d]},z_{[d]}$ represented numbers.
%
%%\subsection{Generic construction}
%
%    In general, when adding more than two variables, the carry bits need to be extended to a categorical variable with more than two states.
%    Let $X^i_{[d]}$ be the $d$ bits of the $i$th number, and let $X^{[n]}_{[d]}$ be all the bit variables (i.e. $n\cdot d$ many) of the $n$ numbers.
%    Then the same construction can be done as above, with cores
%    \begin{align*}
%        \bencodingofat{\exfunction}{Y_{k},X^{[n]}_{k},C_{k-1}},
%        \bencodingofat{\secexfunction}{C_{k},X^{[n]}_{k},Z_{k},C_{k-1}}
%    \end{align*}
%    Note that $C_k$ now takes values in $m_k$ where
%    \begin{align*}
%        m_k = \left\lfloor n \cdot \frac{m_{k-1}}{2}\right\rfloor \, .
%    \end{align*}
%    Further, the result might have more than $d+1$ bits, so we need further basis encoding cores to $\exfunction$ and $\secexfunction$.
%
%
%%\section{Products}
%
%    Products of numbers are decomposable into sums involving two bit variables of the factors, that is
%    \begin{align*}
%        \sum_{k,\tilde{k} \in [d]} 2^{k+\tilde{k}} \cdot (X_k \land Z_{\tilde{k}}) \, .
%    \end{align*}
%    Reordering the sum, we obtain
%    \begin{align*}
%        \sum_{r\in[2d-1]} 2^{r} \left(\sum_{k,\tilde{k} \in [d] \wcols k +\tilde{k}=r}  X_k \land Z_{\tilde{k}} \right) \, .
%    \end{align*}
%    From this, it is obvious that the calculation can be performed in basis calculus with basis encodings of $\land,\exfunction,\secexfunction$.
%    The head variables of the $\land$ encoding are used as the summand variabled in $\exfunction$ (output: bit of the product) and $\secexfunction$ (output: carry bit).
%
%
%%\section{Application}
%
%    Any of these tensor network schemes are considered batch schemes to perform arithmetic operations.
%    Contractions of the representing basis encodings calculate the number of true input-output relations, given e.g. a restriction onto specific outputs and inputs (by adding subset encodings of the numbers of interest).
%    One application is the countdown game, when in addition parametrizing the sum/negation operations with an additional selection variable.

%\end{document}
\end{example}