\begin{example}[Continuation of \exaref{exa:madicRepresentation}]%[Computation of the sum of $\catdim$-adic integers]
    \label{exa:madicPropagation}
    We now show how \algoref{alg:directedBeliefPropagation} can be exploited to compute an efficient message passing algorithm for the digits of the $\catdim$-adic sum.
    Given two numbers in $\catdim$-adic representation by the tuples $\shortcatindices$ and $\tildecatindexof{[\catorder]}$, we add the hyperedges with empty incoming nodes and single outgoing node
    \begin{align*}
        \bigcup_{\catenumeratorin}\left\{(\varnothing,\{\catvariableof{\catenumerator}\}),(\varnothing,\{\tildecatvariableof{\catenumerator}\})\right\}
    \end{align*}
    to the hypergraph of \exaref{exa:madicRepresentation} and decorate them by the digit one-hot encodings $\onehotmapofat{\catindexof{\catenumerator}}{\catvariableof{\catenumerator}}$ and $\onehotmapofat{\tildecatindexof{\catenumerator}}{\tildecatvariableof{\catenumerator}}$ (see \figref{fig:propagationMary}).
    We then apply the Directed Belief Propagation \algoref{alg:directedBeliefPropagation}.
    The initial messages queue then consists of the messages from the digit encoding.
    As sketched in \figref{fig:propagationMary}, to each digit there are three messages (with the exception of the first being two), which can be scheduled in consecutive epochs $\messagesymbol^{(\catenumerator,[3])}$.
    %, namely
%    \begin{align*}
%        \scheduler =&
%        \left\{
%            \big((\varnothing,\{\catvariableof{0}\}),(\{\catvariableof{0},\tildecatvariableof{0}\},\{\headvariableof{0},\thirdcatvariableof{0}\})\big),
%            \big((\varnothing,\{\tildecatvariableof{0}\}),(\{\catvariableof{0},\tildecatvariableof{0}\},\{\headvariableof{0},\thirdcatvariableof{0}\})\big)
%        \right\} \cup \\
%&        \left(\bigcup_{\catenumerator\in\{1,\ldots,\catorder-2\}}\left\{
%                                                               \big((\varnothing,\{\catvariableof{\catenumerator}\}),(\{\catvariableof{\catenumerator},\tildecatvariableof{\catenumerator},\thirdcatvariableof{\catenumerator-1}\},\{\headvariableof{\catenumerator},\thirdcatvariableof{\catenumerator}\})\big),
%                                                               \big((\varnothing,\{\tildecatvariableof{\catenumerator}\}),(\{\catvariableof{\catenumerator},\tildecatvariableof{\catenumerator},\thirdcatvariableof{\catenumerator-1}\},\{\headvariableof{\catenumerator},\thirdcatvariableof{\catenumerator}\})\big)
%        \right\}\right) \\
%        & \cup  \Big\{
%            \big((\varnothing,\{\catvariableof{\catorder-1}\}),(\{\catvariableof{\catorder-1},\tildecatvariableof{\catorder-1},\thirdcatvariableof{\catenumerator-2}\},\{\headvariableof{\catorder-1},\thirdcatvariableof{\catorder-1}\})\big), \\
%        & \big((\varnothing,\{\tildecatvariableof{\catorder-1}\}),(\{\catvariableof{\catorder-1},\tildecatvariableof{\catorder-1},\thirdcatvariableof{\catenumerator-2}\},\{\headvariableof{\catorder-1},\thirdcatvariableof{\catorder-1}\})\big)
%        \Big\}
%    \end{align*}
    We then have by \theref{the:directedBeliefPropagationExactness} for $\catenumerator\in[\catorder-1]$ that
    \begin{align*}
        \contractionof{
            \bencodingofat{\modsumsymbol,\catenumerator}{\headvariableof{\catenumerator},\thirdcatvariableof{\catenumerator},\catvariableof{\catenumerator},\tildecatvariableof{\catenumerator},\thirdcatvariableof{\catenumerator-1}},
            \messagesymbol^{(\catenumerator-1,2)}[\thirdcatvariableof{\catenumerator-1}],
            \messagesymbol^{(\catenumerator,0)}[\catvariableof{\catenumerator}],
            \messagesymbol^{(\catenumerator,1)}[\tildecatvariableof{\catenumerator}]
        }{\thirdcatvariableof{\catenumerator}}
        =
        \onehotmapofat{\thirdcatindexof{\catenumerator}}{\thirdcatvariableof{\catenumerator}}\,,
    \end{align*}
    where $\thirdcatindexof{\catenumerator}$ is the value of the $\catenumerator$-th carry bit.
    The $\catenumerator$-th digit of the sum $\headindexof{\catenumerator}$ can further be read of by the contraction
    \begin{align*}
        \contractionof{
            \bencodingofat{\modsumsymbol,\catenumerator}{\headvariableof{\catenumerator},\thirdcatvariableof{\catenumerator},\catvariableof{\catenumerator},\tildecatvariableof{\catenumerator},\thirdcatvariableof{\catenumerator-1}},
            \messagesymbol^{(\catenumerator-1,2)}[\thirdcatvariableof{\catenumerator-1}],
            \messagesymbol^{(\catenumerator,0)}[\catvariableof{\catenumerator}],
            \messagesymbol^{(\catenumerator,1)}[\tildecatvariableof{\catenumerator}]
        }{\headvariableof{\catenumerator}}
        =
        \onehotmapofat{\headindexof{\catenumerator}}{\headvariableof{\catenumerator}} \, .
    \end{align*}
    % Tree belief also working
    Note that the hypergraph representing this instance is a tree and by \theref{the:treeBeliefPropagationExactness} also the message passing scheme of \algoref{alg:treeBeliefPropagation} is guaranteed to produce the exact contractions.
    We can exploit this fact for example in the efficient computation of averages of the summation digits, when we have an elementary distribution of input digits.
    We emphasize that the directed belief propagation \algoref{alg:treeBeliefPropagation} is exact even if the hypergraph fails to be a tree, provided that we have directed and boolean tensors..

    \begin{figure}[t]
        \begin{center}
            \begin{tikzpicture}[scale=0.5, thick]

                \draw (-1,-1) rectangle (3,1);
                \node[anchor=center] (A) at (1,0) {\corelabelsize $\bencodingof{\modsumsymbol,0}$};
                \draw[->-] (1,1)--(1,3) node[midway,left] {\colorlabelsize $\headvariableof{0}$};
                \draw[-<-] (0,-1)--(0,-2.5) node[midway,left] {\colorlabelsize $\catvariableof{0}$};
                \draw (-0.75,-2.5) rectangle (0.75,-4);
                \node[anchor=center] (A) at (0,-3.25) {\corelabelsize $\onehotmapof{\catindexof{0}}$};
                \draw[\newmessagecolor,dashed, ->] (-1.25,-3.25) to [bend right = -30] (-1.25,-1);
                \node[\newmessagecolor,anchor=center] (A) at (-1.75,-3.6) {\colorlabelsize $\messagesymbol^{(0,0)}$};
                \draw[-<-] (2,-1)--(2,-2.5) node[midway,right] {\colorlabelsize $\tildecatvariableof{0}$};
                \draw (1.25,-2.5) rectangle (2.75,-4);
                \node[anchor=center] (A) at (2,-3.25) {\corelabelsize $\onehotmapof{\tildecatindexof{0}}$};
                \draw[\newmessagecolor,dashed, ->] (3.25,-3.25) to [bend right = 30] (3.25,-1);
                \node[\newmessagecolor,anchor=center] (A) at (3.8,-3.6) {\colorlabelsize $\messagesymbol^{(0,1)}$};
                \draw[->-] (3,0)--(6,0) node[midway,above] {\colorlabelsize $\thirdcatvariableof{0}$};
                \draw[\newmessagecolor,dashed, ->] (3,1.25) to [bend right = -30] (6,1.25);
                \node[\newmessagecolor,anchor=center] (A) at (4.5,2.25) {\colorlabelsize $\messagesymbol^{(0,2)}$};

                \begin{scope}[shift={(7,0)}]
                    \draw (-1,-1) rectangle (3,1);
                    \node[anchor=center] (A) at (1,0) {\corelabelsize $\bencodingof{\modsumsymbol,1}$};
                    \draw[->-] (1,1)--(1,3) node[midway,left] {\colorlabelsize $\headvariableof{1}$};
                    \draw[-<-] (0,-1)--(0,-2.5) node[midway,left] {\colorlabelsize $\catvariableof{1}$};
                    \draw (-0.75,-2.5) rectangle (0.75,-4);
                    \node[anchor=center] (A) at (0,-3.25) {\corelabelsize $\onehotmapof{\catindexof{1}}$};
                    \draw[\newmessagecolor,dashed, ->] (-1.25,-3.25) to [bend right = -30] (-1.25,-1);
                    \node[\newmessagecolor,anchor=center] (A) at (-1.75,-3.6) {\colorlabelsize $\messagesymbol^{(1,0)}$};
                    \draw[-<-] (2,-1)--(2,-2.5) node[midway,right] {\colorlabelsize $\tildecatvariableof{1}$};
                    \draw (1.25,-2.5) rectangle (2.75,-4);
                    \node[anchor=center] (A) at (2,-3.25) {\corelabelsize $\onehotmapof{\tildecatindexof{1}}$};
                    \draw[\newmessagecolor,dashed, ->] (3.25,-3.25) to [bend right = 30] (3.25,-1);
                    \node[\newmessagecolor,anchor=center] (A) at (3.8,-3.6) {\colorlabelsize $\messagesymbol^{(1,1)}$};
                    \draw[->-] (3,0)--(6,0) node[midway,above] {\colorlabelsize $\thirdcatvariableof{1}$};
                    \draw[\newmessagecolor,dashed, ->] (3,1.25) to [bend right = -30] (6,1.25);
                    \node[\newmessagecolor,anchor=center] (A) at (4.5,2.25) {\colorlabelsize $\messagesymbol^{(1,2)}$};
                \end{scope}

                \node[anchor=center] at (15.5,0) {$\cdots$};

                \begin{scope}[shift={(22,0)}]
                    \draw[\newmessagecolor,dashed, ->] (-4,1.25) to [bend right = -30] (-1,1.25);
                    \node[\newmessagecolor,anchor=center] (A) at (-2.5,2.25) {\colorlabelsize $\messagesymbol^{(\catorder\shortminus2,2)}$};

                    \draw[->-] (-4,0)--(-1,0) node[midway,above] {\colorlabelsize $\thirdcatvariableof{\catorder\shortminus 2}$};
                    \draw (-1,-1) rectangle (3,1);
                    \node[anchor=center] (A) at (1,0) {\corelabelsize $\bencodingof{\modsumsymbol,\catorder\shortminus1}$};
                    \draw[->-] (1,1)--(1,3) node[midway,left] {\colorlabelsize $\headvariableof{\catorder\shortminus1}$};
                    \draw[-<-] (0,-1)--(0,-2.5) node[midway,left] {\colorlabelsize $\catvariableof{\catorder\shortminus1}$};
                    \draw (-0.75,-2.5) rectangle (0.75,-4);
                    \node[anchor=center] (A) at (0,-3.25) {\corelabelsize $\onehotmapof{\catindexof{\catorder\shortminus1}}$};
                    \draw[\newmessagecolor,dashed, ->] (-1.5,-3.25) to [bend right = -45] (-1.5,-0.75);
                    \node[\newmessagecolor,anchor=center] (A) at (-2.3,-3.8) {\colorlabelsize $\messagesymbol^{(\catorder\shortminus1,0)}$};
                    \draw[-<-] (2,-1)--(2,-2.5) node[midway,right] {\colorlabelsize $\tildecatvariableof{\catorder\shortminus1}$};
                    \draw (1.25,-2.5) rectangle (2.75,-4);
                    \draw[\newmessagecolor,dashed, ->] (3.5,-3.25) to [bend right = 45] (3.5,-0.75);
                    \node[\newmessagecolor,anchor=center] (A) at (4.3,-3.8) {\colorlabelsize $\messagesymbol^{(\catorder\shortminus1,1)}$};
                    \node[anchor=center] (A) at (2,-3.25) {\corelabelsize $\onehotmapof{\tildecatindexof{\catorder\shortminus1}}$};
                    \draw (3,0)--(4.5,0) -- (4.5,1);
                    \draw[->-] (4.5,1)--(4.5,3) node[midway,left] {\colorlabelsize $\headvariableof{\catorder}$};
                \end{scope}


            \end{tikzpicture}
        \end{center}
        \caption{Computation of the integer sum in $\catdim$-adic representation by the Directed Belief Propagation \algoref{alg:directedBeliefPropagation} (see \exaref{exa:madicPropagation}).
        The summands are represented by one-hot encodings of the digits $\shortcatindices$ and $\tildecatindexof{[\catorder]}$, from which the messages start.
        The $\catenumerator$-th digit (for $\catenumerator\in\{0,\ldots,\catorder-1\}$) of the sum is computed based on the first messages of the epoch labeled by $\messagesymbol^{(\catenumerator,[2])}$,
            The third message $\messagesymbol^{(\catenumerator,2)}$ in each epoch communicates the carry bit to the next digit summation core.
            In the last message epoch the digit $\catorder-1$ and $\catorder$ are computed based.
        }
        \label{fig:propagationMary}
    \end{figure}

\end{example}