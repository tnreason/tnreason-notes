\begin{example}
    We consider a classical example (see \cite[Example~4.3]{koller_probabilistic_2009}):
    A student of intelligence ($\catvariableof{I}$) and SAT score ($\catvariableof{S}$), is assigned a test of difficulty ($\catvariableof{D}$), for which he gets a grade ($\catvariableof{G}$) depending on which he gets a recommendation letter ($\catvariableof{L}$) by its teacher.
    We make the following modelling assumptions:
    \begin{itemize}
        \item "The SAT score depends only on the students intelligence": $\condindependent{\catvariableof{S}}{\catvariableof{\{D,G,L\}}}{\catvariableof{I}}$
        \item "The recommendation letter depends only on the grade": $\condindependent{\catvariableof{L}}{\catvariableof{\{D,I,S\}}}{\catvariableof{G}}$
    \end{itemize}

    These independences are modelled in the hypergraph a) and the clique capturing property is sketched in b):
    \begin{center}
        \begin{tikzpicture}

            \node[anchor=center] (A) at (-2,2.2) {$a)$};

            \node[anchor=center] (A) at (1,0.4) {$\edgeof{0}$};
            \draw[thick] (0,0) -- (1,2/3);
            \draw[thick] (2,0) -- (1,2/3);
            \draw[thick] (1,2) -- (1,2/3);

            \draw[thick] (2,0) -- (3,2) node[midway,right] {$\edgeof{1}$};
            \draw[thick] (1,2) -- (-1,2) node[midway,above] {$\edgeof{2}$};

            \node[circle, draw, thick, fill=\nodegrayscale, minimum size = \nodeminsize] (A) at (0,0) {};
            \node[anchor=center] (A) at (0,0) {\corelabelsize $\catvariableof{D}$};

            \node[circle, draw, thick, fill=\nodegrayscale, minimum size = \nodeminsize] (A) at (2,0) {};
            \node[anchor=center] (A) at (2,0) {\corelabelsize $\catvariableof{I}$};

            \node[circle, draw, thick, fill=\nodegrayscale, minimum size = \nodeminsize] (A) at (1,2) {};
            \node[anchor=center] (A) at (1,2) {\corelabelsize $\catvariableof{G}$};

            \node[circle, draw, thick, fill=\nodegrayscale, minimum size = \nodeminsize] (A) at (3,2) {};
            \node[anchor=center] (A) at (3,2) {\corelabelsize $\catvariableof{S}$};

            \node[circle, draw, thick, fill=\nodegrayscale, minimum size = \nodeminsize] (A) at (-1,2) {};
            \node[anchor=center] (A) at (-1,2) {\corelabelsize $\catvariableof{L}$};

            \begin{scope}[shift={(7,0)}]
                \node[anchor=center] (A) at (-2,2.2) {$b)$};

                %\node[anchor=center] (A) at (1,0.4); % {$\edgeof{0}$};
                \draw[thick] (0,0) -- (2,0);
                \draw[thick] (0,0) -- (1,2);
                \draw[thick] (2,0) -- (1,2);

                \draw[thick, dashed, rounded corners=15pt] (1,3) -- (-0.8,-0.5) -- (2.8,-0.5) -- cycle;

                \node (of) at (0.5,-0.25) {};
                \draw[thick, dashed, rounded corners=10pt]  ($(3.25,2.5)+(of)$) -- ($(3.25,2.5)-(of)$)  -- ($(1.75,-0.5)-(of)$) -- ($(1.75,-0.5)+(of)$) -- cycle;
                \draw[thick] (2,0) -- (3,2); % node[midway,right] {$\edgeof{1}$};

                \node (of) at (0,0.5) {};
                \draw[thick, dashed, rounded corners=10pt]  ($(1.5,2)+(of)$) -- ($(1.5,2)-(of)$)  -- ($(-1.5,2)-(of)$) -- ($(-1.5,2)+(of)$) -- cycle;
                \draw[thick] (1,2) -- (-1,2); % node[midway,above] {$\edgeof{2}$};

                \node[circle, draw, thick, fill=\nodegrayscale, minimum size = \nodeminsize] (A) at (0,0) {};
                \node[anchor=center] (D) at (0,0) {\corelabelsize $\catvariableof{D}$};

                \node[circle, draw, thick, fill=\nodegrayscale, minimum size = \nodeminsize] (A) at (2,0) {};
                \node[anchor=center] (I) at (2,0) {\corelabelsize $\catvariableof{I}$};

                \node[circle, draw, thick, fill=\nodegrayscale, minimum size = \nodeminsize] (A) at (1,2) {};
                \node[anchor=center] (G) at (1,2) {\corelabelsize $\catvariableof{G}$};

                \node[circle, draw, thick, fill=\nodegrayscale, minimum size = \nodeminsize] (A) at (3,2) {};
                \node[anchor=center] (S) at (3,2) {\corelabelsize $\catvariableof{S}$};

                \node[circle, draw, thick, fill=\nodegrayscale, minimum size = \nodeminsize] (A) at (-1,2) {};
                \node[anchor=center] (L) at (-1,2) {\corelabelsize $\catvariableof{L}$};

            \end{scope}
        \end{tikzpicture}
    \end{center}
\end{example}