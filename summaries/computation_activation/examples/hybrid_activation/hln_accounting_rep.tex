\begin{example}[\HybridLogicNetwork{} for a Toy Accounting Model]
    \label{exa:hlnAccountingRep}

    Let us consider a system of three variables $A1$ Account 1 is booked, $A2$ Account 2 is booked, $F$ a feature on an invoice.
    We respect two rules
    \begin{itemize}
        \item \textcolor{\concolor}{Exactly one account must be booked.}
        \item \textcolor{\probcolor}{If feature $\mathrm{F}$ is present on the invoice, the account $\mathrm{A1}$ is typically booked.}
    \end{itemize}
    We formalize this with the statistic
    \begin{align*}
        \hlnstat = (\catvariableof{A1} \oplus \catvariableof{A2}, \catvariableof{F}\Rightarrow \catvariableof{A1})\, .
    \end{align*}
    While the first formula is a hard feature, the second is soft since prone to exceptions.
    We parameterize the first output of the statistic with the hard parameters by setting the set of indices to be initialized with hard logic $A = \{0\}$ and the corresponding initialization $y_0 = 1$ meaning, that the first output of the statistic has to be true for the input to have positive probability.
% \begin{align*}
%     \hardlegset = \{0\} \quad, \quad \headindexof{\hardlegset} = 1.
% \end{align*}
    Then "hard logic activation tensor", should be indifferent to the second part of the statistic, and only impose rules on the first part, leading to
    \begin{align*}
        \textcolor{\concolor}{\kappa^{(A,y_A)}[Y_0,Y_1]} =
        \onehotmapofat{\headindexof{0}}{\headvariableof{0}} \otimes
        \onesat{\headvariableof{1}}
        =\begin{bmatrix}
             0 \\
             1
        \end{bmatrix} \otimes \begin{bmatrix}
                                  1\\1
        \end{bmatrix}.
    \end{align*}
    Since the first feature is hard, the "soft logic activation tensor" should be invariant under the first coordinate of the canonical parameter and we set $\canparamat{\selvariable=0}=0$. We choose the soft parameters as $\theta[L] = [0,\theta[L=1]]^\intercal$ to achieve
    \begin{align*}
        \textcolor{\probcolor}{\alpha^\theta[Y_0,Y_1]} &= \alpha^{0,0}\left[Y_0\right]\otimes \alpha^{1,\theta[L=1]}\left[Y_1\right] = \begin{bmatrix}
                                                                                                                                           1\\1
        \end{bmatrix}\otimes \begin{bmatrix}
                                 1\\ \expof{\canparamat{\selvariable=1}}
        \end{bmatrix}.
        % \\
        % \canparamat{\selvariable} 
        % &= \begin{bmatrix}
        %     0 \\
        %     \canparamat{\selvariable=1}
        % \end{bmatrix} \, .
    \end{align*}
    The activation tensor of the hybrid network then has the from
    \begin{align*}
        \paracttensor[{\headvariableof{0},\headvariableof{1}}]
        = \begin{bmatrix}
              0 \\ 1
        \end{bmatrix}
        \otimes
        \begin{bmatrix}
            1 \\ \expof{\canparamat{\selvariable=1}}
        \end{bmatrix}.
    \end{align*}

    We get a tensor network representation of the \HybridLogicNetwork{} representing the toy accounting example, before normalization to a distribution
    \begin{center}
        \begin{tikzpicture}[scale=0.35, thick, yscale=-1, xscale=-1] % , baseline = -3.5pt

            \begin{scope}[shift={(14,-3)}]
                \draw (-4.5,0) rectangle (10.5,2);
                \node[anchor=center] (A) at (3,1) {\corelabelsize $\probof{(\catvariableof{A1}\oplus \catvariableof{A2}, X_F\Rightarrow \catvariableof{A1}),(\{0\},(1),(0,\theta))}$};
                \draw[-<-]  (-2,2)--(-2,4) node[midway,left] {\colorlabelsize $\catvariableof{A1}$};
                \draw[-<-]  (3,2)--(3,4) node[midway,left] {\colorlabelsize $\catvariableof{A2}$};
                \draw[-<-]  (8,2)--(8,4) node[midway,left] {\colorlabelsize $\catvariableof{F}$};

                \node[anchor=center] (A) at (-6.5,1) {$\parallel$};
            \end{scope}

            \draw[->-]  (3,1.75)--(3,-1) node[midway,left] {\colorlabelsize $\catvariableof{A2}$};

            \draw[->-] (-1.5,1) to[bend right=20] (-3,-1);
            \draw[->-] (-1.5,1) to[bend right=-20] (0,-1);
            \drawvariabledot{-1.5}{1}
            \draw (-1.5,1)--(-1.5,1.75)  node[midway,left] {\colorlabelsize $\catvariableof{A1}$};

            \draw[->-] (-6,1.75)--(-6,-1) node[midway,left] {\colorlabelsize $\catvariableof{F}$};

            \draw (-7,-1) rectangle (-2, -3);
            \node[anchor=center] (text) at (-4.5,-2) {$\bencodingof{\Rightarrow}$};
            \draw[->-] (-4.5,-3) --(-4.5,-5.5) node[midway,left]{\colorlabelsize $\headvariableof{F \Rightarrow A1}$};

            \draw[fill, \probcolor] (-4.5,-5.5) circle (\dotsize);
            \draw[\probcolor] (-4.5,-5.5) -- (-5,-5.5);
            \draw[\probcolor]  (-13, -3.5) rectangle (-5, -6.5);
            \node[anchor=center,\probcolor] (text) at (-9,-5) {$\begin{bmatrix}
                                                                    1 \\
                                                                    \expof{\canparamat{\selvariable=1}}
            \end{bmatrix}$};


            \draw (-1,-1) rectangle (4, -3);
            \node[anchor=center] (text) at (1.5,-2) {$\bencodingof{\oplus}$};
            \draw[->-] (1.5,-3) --(1.5,-5.5) node[midway,left]{\colorlabelsize $\headvariableof{A1 \oplus A2}$};

            \draw[fill, \concolor] (1.5,-5.5) circle (\dotsize);
            \draw[\concolor] (1.5,-5.5) -- (1,-5.5);
            \draw[\concolor]  (-1, -3.5) rectangle (1, -6.5);
            \node[anchor=center,\concolor] (text) at (0,-5) {$\begin{bmatrix}
                                                                  0 \\
                                                                  1
            \end{bmatrix}$};

        \end{tikzpicture}
    \end{center}
    The resulting \HybridLogicNetwork{} is a tensor $\probofat{\hlnstat,\hybridparam}{\catvariableof{A1},\catvariableof{A2},\catvariableof{F}}$ of order $3$.
    With $Y_{F\Rightarrow A_1}=1$ for $F=0$ and any $A_1$ it has the coordinates
    \begin{center} %% FROM THE POSTER
        \begin{tikzpicture}[scale=1.75]
            \node (A) at (-4,0) {\corelabelsize $\probofat{(\catvariableof{A1}\oplus\catvariableof{A2},\catvariableof{F}\Rightarrow\catvariableof{A1}),\big(\{0\},(1),(0,\theta)\big)}{\catvariableof{A1},\catvariableof{A2},\catvariableof{F}} \,\, = \,\, \frac{1}{1+3 \cdot \expof{\canparam}} \, \cdot $};
            \node (A) at (0,0) {
                $\begin{bmatrix}
                     0 & \exp[\canparam] \\
                     \exp[\canparam] & 0
                \end{bmatrix}$
            };
            \node (A) at (1.25,0.3) {
                $\begin{bmatrix}
                     0 & 1 \\
                     \exp[\canparam] & 0
                \end{bmatrix}$
            };
            \draw[->,dashed] (-0.325,0.5) node[below] {\tiny $0$} -- (0.325,0.5) node [midway, above] {\tiny $\catvariableof{A2}$} node[below] {\tiny $1$};
            \draw[<-,dashed] (-0.9,-0.2) node[right] {\tiny $1$} -- (-0.9,0.2) node [midway, left] {\tiny $\catvariableof{A1}$} node[right] {\tiny $0$};
            \draw[->,dashed] (0,-0.5) node[above] {\tiny $0$} -- (1.25,-0.2) node [midway, below] {\tiny $X_F$} node[above] {\tiny $1$};
        \end{tikzpicture}
    \end{center}
\end{example}
