\begin{example}[The $\ttformat$ format]\label{exa:ttFormat}
    The Tensor-Train ($\ttformat$) format (see \cite{oseledets_tensor-train_2011}) corresponds in our notation to a hypergraph (see \figref{fig:TTHypergraph}) defined by
    \begin{itemize}
        \item nodes $\catvariableof{[\catorder]}$ and hidden variables $\decvariableof{[\catorder-1]}$, each decorated by a dimension $\catdimof{[\catorder]}$ and $\decdimof{[\catorder-1]}$,
        \item edges
        \begin{align*}
            \big\{\edgeof{0}=\{\catvariableof{0},\decvariableof{0}\}\big\} \cup
            \big\{\edgeof{\catenumerator}=\{\decvariableof{\catenumerator-1},\catvariableof{\catenumerator},\decvariableof{\catenumerator}\} \wcols \catenumerator\in\{1,\ldots,\catorder-2\}\big\} \cup
            \big\{\edgeof{\catorder-1}=\{\decvariableof{\catorder-2},\catvariableof{\catorder-1}\}\big\}
        \end{align*}
        each decorated by a tensor of order 3 (respectively 2 for $\catenumerator\in\{0,\seldim-1\}$).
    \end{itemize}
    \begin{figure}
        \begin{center}
            \begin{tikzpicture}[scale=0.35,thick]

                \begin{scope}[shift={(-22,-2)}]
                    \coordinate[label=left:$a)$] (A) at (-2,4);

                    \node[circle, draw, thick, fill=\nodegrayscale, minimum size = \nodeminsize] (A) at (0,0) {};
                    \node[anchor=center] (A) at (0,0) {\corelabelsize $\catvariableof{0}$};
                    \node[circle, draw, thick, fill=\nodegrayscale, minimum size = \nodeminsize] (A) at (4,0) {};
                    \node[anchor=center] (A) at (4,0) {\corelabelsize $\catvariableof{1}$};


                    \coordinate[label=below:$\hdots $] (A) at (10.5,4);
                    \coordinate[label=below:$\hdots $] (A) at (10.5,0);

                    \node[circle, draw, thick, fill=\nodegrayscale, minimum size = \nodeminsize] (A) at (2,4) {};
                    \node[anchor=center] (A) at (2,4) {\corelabelsize $\decvariableof{0}$};
                    \node[circle, draw, thick, fill=\nodegrayscale, minimum size = \nodeminsize] (A) at (6,4) {};
                    \node[anchor=center] (A) at (6,4) {\corelabelsize $\decvariableof{1}$};

                    %\coordinate[label=above:$e_0$] (A) at (0.5,1.6);
                    %\node[anchor=south] (A) at (0.5,1.6) {\colorlabelsize $e_0$};
                    \draw (0.5,1) -- (1.5,3) node[midway,left] {\colorlabelsize $e_0$};
%                    \draw (2,2)  to[bend right=30] (0.5,1);
%                    \draw (2,2)  to[bend left=30] (3.5,1);
%                    \draw (2,2) -- (2,2.9);

                    \node[anchor=south] (A) at (4,2) {\colorlabelsize $e_1$};
                    \draw (4,2)  to[bend right=-30] (2.5,3);
                    \draw (4,2)  to[bend left=-30] (5.5,3);
                    \draw (4,2) -- (4,1.1);

                    \node[anchor=south] (A) at (8,2) {\colorlabelsize $e_2$};
                    \draw (8,2)  to[bend right=-30] (6.5,3);
                    \draw (8,2)  to[bend left=-30] (9.5,3);
                    \draw (8,2) -- (8,1.1);

                    \begin{scope}[shift={(3,0)}]

                        \node[anchor=south] (A) at (10,2) {\colorlabelsize $e_{\catorder-2}$};
                        \draw (10,2)  to[bend right=-30] (8.5,3);
                        \draw (10,2)  to[bend left=-30] (11.5,3);
                        \draw (10,2) -- (10,1.1);

                        %\node[anchor=south] (A) at (0.5,1.6) {\colorlabelsize $e_0$};
                        \draw (13.5,1) -- (12.5,3) node[midway,right] {\colorlabelsize $e_{\catorder-1}$};

                        %\coordinate[label=above:$e_{\catorder-1}$] (A) at (11,1.6);
                        %\draw (12,2)  to[bend right=30] (10.5,1);
                        %\draw (12,2)  to[bend left=30] (13.5,1);
                        %\draw (12,2) -- (12,2.9);

                        \node[circle, draw, thick, fill=\nodegrayscale, minimum size = \nodeminsize] (A) at (12,4) {};
                        \node[anchor=center] (text) at (12,4) {\corelabelsize $\decvariableof{\catorder\shortminus2}$};

                        \node[circle, draw, thick, fill=\nodegrayscale, minimum size = \nodeminsize] (A) at (14,0) {};
                        \node[] (text) at (14,0) {\corelabelsize $\catvariableof{\catorder\shortminus1}$};
                    \end{scope}
                \end{scope}

                \coordinate[label=left:$b)$] (A) at (-2,2);

                \draw (-1,-1) rectangle (1,1);
                \node[anchor=center] (A) at (0,0) {\corelabelsize $\hypercoreof{0}$};
                \draw (0,-1)--(0,-2.5) node[midway,right] {\colorlabelsize $\catvariableof{0}$};

                \draw (1,0) -- (3,0) node[midway,above] {\colorlabelsize $\decvariableof{0}$};

                \draw (3,-1) rectangle (5,1);
                \node[anchor=center] (A) at (4,0) {\corelabelsize $\hypercoreof{1}$};
                \draw (4,-1)--(4,-2.5) node[midway,right] {\colorlabelsize $\catvariableof{1}$};

                \draw (5,0) -- (6.5,0) node[midway,above] {\colorlabelsize $\decvariableof{1}$};

%    \coordinate[label=below:$\hdots $] (A) at (9,0.35);
                \node[anchor=center] (text) at (8,0) {$\hdots$};

                \draw (9.5,0) -- (11,0) node[midway,above] {\colorlabelsize $\decvariableof{\catorder\shortminus2}$};;
                \draw (11,-1) rectangle (13,1);
                \node[anchor=center] (A) at (12,0) {\corelabelsize $\hypercoreof{\catorder\shortminus1}$};
                \draw (12,-1)--(12,-2.5) node[midway,right] {\colorlabelsize $\catvariableof{\catorder\shortminus1}$};


            \end{tikzpicture}
        \end{center}
        \caption{Hypergraph to a $\ttformat$ format.
        a) Node-centric design.
        b) Corresponding tensor-network on the edges of the hypergraph.}\label{fig:TTHypergraph}
    \end{figure}

\end{example}