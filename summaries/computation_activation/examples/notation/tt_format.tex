\begin{example}[The $\ttformat$ format]
    The Tensor-Train ($\ttformat$) (see \cite{oseledets_tensor-train_2011}) format corresponds in our notation with a hypergraph (see \figref{fig:TTHypergraph})
    \begin{itemize}
        \item Nodes by $\catvariableof{[\catorder]}$ and hidden variables $\decvariableof{[\catorder-1]}$, each decorated by a dimension $\catdimof{[\catorder]}$ and $\decdimof{[\catorder-1]}$
        \item Edges by
        \begin{align*}
            \left\{\edgeof{0}=(\catvariableof{0},\decvariableof{0})\right\} \cup
            \left\{\edgeof{\catenumerator}=(\decvariableof{\catenumerator-1},\catvariableof{\catenumerator},\decvariableof{\catenumerator+1}) \wcols \catenumerator\in\{1,\ldots,\catorder-2\}\right\} \cup
            \left\{\edgeof{\catdim-1}=(\decvariableof{\catorder-2},\catvariableof{\catorder-1})\right\}
        \end{align*}
        each decorated by a tensor of order 3 (respectively 2 for $\catenumerator\in\{0,\seldim-1\}$).
    \end{itemize}
    \begin{figure}
        \begin{center}
            \input{../tikz_pics/notation_basic_concepts/tt_hypergraph.tex}
        \end{center}
        \caption{Hypergraph to a $\ttformat$ format.
        a) Node-centric design.
        b) Corresponding tensor-network on the edges of the hypergraph.}\label{fig:TTHypergraph}
    \end{figure}

\end{example}