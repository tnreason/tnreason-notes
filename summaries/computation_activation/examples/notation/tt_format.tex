\begin{example}[The $\ttformat$ format]
    The Tensor-Train ($\ttformat$) (see \cite{oseledets_tensor-train_2011}) format corresponds in our notation with a hypergraph (see \figref{fig:TTHypergraph})
    \begin{itemize}
        \item Nodes by $\catvariableof{[\catorder]}$ and hidden variables $\decvariableof{[\catorder-1]}$, each decorated by a dimension $\catdimof{[\catorder]}$ and $\decdimof{[\catorder-1]}$
        \item Edges by
        \begin{align*}
            \left\{\edgeof{0}=(\catvariableof{0},\decvariableof{0})\right\} \cup
            \left\{\edgeof{\catenumerator}=(\decvariableof{\catenumerator-1},\catvariableof{\catenumerator},\decvariableof{\catenumerator+1}) \wcols \catenumerator\in\{1,\ldots,\catorder-2\}\right\} \cup
            \left\{\edgeof{\catdim-1}=(\decvariableof{\catorder-2},\catvariableof{\catorder-1})\right\}
        \end{align*}
        each decorated by a tensor of order 3 (respectively 2 for $\catenumerator\in\{0,\seldim-1\}$).
    \end{itemize}
    \begin{figure}
        \begin{center}
            \begin{tikzpicture}[scale=0.35,thick]

                \begin{scope}[shift={(-19,-2)}]
                    \coordinate[label=left:$a)$] (A) at (-2,4);

                    \node[circle, draw, thick, fill=\nodegrayscale, minimum size = \nodeminsize] (A) at (0,0) {};
                    \node[anchor=center] (A) at (0,0) {\corelabelsize $\catvariableof{0}$};
                    \node[circle, draw, thick, fill=\nodegrayscale, minimum size = \nodeminsize] (A) at (4,0) {};
                    \node[anchor=center] (A) at (4,0) {\corelabelsize $\catvariableof{1}$};


                    \coordinate[label=below:$\hdots $] (A) at (9,0.5);

                    \node[circle, draw, thick, fill=\nodegrayscale, minimum size = \nodeminsize] (A) at (14,0) {};
                    \node[] (text) at (14,0) {\corelabelsize $\catvariableof{\catorder\shortminus1}$};

                    \node[circle, draw, thick, fill=\nodegrayscale, minimum size = \nodeminsize] (A) at (2,4) {};
                    \node[anchor=center] (A) at (2,4) {\corelabelsize $\decvariableof{0}$};
                    \node[circle, draw, thick, fill=\nodegrayscale, minimum size = \nodeminsize] (A) at (6,4) {};
                    \node[anchor=center] (A) at (6,4) {\corelabelsize $\decvariableof{1}$};

                    \node[circle, draw, thick, fill=\nodegrayscale, minimum size = \nodeminsize] (A) at (12,4) {};
                    \node[anchor=center] (text) at (12,4) {\corelabelsize $\decvariableof{\catorder\shortminus1}$};

                    \coordinate[label=above:$e_0$] (A) at (1,1.6);
                    \draw (2,2)  to[bend right=30] (0.5,1);
                    \draw (2,2)  to[bend left=30] (3.5,1);
                    \draw (2,2) -- (2,2.9);

                    \coordinate[label=above:$e_1$] (A) at (5,1.6);
                    \draw (6,2)  to[bend right=30] (4.5,1);
                    \draw (6,2)  to[bend left=30] (7.5,1);
                    \draw (6,2) -- (6,2.9);

                    \coordinate[label=above:$e_{\catorder-1}$] (A) at (11,1.6);
                    \draw (12,2)  to[bend right=30] (10.5,1);
                    \draw (12,2)  to[bend left=30] (13.5,1);
                    \draw (12,2) -- (12,2.9);

                \end{scope}

                \coordinate[label=left:$b)$] (A) at (-2,2);

                \draw (-1,-1) rectangle (1,1);
                \node[anchor=center] (A) at (0,0) {\corelabelsize $\hypercoreof{0}$};
                \draw (0,-1)--(0,-2.5) node[midway,right] {\colorlabelsize $\catvariableof{0}$};

                \draw (1,0) -- (3,0) node[midway,above] {\colorlabelsize $\decvariableof{0}$};

                \draw (3,-1) rectangle (5,1);
                \node[anchor=center] (A) at (4,0) {\corelabelsize $\hypercoreof{1}$};
                \draw (4,-1)--(4,-2.5) node[midway,right] {\colorlabelsize $\catvariableof{1}$};

                \draw (5,0) -- (6.5,0) node[midway,above] {\colorlabelsize $\decvariableof{1}$};

%    \coordinate[label=below:$\hdots $] (A) at (9,0.35);
                \node[anchor=center] (text) at (8,0) {$\hdots$};

                \draw (9.5,0) -- (11,0) node[midway,above] {\colorlabelsize $\decvariableof{\catorder\shortminus1}$};;
                \draw (11,-1) rectangle (13,1);
                \node[anchor=center] (A) at (12,0) {\corelabelsize $\hypercoreof{\catorder\shortminus1}$};
                \draw (12,-1)--(12,-2.5) node[midway,right] {\colorlabelsize $\catvariableof{\catorder\shortminus1}$};


            \end{tikzpicture}
        \end{center}
        \caption{Hypergraph to a $\ttformat$ format.
        a) Node-centric design.
        b) Corresponding tensor-network on the edges of the hypergraph.}\label{fig:TTHypergraph}
    \end{figure}

\end{example}