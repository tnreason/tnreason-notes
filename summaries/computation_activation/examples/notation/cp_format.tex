\begin{example}[The $\cpformat$ format]\label{exa:cpFormat}
    The Candecomp-Parafac ($\cpformat$ (\cite{hitchcock_expression_1927}) tensor format corresponds in our notation to a hypergraph (see \figref{fig:CPHypergraph}) defined by
    \begin{itemize}
        \item Nodes $\catvariableof{[\catorder]}$ and a single hidden variable $\decvariable$, decorated by dimensions $\catdimof{[\catorder]}$ and $\decdim$, respectively \janina{What is n? Equal to the rank?}
        \item Edges
        % \begin{align*}
            $\big\{\edgeof{\catenumerator}=\{\catvariableof{\catenumerator},\decvariable\} \wcols \catenumeratorin \big\}$
        % \end{align*}
        each decorated by a matrix $\hypercoreofat{\edgeof{\catenumerator}}{\catvariableof{\catenumerator},\decvariable}\janina{\in\mathbb{R}^{m\times n}}$.
    \end{itemize}

    \begin{figure}
        \begin{center}
            \begin{tikzpicture}[scale=0.35,thick]

                \begin{scope}[shift={(-19,-2)}]
                    \coordinate[label=left:$a)$] (A) at (-2,4);

                    \node[circle, draw, thick, fill=\nodegrayscale, minimum size = \nodeminsize] (A) at (0,0) {};
                    \node[anchor=center] (A) at (0,0) {\corelabelsize $\catvariableof{0}$};

                    \node[circle, draw, thick, fill=\nodegrayscale, minimum size = \nodeminsize] (A) at (4,0) {};
                    \node[anchor=center] (A) at (4,0) {\corelabelsize $\catvariableof{1}$};

                    \coordinate[label=below:$\hdots $] (A) at (9,0.5);

                    \node[circle, draw, thick, fill=\nodegrayscale, minimum size = \nodeminsize] (A) at (14,0) {};
                    \node[] (text) at (14,0) {\corelabelsize $\catvariableof{\catorder\shortminus1}$};

                    \node[circle, draw, thick, fill=\nodegrayscale, minimum size = \nodeminsize] (A) at (7,4) {\corelabelsize $\decvariable$};

                    \draw (6.25,3.2) -- (0.5,1) node[midway,above] {\colorlabelsize $\edgeof{0}$};
                    \draw (6.75,2.9) -- (4,1.15) node[midway,below] {\colorlabelsize $\edgeof{1}$};
                    \draw (7.75,3.2) -- (13.5,1) node[midway,above] {\colorlabelsize $\edgeof{\catorder\shortminus1}$};

                \end{scope}

                \coordinate[label=left:$b)$] (A) at (-2,2);

                \begin{scope}[shift={(0,-2)}]

                    \draw (-1,-1) rectangle (1,1);
                    \node[anchor=center] (A) at (0,0) {\corelabelsize $\hypercoreof{0}$};
                    \draw (0,-1)--(0,-2.5) node[midway,right] {\colorlabelsize $\catvariableof{0}$};

                    \draw (3,-1) rectangle (5,1);
                    \node[anchor=center] (A) at (4,0) {\corelabelsize $\hypercoreof{1}$};
                    \draw (4,-1)--(4,-2.5) node[midway,right] {\colorlabelsize $\catvariableof{1}$};

                    \node[anchor=center] (text) at (8,0) {$\hdots$};

                    \draw (11,-1) rectangle (13,1);
                    \node[anchor=center] (A) at (12,0) {\corelabelsize $\hypercoreof{\catorder\shortminus1}$};
                    \draw (12,-1)--(12,-2.5) node[midway,right] {\colorlabelsize $\catvariableof{\catorder\shortminus1}$};

                    \drawvariabledot{6}{4}
                    \node[anchor=south] (text) at (6,4) {\colorlabelsize $\decvariable$};

                    \draw (6,4) to[bend right= 20] (0,1);
                    \draw (6,4) to[bend right= 10] (4,1);
                    \draw (6,4) to[bend right= -20] (12,1);

                \end{scope}


            \end{tikzpicture}
        \end{center}
        \caption{Hypergraph to a $\cpformat$ format.
        a) Node-centric design.
        b) Corresponding tensor-network on the edges of the hypergraph.}\label{fig:CPHypergraph}
    \end{figure}

\end{example}