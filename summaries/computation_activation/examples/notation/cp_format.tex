\begin{example}[The $\cpformat$ format]
    The Candecomp-Parafac ($\cpformat$ (\cite{hitchcock_expression_1927}) tensor format corresponds in our notation with a hypergraph (see \figref{fig:CPHypergraph})
    \begin{itemize}
        \item Nodes by $\catvariableof{[\catorder]}$ and a single hidden variable $\decvariable$, decorated by dimensions $\catdimof{[\catorder]}$ and $\decdim$.
        \item Edges by
        \begin{align*}
            \left\{\edgeof{\catenumerator}=(\catvariableof{\catenumerator},\decvariable) \wcols \catenumeratorin \right\}
        \end{align*}
        each decorated by a matrix $\hypercoreofat{\edgeof{\catenumerator}}{\catvariableof{\catenumerator},\decvariable}$.
    \end{itemize}

    \begin{figure}
        \begin{center}
            \input{../tikz_pics/notation_basic_concepts/cp_hypergraph.tex}
        \end{center}
        \caption{Hypergraph to a $\cpformat$ format.
        a) Node-centric design.
        b) Corresponding tensor-network on the edges of the hypergraph.}\label{fig:CPHypergraph}
    \end{figure}

\end{example}