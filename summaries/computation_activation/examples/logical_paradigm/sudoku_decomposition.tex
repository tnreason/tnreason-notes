\begin{example}[Sparse representation of Sudoku rule knowledge base]
    \label{exa:sudokuDecomposition}%{\alex{Attempt to match the above Sudoku example with our notation of boolean variables and the entailment formalism}}
    We exploit \theref{the:kbDecomposition} to find an efficient tensor network representation of the Sudoku knowledge base from \exaref{exa:sudokuEntailment}.
    We directly get that the knowledge base $\sudokukbof{\sudokunum}$ of Sudoku rules is a tensor network of the $4\cdot \sudokunum^4$ constraint formulas using the $\sudokunum^2$-ary connective $\woneoplus$, and the evidence $\sudokustartevidence$ can be encoded by vectors $\tbasisat{\catvariableof{(r_0,r_1,c_0,c_1,i)}}$.
    To get a representation by matrices instead of tensors of order $\sudokunum^2$, we introduce a hidden variable $\decvariable$ taking values in $[\sudokunum^2]$ for each of the constraints, one can further increase the sparsity of the representation.
    Using the matrices
    \begin{align*}
        \hypercoreofat{\catenumerator}{\catvariableof{\catenumerator},\decvariable}
        = \fbasisat{\catvariableof{\catenumerator}} \otimes \onesat{\decvariable} + (\tbasisat{\catvariableof{\catenumerator}}-\fbasisat{\catvariableof{\catenumerator}}) \otimes \onehotmapofat{\catenumerator}{\decvariable},
    \end{align*}
    we have the decomposition
    \begin{align*}
        \woneoplus[\catvariableof{[\sudokunum^2]}]
        = \contractionof{\{\hypercoreofat{\catenumerator}{\catvariableof{\catenumerator},\decvariable} \wcols \catenumerator\in[\sudokunum^2]\}}{\catvariableof{[\sudokunum^2]}} \, ,
    \end{align*}
    which is a $\cpformat$ decomposition (see \exaref{exa:cpFormat}) depicted in \figref{fig:sudokuDecomposition} a).

    %% Alternative TT decomposition
    Alternatively, there is a $\ttformat$ decomposition (see \exaref{exa:ttFormat}) of the constraint $\woneoplus$, which we depict in \figref{fig:sudokuDecomposition} b).
    We introduce for $\catenumerator\in[\catorder-1]$ hidden variables $\decvariable^{\catenumerator}$ of dimension $2$, which are interpreted as the indicator whether one of the variables $\catvariableof{[\catenumerator]}$ is true.
    Following this interpretation we introduce $\ttformat$ cores
    \begin{align*}
        \sechypercoreofat{0}{\catvariableof{0},\secdecvariable^{0}}
        &= \tbasisat{\catvariableof{0}}\otimes\tbasisat{\secdecvariable^{0}}
        + \fbasisat{\catvariableof{0}}\otimes\fbasisat{\secdecvariable^{0}}\,, \\
        \sechypercoreofat{\catorder-1}{\secdecvariable^{\catorder-2},\catvariableof{\catorder-1}}
        &= \fbasisat{\secdecvariable^{\catorder-2}} \otimes \tbasisat{\catvariableof{\catorder-1}}
        + \tbasisat{\secdecvariable^{\catorder-2}} \otimes \fbasisat{\catvariableof{\catorder-1}}
    \end{align*}
    and for $\catenumerator\in\{1,\ldots,\catorder-2\}$
    \begin{align*}
        \sechypercoreofat{\catenumerator}{\secdecvariable^{\catenumerator-1},\catvariableof{\catenumerator},\secdecvariable^{\catenumerator}}
        = \identityat{\secdecvariable^{\catenumerator-1},\secdecvariable^{\catenumerator}}  \otimes \fbasisat{\catvariableof{\catenumerator}}
        + \fbasisat{\secdecvariable^{\catenumerator-1}} \otimes \tbasisat{\catvariableof{\catenumerator}} \otimes \tbasisat{\secdecvariable^{\catenumerator-1}} \, .
    \end{align*}
    Note that the $\ttformat$ decomposition of the constraint $\woneoplus$ introduces $\catorder-1$ many hidden variables of dimension $2$ whereas the $\cpformat$ decomposition introduces a single hidden variable of dimension $\catorder$.
    In the following, we further apply the $\cpformat$ decomposition. % Reason: fewer variables to handle in message passing

    \begin{figure}[t]

        \begin{center}
            \begin{tikzpicture}[scale=0.35,thick]

                \draw (-6,-1) rectangle (-12,1);
                \node[anchor=center] (A) at (-9,0) {\corelabelsize $\woneoplus$};
                \draw (-11.5,-1)--(-11.5,-2.5) node[midway,left] {\colorlabelsize $\catvariableof{0,0,0,0,0}$};
                \draw (-11,-1)--(-11,-2.5) node[midway,right] {\colorlabelsize $\catvariableof{0,0,0,0,1}$};
                \node[anchor=center] (A) at (-9,-2.5) {$\cdots$};
                \draw (-7,-1)--(-7,-2.5) node[midway,right] {\colorlabelsize $\catvariableof{0,0,0,0,\sudokunum^2\shortminus 1}$};

                \node[anchor=center] (A) at (-3.5,1) {$a)$};
                \node[anchor=center] (A) at (-3.5,0) {$=$};

                \draw (-1,-1) rectangle (1,1);
                \node[anchor=center] (A) at (0,0) {\corelabelsize $\hypercoreof{0}$};
                \draw (0,-1)--(0,-2.5) node[midway,right] {\colorlabelsize $\catvariableof{0,0,0,0,0}$};

                \draw (3,-1) rectangle (5,1);
                \node[anchor=center] (A) at (4,0) {\corelabelsize $\hypercoreof{1}$};
                \draw (4,-1)--(4,-2.5) node[midway,right] {\colorlabelsize $\catvariableof{0,0,0,0,1}$};

                \node[anchor=center] (text) at (8,0) {$\hdots$};

                \draw (10.75,-1) rectangle (13.25,1);
                \node[anchor=center] (A) at (12,0) {\corelabelsize $\hypercoreof{\sudokunum^2\shortminus1}$};
                \draw (12,-1)--(12,-2.5) node[midway,right] {\colorlabelsize $\catvariableof{0,0,0,0,\sudokunum^2\shortminus1}$};

                \drawvariabledot{6}{4}
                \node[anchor=south] (text) at (6,4) {\colorlabelsize $\decvariableof{0,0,0,0,:}$};

                \draw (6,4) to[bend right= 20] (0,1);
                \draw (6,4) to[bend right= 10] (4,1);
                \draw (6,4) to[bend right= -20] (12,1);

                \begin{scope}[shift={(0,-7)}]
                    \node[anchor=center] (A) at (-3.5,1) {$b)$};
                    \node[anchor=center] (A) at (-3.5,0) {$=$};

                    \draw (-2,-1) rectangle (0,1);
                    \node[anchor=center] (A) at (-1,0) {\corelabelsize $\sechypercoreof{0}$};
                    \draw (-1,-1)--(-1,-2.5) node[midway,right] {\colorlabelsize $\catvariableof{0,0,0,0,0}$};

                    \draw (0,0) -- (3,0) node [midway,above] {\colorlabelsize $\secdecvariableof{0,0,0,0,:}^0$};

                    \draw (3,-1) rectangle (5,1);
                    \node[anchor=center] (A) at (4,0) {\corelabelsize $\sechypercoreof{1}$};
                    \draw (4,-1)--(4,-2.5) node[midway,right] {\colorlabelsize $\catvariableof{0,0,0,0,1}$};

                    \node[anchor=center] (text) at (6.5,1) {\colorlabelsize $\secdecvariableof{0,0,0,0,:}^{1}$};
                    \draw (5,0) -- (7,0); % node [midway,above] {\colorlabelsize $\secdecvariableof{0,0,0,0,:}^1$};

                    \node[anchor=center] (text) at (8,0) {$\hdots$};

                    \node[anchor=center] (text) at (10.25,1) {\colorlabelsize $\secdecvariableof{0,0,0,0,:}^{\sudokunum^2\shortminus2}$};
                    \draw (9.75,0) -- (11.75,0);

                    \draw (11.75,-1) rectangle (14.25,1);
                    \node[anchor=center] (A) at (13,0) {\corelabelsize $\sechypercoreof{\sudokunum^2\shortminus1}$};
                    \draw (13,-1)--(13,-2.5) node[midway,right] {\colorlabelsize $\catvariableof{0,0,0,0,\sudokunum^2\shortminus1}$};


                \end{scope}

            \end{tikzpicture}
        \end{center}
        \caption{Decomposition of the position constraint $\woneoplus$ at position $(r0,r1,c0,c1)=(0,0,0,0)$ into a) a $\cpformat$ decomposition with hidden variable $\decvariableof{0,0,0,0,:}$ and b) a $\ttformat$ decomposition with $d-1$ hidden variables $\decvariableof{0,0,0,0,:}^k, k\in[d-1]$.}
        \label{fig:sudokuDecomposition}
    \end{figure}

    Given evidence $\sudokustartevidence$ we denote the Sudoku Knowledge Base $\sudokukbof{\sudokunum,\sudokustartevidence}$.
    It is modelled as a tensor network on a hypergraph $\graphof{\mathrm{Sudoku},n}$ consisting of
    \begin{itemize}
        \item $\sudokunum^6+4\cdot \sudokunum^4$ nodes by $\sudokunum^6$ categorical variables $\catvariableof{(r0,r1,c0,c1,i)}$ and by $4\cdot \sudokunum^4$ decomposition variables to the constraints
        \item $5\cdot \sudokunum^6$ edges
        \begin{align*}
            \edges=
            \bigcup_{r0,r1,c0,c1\in[\sudokunum]}
            \big\{
            &\{\catvariableof{(r0,r1,c0,c1,i)}\},\{\catvariableof{(r0,r1,c0,c1,i)},\decvariableof{r0,r1,c0,c1,:}\},\{\catvariableof{(r0,r1,c0,c1,i)},\decvariableof{r0,r1,:,:,i}\},\\
            &\{\catvariableof{(r0,r1,c0,c1,i)},\decvariableof{:,:,c0,c1,i}\},
            \{\catvariableof{(r0,r1,c0,c1,i)},\decvariableof{r0,:,c0,:,i}\}\big\}
        \end{align*}
        We denote the decomposition variables to the position, row, column and square constraints by $\decvariableof{r0,r1,c0,c1,:},\decvariableof{r0,r1,:,:,i},\decvariableof{:,:,c0,c1,i}$ and $\decvariableof{r0,:,c0,:,i}$.
    \end{itemize}
    Each edge containing a decomposition variable is decorated by a matrix $\hypercoreofat{\catenumerator}{\catvariable,\decvariable}$ corresponding to a core in the $\cpformat$ decomposition of a constraint.
    Here, $\catenumerator$ is determined by the tuple $(r0,r1,c0,c1,i)$ and the type of the constraint (for example, for the variable $\catvariableof{(0,1,1,2,1)}$ and the row constraint $\decvariableof{(0,1,:,:,1)}$ we have $\catenumerator=1\cdot n + 2$.
    We further assign to each edge containing a single variable $\{\catvariableof{(r0,r1,c0,c1,i)}\}$ either the vector $\tbasisat{\catvariableof{(r0,r1,c0,c1,i)}}$ if $(r0,r1,c0,c1,i)\in \sudokustartevidence$ or the trivial vector $\onesat{\catvariableof{(r0,r1,c0,c1,i)}}$.
\end{example}