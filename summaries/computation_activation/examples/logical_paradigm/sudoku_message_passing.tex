\begin{example}[Constraint Propagation for the Sudoku of \exaref{exa:sudokuEntailment}]
    \label{exa:sudokuMessagePassing}
    We iteratively solve a Sudoku puzzle by determining a possible value based on neighboring cells, rows and squares (using \theref{the:monotonicityPL}) and adding to our knowledge (using \theref{the:addingEntailed}).
    For example, consider the following $\sudokunum=2$ Sudoku puzzle, where a first entailment step uses only the knowledge of the rules and the \textcolor{\concolor}{blue} cells to determine the value $3$ in the first square:
    \begin{center}
        \begin{sudoku4x4}
            \matrix[sudokumatrix] (M) at (0,0) {
                1 & \ & \textcolor{\concolor}{3} & 2 \\
                \ & \textcolor{\concolor}{2} & \  & \  \\
                \ & \ & 4 & \ \\
                4 & 3 &  \ & \  \\
            };
            \draw[thick]([yshift=9.5pt,xshift=-0.6pt]M-1-2.east) -- ([yshift=-9.5pt,xshift=-0.6pt]M-4-2.east);
            \draw[thick]([xshift=-9.5pt,yshift=0.6pt]M-2-1.south) -- ([xshift=9.5pt,yshift=0.6pt]M-2-4.south);

            \node[anchor=center] (ist) at (1.75,0) {$=$};

            \matrix[sudokumatrix] (M) at (3.5,0) {
                1 & \ & 3 & 2 \\
                \textcolor{\probcolor}{3} & 2 & \  & \  \\
                \ & \ & 4 & \ \\
                4 & 3 &  \ & \  \\
            };
            \draw[thick]([yshift=9.5pt,xshift=-0.6pt]M-1-2.east) -- ([yshift=-9.5pt,xshift=-0.6pt]M-4-2.east);
            \draw[thick]([xshift=-9.5pt,yshift=0.6pt]M-2-1.south) -- ([xshift=9.5pt,yshift=0.6pt]M-2-4.south);

            \node[anchor=center] (ist) at (6.25,0) {$= \quad \ldots \quad =$};

            \matrix[sudokumatrix] (M) at (9,0) {
                1 & \textcolor{\probcolor}{4} & 3 & 2 \\
                \textcolor{\probcolor}{3} & 2 & \textcolor{\probcolor}{1} & \textcolor{\probcolor}{4}  \\
                \textcolor{\probcolor}{2} & \textcolor{\probcolor}{1} & 4 & \textcolor{\probcolor}{3} \\
                4 & 3 & \textcolor{\probcolor}{2} & \textcolor{\probcolor}{1}  \\
            };
            \draw[thick]([yshift=9.5pt,xshift=-0.6pt]M-1-2.east) -- ([yshift=-9.5pt,xshift=-0.6pt]M-4-2.east);
            \draw[thick]([xshift=-9.5pt,yshift=0.6pt]M-2-1.south) -- ([xshift=9.5pt,yshift=0.6pt]M-2-4.south);
        \end{sudoku4x4}
    \end{center}

    To illustrate the first reasoning step of assigning $\textcolor{\probcolor}{\catvariableof{0,1,0,0,2}}$, we make the following entailment steps applying \theref{the:monotonicityPL}.
    We also depict in \figref{fig:contractionPropagationSudoku} the corresponding messages in the Constraint Propagation Algorithm on the hypergraph $\graph^{\mathrm{Sudoku},n}$.
    \begin{itemize}
        \item From $\textcolor{\concolor}{\catvariableof{0,1,0,1,1}}$ (i.e. the $2$ in the cell $(0,1,0,1)$) and the Sudoku rule that at the cell $(0,1,0,1)$ exactly one number is assigned, we get
        \begin{align*}
            \left( \woneoplus_{i\in[\sudokunum^2]} \catvariableof{0,1,0,1,i} \right) \land \textcolor{\concolor}{\catvariableof{0,1,0,1,1}} \models \lnot\catvariableof{0,1,0,1,2} \, ,
        \end{align*}
        That is, that the number $3$ is not in the cell $(0,1,0,1)$.
        This entailment step is performed by three consecutive messages (see $\messagesymbol^{(0,[3])}$ in \figref{fig:contractionPropagationSudoku}) along the directions %involving decomposition cores of the position constraint in the position $(r_0,r_1,c_0,c_1)=(0,0,0,0)$.
        \begin{align*}
        (\sedge,\redge)
            \in \big[&(\{\catvariableof{0,1,0,1,1}\},\{\catvariableof{0,1,0,1,1},\decvariableof{0,1,0,1,:}\}),
                (\{\catvariableof{0,1,0,1,1},\decvariableof{0,1,0,1,:}\},\{\catvariableof{0,1,0,1,2},\decvariableof{0,1,0,1,:}\}), \\
                &(\{\catvariableof{0,1,0,1,2},\decvariableof{0,1,0,1,:}\},\{\catvariableof{0,1,0,1,2},\decvariableof{0,:,0,:,2}\})\big] \, .
        \end{align*}
        Intuitively, the messages commmunicate to the square constraint $\decvariableof{0,:,0,:,2}$, that by the position constraint $\decvariableof{0,1,0,1,:}$ the variable $3$ cannot be assigned at $(0,1,0,1)$.
        %This entailment step is performed by two consecutive messages along the directions $(\catvariableof{0,1,0,1,1},\decvariableof{0,1,0,1,:})$ and $(\decvariableof{0,1,0,1,:},\catvariableof{0,1,0,1,2})$.
        \item From $\textcolor{\concolor}{\catvariableof{0,0,1,0,2}}$ (i.e. the $3$ in the cell $(0,0,1,0)$) and the Sudoku rule that at the row $(0,0)$ exactly one number is assigned, we get
        \begin{align*}
            \left( \woneoplus_{c0,c1\in[\sudokunum]} \catvariableof{0,0,c0,c1,2} \right) \land \textcolor{\concolor}{\catvariableof{0,0,1,0,2}} \models \lnot\catvariableof{0,0,0,0,2}\land \lnot\catvariableof{0,0,0,1,2} \, ,
        \end{align*}
        meaning that the number $3$ is neither in the cell $(0,0,0,0)$ nor in $(0,0,0,1)$.
        This entailment step is performed by five consecutive messages (see $\messagesymbol^{(1,[5])}$ in \figref{fig:contractionPropagationSudoku}) along the directions
        %This entailment step is performed by three consecutive messages along the directions $(\catvariableof{0,0,1,0,2},\decvariableof{0,0,:,:,2})$, $(\decvariableof{0,0,:,:,2},\catvariableof{0,0,0,0,2})$ and $(\decvariableof{0,0,:,:,2},\catvariableof{0,0,0,1,2})$.
        \begin{align*}
        (\sedge,\redge)
            \in \big[
            &(\{\catvariableof{0,0,1,0,2}\},\{\catvariableof{0,0,1,0,2},\decvariableof{0,0,:,:,2}\}),
                (\{\catvariableof{0,0,1,0,2},\decvariableof{0,0,:,:,2}\},\{\catvariableof{0,0,0,0,2},\decvariableof{0,0,:,:,2}\}), \\
                &(\{\catvariableof{0,0,1,0,2},\decvariableof{0,0,:,:,2}\},\{\catvariableof{0,0,0,1,2},\decvariableof{0,0,:,:,2}\}),
                (\{\catvariableof{0,0,0,0,2},\decvariableof{0,0,:,:,2}\},\{\catvariableof{0,0,0,0,2},\decvariableof{0,:,0,:,2}\}) \\
                &(\{\catvariableof{0,0,0,1,2},\decvariableof{0,0,:,:,2}\},\{\catvariableof{0,0,0,1,2},\decvariableof{0,:,0,:,2}\})
                \big] \, .
        \end{align*}
        The messages communicate that based on the decomposition cores of the constraint to the number $i=3$ in the first row $(r_0,r_1)=(0,0)$, the number $3$ cannot be assigned at $(0,0,0,0)$ and $(0,0,0,1)$.
    \end{itemize}
    We add these formulas to our knowledge base (justified by \theref{the:addingEntailed}) and use the rule that $3$ appears exactly once in the first square
    \begin{align*}
        &\left( \woneoplus_{r1,c1\in[\sudokunum]} \catvariableof{0,r1,0,c1,2} \right)
        \land (\lnot\catvariableof{0,1,0,1,2})
        \land (\lnot\catvariableof{0,0,0,0,2}\land \lnot\catvariableof{0,0,0,1,2})
        \models \textcolor{\probcolor}{\catvariableof{0,1,0,0,2}} \, .
    \end{align*}
    We hence conclude that the number $3$ must be in the cell $(0,1,0,0)$.
    This information is also included in the updated knowledge base for further reasoning steps.
    This last entailment step is performed by four consecutive messages (see $\messagesymbol^{(2,[4])}$ in \figref{fig:contractionPropagationSudoku}) along the directions
    \begin{align*}
    (\sedge,\redge)
        \in \big[
            &(\{\catvariableof{0,1,0,1,2},\decvariableof{0,:,0,:,2}\},\{\catvariableof{0,1,0,0,2},\decvariableof{0,:,0,:,2}\}),
            (\{\catvariableof{0,0,0,1,2},\decvariableof{0,:,0,:,2}\},\{\catvariableof{0,1,0,0,2},\decvariableof{0,:,0,:,2}\}), \\
            &(\{\catvariableof{0,0,1,0,2},\decvariableof{0,:,0,:,2}\},\{\catvariableof{0,1,0,0,2},\decvariableof{0,:,0,:,2}\}),
            (\{\catvariableof{0,1,0,0,2},\decvariableof{0,:,0,:,2}\},\{\catvariableof{0,1,0,0,2}\})\big]
    \end{align*}
    The first three messages communicate that the $3$ is not possible in positions $(0,1,0,1),(0,0,0,1)$ and $(0,0,1,0)$ and the fourth message concludes that the $3$ then has to be at position $(0,1,0,0)$.

    %% Further reasoning steps
    We now iteratively apply similar reasoning steps and store the entailed variables in \textcolor{\probcolor}{$E^{\mathrm{entailed}}$} until we arrive at the right side of the above sketch.
    \begin{align*}
        \sudokukbof{2} \land \left(\bigwedge_{(r_0,r_1,c_0,c_1,i)\in \sudokustartevidence} \catvariableof{r0,r1,c0,c1,i} \right)
        \models \textcolor{\probcolor}{\left(\bigwedge_{(r_0,r_1,c_0,c_1,i)\in E^{\mathrm{entailed}}} \catvariableof{r0,r1,c0,c1,i} \right)} \, .
    \end{align*}
    Since all Sudoku rules are satisfied in the final assignment and to each cell $(r_0,r_1,c_0,c_1)$ we found exactly one $i\in[\sudokunum^2]$ such that $(r_0,r_1,c_0,c_1,i)\in \sudokustartevidence\cup\textcolor{\probcolor}{E^{\mathrm{entailed}}}$, there is a unique solution of the puzzle and we conclude that
    \begin{align*}
        &\sudokukbof{2} \land \left(\bigwedge_{(r_0,r_1,c_0,c_1,i)\in \sudokustartevidence} \catvariableof{r0,r1,c0,c1,i} \right) \\
        &\quad= \left(\bigwedge_{(r_0,r_1,c_0,c_1,i)\in \sudokustartevidence} \catvariableof{r0,r1,c0,c1,i} \right)
        \land \textcolor{\probcolor}{\left(\bigwedge_{(r_0,r_1,c_0,c_1,i)\in E^{\mathrm{entailed}}} \catvariableof{r0,r1,c0,c1,i} \right)} \, .
    \end{align*}
\end{example}

\begin{figure}[t]
    \begin{center}
        \begin{tikzpicture}[scale=0.35,thick]

            %% I 0:0:2 constraint (3 is in the 00 square)
            \draw (-1,-1) rectangle (1,1);
            \node[anchor=center] (A) at (0,0) {\corelabelsize $\hypercoreof{0}$};
            \draw (0,-1)--(0,-3) node[midway,right] {\colorlabelsize $\catvariableof{0,0,0,0,2}$};

            \draw (3,-1) rectangle (5,1);
            \node[anchor=center] (A) at (4,0) {\corelabelsize $\hypercoreof{1}$};
            \draw (4,-1)--(4,-3) node[midway,right] {\colorlabelsize $\catvariableof{0,0,0,1,2}$};

            \draw (7,-1) rectangle (9,1);
            \node[anchor=center] (A) at (8,0) {\corelabelsize $\hypercoreof{2}$};
            \draw (8,-1)--(8,-3); % node[midway,right] {\colorlabelsize $\catvariableof{0,1,0,0,2}$};

            \drawvariabledot{8}{-3}
            \draw (8,-3) -- (7.25,-3);
            \draw[\probcolor] (7,-5) rectangle (9,-7);
            \node[anchor=center, \probcolor] (A) at (8,-6) {\corelabelsize $\tbasis$};
            \draw[\probcolor] (8,-5)--(8,-3) node[midway,right] {\colorlabelsize $\catvariableof{0,1,0,0,2}$};

            \draw[\newmessagecolor,dashed, ->] (6.5,-1) to [bend right = 30] (6.5,-5);
            \node[\newmessagecolor,anchor=center] (A) at (5.25,-4) {\colorlabelsize $\messagesymbol^{(2,3)}$};

            \draw[\newmessagecolor,dashed, ->] (11,1.25) to [bend right = 30] (9,1.25);
            \node[\newmessagecolor,anchor=center] (A) at (9.75,2) {\colorlabelsize $\messagesymbol^{(2,0)}$};

            \draw[\newmessagecolor,dashed, ->] (5,1.5) to [bend right = -40] (6.8,1.1);
            \node[\newmessagecolor,anchor=center] (A) at (6,0.75) {\colorlabelsize $\messagesymbol^{(2,1)}$};

            \draw[\newmessagecolor,dashed, ->] (1,1.5) to [bend right = -40] (7,1.75);
            \node[\newmessagecolor,anchor=center] (A) at (3.5,2) {\colorlabelsize $\messagesymbol^{(2,2)}$};


            \draw (11,-1) rectangle (13,1);
            \node[anchor=center] (A) at (12,0) {\corelabelsize $\hypercoreof{3}$};
            \draw (12,-1)--(12,-2.5) node[midway,right] {\colorlabelsize $\catvariableof{0,1,0,1,2}$};

            \drawvariabledot{6}{4}
            \node[anchor=south] (text) at (6,4) {\colorlabelsize $\decvariableof{0,:,0,:,2}$};

            \draw (6,4) to[bend right= 20] (0,1);
            \draw (6,4) to[bend right= 10] (4,1);
            \draw (6,4) to[bend right= -10] (8,1);
            \draw (6,4) to[bend right= -20] (12,1);


            %% I 00::2 constraint (3 in the first row)
            \draw (3,-7) rectangle (5,-5);
            \node[anchor=center] (A) at (4,-6) {\corelabelsize $\hypercoreof{1}$};
            \drawvariabledot{4}{-3}
            \draw (4,-3) -- (3.25,-3);
            \draw (4,-5)--(4,-1);

            \draw (-1,-7) rectangle (1,-5);
            \node[anchor=center] (A) at (0,-6) {\corelabelsize $\hypercoreof{0}$};
            \drawvariabledot{0}{-3}
            \draw (0,-3) -- (-0.75,-3);
            \draw (0,-5)--(0,-1);

            \draw (-5,-7) rectangle (-3,-5);
            \node[anchor=center] (A) at (-4,-6) {\corelabelsize $\hypercoreof{2}$};
            \draw (-4,-5)--(-4,-3);

            \drawvariabledot{-4}{-3}
            \draw (-4,-3) -- (-4.75,-3);

            \draw[\concolor] (-5,-1) rectangle (-3,1);
            \node[anchor=center, \concolor] (A) at (-4,0) {\corelabelsize $\tbasis$};
            \draw[\concolor] (-4,-1)--(-4,-3) node[midway,right] {\colorlabelsize $\catvariableof{0,0,1,0,2}$};

            %% Messages from 3 in row 0,0
            \draw[\newmessagecolor,dashed, ->] (-5.5,-1) to [bend right = 30] (-5.5,-5);
            \node[\newmessagecolor,anchor=center] (A) at (-6.75,-2) {\colorlabelsize $\messagesymbol^{(1,0)}$};

            \draw[\newmessagecolor,dashed, ->] (-3,-7.25) to [bend right = 30] (-1,-7.25);
            \node[\newmessagecolor,anchor=center] (A) at (-2,-6.75) {\colorlabelsize $\messagesymbol^{(1,1)}$};

            \draw[\newmessagecolor,dashed, ->] (-3,-7.5) to [bend right = 40] (3,-7.25);
            \node[\newmessagecolor,anchor=center] (A) at (0,-9) {\colorlabelsize $\messagesymbol^{(1,2)}$};

            \draw[\newmessagecolor,dashed, <-] (-1.5,-1) to [bend right = 30] (-1.5,-5);
            \node[\newmessagecolor,anchor=center] (A) at (-2.75,-4) {\colorlabelsize $\messagesymbol^{(1,3)}$};

            \draw[\newmessagecolor,dashed, <-] (2.5,-1) to [bend right = 30] (2.5,-5);
            \node[\newmessagecolor,anchor=center] (A) at (1.25,-4) {\colorlabelsize $\messagesymbol^{(1,4)}$};

            \draw (-9,-7) rectangle (-7,-5);
            \node[anchor=center] (A) at (-8,-6) {\corelabelsize $\hypercoreof{3}$};
            \draw (-8,-5)--(-8,-3) node[midway,left] {\colorlabelsize $\catvariableof{0,0,1,1,2}$};

            \drawvariabledot{-2}{-10}
            \node[anchor=north] (text) at (-2,-10) {\colorlabelsize $\decvariableof{0,0,:,:,2}$};

            \draw (-2,-10) to[bend right= -20] (-8,-7);
            \draw (-2,-10) to[bend right= -10] (-4,-7);
            \draw (-2,-10) to[bend right= 10] (0,-7);
            \draw (-2,-10) to[bend right= 20] (4,-7);

            % I 0001: (position 0001)
            \draw (11,-7) rectangle (13,-5);
            \node[anchor=center] (A) at (12,-6) {\corelabelsize $\hypercoreof{2}$};
            \draw (12,-5)--(12,-2.5);

            \drawvariabledot{12}{-3}
            \draw (12,-3) -- (12.5,-3);

            \draw (15,-7) rectangle (17,-5);
            \node[anchor=center] (A) at (16,-6) {\corelabelsize $\hypercoreof{0}$};
            \draw (16,-5)--(16,-3) node[midway,right] {\colorlabelsize $\catvariableof{0,1,0,1,0}$};

            \draw (19,-7) rectangle (21,-5);
            \node[anchor=center] (A) at (20,-6) {\corelabelsize $\hypercoreof{1}$};
            \draw (20,-5)--(20,-3) node[midway,right] {\colorlabelsize $\catvariableof{0,1,0,1,1}$};

            \draw[\concolor] (19,-1) rectangle (21,1);
            \node[anchor=center, \concolor] (A) at (20,0) {\corelabelsize $\tbasis$};
            \draw[\concolor] (20,-1)--(20,-3) node[midway,right] {\colorlabelsize $\catvariableof{0,1,0,1,1}$};

            \drawvariabledot{20}{-3}
            \draw (20,-3) -- (20.5,-3);

            %% Messages from 2 at position 2,2,2,2
            \draw[\newmessagecolor,dashed, ->] (18.5,-1) to [bend right = 30] (18.5,-5);
            \node[\newmessagecolor,anchor=center] (A) at (17.25,-2) {\colorlabelsize $\messagesymbol^{(0,0)}$};

            \draw[\newmessagecolor,dashed, ->] (19,-7.5) to [bend right = -40] (13,-7.25);
            \node[\newmessagecolor,anchor=center] (A) at (16,-9) {\colorlabelsize $\messagesymbol^{(0,1)}$};

            \draw[\newmessagecolor,dashed, <-] (13.5,-1) to [bend right = -30] (13.5,-5);
            \node[\newmessagecolor,anchor=center] (A) at (14.9,-4) {\colorlabelsize $\messagesymbol^{(0,2)}$};

            \draw (23,-7) rectangle (25,-5);
            \node[anchor=center] (A) at (24,-6) {\corelabelsize $\hypercoreof{3}$};
            \draw (24,-5)--(24,-3) node[midway,right] {\colorlabelsize $\catvariableof{0,1,0,1,3}$};

            \drawvariabledot{18}{-10}
            \node[anchor=north] (text) at (18,-10) {\colorlabelsize $\decvariableof{0,1,0,1,:}$};

            \draw (18,-10) to[bend right= -20] (12,-7);
            \draw (18,-10) to[bend right= -10] (16,-7);
            \draw (18,-10) to[bend right= 10] (20,-7);
            \draw (18,-10) to[bend right= 20] (24,-7);

        \end{tikzpicture}
    \end{center}
    \caption{
        The tensor network decomposition of $3$ out of $4\cdot2^2=64$ rules in the $2^2\times2^2$ Sudoku knowledge base (see \exaref{exa:sudokuDecomposition}),  namely to the number $3$ appearing once in the $(0,0)$-square (top), the number $3$ appearing once in the $(0,0)$-row (bottom left) and a unique number appearing at the $(0,1,0,1)$-position (bottom right).
        The evidence of the number $3$ already being assigned at the position $(0,0,1,0)$ is sketched by a basis vector $\textcolor{\concolor}{\tbasis}$ on the left side, and the number $2$ assigned at position $(0,1,0,1)$ analogously on the right side.
        During Constraint Propagation \algoref{alg:constraintPropagation} on the hypergraph of Sudoku rules and evidence (see \exaref{exa:sudokuMessagePassing}), this evidence is in three epochs of messages propagated to the constraints by partial entailment steps and imply that $\textcolor{\probcolor}{\catvariableof{0,1,0,0,2}}$ is true, i.e. that at the position $(0,1,0,0)$ the number $3$ needs to be assigned.
        We depict the messages between the cores by dashed lines labeled by $\messagesymbol^{(0,[3])},\messagesymbol^{(1,[5])}$ and $\messagesymbol^{(2,[4])}$ and provide further interpretation in \exaref{exa:sudokuMessagePassing}.
    }\label{fig:contractionPropagationSudoku}
\end{figure}

