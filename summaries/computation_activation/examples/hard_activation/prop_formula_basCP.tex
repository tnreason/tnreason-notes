\begin{example}
    For the formula described in \exaref{exa:propFormulaCoordinatewise}, we have
    \begin{align*}
        \formulaat{\catvariableof{[3]}}
        &= \left(\tbasisat{\catvariableof{0}} \otimes \fbasisat{\catvariableof{1}} \otimes \fbasisat{\catvariableof{2}}\right)
        + (\fbasisat{\catvariableof{0}} \otimes \tbasisat{\catvariableof{1}} \otimes \fbasisat{\catvariableof{2}}) \\
        &\quad+ (\tbasisat{\catvariableof{0}} \otimes \tbasisat{\catvariableof{1}} \otimes \fbasisat{\catvariableof{2}}),
    \end{align*}
    where we denote the vectors $\tbasisat{Y} = [0,1]^T$ and $\fbasisat{Y} = [1,0]^T$.
    Then for the model $\catindexof{[3]} = (1,1,0)$ it holds
    \begin{align*}
        \formulaat{\indexedcatvariableof{[3]}}
        &= \left(\tbasisat{\catvariableof{0}=1} \otimes \fbasisat{\catvariableof{1}=1} \otimes \fbasisat{\catvariableof{2}=0}\right) \\
       &\quad + (\fbasisat{\catvariableof{0}=1} \otimes \tbasisat{\catvariableof{1}=1} \otimes \fbasisat{\catvariableof{2}=0}) \\
       &\quad + (\tbasisat{\catvariableof{0}=1} \otimes \tbasisat{\catvariableof{1}=1} \otimes \fbasisat{\catvariableof{2}=0}) \\
        &=  1\cdot 0 \cdot 1 + 0\cdot 1 \cdot 1 + 1 \cdot 1 \cdot 1 = 1.
    \end{align*}
\end{example}