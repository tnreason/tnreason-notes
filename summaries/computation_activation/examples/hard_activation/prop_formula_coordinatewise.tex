\begin{example}
    \label{exa:propFormulaCoordinatewise}
    Let there be $\catorder=3$ boolean variables $\catvariableof{[3]}$ and a propositional formula
    \begin{align*}
        \formulaat{\catvariableof{[3]}} = (\catvariableof{0} \lor \catvariableof{1}) \land \lnot \catvariableof{2} \, .
    \end{align*}
    In a graphical depiction and in the coordinatewise representation this formula can be represented as
    \begin{center}
        \begin{tikzpicture}[scale=1]

            \begin{scope}[shift={(-4,-0.2)}]
                \node[anchor=east] (A) at (-0.25,0.2) {$\formulaat{\catvariableof{[3]}}\,=$};
                \draw (0,0) rectangle (1.6,0.8);
                \node[anchor=center] (A) at (0.8,0.4) {$\formula$};
                \draw (0.2,0) -- (0.2,-0.6) node[midway,left] {\tiny $\catvariableof{0}$};
                \draw (0.8,0) -- (0.8,-0.6) node[midway,left] {\tiny $\catvariableof{1}$};
                \draw (1.4,0) -- (1.4,-0.6) node[midway,left] {\tiny $\catvariableof{2}$};
            \end{scope}

            \node[anchor=east] (A) at (-1.5,0) {$=$};
            \node (A) at (0,0) {
                $\begin{bmatrix}
                     0 & 1 \\
                     1 & 1
                \end{bmatrix}$
            };
            \node (A) at (1.25,0.3) {
                $\begin{bmatrix}
                     0 & 0 \\
                     0 & 0
                \end{bmatrix}$
            };
            \draw[<-,dashed] (-0.9,-0.275) node[right] {\tiny $1$} -- (-0.9,0.275) node [midway, left] {\tiny $\catvariableof{0}$} node[right] {\tiny $0$};
            \draw[->,dashed] (-0.3,0.85) node[below] {\tiny $0$} -- (0.3,0.85) node [midway, above] {\tiny $\catvariableof{1}$} node[below] {\tiny $1$};
            \draw[->,dashed] (0,-0.85) node[above] {\tiny $0$} -- (1.25,-0.55) node [midway, below] {\tiny $\catvariableof{2}$} node[above] {\tiny $1$};

            \node[anchor=east] (A) at (2.25,-0.8) {$\cdot$};
        \end{tikzpicture}
    \end{center}
    In the state set $\atomstates = \{0,1\}\times \{0,1\} \times \{0,1\}$ we have three models of the formula by the positions of the non-zero entries in the tensor, i.e. $\formulaat{\indexedcatvariableof{[3]}}=1$ if and only if
    \begin{align*}
        \catindexof{[3]}\in\{(1,0,0),(0,1,0),(1,1,0)\} \, .
    \end{align*}
    The formula $\formula$ is therefore satisfiable.
\end{example}