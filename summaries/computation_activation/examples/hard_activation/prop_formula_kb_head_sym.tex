\begin{example}[Computation Network Symmetries]
    %see the notebook: \url{https://colab.research.google.com/drive/1p2wp61fFMu0otnfFhKoNsLiCNfWpuEsn?usp=sharing}
    For the propositional formula $\formulaat{\catvariableof{[3]}}={(\catvariableof{0} \lor \catvariableof{1}) \land \lnot \catvariableof{2}}$ (see \exaref{exa:propFormulaCoordinatewise}), we can write the formula in terms of a \ComputationActivationNetwork{} with activation tensor $\tbasis$ and computation network decomposed by the basis encodings. First, it is written with one activation vector. Second, we see that it can also be interpreted with multiple features.
    \begin{center}
        \begin{tikzpicture}[scale=0.4, yscale=-1, thick] % , baseline = -3.5pt

            \draw[] (-2,1)--(-2,-1) node[midway,left] {\colorlabelsize $\catvariableof{0}$};
            \draw[] (0.5,1)--(0.5,-1) node[midway,right] {\colorlabelsize $\catvariableof{1}$};
            \draw[] (3,1)--(3,-1) node[midway,right] {\colorlabelsize $\catvariableof{2}$};
            \draw (-3,-1) rectangle (4, -3);
            \node[anchor=center] (text) at (0.5,-2) {\corelabelsize $(\catvariableof{0} \lor \catvariableof{1}) \land \lnot \catvariableof{2}$};
            %\draw[->-] (1.5,-3)--(1.5,-5) node[midway,right] {\colorlabelsize $\headvariableof{a \land b \land \lnot c}$};

            \node[anchor=center] (text) at (5,-2) {${=}$};


            \begin{scope}
            [shift={(7,0)}]

                \draw[->-] (0,1)--(0,-1) node[midway,left] {\colorlabelsize $\catvariableof{0}$};
                \draw[->-] (3,1)--(3,-1) node[midway,right] {\colorlabelsize $\catvariableof{1}$};
                \draw[->-] (6,1)--(6,-1) node[midway,right] {\colorlabelsize $\catvariableof{2}$};

                \draw (-1,-1) rectangle (4, -3);
                \node[anchor=center] (text) at (1.5,-2) {\corelabelsize $\bencodingof{\lor}$};

                \draw[->-] (1.5,-3) --(1.5,-5) node[midway,right]{\colorlabelsize $\headvariableof{0 \lor 1}$};

                \draw (5,-1) rectangle (7, -3);
                \node[anchor=center] (text) at (6,-2) {\corelabelsize $\bencodingof{\lnot}$};

                \draw[->-] (6,-3) --(6,-5) node[midway,right]{\colorlabelsize $\headvariableof{\lnot 2}$};

                \draw (0.5,-5) rectangle (6.5,-7);
                \node[anchor=center] (text) at (3.5,-6) {\corelabelsize $\bencodingof{\land}$};

                \draw[->-] (4,-7) -- (4,-8.5) node[right] {\colorlabelsize $\headvariableof{(0 \lor 1) \land \lnot 2}$};
                \drawvariabledot{4}{-8}
                \draw[] (4,-8) -- (4,-9);
                \draw (3,-9) rectangle (5,-11);
                \node[anchor=center] (text) at (4,-10) {$\tbasis$};

            \end{scope}

            \node[anchor=center] (text) at (15,-2) {${=}$};

            \begin{scope}
            [shift={(17,0)}]

                \draw[->-] (0,1)--(0,-1) node[midway,left] {\colorlabelsize $\catvariableof{0}$};
                \draw[->-] (3,1)--(3,-1) node[midway,right] {\colorlabelsize $\catvariableof{1}$};
                \draw[] (7,1)--(7,-1) node[midway,right] {\colorlabelsize $\catvariableof{2}$};

                \draw (-1,-1) rectangle (4, -3);
                \node[anchor=center] (text) at (1.5,-2) {\corelabelsize $\bencodingof{\lor}$};

                \draw (1.5,-4.5) -- (1.5,-5);
                \draw[->-] (1.5,-3) --(1.5,-4.5) node[midway,right]{\colorlabelsize $\headvariableof{0 \lor 1}$};

                \drawvariabledot{1.5}{-4}
                \draw (0.5,-5) rectangle (2.5,-7);
                \node[anchor=center] (text) at (1.5,-6) {\corelabelsize $\tbasis$};

                \node[anchor=center] (text) at (5,-2) {$\otimes$};


                \draw (6,-1) rectangle (8, -3);
                \node[anchor=center] (text) at (7,-2) {\corelabelsize $\fbasis$};

                %\draw[->-] (6,-3) --(6,-5) node[midway,right]{\colorlabelsize $\headvariableof{\lnot c}$};


                %\draw (0.5,-5) rectangle (6.5,-7);
                %\node[anchor=center] (text) at (3.5,-6) {\corelabelsize $\bencodingof{\land}$};

                %\draw[->-] (4,-7) -- (4,-8.5) node[right] {\colorlabelsize $\headvariableof{(a \lor b) \land \lnot c}$};
                %\drawvariabledot{4}{-8}
                %\draw[] (4,-8) -- (4,-9);
                %\draw (3,-9) rectangle (5,-11);
                %\node[anchor=center] (text) at (4,-10) {$\tbasis$};

            \end{scope}

        \end{tikzpicture}
    \end{center}
\end{example}
