\begin{example}[Exponential Family of Coin Tosses]
    \label{exa:coinTossExp}
    Let us recall the family of distributions of boolean $\shortcatvariables$ from \exaref{exa:coinTossFN}, which has the order statistic $\sstatof{+}$ as a sufficient statistic.
    We now in addition assume, that the variables $\shortcatvariables$ are i.i.d. with respect to any member of the family (see \exaref{exa:coinTossFN}).
    For the variables to be i.i.d. we need $\basemeasurewith=\oneswith$ and can thus choose a representation such that for $\thirdcatindex\in\valof{\thirdcatvariable}$
    \begin{align*}
        \condprobat{\shortcatvariables}{\indexedthirdcatvariable}
        = \contractionof{\bencodingofat{\sstatof{+}}{\headvariableof{+},\shortcatvariables},\acttensorat{\headvariableof{+},\indexedthirdcatvariable}}{\shortcatvariables}
    \end{align*}
    where for each $\catenumerator\in[\catorder+1]$
    % The family of distributions (see \exaref{exa:coinTossFN}), such that the variables $\shortcatvariables$ are i.i.d. with respect to each other (see \exaref{exa:coinTossHC}) are the special case, where $\basemeasurewith=\onesat{\shortcatvariables}$ and the activation tensors are %and the family is labeled by $\theta\in[0,1]$ such that for $\theta\in(0,1)$ and $\catenumerator\in[\catorder+1]$
    \begin{align*}
        \acttensorat{\headvariableof{+}=\catenumerator,\indexedthirdcatvariable}
        = (1-\thirdcatindex)^{\catorder-\catenumerator} \cdot \thirdcatindex^{\catenumerator} \, .
    \end{align*}
    The marginal distribution $\condprobof{\headvariableof{+}}{\indexedthirdcatvariable}$ is then the binominal distribution $B(\catorder,\thirdcatindex)$.
    When excluding the case of $\thirdcatindex\in\{0,1\}$, the family is a subset of the exponential family with the head count statistic, where each member is reparametrized by
    \begin{align*}
        \canparam \coloneqq \lnof{\frac{\thirdcatindex}{1-\thirdcatindex}} \, .
    \end{align*}
    To see that this is true, we notice that the coordinate $\headindexof{+}\in[\catorder+1]$ of the activation tensor of $\expdistof{\sstatof{+},\canparam,\trivbm}$ is
    \begin{align*}
        \softacttensorat{\indexedheadvariableof{+}}
        = \expof{\headindexof{+}\cdot\canparam}
        = \frac{\thirdcatindex^{\headindexof{+}}}{(1-\thirdcatindex)^{\headindexof{+}}} \, .
    \end{align*}
    Now with $\partitionfunctionof{\canparam}=\frac{1}{\thirdcatindex^{\catorder}}$ we have for any $\shortcatindices$ with $\sum_{\catenumeratorin}\catindexof{\catenumerator}=\headindexof{+}$ that
    \begin{align*}
        \frac{1}{\partitionfunctionof{\canparam}} \cdot \contraction{\bencodingofat{\sstatof{+}}{\headvariableof{+},\indexedshortcatvariables},\softacttensorat{\headvariableof{+}}}
        = \thirdcatindex^{\catorder} \cdot \frac{\thirdcatindex^{\headindexof{+}}}{(1-\thirdcatindex)^{\headindexof{+}}}
        =  \thirdcatindex^{\headindexof{+}} \cdot (1-\thirdcatindex)^{\catorder-\headindexof{+}}  \, .
    \end{align*}
    Comparing with the activation tensor $\acttensorat{\headvariableof{+}}$ above, we notice that $\partitionfunctionof{\canparam}$ is indeed the partition function of the exponential family and $\expdistofat{\sstatof{+},\canparam,\trivbm}{\shortcatvariables}$ coincides with the member $\condprobat{\shortcatvariables}{\indexedthirdcatvariable}$.
\end{example}