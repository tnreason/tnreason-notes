\begin{example}[Exponential Family of Coin Tosses]
    \label{exa:coinTossExp}
    The family of distributions (see \exaref{exa:coinToss}), such that the variables $\shortcatvariables$ are i.i.d. with respect to each other (see \exaref{exa:ctHc}) are the special case, where $\basemeasurewith=\onesat{\shortcatvariables}$ and the activation tensors are %and the family is labeled by $\theta\in[0,1]$ such that for $\theta\in(0,1)$ and $\catenumerator\in[\catorder+1]$
    \begin{align*}
        \acttensorat{\headvariableof{+}=\catenumerator,\indexedthirdcatvariable}
        = (1-\thirdcatindex)^{\catorder-\catenumerator} \cdot \thirdcatindex^{\catenumerator} \, .
    \end{align*}
%    and for $\theta\in\{0,1\}$
%    \begin{align*}
%        \acttensorofat{\theta}{\headvariableof{+}}
%        = \begin{cases}
%              \onehotmapofat{0}{\headvariableof{+}} & \ifspace \theta=0 \\
%              \onehotmapofat{\catorder}{\headvariableof{+}} & \ifspace \theta=1
%        \end{cases}\, .
%    \end{align*}
    The marginal distribution $\condprobof{\headvariableof{+}}{\indexedthirdcatvariable}$ is then the binominal distribution $B(\catorder,\theta)$.
    When excluding the case of $\thirdcatindex\in\{0,1\}$, this is the exponential family with the head count statistic, when reparametrizing
    \begin{align*}
        \canparam = \lnof{\frac{\thirdcatindex}{1-\thirdcatindex}} \, .
    \end{align*}
\end{example}