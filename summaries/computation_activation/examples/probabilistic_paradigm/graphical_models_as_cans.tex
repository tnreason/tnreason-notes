\begin{example}[Graphical Models as a special case of \CompActNets{}]
%    Recall \defref{def:compactnets}.
%    Given a statistic $\sstat:\facstates \to \parspace$ and a hypergraph $\graph=(\nodes,\edges)$ on the image coordinates $\headvariables$, any by $\sstat$ computable and by $\graph$ activated \CompActNets{} has the form
%    \begin{align*}
%        \probwith
%        = \normalizationof{\acttensorwith,\bencsstatwith}{\shortcatvariables}
%    \end{align*}
%    where $\acttensorwith$ is an arbitrary non-negative tensor.
    For graphical models we take the \emph{identity statistic}
    \begin{align*}
        \identity\big(\shortcatindices\big)
        = \shortcatindices \, ,
    \end{align*}
    so that the image coordinates coincide with the variables and there are no non-trivial computation cores.
    The associated basis encoding is just the identity tensor
    \begin{align*}
        \bencodingofat{\identity}{\headvariableof{[\catorder]},\shortcatvariables}
        = \identityat{\shortcatvariables,\headvariableof{[\catorder]}} \, .
    \end{align*}
    and therefore, for any activation tensor $\acttensorwith$ we obtain
    \begin{align*}
        \probwith
        &= \normalizationof{\acttensorwith,\bencodingofat{\identity}{\headvariableof{[\catorder]},\shortcatvariables}}{\shortcatvariables}
        & = \normalizationof{\acttensorat{\shortcatvariables}}{\shortcatvariables}
    \end{align*}
    In other words, in the graphical–model case the activation tensor coincides with the joint distribution tensor.
    In this setting, structural properties of the distribution such as (conditional) independences can be read off as algebraic factorization patterns of the activation (and hence joint) tensor.
\end{example}