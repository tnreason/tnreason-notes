\begin{example}[Family of independent coin tosses]
    Consider tossing a coin with head probability $\thirdcatindex\in[0,1]$ and repeating the experiment independently $\catorder\in\nn$ times.
    We define a variable $\thirdcatvariable$ taking values in $\mathrm{val}(\thirdcatvariable)=[0,1]$ and denote by $\shortcatvariables$ $\catorder$ boolean variables.
    Then, the family of coin toss distributions is the tensor $\condprobat{\shortcatvariables}{\thirdcatvariable}$ with coordinates $\shortcatindices\in\atomstates$ and $\thirdcatindex\in[0,1]$ defined by
    \begin{align*}
        \condprobat{\indexedshortcatvariables}{\thirdcatvariable=\thirdcatindex}
        = \prod_{\catenumeratorin} \thirdcatindex^{\catindexof{\catenumerator}} (1-\thirdcatindex)^{1-\catindexof{\catenumerator}}
        = \thirdcatindex^{\sum_{\catenumeratorin} \catindexof{\catenumerator}} (1-\thirdcatindex)^{\catorder - \sum_{\catenumeratorin} \catindexof{\catenumerator}} \, .
    \end{align*}
    Note that for each slice with respect to $\thirdcatindex\in[0,1]$ by the binomial theorem we have $\contraction{\probat{\shortcatvariables,\thirdcatvariable=\thirdcatindex}}=1$ and thus $\probat{\shortcatvariables,\thirdcatvariable}$ is indeed a family of probability distributions.
    For $\catorder=2$ we have more explicitly for any $\thirdcatindex\in[0,1]$ that
    \begin{center}
        \begin{tikzpicture}[scale=1]
            \node[anchor=east] (A) at (-2,0) {$\condprobat{\catvariableof{[2]}}{\thirdcatvariable=\thirdcatindex}\,=$};

            \node (A) at (1,0) {
                $\begin{bmatrix}
                (1-\thirdcatindex)
                     ^2 & \thirdcatindex \cdot (1-\thirdcatindex) \\
                     \thirdcatindex \cdot (1-\thirdcatindex) & \thirdcatindex^2
                \end{bmatrix}$
            };
            \draw[<-,dashed] (-1.3,-0.275) node[right] {\tiny $1$} -- (-1.3,0.275) node [midway, left] {\tiny $\catvariableof{0}$} node[right] {\tiny $0$};
            \draw[->,dashed] (0,0.85) node[below] {\tiny $0$} -- (2,0.85) node [midway, above] {\tiny $\catvariableof{1}$} node[below] {\tiny $1$};
            \node[anchor=east] (A) at (3.5,-0.55) {$\cdot$};
        \end{tikzpicture}
    \end{center}
\end{example}