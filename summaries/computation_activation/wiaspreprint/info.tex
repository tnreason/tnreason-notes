% \title*{Neural and tensor networks for high-dimensional parametric PDEs and sampling}
% \titlerunning{NNs \& TTs for pPDEs and sampling}
% % Use \titlerunning{Short Title} for an abbreviated version of
% % your contribution title if the original one is too long
% \author{Martin Eigel, Lars Grasedyck, Thong Le, Janina Schütte}
% % Use \authorrunning{Short Title} for an abbreviated version of
% % your contribution title if the original one is too long
% \institute{M. Eigel, J. Schütte \at WIAS Berlin, Mohrenstraße 39, 10117 Berlin, \\\email{eigel@wias-berlin.de}, \url{schuette@wias-berlin.de}
% \and L. Grasedyck, T. Le \at RWTH Aachen, Templergraben 55, 52062 Aachen \\\email{lgr@igpm.rwth-aachen.de}, \url{le@igpm.rwth-aachen.de}}


\newcommand{\ownTitle}{A tensor network formalism for neuro-symbolic AI}

\newcommand{\ownShortTitle}{A tensor network formalism for neuro-symbolic AI}

\newcommand{\ownThanks}{AG and ME acknowledge funding from the German Federal Ministry of Education and Research (BMBF), grant number FKZ 13N17160, "Verbundprojekt: Quantum Read-Once-Memory - Verwandlung von klassischen Daten zu Quantenzuständen - Teilvorhaben: Quantenschaltkreis-Optimierung von Quantenzuständen durch Tensor-Netzwerke (QOQ-tn)".
JS and ME acknowledge funding from the Deutsche Forschungsgemeinschaft (DFG, German Research Foundation) under Germany´s
Excellence Strategy – The Berlin Mathematics Research Center MATH+ (EXC-2046/2, project ID: 390685689) as Project PaA-7}

\newcommand{\ownKeywords}{Neuro-symbolic AI, tensor networks}


% 65N21, 62F15, 65N75, 65C30, 90C56

\newcommand{\ownAMS}{
  % 62F15,   	%Bayesian inference
  % 65N75,   	%Probabilistic methods, particle methods, etc. for boundary value problems involving PDEs
  % 65C30,   	%Numerical solutions to stochastic differential and integral equations
  %35Q84,   	%Fokker-Planck equations
  %65K10, %Numerical optimization and variational techniques
  % 35R60,  %	PDEs with randomness, stochastic partial differential equations
  % 60H35,  %Computational methods for stochastic equations (aspects of stochastic analysis)
  %65C20,
  %65N12, %Stability and convergence of numerical methods for boundary value problems involving PDEs
  % 65N22, % 	Numerical solution of discretized equations for boundary value problems involving PDEs
  % 65J10, %Numerical solutions to equations with linear operators
  %74B05, %Classical linear elasticity
  %97N50%Interpolation and approximation}
  % 62H12, %Estimation in multivariate analysis
  % 65C05, %Monte Carlo methods 
  % 60H35, %Computational methods for stochastic equations
  % 68T07%Artificial neural networks and deep learning
   68T27,   %	Logic in artificial intelligence
   68T30,   %	Knowledge representation
   68T37,   %	Reasoning under uncertainty in the context of artificial intelligence
   15A69,   %	Multilinear algebra, tensor calculus
   65F99%	None of the above, but in this section: Numerical linear algebra
}
% Used by Reich/Nüsken:
%AMS subject classifications. 65N21, 62F15, 65N75, 65C30, 90C56
% Used by Ding/Li
% AMS subject classifications.35Q84, 35Qxx, 62Dxx, 35R30

