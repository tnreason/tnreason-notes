\section{Implementation of the algorithms and examples}\label{sec:algExaImplementation}

\setminted{
    fontsize=\fontsize{8.5}{9}\selectfont,
    linenos,
    breaklines
}


The implementations of the algorithms and concepts are available at \href{https://github.com/tnreason/tnreason-py/tree/version2/demonstrations/comp_act_nets} and implemented with \tnreason{} in the version \curvertnreason{}.

\subsection{Algorithm~\ref{alg:beliefPropagation} and \ref{alg:constraintPropagation} (Belief and Constraint Propagation)}

\inputminted{python}{../../../../tnreason-py/demonstrations/comp_act_nets/algorithms/propagation.py}

\subsection{Algorithm~\ref{alg:AMM_HLN} (Alternating Moment Matching)}

\inputminted{python}{../../../../tnreason-py/demonstrations/comp_act_nets/algorithms/moment_matching.py}

\subsection{Example~\ref{exa:marySum} (Integer Summation in $\catdim$-ary Representation)}

Following the decomposition of summations in $\catdim$-ary into local summations, the function \mintinline{python}{get_sum_tn} produces a corresponding tensor network of basis encodings.
We test by coordinate retrieval operations, whether the summation is performed correctly.

\inputminted{python}{../../../../tnreason-py/demonstrations/comp_act_nets/mary_sum_example.py}

\subsection{Example~\ref{exa:studentHC} and \ref{exa:studentBP} (Student Markov Network)}

We here implement the Markov Network on the hypergraph of \exaref{exa:studentHC}, with tensors having independent random coordinates drawn from the uniform distribution on $[0,1]$.
We test in a final \mintinline{python}{assert} statement, whether the messages resulting from \algoref{alg:beliefPropagation} in a tree implementation contract to the marginal distribution, which we directly compute for comparison.

implications of \theref{the:treeBeliefPropagationExactness} hold

\inputminted{python}{../../../../tnreason-py/demonstrations/comp_act_nets/student_example.py}

\subsection{Example~\ref{exa:sudokuEntailment}, \ref{exa:sudokuDecomposition} and \ref{exa:sudokuMessagePassing} (Sudoku Game)}

We implement the $2^2\times2^2$ Sudoku with the start assignment given in \exaref{exa:sudokuEntailment} and apply the Constraint Propagation \algoref{alg:constraintPropagation} to deduce the full assignment.
We then test whether the correct board assignment (given in \exaref{exa:sudokuMessagePassing}) has been found.

\inputminted{python}{../../../../tnreason-py/demonstrations/comp_act_nets/sudoku_example.py}

\subsection{Example~\ref{exa:hlnAccountingRep} and \ref{exa:hlnAccountingAMM} (Toy Accounting Model)}

\inputminted{python}{../../../../tnreason-py/demonstrations/comp_act_nets/accounting_example.py}

