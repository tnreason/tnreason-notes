\section{Decompositions based on Propositional Syntax}\label{sec:log_rep}

A tensor-based representation of propositional logic is developed by encoding boolean variables into vectors, defining formulas as boolean tensors, and showing how logical connectives and normal forms can be expressed as tensor contractions.

\subsection{Propositional Semantics by Boolean Tensors}

%We define
%\begin{itemize}
%    \item a \emph{propositional formula} as a boolean-valued tensor
%    \begin{align*}
%        \formulaat{\shortcatvariables} \defcols \atomstates \rightarrow \ozset \subset \rr \, ,
%    \end{align*}
%    \item a \emph{model} of a propositional formula as a state $\shortcatindices \in \atomstates$, which fulfills
%    $% \begin{align*}
%    \formulaat{\indexedshortcatvariables}=1 \, ,
%    $% \end{align*}
%    where we associate $\text{True}\leftrightarrow 1$ and $\text{False}\leftrightarrow 0$,
%    \item and the propositional formula to be \emph{satisfiable}, if there is a model.
%\end{itemize}

\begin{definition}
    \label{def:formulas}
    A \emph{propositional formula} $\formulaat{\shortcatvariables}$ depending on $\atomorder$ boolean variables $\catvariableof{\atomenumerator}$ is a boolean-valued tensor
    \begin{align*}
        \formulaat{\shortcatvariables} \defcols \atomstates \rightarrow \ozset \subset \rr \, .
    \end{align*}
    We call a state $\shortcatindices \in \atomstates$ a \emph{model} of a propositional formula $\formula$, if
    \begin{align*}
        \formulaat{\indexedshortcatvariables}=1 \, ,
    \end{align*}
    where we associate $\mathrm{True}\leftrightarrow 1$ and $\mathrm{False}\leftrightarrow 0$.
    If there is a model to a propositional formula, we say the formula is \emph{satisfiable}.
\end{definition}

\begin{example}
    \label{exa:propFormulaCoordinatewise}
    Let there be $\catorder=3$ boolean variables $\catvariableof{[3]}$ and a propositional formula
    \begin{align*}
        \formulaat{\catvariableof{[3]}} = (\catvariableof{0} \lor \catvariableof{1}) \land \lnot \catvariableof{2} \, .
    \end{align*}
    In a graphical depiction and in the coordinatewise representation this formula can be represented as
    \begin{center}
        \begin{tikzpicture}[scale=1]

            \begin{scope}[shift={(-4,-0.2)}]
                \node[anchor=east] (A) at (-0.25,0.2) {$\formulaat{\catvariableof{[3]}}\,=$};
                \draw (0,0) rectangle (1.6,0.8);
                \node[anchor=center] (A) at (0.8,0.4) {$\formula$};
                \draw (0.2,0) -- (0.2,-0.6) node[midway,left] {\tiny $\catvariableof{0}$};
                \draw (0.8,0) -- (0.8,-0.6) node[midway,left] {\tiny $\catvariableof{1}$};
                \draw (1.4,0) -- (1.4,-0.6) node[midway,left] {\tiny $\catvariableof{2}$};
            \end{scope}

            \node[anchor=east] (A) at (-1.5,0) {$=$};
            \node (A) at (0,0) {
                $\begin{bmatrix}
                     0 & 1 \\
                     1 & 1
                \end{bmatrix}$
            };
            \node (A) at (1.25,0.3) {
                $\begin{bmatrix}
                     0 & 0 \\
                     0 & 0
                \end{bmatrix}$
            };
            \draw[<-,dashed] (-0.9,-0.275) node[right] {\tiny $1$} -- (-0.9,0.275) node [midway, left] {\tiny $\catvariableof{0}$} node[right] {\tiny $0$};
            \draw[->,dashed] (-0.3,0.85) node[below] {\tiny $0$} -- (0.3,0.85) node [midway, above] {\tiny $\catvariableof{1}$} node[below] {\tiny $1$};
            \draw[->,dashed] (0,-0.85) node[above] {\tiny $0$} -- (1.25,-0.55) node [midway, below] {\tiny $\catvariableof{2}$} node[above] {\tiny $1$};

            \node[anchor=east] (A) at (2.25,-0.8) {$\cdot$};
        \end{tikzpicture}
    \end{center}
    In the state set $\atomstates = \{0,1\}\times \{0,1\} \times \{0,1\}$ we have three models of the formula by the positions of the non-zero entries in the tensor, i.e. $\formulaat{\indexedcatvariableof{[3]}}=1$ if and only if
    \begin{align*}
        \catindexof{[3]}\in\{(1,0,0),(0,1,0),(1,1,0)\} \, .
    \end{align*}
    The formula $\formula$ is therefore satisfiable.
\end{example}

%\begin{example}
%    \label{ex:propform}
%    The propositional formula defined for $d=3$ and $x_{[3]} = (\catvariableof{0},\catvariableof{1},\catvariableof{2})\in \atomstates = \{0,1\}\times \{0,1\} \times \{0,1\}$ by
%    \begin{align*}
%        \formulaat{\shortcatvariables = x_{[3]}} = \catvariableof{0} \wedge (\catvariableof{1} \vee \catvariableof{2})
%    \end{align*}
%    is satisfiable, since $\formulaat{\shortcatvariables = (1,1,1)} = 1$, $\formulaat{\shortcatvariables = (1,0,1)} = 1$, and $\formulaat{\shortcatvariables = (1,1,0)} = 1$ and therefore $x=(1,1,1)$, $x=(1,0,1)$, and $x=(1,1,0)$ are models of $\formulaat{\shortcatvariables}$.
%\end{example}

\paragraph{CP decomposition}
Since the tensor $\formulaat{\shortcatvariables}$ is equal to one at index $x_{[d]}$ if and only if $x_{[d]}$ is a model of $\formula$, i.e. fulfills the formula, a propositional formula can be written as the sum over the one-hot encodings of its models.
\begin{center}
    \begin{tikzpicture}[scale=0.35, thick]

    \draw (-1,-1) rectangle (5,-3);
    \node[anchor=center] (text) at (2,-2) {\corelabelsize ${\exformula}$};
    \draw[] (0,-3)--(0,-5) node[midway,left] {\colorlabelsize $\catvariableof{0}$};
    \draw[] (1.5,-3)--(1.5,-5) node[midway,left] {\colorlabelsize $\catvariableof{1}$};
    \node[anchor=center] (text) at (3,-4) {$\cdots$};
    \draw[] (4,-3)--(4,-5) node[midway,right] {\colorlabelsize $\catvariableof{\atomorder\shortminus1}$};


    \node[anchor=center] (text) at (7,-2) {${=}$};

    \node[anchor=center] (text) at (12,-2.5) {${\sum\limits_{\atomindices\in\atomstates}}$};
    \node[anchor=center] (text) at (12,-4) {\colorlabelsize $\exformula(\atomindices)=1$};

    \begin{scope}
        [shift={(19.5,1)}]

        \draw (-3,-2) rectangle (-1,-4);
        \node[anchor=center] (text) at (-2,-3) {\corelabelsize $\onehotmapof{\atomlegindexof{0}}$};
        \draw[->-] (-2,-4)--(-2,-6) node[midway,right] {\colorlabelsize $\catvariableof{0}$};

        \node[anchor=center] (text) at (1,-3) {\corelabelsize $\cdots$};

        \draw (3,-2) rectangle (5,-4);
        \node[anchor=center] (text) at (4,-3) {\corelabelsize $\onehotmapof{\atomlegindexof{\atomorder\shortminus1}}$};
        \draw[->-] (4,-4)--(4,-6) node[midway,right] {\colorlabelsize $\catvariableof{\atomorder\shortminus1}$};

    \end{scope}

\end{tikzpicture}
\end{center}
This decomposition corresponds to the CP decomposition of a tensor.

\begin{example}\label{exa:propFormulaBasCP}
    For the formula described in \exaref{exa:propFormulaCoordinatewise}, we have
    \begin{align*}
        \formulaat{\catvariableof{[3]}}
        &= \left(\tbasisat{\catvariableof{0}} \otimes \fbasisat{\catvariableof{1}} \otimes \fbasisat{\catvariableof{2}}\right)
        + (\fbasisat{\catvariableof{0}} \otimes \tbasisat{\catvariableof{1}} \otimes \fbasisat{\catvariableof{2}}) \\
        &\quad+ (\tbasisat{\catvariableof{0}} \otimes \tbasisat{\catvariableof{1}} \otimes \fbasisat{\catvariableof{2}}) \, .
    \end{align*}
    Note that we have $\contraction{\formulawith}=3$ and we can interpret this sum as a $\cpformat$ decomposition of $\formula$ with rank $3$.
%    where we denote the vectors $\tbasisat{Y} = [0,1]^T$ and $\fbasisat{Y} = [1,0]^T$.
    We use the decomposition to evaluate the formula $\formula$ at $\catindexof{[3]} = (1,1,0)$ and get
    \begin{align*}
        \formulaat{\indexedcatvariableof{[3]}}
        &= \left(\tbasisat{\catvariableof{0}=1} \otimes \fbasisat{\catvariableof{1}=1} \otimes \fbasisat{\catvariableof{2}=0}\right) \\
       &\quad + (\fbasisat{\catvariableof{0}=1} \otimes \tbasisat{\catvariableof{1}=1} \otimes \fbasisat{\catvariableof{2}=0}) \\
       &\quad + (\tbasisat{\catvariableof{0}=1} \otimes \tbasisat{\catvariableof{1}=1} \otimes \fbasisat{\catvariableof{2}=0}) \\
        &=  1\cdot 0 \cdot 1 + 0\cdot 1 \cdot 1 + 1 \cdot 1 \cdot 1 = 1 \, ,
    \end{align*}
    which verifies that $\catindexof{[3]} = (1,1,0)$ is a model of the formula $\formula$.
\end{example}

\paragraph{Model counts by contraction}
Each coordiante of the propositional formula is either a $1$ or $0$ encoding if the indexed state is a model of the formula or not.
In this way, the contraction $\contraction{\formula}$ counts the number of models of the propositional formula $\formula$.
One can therefore decide the satisfiability of a formula by checking if $\contraction{\formula}>0$.

\paragraph{Basis encoding}
Representing booleans by elements in $\{0,1\}$ leads to the problem, that negation is an affine transformation and can not be represented by multilinear tensors. %~\cite[Section 4.1.1]{goessmann_tensor-network_2025}.
Therefore, instead of using this \emph{coordinate calculus} an approach based on \emph{basis calculus} is employed, which is explained in this section.
To be able to express different kinds of connectives and finally any propositional formula by multi-linear tensors, booleans are encoded by one-hot encodings as defined in \defref{def:onehotenc}.
Propositional formulas $\formula$ can be expressed in terms of a tensor describing the mapping and its negation by
\begin{align}
    \label{eq:basisencboolean}
    \bencodingofat{\formula}{\indexedheadvariableof{\formula},\indexedshortcatvariables}
    = \begin{cases}
          1 & \ifspace \formulaat{\indexedshortcatvariables} = \headindexof{\formula}\\
          0 & \text{else}
    \end{cases}.
\end{align}
This basis encoding $\bencodingofat{\formula}{\headvariableof{\formula},\shortcatvariables} \in \{0,1\}^{2\times 2^d}$ then has the form
\begin{align}
    \label{eq:basisencnegsum}
    \bencodingofat{\formula}{\headvariableof{\formula},\shortcatvariables}
    = \tbasisat{\headvariableof{\formula}} \otimes \formulaat{\shortcatvariables}
    + \fbasisat{\headvariableof{\formula}} \otimes \lnot\formulaat{\shortcatvariables} \, .
\end{align}
In our graphical notation this property is visualized by
\begin{center}
    \begin{tikzpicture}[scale=0.35, thick] % , baseline = -3.5pt

    \draw[->-] (2,-1)--(2,1) node[midway,right] {\colorlabelsize $\formulavar$};
    \draw (-1,-1) rectangle (5,-3);
    \node[anchor=center] (text) at (2,-2) {\corelabelsize $\bencodingof{\exformula}$};
    \draw[-<-] (0,-3)--(0,-5) node[midway,left] {\colorlabelsize $\catvariableof{0}$};
    \draw[-<-] (1.5,-3)--(1.5,-5) node[midway,left] {\colorlabelsize $\catvariableof{1}$};
    \node[anchor=center] (text) at (3,-4) {$\cdots$};
    \draw[-<-] (4,-3)--(4,-5) node[midway,right] {\colorlabelsize $\catvariableof{\atomorder\shortminus1}$};


    \node[anchor=center] (text) at (7,-2) {${=}$};

    \node[anchor=center] (text) at (10,-2.5) {${\sum\limits_{\shortcatindices\in\atomstates}}$};

    \begin{scope}
    [shift={(15.5,-0.5)}]

        \draw (-2,1) rectangle (4,-1);
        \node[anchor=center] (text) at (1,0) {\corelabelsize $\onehotmapof{\exformulaat{\indexedshortcatvariables}}$};
        \draw[->-] (1,1)--(1,2.5) node[midway,right] {\colorlabelsize $\formulavar$};

        \draw (-2,-2) rectangle (4,-4);
        \node[anchor=center] (text) at (1,-3) {\corelabelsize $\onehotmapof{\shortcatindices}$};

        \draw[->-] (-1.5,-4)--(-1.5,-5.5) node[midway,left] {\colorlabelsize $\catvariableof{0}$};
        \draw[->-] (0.5,-4)--(0.5,-5.5) node[midway,left] {\colorlabelsize $\catvariableof{1}$};
        \node[anchor=center] (text) at (2,-5) {$\cdots$};
        \draw[->-] (3.5,-4)--(3.5,-5.5) node[midway,right] {\colorlabelsize $\catvariableof{\atomorder\shortminus1}$};

    \end{scope}

    \node[anchor=center] (text) at (21.25,-2) {${=}$};

    \begin{scope} [shift={(25.5,-0.5)}]

        \draw (-2,1) rectangle (4,-1);
        \node[anchor=center] (text) at (1,0) {\corelabelsize $\onehotmapof{0}$};
        \draw[->-] (1,1)--(1,2.5) node[midway,right] {\colorlabelsize $\formulavar$};

        \draw (-2,-2) rectangle (4,-4);
        \node[anchor=center] (text) at (1,-3) {\corelabelsize $\lnot\formula$};%$\sum_{\shortcatindices \wcols \formula(\shortcatindices)=0}\onehotmapof{\shortcatindices}$};

        \draw[] (-1.5,-4)--(-1.5,-5.5) node[midway,left] {\colorlabelsize $\catvariableof{0}$};
        \draw[] (0.5,-4)--(0.5,-5.5) node[midway,left] {\colorlabelsize $\catvariableof{1}$};
        \node[anchor=center] (text) at (2,-5) {$\cdots$};
        \draw[] (3.5,-4)--(3.5,-5.5) node[midway,right] {\colorlabelsize $\catvariableof{\atomorder\shortminus1}$};

    \end{scope}


    \node[anchor=center] (text) at (31.25,-2) {${+}$};

    \begin{scope} [shift={(35.5,-0.5)}]

        \draw (-2,1) rectangle (4,-1);
        \node[anchor=center] (text) at (1,0) {\corelabelsize $\onehotmapof{1}$};
        \draw[->-] (1,1)--(1,2.5) node[midway,right] {\colorlabelsize $\formulavar$};

        \draw (-2,-2) rectangle (4,-4);
        \node[anchor=center] (text) at (1,-3) {\corelabelsize $\formula$};%$\sum_{\shortcatindices \wcols \formula(\shortcatindices)=1}\onehotmapof{\shortcatindices}$};

        \draw[] (-1.5,-4)--(-1.5,-5.5) node[midway,left] {\colorlabelsize $\catvariableof{0}$};
        \draw[] (0.5,-4)--(0.5,-5.5) node[midway,left] {\colorlabelsize $\catvariableof{1}$};
        \node[anchor=center] (text) at (2,-5) {$\cdots$};
        \draw[] (3.5,-4)--(3.5,-5.5) node[midway,right] {\colorlabelsize $\catvariableof{\atomorder\shortminus1}$};

    \end{scope}

\end{tikzpicture}
\end{center}
We further provide a more detailed example in coordinate sensitive notation in the following.
\begin{example}[Logical Negation and Conjunction]
    \label{exa:bencodingNegCon} %\cite[Example 4.9]{goessmann_tensor-network_2025}
    The basis encodings of the negation $\notucon: [2]\rightarrow [2]$ is the matrix
    \begin{center}
        \begin{tikzpicture}[scale=1]
            \node (A) at (-2.5,0) {$\bencodingofat{\lnot}{\headvariableof{\lnot},\catvariable}$=};
            \node (A) at (0,0) {
                $\begin{bmatrix}
                     0 & 1 \\
                     1 & 0
                \end{bmatrix}$
            };
            \draw[<-,dashed] (-0.9,-0.275) node[right] {\tiny $1$} -- (-0.9,0.275) node [midway, left] {\tiny $\catvariableof{0}$} node[right] {\tiny $0$};
            \draw[->,dashed] (-0.3,0.85) node[below] {\tiny $0$} -- (0.3,0.85) node [midway, above] {\tiny $\headvariableof{\lnot}$} node[below] {\tiny $1$};
        \end{tikzpicture}
    \end{center}
    The $2$-ary conjunctions $\land:  [2]\times[2] \rightarrow[2]$ is encoded by the order-$3$ tensor
    \begin{center}
        \begin{tikzpicture}[scale=1]
            \node (A) at (-4.5,0) {$\bencodingofat{\land}{\headvariableof{\land},\catvariableof{0},\catvariableof{1}}$=};

            \begin{scope}[shift={(0,0)}]
                \node(B) at (-2,0){
                    $\begin{bmatrix}
                         1 \\
                         0
                    \end{bmatrix}$
                };
                \draw[<-,dashed] (-2.5,-0.275) node[right] {\tiny $1$} -- (-2.5,0.275) node [midway, left] {\tiny $\headvariableof{\land}$} node[right] {\tiny $0$};
                \node (A) at (-1.6,0) {$\otimes$};
                \node (A) at (0,0) {
                    $\begin{bmatrix}
                         1 & 1 \\
                         1 & 0
                    \end{bmatrix}$
                };
                \draw[<-,dashed] (-0.9,-0.275) node[right] {\tiny $1$} -- (-0.9,0.275) node [midway, left] {\tiny $\catvariableof{0}$} node[right] {\tiny $0$};
                \draw[->,dashed] (-0.3,0.85) node[below] {\tiny $0$} -- (0.3,0.85) node [midway, above] {\tiny $\catvariableof{1}$} node[below] {\tiny $1$};
            \end{scope}

            \begin{scope}[shift={(4.25,0)}]

                \node[anchor=center] (A) at (-3.25,0) {$+$};

                \node(B) at (-2,0){
                    $\begin{bmatrix}
                         0 \\
                         1
                    \end{bmatrix}$
                };
                \draw[<-,dashed] (-2.5,-0.275) node[right] {\tiny $1$} -- (-2.5,0.275) node [midway, left] {\tiny $\headvariableof{\land}$} node[right] {\tiny $0$};
                \node (A) at (-1.6,0) {$\otimes$};
                \node (A) at (0,0) {
                    $\begin{bmatrix}
                         0 & 0 \\
                         0 & 1
                    \end{bmatrix}$
                };
                \draw[<-,dashed] (-0.9,-0.275) node[right] {\tiny $1$} -- (-0.9,0.275) node [midway, left] {\tiny $\catvariableof{0}$} node[right] {\tiny $0$};
                \draw[->,dashed] (-0.3,0.85) node[below] {\tiny $0$} -- (0.3,0.85) node [midway, above] {\tiny $\catvariableof{1}$} node[below] {\tiny $1$};
            \end{scope}

            \begin{scope}[shift={(7,0)}]
                \node[anchor=center] (A) at (-1.75,0) {$=$};

                \node (A) at (0,0) {
                    $\begin{bmatrix}
                         1 & 1 \\
                         1 & 0
                    \end{bmatrix}$
                };
                \node (A) at (1.25,0.3) {
                    $\begin{bmatrix}
                         0 & 0 \\
                         0 & 1
                    \end{bmatrix}$
                };
                \draw[<-,dashed] (-0.9,-0.275) node[right] {\tiny $1$} -- (-0.9,0.275) node [midway, left] {\tiny $\catvariableof{0}$} node[right] {\tiny $0$};
                \draw[->,dashed] (-0.3,0.85) node[below] {\tiny $0$} -- (0.3,0.85) node [midway, above] {\tiny $\catvariableof{1}$} node[below] {\tiny $1$};
                \draw[->,dashed] (0,-0.85) node[above] {\tiny $0$} -- (1.25,-0.55) node [midway, below] {\tiny $\headvariableof{\land}$} node[above] {\tiny $1$};
            \end{scope}
        \end{tikzpicture}
    \end{center}
    Further, the $2$-ary disjunction $\lor:  [2]\times[2] \rightarrow[2]$ is encoded by the order-$3$ tensor
    \begin{center}
        \begin{tikzpicture}[scale=1]
            \node (A) at (-3,0) {$\bencodingofat{\lor}{\headvariableof{\lor},\catvariableof{0},\catvariableof{1}}$=};
            \node (A) at (0,0) {
                $\begin{bmatrix}
                     1 & 0 \\
                     0 & 0
                \end{bmatrix}$
            };
            \node (A) at (1.25,0.3) {
                $\begin{bmatrix}
                     0 & 1 \\
                     1 & 1
                \end{bmatrix}$
            };
            \draw[<-,dashed] (-0.9,-0.275) node[right] {\tiny $1$} -- (-0.9,0.275) node [midway, left] {\tiny $\catvariableof{0}$} node[right] {\tiny $0$};
            \draw[->,dashed] (-0.3,0.85) node[below] {\tiny $0$} -- (0.3,0.85) node [midway, above] {\tiny $\catvariableof{1}$} node[below] {\tiny $1$};
            \draw[->,dashed] (0,-0.85) node[above] {\tiny $0$} -- (1.25,-0.55) node [midway, below] {\tiny $\headvariableof{\lor}$} node[above] {\tiny $1$};
        \end{tikzpicture}
    \end{center}
\end{example}

\paragraph{Interpretation as \CompActNets{}}
The propositional formula and its negation can be represented by that tensor by
\begin{align*}
    \formulaat{\shortcatvariables}
    = \contractionof{\tbasisat{\headvariableof{\formula}},\bencodingofat{\formula}{\headvariableof{\formula},\shortcatvariables}}{\shortcatvariables}
    \andspace
    \lnot\formulaat{\shortcatvariables}
    = \contractionof{\fbasisat{\headvariableof{\formula}},\bencodingofat{\formula}{\headvariableof{\formula},\shortcatvariables}}{\shortcatvariables} \, .
\end{align*}
Both $\formula$ and $\lnot\formula$ are thus \ComputationActivationNetworks{} to the statistic $\{\formula\}$ and the hard activation tensor $\tbasisat{\headvariableof{\formula}}$, respectively $\fbasisat{\headvariableof{\formula}}$.
This representation of propositional formulas with respect to basis encoding thus leads to \ComputationActivationNetworks{}, which were also used to describe probability distributions in the last section.
In this way the soft and hard logic can be combined in one framework.

\subsection{Decomposition of Propositional Formulas}

We now show, that the propositional formula allows for a decomposition into connective formulas, its basis encoding decomposes into the basis encodings of the connective formulas.

\begin{lemma}
    \label{lem:formulaDecomp}
    Let $\formulaat{\shortcatvariables}$ be a composition of a $\seldim$-ary connective formula $\exconnective$ and propositional formulas $\formulaofat{\selindex}{\shortcatvariables}$, where $\selindexin$, i.e. for $\shortcatindices\in\atomstates$ we have
    \begin{align*}
        \formulaat{\indexedshortcatvariables}
        = \exconnective\left(\formulaofat{0}{\indexedshortcatvariables}, \dots, \formulaofat{\seldim-1}{\indexedshortcatvariables}\right) \, .
    \end{align*}
    Then we have
    \begin{align*}
        \bencodingofat{\formula}{\headvariableof{\formula},\shortcatvariables}
        = \contractionof{
            \{\bencodingofat{\exconnective}{\headvariableof{\formula},\headvariableof{[\seldim]}}\}
            \cup \{\bencodingofat{\formulaof{\selindex}}{\headvariableof{\selindex},\shortcatvariables} \wcols \selindexin\}
        }{\headvariableof{\formula},\shortcatvariables} \, .
    \end{align*}
\end{lemma}
\begin{proof}
    This can be shown on each index $\shortcatindices$.
\end{proof}

For the composition of two propositional formulas $\formulaat{\shortcatvariables}$ and $\secexformula\left[\shortcatvariables\right]$ the composition by some binary connective is pictured by:
\begin{center}
    \begin{tikzpicture}[scale=0.4, thick]
    \begin{scope}

        % \node[anchor=center] (text) at (-6,-4) {$b)$};

        \begin{scope}
        [shift={(-8,0)}]
            \draw[->-] (5.5,-9)--(5.5,-7) node[midway,right] {\colorlabelsize $\headvariableof{\exformula}$};
            \drawatomcore{3.5}{-8}{$\bencodingof{\exformula}$}
            \drawatomindices{3.5}{-12}
        \end{scope}

        \node[anchor=center] (text) at (1.5,-10) {${=}$};

        \draw[->-] (9.5,-5)--(9.5,-3) node[midway,right] {\colorlabelsize $\headvariableof{\exformula}$};

        \node[anchor=center] (text) at (9.5,-6) {$\bencodingof{\chainingfunction}$};
        \draw (4.5,-7) rectangle (14.5,-5);

        \draw[->-] (5.5,-9)--(5.5,-7) node[midway,right] {\colorlabelsize $\headvariableof{0}$};

        \node[anchor=center] at (9.5,-8) {$\cdots$};

        \drawatomcore{3.5}{-8}{$\bencodingof{\formulaof{0}}$}
        \drawatomindices{3.5}{-12}

        \begin{scope}
        [shift={(8,0)}]

            \draw[->-] (5.5,-9)--(5.5,-7) node[midway,right] {\colorlabelsize $\headvariableof{\seldim-1}$};

            \drawatomcore{3.5}{-8}{$\bencodingof{\formulaof{\seldim-1}}$}
            \drawatomindices{3.5}{-12}

        \end{scope}

        \draw[fill] (7.5,-15) circle (\dotsize);
        \draw[] (7.5,-15) to[bend left=25] (3.5,-13);
        \draw[] (7.5,-15) to[bend right=25] (11.5,-13);

        \draw[fill] (9,-15.25) circle (\dotsize);
        \draw[] (9,-15.25) to[bend left=25] (5,-13);
        \draw[] (9,-15.25) to[bend right=25] (13,-13);

        \draw[fill] (11.5,-15) circle (\dotsize);
        \draw[] (11.5,-15) to[bend left=25] (7.5,-13);
        \draw[] (11.5,-15) to[bend right=25] (15.5,-13);



        \draw[] (7.5,-15)--(7.5,-17) node[midway,left] {\colorlabelsize $\catvariableof{0}$};
        \draw[] (9,-15.25)--(9,-17) node[midway,left] {\colorlabelsize $\catvariableof{1}$};
        \node[anchor=center] (text) at (10.5,-16.5) {$\cdots$};
        \draw[] (11.5,-15)--(11.5,-17) node[midway,right] {\colorlabelsize $\catvariableof{\atomorder-1}$};

    \end{scope}
\end{tikzpicture}
\end{center}

Let us now define a more generic syntactical decomposition of propositional formulas.

\begin{definition}
    \label{def:formulaDecomposition}
    A syntactical hypergraph is a directed acyclic hypergraph $\graph=(\nodes,\edges)$ such that
    \begin{itemize}
        \item each hyperedge $\edge=(\incomingnodes,\outgoingnodes)$ has exactly one outgoing node, i.e. $\cardof{\outgoingnodes}=1$
        \item each node $\nodein$ carries a boolean variable $\headvariableof{\node}$ and appears at most once as the outgoing node of a hyperedge % well-definedness
        \item each hyperedge $(\incomingnodes,\{\node\})$ with $\incomingnodes\neq\varnothing$ is decorated by a propositional formula
        \begin{align*}
            \connectiveofat{\node}{\headvariableof{\incomingnodes}} \defcols \bigtimes_{\node\in\incomingnodes} [2] \rightarrow [2]
        \end{align*}
        \item the node not appearing as an outgoing node are labeled by $[\atomorder]$
    \end{itemize}
    We say that the syntactical hypergraph is single-rooted, if exactly one node $\secnode$ does not appear as an incoming node of a hyperedge.
    In this case this unique node is called the root node. % head node
    We assign atomic formulas to the nodes $[\atomorder]$ and recursively assign to each further node $\node$ a node formula % connective $\connectiveofat{\node}{\headvariableof{\incomingnodes}}$.
    \begin{align*}
        \formulaofat{\node}{\indexedshortcatvariables}
        = \connectiveofat{\node}{[\formulaofat{\thirdnode}{\indexedshortcatvariables}\wcols\thirdnode\in\incomingnodes]} \quad \forall\shortcatindicesin\, ,
    \end{align*}
    where $\incomingnodes$ are the incoming nodes in the unique hyperedge with outgoing nodes $\{\node\}$.
    We call the formula $\exformulaat{\shortcatvariables}\coloneqq\formulaofat{\secnode}{\shortcatvariables}$ to the root note $\secnode$ the syntactical composition of $\graph$ and $\graph$ is a syntactical decomposition of $\exformula$.
\end{definition}

\begin{theorem}
    \label{the:formulaDecompositionRep}
    For any syntactical hypergraph $\graph$ with composition $\exformula$ we have
    \begin{align*}
        \exformulaat{\shortcatvariables}
        = \breakablecontractionof{
            &\left\{
                 \bencodingofat{\connectiveof{\node}}{\headvariableof{\node},\headvariableof{\incomingnodes}} \wcols (\incomingnodes,\{\node\})\in\edges
            \right\} \cup \\
            & \{\identityat{\headvariableof{\atomenumerator},\catvariableof{\atomenumerator}} \wcols \atomenumeratorin\}
            \cup \{\tbasisat{\headvariableof{\secnode}}\}
        }{\shortcatvariables} \, .
    \end{align*}
\end{theorem}
\begin{proof}
    One can show this theorem by induction over the node formulas of the syntactical hypergraph, from the leafs to the root and iteratively applying \lemref{lem:formulaDecomp}.
\end{proof}

Thus we have a tensor network representation of any propositional formula based on its syntactical decomposition, where the hypergraph of the syntactical decomposition equals the hypergraph of the representing tensor network.

\subsection{Contractions to decide entailment}

We have already seen that the contraction of a propositional formula counts its models.
This allows to define entailment between two propositional formulas as follows.

\begin{definition}[Entailment of propositional formulas]
    \label{def:logicalEntailment}
    Given two propositional formulas $\kb$ and $\exformula$ we say that $\kb$ entails $\exformula$, denoted by $\kb\models\exformula$, if any model of $\kb$ is also a model of $\exformula$, that is
    \begin{align*}
        \contraction{\kb,\lnot\exformula}=0 \, .
    \end{align*}
    If $\kb\models\lnot\exformula$ holds (i.e. $\contraction{\kb,\exformula}$=0), we say that $\kb$ contradicts $\exformula$.
\end{definition}

% Relation to classical definition of entailment
Classically (see e.g. \cite{russell_artificial_2021}) entailment in propositional logics is defined as the the unsatisfiability of $\kb\land\lnot\exformula$.
This is equivalent to \defref{def:logicalEntailment}, since $\contraction{\kb,\lnot\exformula}=0$ is equivalent to $\contraction{\kb \land (\lnot\exformula)}=0$, which is the unsatisfiability of $\kb\land\lnot\exformula$.

Entailment is the central operation of "logical inference", i.e. deduce true statements from known statements.
In the tensor network representation, these entailments can be decided by contracting the whole representing tensor with the statement, that needs to be checked.

\begin{example}[$\sudokunum^2\,\times \,\sudokunum^2$ Sudoku]
    \label{exa:sudokuEntailment}%{\alex{Attempt to match the above Sudoku example with our notation of boolean variables and the entailment formalism}}
    We index the rows and the columns by tuples $(r0,r1)$ and $(co,c1)$, where $r0,r1,c0,c1\in[\sudokunum]$. The first index indicates the block and the second counts the row or column inside that block.
    For each $r0,r1,c0,c1\in[\sudokunum]$ and $i\in[\sudokunum^2]$ we then define an atomic variable $\catvariableof{r0,r1,c0,c1,i}\in\{0,1\}$ indicating whether in the row $(r0,r1)$ and column $(co,c1)$ the number $i$ is written.
    The Sudoku rules then amount to the formula
    \begin{align*}
        \sudokukbof{\sudokunum}  \coloneqq
        &\left( \bigwedge_{r0,r1,c0,c1\in[\sudokunum]} \left( \woneoplus_{i\in[\sudokunum^2]} \catvariableof{r0,r1,c0,c1,i} \right) \right) \land
        \left( \bigwedge_{r0,r1\in[\sudokunum], i\in[\sudokunum^2]} \left( \woneoplus_{c0,c1\in[\sudokunum]} \catvariableof{r0,r1,c0,c1,i} \right) \right) \land \\
        &\left( \bigwedge_{c0,c1\in[\sudokunum], i\in[\sudokunum^2]} \left( \woneoplus_{c0,c1\in[\sudokunum]} \catvariableof{r0,r1,c0,c1,i} \right) \right) \land
        \left( \bigwedge_{r0,c0\in[\sudokunum], i\in[\sudokunum^2]} \left( \woneoplus_{r1,c1\in[\sudokunum]} \catvariableof{r0,r1,c0,c1,i} \right) \right) \, ,
    \end{align*}
    where $\woneoplus$ is the $\sudokunum^2$-ary exclusive or connective (that is $1$ if and only if exactly one of the arguments is $1$).
    The four outer brackets in $\kb$ mark the constraints that at each position exactly one number is assigned, further that in each row each number is assigned once, and similar for the columns and the squares of the board.
    When solving a specific Sudoku instance, one typically knows from an initial board assignment $\sudokustartevidence$ a collection of atomic variables, which hold, and needs to find further atomic variables, which are entailed.
    This means, we need to decide for each $(r_0,r_1,c_0,c_1,i)\notin \sudokustartevidence$ whether the Sudoku rules and the initial board imply that the atomic variable $\catvariableof{r0,r1,c0,c1,i}$ (i.e. assignment to the board) is true
    \begin{align*}
        \sudokukbof{\sudokunum} \land \left(\bigwedge_{(r_0,r_1,c_0,c_1,i)\in \sudokustartevidence} \catvariableof{r0,r1,c0,c1,i} \right) \models \catvariableof{r0,r1,c0,c1,i}
    \end{align*}
    or false
    \begin{align*}
%        \label{eq:sudokukb}
        \kb \land \left(\bigwedge_{(r_0,r_1,c_0,c_1,i)\in \sudokustartevidence} \catvariableof{r0,r1,c0,c1,i} \right) \models \lnot\catvariableof{r0,r1,c0,c1,i} \, .
    \end{align*}
%    In other words, for each assignment to the board, that fulfills the Sudoku rules and the initial board, do we write the number $n$ in row $(r0,r1)$ and column $(c0,c1)$?
    If and only if the Sudoku has a unique solution given the initial board assignment $\sudokustartevidence$, exactly one of these entailment statements holds for each $(r_0,r_1,c_0,c_1,i)\notin \sudokustartevidence$.
    Deciding which is equivalent to solving the Sudoku.
%    We model Sudoku as a hypergraph of
%    \begin{itemize}
%        \item Nodes labeled by $\sudokunum^6$ tuples $(r0,r1,c0,c1,i)$:
%        \begin{align*}
%            \nodes = \{(r0,r1,c0,c1,i) \wcols r0,r1,c0,c1 \in [\sudokunum]\ncond i \in [\sudokunum^2]\}
%        \end{align*}
%        \item Hyperedges by the $4\cdot \sudokunum^4$ constraints (implementing the position, row, columns and square contraints):
%        \begin{align*}
%            \edges =& \big\{\{(r0,r1,c0,c1,i) \wcols r0,r1,c0,c1\in[\sudokunum]\} \wcols r0,r1,c0,c1\in[\sudokunum]\big\} \cup \\
%            &\big\{\{(r0,r1,c0,c1,i) \wcols r0,r1,c0,c1\in[\sudokunum]\} \wcols r0,r1\in[\sudokunum], i\in[\sudokunum^2]\big\} \cup \\
%            &\big\{\{(r0,r1,c0,c1,i) \wcols r0,r1,c0,c1\in[\sudokunum]\} \wcols c0,c1\in[\sudokunum], i\in[\sudokunum^2]\big\} \cup \\
%            &\big\{\{(r0,r1,c0,c1,i) \wcols r0,r1,c0,c1\in[\sudokunum]\} \wcols r0,c0\in[\sudokunum], i\in[\sudokunum^2]\big\} \cup
%        \end{align*}
%        Each hypercore is carried by a tensor representing the logical formula $\woneoplus$ on $\sudokunum^2$ boolean variables.
%    \end{itemize}

    For a more concrete example, let $n=2$ and
    \begin{align*}
        \sudokustartevidence = \{&(0,0,0,0,0),(0,0,1,0,2),(0,0,1,1,1), %first row
        (0,1,0,1,1), \\ %second row
        &(1,0,1,0,3), %third row
        (1,1,0,0,3),(1,1,0,1,2) %fourth row
        \} \, .
    \end{align*}
    We visualize this evidence by writing $i+1$ in a grid cell $(r0,r1,c0,c1)$ to indicate that $(r0,r1,c0,c1,i)\in \sudokustartevidence$:
    \begin{center}
        \begin{sudoku4x4}
            \matrix[sudokumatrix] (M) at (0,0) {
                1 & \ & 3 & 2 \\
                \ & 2 & \  & \  \\
                \ & \ & 4 & \ \\
                4 & 3 &  \ & \  \\
            };
            \draw[thick]([yshift=9.5pt,xshift=-0.6pt]M-1-2.east) -- ([yshift=-9.5pt,xshift=-0.6pt]M-4-2.east);
            \draw[thick]([xshift=-9.5pt,yshift=0.6pt]M-2-1.south) -- ([xshift=9.5pt,yshift=0.6pt]M-2-4.south);
        \end{sudoku4x4}
    \end{center}
    After deriving a sparse tensor network representations in \exaref{exa:sudokuDecomposition}, we demonstrate a solution algorithm to solve this instance in \exaref{exa:sudokuEntailment}.
\end{example}

\subsection{Efficient Representation of Knowledge Bases}

We now investigate the representation of knowledge bases, which are conjunctions
\begin{align*}
    \kbwith
    = \bigwedge_{\selindexin} \formulaofat{\selindex}{\shortcatvariables} \, .
\end{align*}
To show efficient repesentations we will use the following identities.

\begin{lemma}[Computation Network Symmetries]
    \label{lem:comNetSymmetries}
    We have for the $\catorder$-ary $\land$-connective (where $\catorder\in\nn$) and the unary $\lnot$-connective that
    \begin{align*}
        \contractionof{\tbasisat{\headvariable},\bencodingofat{\land}{\headvariable,\shortcatvariables}}{\shortcatvariables}
        =
        \bigotimes_{\catenumeratorin} \tbasisat{\catvariableof{\catenumerator}}
        \andspace
        \contractionof{\tbasisat{\headvariable},\bencodingofat{\lnot}{\headvariable,\catvariable}}{\catvariable}
        =
        \fbasisat{\catvariable} \, .
    \end{align*}
\end{lemma}
\begin{proof}
    Follows directly from the definitions of the basis encodings and the connectives.
\end{proof}

\begin{example}[Computation Network Symmetries]
    %see the notebook: \url{https://colab.research.google.com/drive/1p2wp61fFMu0otnfFhKoNsLiCNfWpuEsn?usp=sharing}
    For the propositional formula $\formulaat{\catvariableof{[3]}}={(\catvariableof{0} \lor \catvariableof{1}) \land \lnot \catvariableof{2}}$ (see \exaref{exa:propFormulaCoordinatewise}), we can write the formula in terms of a \ComputationActivationNetwork{} with activation tensor $\tbasis$ and computation network decomposed by the basis encodings. First, it is written with one activation vector. Second, we see that it can also be interpreted with multiple features.
    \begin{center}
        \begin{tikzpicture}[scale=0.4, yscale=-1, thick] % , baseline = -3.5pt

            \draw[] (-2,1)--(-2,-1) node[midway,left] {\colorlabelsize $\catvariableof{0}$};
            \draw[] (0.5,1)--(0.5,-1) node[midway,right] {\colorlabelsize $\catvariableof{1}$};
            \draw[] (3,1)--(3,-1) node[midway,right] {\colorlabelsize $\catvariableof{2}$};
            \draw (-3,-1) rectangle (4, -3);
            \node[anchor=center] (text) at (0.5,-2) {\corelabelsize $(\catvariableof{0} \lor \catvariableof{1}) \land \lnot \catvariableof{2}$};
            %\draw[->-] (1.5,-3)--(1.5,-5) node[midway,right] {\colorlabelsize $\headvariableof{a \land b \land \lnot c}$};

            \node[anchor=center] (text) at (5,-2) {${=}$};


            \begin{scope}
            [shift={(7,0)}]

                \draw[->-] (0,1)--(0,-1) node[midway,left] {\colorlabelsize $\catvariableof{0}$};
                \draw[->-] (3,1)--(3,-1) node[midway,right] {\colorlabelsize $\catvariableof{1}$};
                \draw[->-] (6,1)--(6,-1) node[midway,right] {\colorlabelsize $\catvariableof{2}$};

                \draw (-1,-1) rectangle (4, -3);
                \node[anchor=center] (text) at (1.5,-2) {\corelabelsize $\bencodingof{\lor}$};

                \draw[->-] (1.5,-3) --(1.5,-5) node[midway,right]{\colorlabelsize $\headvariableof{0 \lor 1}$};

                \draw (5,-1) rectangle (7, -3);
                \node[anchor=center] (text) at (6,-2) {\corelabelsize $\bencodingof{\lnot}$};

                \draw[->-] (6,-3) --(6,-5) node[midway,right]{\colorlabelsize $\headvariableof{\lnot 2}$};

                \draw (0.5,-5) rectangle (6.5,-7);
                \node[anchor=center] (text) at (3.5,-6) {\corelabelsize $\bencodingof{\land}$};

                \draw[->-] (4,-7) -- (4,-8.5) node[right] {\colorlabelsize $\headvariableof{(0 \lor 1) \land \lnot 2}$};
                \drawvariabledot{4}{-8}
                \draw[] (4,-8) -- (4,-9);
                \draw (3,-9) rectangle (5,-11);
                \node[anchor=center] (text) at (4,-10) {$\tbasis$};

            \end{scope}

            \node[anchor=center] (text) at (15,-2) {${=}$};

            \begin{scope}
            [shift={(17,0)}]

                \draw[->-] (0,1)--(0,-1) node[midway,left] {\colorlabelsize $\catvariableof{0}$};
                \draw[->-] (3,1)--(3,-1) node[midway,right] {\colorlabelsize $\catvariableof{1}$};
                \draw[] (7,1)--(7,-1) node[midway,right] {\colorlabelsize $\catvariableof{2}$};

                \draw (-1,-1) rectangle (4, -3);
                \node[anchor=center] (text) at (1.5,-2) {\corelabelsize $\bencodingof{\lor}$};

                \draw (1.5,-4.5) -- (1.5,-5);
                \draw[->-] (1.5,-3) --(1.5,-4.5) node[midway,right]{\colorlabelsize $\headvariableof{0 \lor 1}$};

                \drawvariabledot{1.5}{-4}
                \draw (0.5,-5) rectangle (2.5,-7);
                \node[anchor=center] (text) at (1.5,-6) {\corelabelsize $\tbasis$};

                \node[anchor=center] (text) at (5,-2) {$\otimes$};


                \draw (6,-1) rectangle (8, -3);
                \node[anchor=center] (text) at (7,-2) {\corelabelsize $\fbasis$};

                %\draw[->-] (6,-3) --(6,-5) node[midway,right]{\colorlabelsize $\headvariableof{\lnot c}$};


                %\draw (0.5,-5) rectangle (6.5,-7);
                %\node[anchor=center] (text) at (3.5,-6) {\corelabelsize $\bencodingof{\land}$};

                %\draw[->-] (4,-7) -- (4,-8.5) node[right] {\colorlabelsize $\headvariableof{(a \lor b) \land \lnot c}$};
                %\drawvariabledot{4}{-8}
                %\draw[] (4,-8) -- (4,-9);
                %\draw (3,-9) rectangle (5,-11);
                %\node[anchor=center] (text) at (4,-10) {$\tbasis$};

            \end{scope}

        \end{tikzpicture}
    \end{center}
\end{example}


We use this to decompose knowledge bases into their individual formulas as follows.

\begin{theorem}\label{the:kbDecomposition}
    For any knowledge base $\kbwith = \bigwedge_{\selindexin} \formulaofat{\selindex}{\shortcatvariables}$ it holds that
    \begin{align*}
        \kbwith
        = \contractionof{\{\formulaofat{\selindex}{\shortcatvariables} \wcols \selindexin\}}{\shortcatvariables} \, .
    \end{align*}
\end{theorem}
\begin{proof}
    With \lemref{lem:comNetSymmetries} we have
    \begin{align*}
        \kbwith
        &= \contractionof{\{\tbasisat{\headvariableof{\land}},\bencodingofat{\land}{\headvariableof{\land},\headvariables}\}
            \cup \{\bencodingofat{\formulaof{\selindex}}{\headvariableof{\selindex},\shortcatvariables}\wcols \selindexin\}}{\shortcatvariables} \\
        &= \contractionof{
            \bigcup_{\selindexin} \{\tbasisat{\headvariableof{\selindex}},\bencodingofat{\formulaof{\selindex}}{\headvariableof{\selindex},\shortcatvariables}\wcols \selindexin\}}{\shortcatvariables} \\
        &= \contractionof{\{\formulaofat{\selindex}{\shortcatvariables} \wcols \selindexin\}}{\shortcatvariables} \, .
    \end{align*}
\end{proof}



\begin{example}[$\sudokunum^2\,\times \,\sudokunum^2$ Sudoku]
    \label{exa:sudokuEntailment}%{\alex{Attempt to match the above Sudoku example with our notation of boolean variables and the entailment formalism}}
    We index the rows and the columns by tuples $(r0,r1)$ and $(co,c1)$, where $r0,r1,c0,c1\in[\sudokunum]$. The first index indicates the block and the second counts the row or column inside that block.
    For each $r0,r1,c0,c1\in[\sudokunum]$ and $i\in[\sudokunum^2]$ we then define an atomic variable $\catvariableof{r0,r1,c0,c1,i}\in\{0,1\}$ indicating whether in the row $(r0,r1)$ and column $(co,c1)$ the number $i$ is written.
    The Sudoku rules then amount to the formula
    \begin{align*}
        \sudokukbof{\sudokunum}  \coloneqq
        &\left( \bigwedge_{r0,r1,c0,c1\in[\sudokunum]} \left( \woneoplus_{i\in[\sudokunum^2]} \catvariableof{r0,r1,c0,c1,i} \right) \right) \land
        \left( \bigwedge_{r0,r1\in[\sudokunum], i\in[\sudokunum^2]} \left( \woneoplus_{c0,c1\in[\sudokunum]} \catvariableof{r0,r1,c0,c1,i} \right) \right) \land \\
        &\left( \bigwedge_{c0,c1\in[\sudokunum], i\in[\sudokunum^2]} \left( \woneoplus_{c0,c1\in[\sudokunum]} \catvariableof{r0,r1,c0,c1,i} \right) \right) \land
        \left( \bigwedge_{r0,c0\in[\sudokunum], i\in[\sudokunum^2]} \left( \woneoplus_{r1,c1\in[\sudokunum]} \catvariableof{r0,r1,c0,c1,i} \right) \right) \, ,
    \end{align*}
    where $\woneoplus$ is the $\sudokunum^2$-ary exclusive or connective (that is $1$ if and only if exactly one of the arguments is $1$).
    The four outer brackets in $\kb$ mark the constraints that at each position exactly one number is assigned, further that in each row each number is assigned once, and similar for the columns and the squares of the board.
    When solving a specific Sudoku instance, one typically knows from an initial board assignment $\sudokustartevidence$ a collection of atomic variables, which hold, and needs to find further atomic variables, which are entailed.
    This means, we need to decide for each $(r_0,r_1,c_0,c_1,i)\notin \sudokustartevidence$ whether the Sudoku rules and the initial board imply that the atomic variable $\catvariableof{r0,r1,c0,c1,i}$ (i.e. assignment to the board) is true
    \begin{align*}
        \sudokukbof{\sudokunum} \land \left(\bigwedge_{(r_0,r_1,c_0,c_1,i)\in \sudokustartevidence} \catvariableof{r0,r1,c0,c1,i} \right) \models \catvariableof{r0,r1,c0,c1,i}
    \end{align*}
    or false
    \begin{align*}
%        \label{eq:sudokukb}
        \kb \land \left(\bigwedge_{(r_0,r_1,c_0,c_1,i)\in \sudokustartevidence} \catvariableof{r0,r1,c0,c1,i} \right) \models \lnot\catvariableof{r0,r1,c0,c1,i} \, .
    \end{align*}
%    In other words, for each assignment to the board, that fulfills the Sudoku rules and the initial board, do we write the number $n$ in row $(r0,r1)$ and column $(c0,c1)$?
    If and only if the Sudoku has a unique solution given the initial board assignment $\sudokustartevidence$, exactly one of these entailment statements holds for each $(r_0,r_1,c_0,c_1,i)\notin \sudokustartevidence$.
    Deciding which is equivalent to solving the Sudoku.
%    We model Sudoku as a hypergraph of
%    \begin{itemize}
%        \item Nodes labeled by $\sudokunum^6$ tuples $(r0,r1,c0,c1,i)$:
%        \begin{align*}
%            \nodes = \{(r0,r1,c0,c1,i) \wcols r0,r1,c0,c1 \in [\sudokunum]\ncond i \in [\sudokunum^2]\}
%        \end{align*}
%        \item Hyperedges by the $4\cdot \sudokunum^4$ constraints (implementing the position, row, columns and square contraints):
%        \begin{align*}
%            \edges =& \big\{\{(r0,r1,c0,c1,i) \wcols r0,r1,c0,c1\in[\sudokunum]\} \wcols r0,r1,c0,c1\in[\sudokunum]\big\} \cup \\
%            &\big\{\{(r0,r1,c0,c1,i) \wcols r0,r1,c0,c1\in[\sudokunum]\} \wcols r0,r1\in[\sudokunum], i\in[\sudokunum^2]\big\} \cup \\
%            &\big\{\{(r0,r1,c0,c1,i) \wcols r0,r1,c0,c1\in[\sudokunum]\} \wcols c0,c1\in[\sudokunum], i\in[\sudokunum^2]\big\} \cup \\
%            &\big\{\{(r0,r1,c0,c1,i) \wcols r0,r1,c0,c1\in[\sudokunum]\} \wcols r0,c0\in[\sudokunum], i\in[\sudokunum^2]\big\} \cup
%        \end{align*}
%        Each hypercore is carried by a tensor representing the logical formula $\woneoplus$ on $\sudokunum^2$ boolean variables.
%    \end{itemize}

    For a more concrete example, let $n=2$ and
    \begin{align*}
        \sudokustartevidence = \{&(0,0,0,0,0),(0,0,1,0,2),(0,0,1,1,1), %first row
        (0,1,0,1,1), \\ %second row
        &(1,0,1,0,3), %third row
        (1,1,0,0,3),(1,1,0,1,2) %fourth row
        \} \, .
    \end{align*}
    We visualize this evidence by writing $i+1$ in a grid cell $(r0,r1,c0,c1)$ to indicate that $(r0,r1,c0,c1,i)\in \sudokustartevidence$:
    \begin{center}
        \begin{sudoku4x4}
            \matrix[sudokumatrix] (M) at (0,0) {
                1 & \ & 3 & 2 \\
                \ & 2 & \  & \  \\
                \ & \ & 4 & \ \\
                4 & 3 &  \ & \  \\
            };
            \draw[thick]([yshift=9.5pt,xshift=-0.6pt]M-1-2.east) -- ([yshift=-9.5pt,xshift=-0.6pt]M-4-2.east);
            \draw[thick]([xshift=-9.5pt,yshift=0.6pt]M-2-1.south) -- ([xshift=9.5pt,yshift=0.6pt]M-2-4.south);
        \end{sudoku4x4}
    \end{center}
    After deriving a sparse tensor network representations in \exaref{exa:sudokuDecomposition}, we demonstrate a solution algorithm to solve this instance in \exaref{exa:sudokuEntailment}.
\end{example}

% BAD NOTATION!
%Noting that for example for $x_a=x_b=\epsilon_1=[0,1]^\intercal$
%\begin{center}
%    \input{../tikz_pics/logic_representation/and_decomposition}
%\end{center}
%while for all other vectors $x_a,x_b$, all parts of the equations amount to $0$. This yields that a knowledge base consisting of multiple formulas connected by a $\land$ has an efficient representation by decomposing the the tensor into its individual formulas.
%

\subsection{Message-passing for Entailment}

% Infeasible constractions
Since contracting the whole tensor is often infeasible and for instance for the Sudoku example would correspond to solving the whole problem, local contractions can be considered to decide in some cases.
Here a local contraction describes the calculation of contractions along few closely connected legs in the tensor network. Now, if the local contraction of any legs leads to a zero-tensor in the network decomposition, the whole contraction amounts to zero, and the knowledge base entails $f$.

\begin{theorem}[Monotonicity of Propositional Logics]
    \label{the:monotonicityPL}
    If $\seckb\subset\kb$ and $\seckb\models\formula$ then also $\kb\models\formula$.
\end{theorem}
\begin{proof}
    Since $\seckb\models\formula$ it holds that $\contraction{\seckb,\lnot\formula}=0$ and thus  $\contractionof{\seckb,\lnot\formula}{\shortcatvariables}=\zerosat{\shortcatvariables}$.
    Denoting by $\kb/\seckb$ the conjunctions of formulas in $\kb$ not in $\seckb$, we have
    \begin{align*}
        \contraction{\kbwith,\lnot\formulawith}
        &= \contraction{\kb/\seckb[\shortcatvariables],\seckb,\lnot\formulawith} \\
        &= \contraction{\kb/\seckb[\shortcatvariables],\contractionof{\seckb[\shortcatvariables],\lnot\formulawith}{\shortcatvariables}} \\
        &= \contraction{\kb/\seckb[\shortcatvariables],\zerosat{\shortcatvariables}} \\
        &= 0 \, .
    \end{align*}
\end{proof}

To decide entailment, we can therefore investigate entailment on smaller parts of the knowledge base.
This is sound by the above theorem, but not complete, since it can happen that no smaller part of the knowledge base entails the formula, but the whole knowledge base does.

We can futhermore add entailed formulas to the knowledge base without the latter, as we show next.

\begin{theorem}[Invariance of adding Entailed Formulas]
    \label{the:addingEntailed}
    If $\kb\models\formula$ then % Actually can extend to iff -> Modify the proof for that?
    \begin{align*}
        \kbwith
        = \contractionof{\kbwith,\formulawith}{\shortcatvariables} \, .
    \end{align*}
\end{theorem}
\begin{proof}
    We use that $\formulawith+\lnot\formulawith=\onesat{\shortcatvariables}$ and thus
    \begin{align*}
        \kbwith
        &= \contractionof{\kbwith,(\formulawith+\lnot\formulawith)}{\shortcatvariables} \\
        &= \contractionof{\kbwith,\formulawith}{\shortcatvariables}  + \contractionof{\kbwith,\lnot\formulawith}{\shortcatvariables}  \\
        &= \contractionof{\kbwith,\formulawith}{\shortcatvariables} \, .
    \end{align*}
\end{proof}

% Interpreting entailment
One can understand entailment as "making the knowledge base more accessible":
Adding deduced statements to a knowledge base does not change the knowledge base as a tensor, but one can interpret it in an easier way.
\theref{the:addingEntailed} justifies this intuition in our tensor network formalism.

% Constraint Propagation
This motivates an message-passing approach to decide entailment by iteratively adding entailed formulas to the knowledge base and checking entailment on smaller parts of the knowledge base.
Let us now refine \algoref{alg:beliefPropagation} to this situation.
Since we are only interested in the support of the contractions, we schedule new messages in the direction $(\sedge,\redge)$, once the support of a message received at $\sedge$ has been changed.
Note that such a scheduling system is guaranteed to converge, since there can only be a finite number of message changes.
We further directly reduce the computation of messages to their support and call the resulting \algoref{alg:constraintPropagation} Constraint Propagation.

\begin{algorithm}[hbt!]
    \caption{Constraint Propagation}\label{alg:constraintPropagation}
    \begin{algorithmic}
        \Require Tensor network $\extnet$ on a hypergraph $\graph$
        %\Ensure Scheduler $\scheduler$
        \iosepline
        \State Initialize a queue $\scheduler = \dirovedges$ of message directions
        \State Initialize messages $\messagewith = \onesat{\catvariableof{\sedge\cap\redge}}$ for $(\sedge,\redge)\in\dirovedges$
        \While{$\scheduler$ not empty}
            \State Pop a $(\sedge,\redge)$ pair from $\scheduler$
            \State Compute
            \begin{align*}
                \hypercoreat{\catvariableof{\sedge\cap \redge}}
                = \nonzeroof{\contractionof{\{\hypercoreofat{\sedge}{\catvariableof{\sedge}}\}
                    \cup \{\mesfromtoat{\secsedge}{\sedge}{\catvariableof{\secsedge\cap\sedge}} \wcols (\secsedge,\sedge)\in\dirovedges \ncond \secsedge\neq \redge\}
                }{\catvariableof{\sedge\cap \redge}}}
            \end{align*}
            \If{$\hypercoreat{\catvariableof{\sedge\cap \redge}}\neq\messagewith$}
                \State Update the message: $\messagewith\coloneqq\hypercoreat{\catvariableof{\sedge\cap \redge}}$
                \State Add $\scheduler = \scheduler \cup \{(\redge,\secsedge) \wcols (\redge,\secsedge)\in\dirovedges\}$ % Clear?
            \EndIf
        \EndWhile
        \State \Return Messages $\{\messagewith\wcols(\secsedge,\sedge)\in\dirovedges\}$
    \end{algorithmic}
\end{algorithm}

\begin{example}[Constraint Propagation for the Sudoku of \exaref{exa:sudokuEntailment}]
    \label{exa:sudokuMessagePassing}
    We iteratively solve a Sudoku puzzle by determining a possible value based on neighboring cells, rows and squares (using \theref{the:monotonicityPL}) and adding to our knowledge (using \theref{the:addingEntailed}).
    For example, consider the following $\sudokunum=2$ Sudoku puzzle, where a first entailment step uses only the knowledge of the rules and the \textcolor{\concolor}{blue} cells to determine the value $3$ in the first square:
    \begin{center}
        \begin{sudoku4x4}
            \matrix[sudokumatrix] (M) at (0,0) {
                1 & \ & \textcolor{\concolor}{3} & 2 \\
                \ & \textcolor{\concolor}{2} & \  & \  \\
                \ & \ & 4 & \ \\
                4 & 3 &  \ & \  \\
            };
            \draw[thick]([yshift=9.5pt,xshift=-0.6pt]M-1-2.east) -- ([yshift=-9.5pt,xshift=-0.6pt]M-4-2.east);
            \draw[thick]([xshift=-9.5pt,yshift=0.6pt]M-2-1.south) -- ([xshift=9.5pt,yshift=0.6pt]M-2-4.south);

            \node[anchor=center] (ist) at (1.75,0) {$=$};

            \matrix[sudokumatrix] (M) at (3.5,0) {
                1 & \ & 3 & 2 \\
                \textcolor{\probcolor}{3} & 2 & \  & \  \\
                \ & \ & 4 & \ \\
                4 & 3 &  \ & \  \\
            };
            \draw[thick]([yshift=9.5pt,xshift=-0.6pt]M-1-2.east) -- ([yshift=-9.5pt,xshift=-0.6pt]M-4-2.east);
            \draw[thick]([xshift=-9.5pt,yshift=0.6pt]M-2-1.south) -- ([xshift=9.5pt,yshift=0.6pt]M-2-4.south);

            \node[anchor=center] (ist) at (6.25,0) {$= \quad \ldots \quad =$};

            \matrix[sudokumatrix] (M) at (9,0) {
                1 & \textcolor{\probcolor}{4} & 3 & 2 \\
                \textcolor{\probcolor}{3} & 2 & \textcolor{\probcolor}{1} & \textcolor{\probcolor}{4}  \\
                \textcolor{\probcolor}{2} & \textcolor{\probcolor}{1} & 4 & \textcolor{\probcolor}{3} \\
                4 & 3 & \textcolor{\probcolor}{2} & \textcolor{\probcolor}{1}  \\
            };
            \draw[thick]([yshift=9.5pt,xshift=-0.6pt]M-1-2.east) -- ([yshift=-9.5pt,xshift=-0.6pt]M-4-2.east);
            \draw[thick]([xshift=-9.5pt,yshift=0.6pt]M-2-1.south) -- ([xshift=9.5pt,yshift=0.6pt]M-2-4.south);
        \end{sudoku4x4}
    \end{center}

    To illustrate the first reasoning step of assigning $\textcolor{\probcolor}{\catvariableof{0,1,0,0,2}}$, we make the following entailment steps applying \theref{the:monotonicityPL}.
    We also depict in \figref{fig:contractionPropagationSudoku} the corresponding messages in the Constraint Propagation Algorithm on the hypergraph $\graph^{\mathrm{Sudoku},n}$.
    \begin{itemize}
        \item From $\textcolor{\concolor}{\catvariableof{0,1,0,1,1}}$ (i.e. the $2$ in the cell $(0,1,0,1)$) and the Sudoku rule that at the cell $(0,1,0,1)$ exactly one number is assigned, we get
        \begin{align*}
            \left( \woneoplus_{i\in[\sudokunum^2]} \catvariableof{0,1,0,1,i} \right) \land \textcolor{\concolor}{\catvariableof{0,1,0,1,1}} \models \lnot\catvariableof{0,1,0,1,2} \, ,
        \end{align*}
        That is, that the number $3$ is not in the cell $(0,1,0,1)$.
        This entailment step is performed by three consecutive messages (see $\messagesymbol^{(0,[3])}$ in \figref{fig:contractionPropagationSudoku}) along the directions %involving decomposition cores of the position constraint in the position $(r_0,r_1,c_0,c_1)=(0,0,0,0)$.
        \begin{align*}
        (\sedge,\redge)
            \in \big[&(\{\catvariableof{0,1,0,1,1}\},\{\catvariableof{0,1,0,1,1},\decvariableof{0,1,0,1,:}\}),
                (\{\catvariableof{0,1,0,1,1},\decvariableof{0,1,0,1,:}\},\{\catvariableof{0,1,0,1,2},\decvariableof{0,1,0,1,:}\}), \\
                &(\{\catvariableof{0,1,0,1,2},\decvariableof{0,1,0,1,:}\},\{\catvariableof{0,1,0,1,2},\decvariableof{0,:,0,:,2}\})\big] \, .
        \end{align*}
        Intuitively, the messages commmunicate to the square constraint $\decvariableof{0,:,0,:,2}$, that by the position constraint $\decvariableof{0,1,0,1,:}$ the variable $3$ cannot be assigned at $(0,1,0,1)$.
        %This entailment step is performed by two consecutive messages along the directions $(\catvariableof{0,1,0,1,1},\decvariableof{0,1,0,1,:})$ and $(\decvariableof{0,1,0,1,:},\catvariableof{0,1,0,1,2})$.
        \item From $\textcolor{\concolor}{\catvariableof{0,0,1,0,2}}$ (i.e. the $3$ in the cell $(0,0,1,0)$) and the Sudoku rule that at the row $(0,0)$ exactly one number is assigned, we get
        \begin{align*}
            \left( \woneoplus_{c0,c1\in[\sudokunum]} \catvariableof{0,0,c0,c1,2} \right) \land \textcolor{\concolor}{\catvariableof{0,0,1,0,2}} \models \lnot\catvariableof{0,0,0,0,2}\land \lnot\catvariableof{0,0,0,1,2} \, ,
        \end{align*}
        meaning that the number $3$ is neither in the cell $(0,0,0,0)$ nor in $(0,0,0,1)$.
        This entailment step is performed by five consecutive messages (see $\messagesymbol^{(1,[5])}$ in \figref{fig:contractionPropagationSudoku}) along the directions
        %This entailment step is performed by three consecutive messages along the directions $(\catvariableof{0,0,1,0,2},\decvariableof{0,0,:,:,2})$, $(\decvariableof{0,0,:,:,2},\catvariableof{0,0,0,0,2})$ and $(\decvariableof{0,0,:,:,2},\catvariableof{0,0,0,1,2})$.
        \begin{align*}
        (\sedge,\redge)
            \in \big[
            &(\{\catvariableof{0,0,1,0,2}\},\{\catvariableof{0,0,1,0,2},\decvariableof{0,0,:,:,2}\}),
                (\{\catvariableof{0,0,1,0,2},\decvariableof{0,0,:,:,2}\},\{\catvariableof{0,0,0,0,2},\decvariableof{0,0,:,:,2}\}), \\
                &(\{\catvariableof{0,0,1,0,2},\decvariableof{0,0,:,:,2}\},\{\catvariableof{0,0,0,1,2},\decvariableof{0,0,:,:,2}\}),
                (\{\catvariableof{0,0,0,0,2},\decvariableof{0,0,:,:,2}\},\{\catvariableof{0,0,0,0,2},\decvariableof{0,:,0,:,2}\}) \\
                &(\{\catvariableof{0,0,0,1,2},\decvariableof{0,0,:,:,2}\},\{\catvariableof{0,0,0,1,2},\decvariableof{0,:,0,:,2}\})
                \big] \, .
        \end{align*}
        The messages communicate that based on the decomposition cores of the constraint to the number $i=3$ in the first row $(r_0,r_1)=(0,0)$, the number $3$ cannot be assigned at $(0,0,0,0)$ and $(0,0,0,1)$.
    \end{itemize}
    We add these formulas to our knowledge base (justified by \theref{the:addingEntailed}) and use the rule that $3$ appears exactly once in the first square
    \begin{align*}
        &\left( \woneoplus_{r1,c1\in[\sudokunum]} \catvariableof{0,r1,0,c1,2} \right)
        \land (\lnot\catvariableof{0,1,0,1,2})
        \land (\lnot\catvariableof{0,0,0,0,2}\land \lnot\catvariableof{0,0,0,1,2})
        \models \textcolor{\probcolor}{\catvariableof{0,1,0,0,2}} \, .
    \end{align*}
    We hence conclude that the number $3$ must be in the cell $(0,1,0,0)$.
    This information is also included in the updated knowledge base for further reasoning steps.
    This last entailment step is performed by four consecutive messages (see $\messagesymbol^{(2,[4])}$ in \figref{fig:contractionPropagationSudoku}) along the directions
    \begin{align*}
    (\sedge,\redge)
        \in \big[
            &(\{\catvariableof{0,1,0,1,2},\decvariableof{0,:,0,:,2}\},\{\catvariableof{0,1,0,0,2},\decvariableof{0,:,0,:,2}\}),
            (\{\catvariableof{0,0,0,1,2},\decvariableof{0,:,0,:,2}\},\{\catvariableof{0,1,0,0,2},\decvariableof{0,:,0,:,2}\}), \\
            &(\{\catvariableof{0,0,1,0,2},\decvariableof{0,:,0,:,2}\},\{\catvariableof{0,1,0,0,2},\decvariableof{0,:,0,:,2}\}),
            (\{\catvariableof{0,1,0,0,2},\decvariableof{0,:,0,:,2}\},\{\catvariableof{0,1,0,0,2}\})\big]
    \end{align*}
    The first three messages communicate that the $3$ is not possible in positions $(0,1,0,1),(0,0,0,1)$ and $(0,0,1,0)$ and the fourth message concludes that the $3$ then has to be at position $(0,1,0,0)$.

    %% Further reasoning steps
    We now iteratively apply similar reasoning steps and store the entailed variables in \textcolor{\probcolor}{$E^{\mathrm{entailed}}$} until we arrive at the right side of the above sketch.
    \begin{align*}
        \sudokukbof{2} \land \left(\bigwedge_{(r_0,r_1,c_0,c_1,i)\in \sudokustartevidence} \catvariableof{r0,r1,c0,c1,i} \right)
        \models \textcolor{\probcolor}{\left(\bigwedge_{(r_0,r_1,c_0,c_1,i)\in E^{\mathrm{entailed}}} \catvariableof{r0,r1,c0,c1,i} \right)} \, .
    \end{align*}
    Since all Sudoku rules are satisfied in the final assignment and to each cell $(r_0,r_1,c_0,c_1)$ we found exactly one $i\in[\sudokunum^2]$ such that $(r_0,r_1,c_0,c_1,i)\in \sudokustartevidence\cup\textcolor{\probcolor}{E^{\mathrm{entailed}}}$, there is a unique solution of the puzzle and we conclude that
    \begin{align*}
        &\sudokukbof{2} \land \left(\bigwedge_{(r_0,r_1,c_0,c_1,i)\in \sudokustartevidence} \catvariableof{r0,r1,c0,c1,i} \right) \\
        &\quad= \left(\bigwedge_{(r_0,r_1,c_0,c_1,i)\in \sudokustartevidence} \catvariableof{r0,r1,c0,c1,i} \right)
        \land \textcolor{\probcolor}{\left(\bigwedge_{(r_0,r_1,c_0,c_1,i)\in E^{\mathrm{entailed}}} \catvariableof{r0,r1,c0,c1,i} \right)} \, .
    \end{align*}
\end{example}

\begin{figure}[t]
    \begin{center}
        \begin{tikzpicture}[scale=0.35,thick]

            %% I 0:0:2 constraint (3 is in the 00 square)
            \draw (-1,-1) rectangle (1,1);
            \node[anchor=center] (A) at (0,0) {\corelabelsize $\hypercoreof{0}$};
            \draw (0,-1)--(0,-3) node[midway,right] {\colorlabelsize $\catvariableof{0,0,0,0,2}$};

            \draw (3,-1) rectangle (5,1);
            \node[anchor=center] (A) at (4,0) {\corelabelsize $\hypercoreof{1}$};
            \draw (4,-1)--(4,-3) node[midway,right] {\colorlabelsize $\catvariableof{0,0,0,1,2}$};

            \draw (7,-1) rectangle (9,1);
            \node[anchor=center] (A) at (8,0) {\corelabelsize $\hypercoreof{2}$};
            \draw (8,-1)--(8,-3); % node[midway,right] {\colorlabelsize $\catvariableof{0,1,0,0,2}$};

            \drawvariabledot{8}{-3}
            \draw (8,-3) -- (7.25,-3);
            \draw[\probcolor] (7,-5) rectangle (9,-7);
            \node[anchor=center, \probcolor] (A) at (8,-6) {\corelabelsize $\tbasis$};
            \draw[\probcolor] (8,-5)--(8,-3) node[midway,right] {\colorlabelsize $\catvariableof{0,1,0,0,2}$};

            \draw[\newmessagecolor,dashed, ->] (6.5,-1) to [bend right = 30] (6.5,-5);
            \node[\newmessagecolor,anchor=center] (A) at (5.25,-4) {\colorlabelsize $\messagesymbol^{(2,3)}$};

            \draw[\newmessagecolor,dashed, ->] (11,1.25) to [bend right = 30] (9,1.25);
            \node[\newmessagecolor,anchor=center] (A) at (9.75,2) {\colorlabelsize $\messagesymbol^{(2,0)}$};

            \draw[\newmessagecolor,dashed, ->] (5,1.5) to [bend right = -40] (6.8,1.1);
            \node[\newmessagecolor,anchor=center] (A) at (6,0.75) {\colorlabelsize $\messagesymbol^{(2,1)}$};

            \draw[\newmessagecolor,dashed, ->] (1,1.5) to [bend right = -40] (7,1.75);
            \node[\newmessagecolor,anchor=center] (A) at (3.5,2) {\colorlabelsize $\messagesymbol^{(2,2)}$};


            \draw (11,-1) rectangle (13,1);
            \node[anchor=center] (A) at (12,0) {\corelabelsize $\hypercoreof{3}$};
            \draw (12,-1)--(12,-2.5) node[midway,right] {\colorlabelsize $\catvariableof{0,1,0,1,2}$};

            \drawvariabledot{6}{4}
            \node[anchor=south] (text) at (6,4) {\colorlabelsize $\decvariableof{0,:,0,:,2}$};

            \draw (6,4) to[bend right= 20] (0,1);
            \draw (6,4) to[bend right= 10] (4,1);
            \draw (6,4) to[bend right= -10] (8,1);
            \draw (6,4) to[bend right= -20] (12,1);


            %% I 00::2 constraint (3 in the first row)
            \draw (3,-7) rectangle (5,-5);
            \node[anchor=center] (A) at (4,-6) {\corelabelsize $\hypercoreof{1}$};
            \drawvariabledot{4}{-3}
            \draw (4,-3) -- (3.25,-3);
            \draw (4,-5)--(4,-1);

            \draw (-1,-7) rectangle (1,-5);
            \node[anchor=center] (A) at (0,-6) {\corelabelsize $\hypercoreof{0}$};
            \drawvariabledot{0}{-3}
            \draw (0,-3) -- (-0.75,-3);
            \draw (0,-5)--(0,-1);

            \draw (-5,-7) rectangle (-3,-5);
            \node[anchor=center] (A) at (-4,-6) {\corelabelsize $\hypercoreof{2}$};
            \draw (-4,-5)--(-4,-3);

            \drawvariabledot{-4}{-3}
            \draw (-4,-3) -- (-4.75,-3);

            \draw[\concolor] (-5,-1) rectangle (-3,1);
            \node[anchor=center, \concolor] (A) at (-4,0) {\corelabelsize $\tbasis$};
            \draw[\concolor] (-4,-1)--(-4,-3) node[midway,right] {\colorlabelsize $\catvariableof{0,0,1,0,2}$};

            %% Messages from 3 in row 0,0
            \draw[\newmessagecolor,dashed, ->] (-5.5,-1) to [bend right = 30] (-5.5,-5);
            \node[\newmessagecolor,anchor=center] (A) at (-6.75,-2) {\colorlabelsize $\messagesymbol^{(1,0)}$};

            \draw[\newmessagecolor,dashed, ->] (-3,-7.25) to [bend right = 30] (-1,-7.25);
            \node[\newmessagecolor,anchor=center] (A) at (-2,-6.75) {\colorlabelsize $\messagesymbol^{(1,1)}$};

            \draw[\newmessagecolor,dashed, ->] (-3,-7.5) to [bend right = 40] (3,-7.25);
            \node[\newmessagecolor,anchor=center] (A) at (0,-9) {\colorlabelsize $\messagesymbol^{(1,2)}$};

            \draw[\newmessagecolor,dashed, <-] (-1.5,-1) to [bend right = 30] (-1.5,-5);
            \node[\newmessagecolor,anchor=center] (A) at (-2.75,-4) {\colorlabelsize $\messagesymbol^{(1,3)}$};

            \draw[\newmessagecolor,dashed, <-] (2.5,-1) to [bend right = 30] (2.5,-5);
            \node[\newmessagecolor,anchor=center] (A) at (1.25,-4) {\colorlabelsize $\messagesymbol^{(1,4)}$};

            \draw (-9,-7) rectangle (-7,-5);
            \node[anchor=center] (A) at (-8,-6) {\corelabelsize $\hypercoreof{3}$};
            \draw (-8,-5)--(-8,-3) node[midway,left] {\colorlabelsize $\catvariableof{0,0,1,1,2}$};

            \drawvariabledot{-2}{-10}
            \node[anchor=north] (text) at (-2,-10) {\colorlabelsize $\decvariableof{0,0,:,:,2}$};

            \draw (-2,-10) to[bend right= -20] (-8,-7);
            \draw (-2,-10) to[bend right= -10] (-4,-7);
            \draw (-2,-10) to[bend right= 10] (0,-7);
            \draw (-2,-10) to[bend right= 20] (4,-7);

            % I 0001: (position 0001)
            \draw (11,-7) rectangle (13,-5);
            \node[anchor=center] (A) at (12,-6) {\corelabelsize $\hypercoreof{2}$};
            \draw (12,-5)--(12,-2.5);

            \drawvariabledot{12}{-3}
            \draw (12,-3) -- (12.5,-3);

            \draw (15,-7) rectangle (17,-5);
            \node[anchor=center] (A) at (16,-6) {\corelabelsize $\hypercoreof{0}$};
            \draw (16,-5)--(16,-3) node[midway,right] {\colorlabelsize $\catvariableof{0,1,0,1,0}$};

            \draw (19,-7) rectangle (21,-5);
            \node[anchor=center] (A) at (20,-6) {\corelabelsize $\hypercoreof{1}$};
            \draw (20,-5)--(20,-3) node[midway,right] {\colorlabelsize $\catvariableof{0,1,0,1,1}$};

            \draw[\concolor] (19,-1) rectangle (21,1);
            \node[anchor=center, \concolor] (A) at (20,0) {\corelabelsize $\tbasis$};
            \draw[\concolor] (20,-1)--(20,-3) node[midway,right] {\colorlabelsize $\catvariableof{0,1,0,1,1}$};

            \drawvariabledot{20}{-3}
            \draw (20,-3) -- (20.5,-3);

            %% Messages from 2 at position 2,2,2,2
            \draw[\newmessagecolor,dashed, ->] (18.5,-1) to [bend right = 30] (18.5,-5);
            \node[\newmessagecolor,anchor=center] (A) at (17.25,-2) {\colorlabelsize $\messagesymbol^{(0,0)}$};

            \draw[\newmessagecolor,dashed, ->] (19,-7.5) to [bend right = -40] (13,-7.25);
            \node[\newmessagecolor,anchor=center] (A) at (16,-9) {\colorlabelsize $\messagesymbol^{(0,1)}$};

            \draw[\newmessagecolor,dashed, <-] (13.5,-1) to [bend right = -30] (13.5,-5);
            \node[\newmessagecolor,anchor=center] (A) at (14.9,-4) {\colorlabelsize $\messagesymbol^{(0,2)}$};

            \draw (23,-7) rectangle (25,-5);
            \node[anchor=center] (A) at (24,-6) {\corelabelsize $\hypercoreof{3}$};
            \draw (24,-5)--(24,-3) node[midway,right] {\colorlabelsize $\catvariableof{0,1,0,1,3}$};

            \drawvariabledot{18}{-10}
            \node[anchor=north] (text) at (18,-10) {\colorlabelsize $\decvariableof{0,1,0,1,:}$};

            \draw (18,-10) to[bend right= -20] (12,-7);
            \draw (18,-10) to[bend right= -10] (16,-7);
            \draw (18,-10) to[bend right= 10] (20,-7);
            \draw (18,-10) to[bend right= 20] (24,-7);

        \end{tikzpicture}
    \end{center}
    \caption{
        The tensor network decomposition of $3$ out of $4\cdot2^2=64$ rules in the $2^2\times2^2$ Sudoku knowledge base (see \exaref{exa:sudokuDecomposition}),  namely to the number $3$ appearing once in the $(0,0)$-square (top), the number $3$ appearing once in the $(0,0)$-row (bottom left) and a unique number appearing at the $(0,1,0,1)$-position (bottom right).
        The evidence of the number $3$ already being assigned at the position $(0,0,1,0)$ is sketched by a basis vector $\textcolor{\concolor}{\tbasis}$ on the left side, and the number $2$ assigned at position $(0,1,0,1)$ analogously on the right side.
        During Constraint Propagation \algoref{alg:constraintPropagation} on the hypergraph of Sudoku rules and evidence (see \exaref{exa:sudokuMessagePassing}), this evidence is in three epochs of messages propagated to the constraints by partial entailment steps and imply that $\textcolor{\probcolor}{\catvariableof{0,1,0,0,2}}$ is true, i.e. that at the position $(0,1,0,0)$ the number $3$ needs to be assigned.
        We depict the messages between the cores by dashed lines labeled by $\messagesymbol^{(0,[3])},\messagesymbol^{(1,[5])}$ and $\messagesymbol^{(2,[4])}$ and provide further interpretation in \exaref{exa:sudokuMessagePassing}.
    }\label{fig:contractionPropagationSudoku}
\end{figure}

