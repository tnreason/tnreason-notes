\section{The Neural Paradigm}\label{sec:neurPar}

The neural paradigm of artificial intelligence exploits the decomposition of functions into neurons, which are aligned in a directed acyclic graph.
We show in this section how functions decomposeable into neurons can be represented by tensor networks.
To this end we formalize discrete neural models in decomposition graphs and formally proof the corresponding decomposition of their basis encodings.
Particular examples will be presented in the next section by propositional formulas with sparse syntactical descriptions.

\subsection{Function Decomposition}

As a main principle of tensor decompositions we now show that basis encodings of composition functions is the contraction of basis encodings to the components.

\begin{lemma}
    \label{lem:formulaDecomp}
    Let $\formulaat{\shortcatvariables}$ be a composition of a $\seldim$-ary connective function $\chainingfunction$ and functions $\formulaofat{\selindex}{\shortcatvariables}$, where $\selindexin$, i.e. for $\shortcatindices\in\atomstates$ we have
    \begin{align*}
        \formula(\shortcatindices)
        = \chainingfunction\left(\formulaofat{0}{\shortcatindices}, \dots, \formulaofat{\seldim-1}{\shortcatindices}\right) \, .
    \end{align*}
    Then we have (see \figref{fig:functionDecomposition})
    \begin{align*}
        \bencodingofat{\formula}{\headvariableof{\formula},\shortcatvariables}
        = \contractionof{
            \{\bencodingofat{\chainingfunction}{\headvariableof{\formula},\headvariables}\}
            \cup \{\bencodingofat{\formulaof{\selindex}}{\headvariableof{\selindex},\shortcatvariables} \wcols \selindexin\}
        }{\headvariableof{\formula},\shortcatvariables} \, .
    \end{align*}
\end{lemma}
\begin{proof}
    For any $\shortcatindicesin$ we have
    \begin{align*}
        &\contractionof{
            \{\bencodingofat{\chainingfunction}{\headvariableof{\formula},\headvariables}\}
            \cup \{\bencodingofat{\formulaof{\selindex}}{\headvariableof{\selindex},\shortcatvariables} \wcols \selindexin\}
        }{\headvariableof{\formula},\indexedshortcatvariables} \\
        &\quad\quad= \contractionof{
            \{\bencodingofat{\chainingfunction}{\headvariableof{\formula},\headvariables}\}
            \cup \{\bencodingofat{\formulaof{\selindex}}{\headvariableof{\selindex},\indexedshortcatvariables} \wcols \selindexin\}
        }{\headvariableof{\formula}} \\
        &\quad\quad= \contractionof{
            \{\bencodingofat{\chainingfunction}{\headvariableof{\formula},\headvariables}\}
            \cup \{\onehotmapofat{\formulaof{\selindex}(\shortcatindices)}{\headvariableof{\selindex}} \wcols \selindexin\}
        }{\headvariableof{\formula}} \\
        &\quad\quad= \onehotmapofat{\formula(\shortcatindices)}{\headvariableof{\formula}}\\
        &\quad\quad= \bencodingofat{\formula}{\headvariableof{\formula},\indexedshortcatvariables} \, .
    \end{align*}
    Thus the tensors on both sides of the equation coincide in all slides to $\shortcatvariables$ and are thus equal.
\end{proof}

\begin{figure}[t]
    \begin{center}
        \begin{tikzpicture}[scale=0.4, thick]
    \begin{scope}

        % \node[anchor=center] (text) at (-6,-4) {$b)$};

        \begin{scope}
        [shift={(-8,0)}]
            \draw[->-] (5.5,-9)--(5.5,-7) node[midway,right] {\colorlabelsize $\headvariableof{\exformula}$};
            \drawatomcore{3.5}{-8}{$\bencodingof{\exformula}$}
            \drawatomindices{3.5}{-12}
        \end{scope}

        \node[anchor=center] (text) at (1.5,-10) {${=}$};

        \draw[->-] (9.5,-5)--(9.5,-3) node[midway,right] {\colorlabelsize $\headvariableof{\exformula}$};

        \node[anchor=center] (text) at (9.5,-6) {$\bencodingof{\chainingfunction}$};
        \draw (4.5,-7) rectangle (14.5,-5);

        \draw[->-] (5.5,-9)--(5.5,-7) node[midway,right] {\colorlabelsize $\headvariableof{0}$};

        \node[anchor=center] at (9.5,-8) {$\cdots$};

        \drawatomcore{3.5}{-8}{$\bencodingof{\formulaof{0}}$}
        \drawatomindices{3.5}{-12}

        \begin{scope}
        [shift={(8,0)}]

            \draw[->-] (5.5,-9)--(5.5,-7) node[midway,right] {\colorlabelsize $\headvariableof{\seldim-1}$};

            \drawatomcore{3.5}{-8}{$\bencodingof{\formulaof{\seldim-1}}$}
            \drawatomindices{3.5}{-12}

        \end{scope}

        \draw[fill] (7.5,-15) circle (\dotsize);
        \draw[] (7.5,-15) to[bend left=25] (3.5,-13);
        \draw[] (7.5,-15) to[bend right=25] (11.5,-13);

        \draw[fill] (9,-15.25) circle (\dotsize);
        \draw[] (9,-15.25) to[bend left=25] (5,-13);
        \draw[] (9,-15.25) to[bend right=25] (13,-13);

        \draw[fill] (11.5,-15) circle (\dotsize);
        \draw[] (11.5,-15) to[bend left=25] (7.5,-13);
        \draw[] (11.5,-15) to[bend right=25] (15.5,-13);



        \draw[] (7.5,-15)--(7.5,-17) node[midway,left] {\colorlabelsize $\catvariableof{0}$};
        \draw[] (9,-15.25)--(9,-17) node[midway,left] {\colorlabelsize $\catvariableof{1}$};
        \node[anchor=center] (text) at (10.5,-16.5) {$\cdots$};
        \draw[] (11.5,-15)--(11.5,-17) node[midway,right] {\colorlabelsize $\catvariableof{\atomorder-1}$};

    \end{scope}
\end{tikzpicture}
    \end{center}
    \caption{Tensor network decomposition of a the basis encoding of a function $\exformula$, which is the composition of the functions $\formulaof{0},\ldots,\formulaof{\seldim-1}$ with a function $\chainingfunction$.}
    \label{fig:functionDecomposition}
\end{figure}


Let us now define a more generic decomposition of discrete functions.

\begin{definition}
    \label{def:decompositionHypergraph}
    A \emph{decomposition hypergraph} is a directed acyclic hypergraph $\graph=(\nodes,\edges)$ such that
    \begin{itemize}
        \item Each node $\nodein$ is decorated by a set $\arbsetof{\node}$ of finite cardinality $\catdimof{\node}$, a variable $\catvariableof{\node}$ and an index interpretation function
        \begin{align*}
            \indexinterpretationof{\node} \defcols [\catdimof{\node}] \rightarrow \arbsetof{\node} \, .
        \end{align*}
        \item Each directed hyperedge $(\incomingnodes,\outgoingnodes)$ has at least one outgoing node, i.e. $\outgoingnodes\neq\varnothing$ and is decorated by an activation function % propositional formula
        \begin{align*}
            \secexfunctionof{\edge} \defcols
            \bigtimes_{\node\in\incomingnodes} \arbsetof{\node}
            \rightarrow \bigtimes_{\node\in\outgoingnodes} \arbsetof{\node} \, .
        \end{align*}
        \item Each node $\nodein$ appears at most once as an outgoing node. % Well-defined function
        \item The nodes not appearing as an outgoing node are enumerated by $\node^{\insymbol}_{[\atomorder]}$.
        We abbreviate the corresponding variables by $\catvariableof{\node^{\insymbol}_{[\atomorder]}}=\shortcatvariables$. % and labeled by $\nodes^{\insymbol}_{[\atomorder]}$.
        \item The nodes not appearing as an incoming node are enumerated by $\node^{\outsymbol}_{[\seldim]}$.
        We abbreviate the variables by $\catvariableof{\node^{\outsymbol}_{[\selindex]}}=\headvariables$. %$[\seldim]$ are labeled by $\nodes^{\outsymbol}_{[\seldim]}$.
    \end{itemize}
    We assign for each $\catenumeratorin$ restriction functions
    \begin{align*}
        \restrictionofto{\cdot}{\node^{\insymbol}_\catenumerator}
        \defcols \bigtimes_{\seccatenumerator\in[\catorder]} \arbsetof{\node^{\insymbol}_{\seccatenumerator}} \rightarrow \arbsetof{\node^{\insymbol}_\catenumerator}
        \quad,\quad  \restrictionofto{\catindexof{[\catorder]}}{\catenumerator} = \catindexof{\catenumerator}
    \end{align*}
    to the nodes $\node^{\insymbol}_{[\atomorder]}$ and recursively assign to each further node $\node$ a node function % connective $\connectiveofat{\node}{\headvariableof{\incomingnodes}}$.
    \begin{align*}
        \exfunctionof{\node} \wcols \bigtimes_{\catenumeratorin} \arbsetof{\node^{\insymbol}_\catenumerator} \rightarrow \bigtimes_{\node\in\outgoingnodes} \arbsetof{\node}
        \quad,\quad
        \exfunctionof{\node}(\catindexof{[\catorder]})
        = \restrictionofto{\secexfunctionof{\edge^\node}
        \left(
            \bigtimes_{\node\in\incomingnodes}\exfunctionof{\node}(\catindexof{[\catorder]})
        \right)}{\node} \, ,
    \end{align*}
    where $\edgeof{\node}$ is to each $\node\in\incomingnodes$ the unique hyperedge with outgoing nodes $\{\node\}$.
    We then call the function
    \begin{align*}
        \exfunctionof{\graph} \wcols \bigtimes_{\catenumeratorin} \arbsetof{\node^{\insymbol}_\catenumerator} \rightarrow \bigtimes_{\selindexin} \arbsetof{\node^{\outsymbol}_\selindex}
        \quad,\quad
        \exfunctionof{\graph} = \bigtimes_{\selindexin} \exfunctionof{\node^{\outsymbol}_\selindex}
    \end{align*}
    the composition formula to the decomposition hypergraph $\graph$.
    %We call the formula $\exformulaat{\shortcatvariables}\coloneqq\formulaofat{\secnode}{\shortcatvariables}$ to the root note $\secnode$ the syntactical composition of $\graph$ and $\graph$ is a syntactical decomposition of $\exformula$.
\end{definition}

% Neural Paradigm
The neural paradigm in AI can be modelled by the existence of decomposition hypergraphs for functions on large sets.
Let us now show how decomposition hypergraphs enable the sparse representation of composition functions by tensor networks.

\begin{theorem}
    \label{the:functionDecompositionRep}
%    \red{Merge into the message passing?}
    For any decomposition hypergraph $\graph$ with composition formula $\exfunctionof{\graph}$ we have
    \begin{align*}
        \bencodingofat{\exfunctionof{\graph}}{\headvariables,\shortcatvariables}
        = \contractionof{\left\{\bencodingofat{\secexfunctionof{\edge}}{\catvariableof{\outnodes},\catvariableof{\innodes}} \wcols \edge=(\innodes,\outnodes)\in\edges\right\}
        }{\headvariables,\shortcatvariables} \, .
    \end{align*}
\end{theorem}
\begin{proof}
    The claim follows by induction from the leafs to the root and iteratively applying \lemref{lem:formulaDecomp}.
\end{proof}

% Weights
When neurons have tunable parameters, we can discretize those by sets $\arbsetof{\catenumerator}$ and understand them as additional input variables.


\begin{example}[Sum of integers in $\catdim$-adic representation]
    \label{exa:madicRepresentation}
    Let us develop a tensor network representation of integer summations on the set $[\catdim^{\catorder}]=\{0,\ldots,\catdim^{\catorder}-1\}$, where $\catdim,\catorder\in\nn$,
    \begin{align*}
        + \defcols [\catdim^\catorder] \times [\catdim^\catorder] \rightarrow [\catdim^{\catorder+1}]
        \quad,\quad
        +(i,\tilde{i}) = i+\tilde{i} \,
    \end{align*}
    which have a $\catdim$-adic representation of length $\catorder$.
    We define an index interpretation map
    \begin{align*}
        \indexinterpretation \defcols \bigtimes_{\catenumeratorin}[\catdim] \rightarrow [\catdim^{\catorder}]
        \quad,\quad
        \indexinterpretationat{\shortcatindices}
        = \sum_{\catenumeratorin} \catindexof{\catenumerator} \cdot \catdim^{\catenumerator} \, ,
    \end{align*}
    which enables the parameterization of $[\catdim^{\catorder}]$ as the states of $\catorder$ categorical variables $\shortcatvariables$ of dimension $\catdim$.
    We analogously represent a second set $[\catdim^{\catorder}]$ by variables $\tildecatvariableof{[\catorder]}$ and the set $[\catdim^{\catorder+1}]$ of possible sums by $\headvariableof{[\catorder+1]}$.
    The basis encoding of the sum is then
    \begin{align*}
        \bencodingofat{+}{\headvariableof{[\catorder+1]},\shortcatvariables,\seccatvariableof{[\catorder]}}
        = \sum_{\shortcatindices,\tildecatindexof{[\catorder]}} \onehotmapofat{\invindexinterpretationat{
            \indexinterpretationat{\shortcatindices} + \indexinterpretationat{\tildecatindexof{[\catorder]}}
        }}{\headvariableof{[\catorder+1]}}
        \otimes \onehotmapofat{\shortcatindices}{\shortcatvariables}
        \otimes \onehotmapofat{\tildecatindexof{[\catorder]}}{\tildecatvariableof{[\catorder]}} \, .
    \end{align*}
    We notice that the tensor space of $\bencodingof{+}$ is of dimension $\catdim^{3\cdot\catorder+1}$ increasing exponentially in $\catorder$.
    Feasible representation of this tensor for large $\catorder$ therefore requires tensor network decompositions, which we now provide based on a decomposition hypergraph.
    The targeted function to be decomposed is the representation of the integer sum by
    \begin{align*}
        \marysumsymbol \defcols \left(\bigtimes_{\catenumeratorin}[\catdim]\right) \times \left(\bigtimes_{\catenumeratorin}[\catdim]\right) \rightarrow \bigtimes_{\catenumerator\in[\catorder+1]}[\catdim]
        \quad,\quad
        \marysumsymbol(\shortcatindices,\tildecatindexof{[\catorder]}) =
        \invindexinterpretationat{\indexinterpretationat{\shortcatindices}+\indexinterpretationat{\tildecatindexof{[\catorder]}}} \, .
    \end{align*}
    We build a decomposition hypergraph $\graph=(\nodes,\edges)$ (see \defref{def:decompositionHypergraph}) consistent in $4\cdot \catorder$ nodes (see \figref{fig:decompositionMarySum}a) .
    The nodes carry the $(3\cdot\catorder +1)$ variables $\shortcatvariables,\seccatvariableof{[\catorder]},\headvariableof{[\catorder+1]}$ of dimension $\catdim$ constructed above and $\catorder-1$ auxiliary variables $\thirdcatvariableof{[\catorder-1]}$ of dimension $2$ representing carry bits.
    The directed hyperedges of $\graph$ are
    \begin{align*}
        \edges
        =&\left\{(\{\catvariableof{0},\tildecatvariableof{0}\},\{\headvariableof{0},\thirdcatvariableof{0}\})\right\}
        \cup \left\{(\{\thirdcatvariableof{\catenumerator-1},\catvariableof{\catenumerator},\tildecatvariableof{\catenumerator}\},\{\headvariableof{\catenumerator},\thirdcatvariableof{\catenumerator}\})
                 \wcols \catenumerator\in\{1,\ldots,\catorder-2\}\right\} \\
        &\cup \left\{(\{\thirdcatvariableof{\catorder-2},\catvariableof{\catorder-1},\tildecatvariableof{\catorder-1}\},\{\headvariableof{\catorder-1},\headvariableof{\catorder}\})\right\}
    \end{align*}
    and are decorated by local summation functions
    \begin{align*}
        \modsumsymbol \defcols [2] \times [\catdim] \times [\catdim] \rightarrow [\catdim] \times [2]
        \quad,\quad
        \modsumsymbol(\thirdcatindex,\catindex,\tildecatindex)
        = \left((\thirdcatindex + \catindex + \tildecatindex) \modspace\catdim,
              \left\lfloor\frac{\thirdcatindex + \catindex + \tildecatindex}{\catdim}\right\rfloor\right) \, .
    \end{align*}
    Since to the first hyperedge we do not have a carry bit, the decorating function is the restriction of the first argument to $0$.

    It is known that the composition of the local summations $\modsumsymbol$ is the global summation $\marysumsymbol$ of integers in $\catdim$-adic representation.
    Thus, the composition function $\exfunctionof{\graph}$ is $\marysumsymbol$.
    By \theref{the:functionDecompositionRep} we have a decomposition of the basis encoding to $\exfunctionof{\graph}$ (see \figref{fig:decompositionMarySum}b) as
    \begin{align*}
        \bencodingofat{\marysumsymbol}{\headvariableof{[\catorder+1]},\shortcatvariables,\tildecatvariableof{[\catorder]}}
        = \breakablecontractionof{
            &\{\bencodingofat{\modsumsymbol,0}{\headvariableof{0},\thirdcatvariableof{0},\catvariableof{0},\tildecatvariableof{0}}\} \cup \\
            &\{\bencodingofat{\modsumsymbol,\catenumerator}{
            \headvariableof{\catenumerator},\thirdcatvariableof{\catenumerator},\catvariableof{\catenumerator},\tildecatvariableof{\catenumerator},\thirdcatvariableof{\catenumerator-1}
            }\wcols \catenumerator\in\{1,\ldots,\catorder-2\}\} \cup \\
            &\{\bencodingofat{\modsumsymbol,\catorder-2}{\headvariableof{\catorder-1},\headvariableof{\catorder},\catvariableof{\catorder-1},\tildecatvariableof{\catorder-1},\thirdcatvariableof{\catorder-2}}\}
        }{\headvariableof{[\catorder+1]},\shortcatvariables,\tildecatvariableof{[\catorder]}} \, .
    \end{align*}

    \begin{figure}[t]
        \begin{center}
            \begin{tikzpicture}[scale=0.35, thick]

                \begin{scope}[shift={(-0.5,10)}]

                    \node[anchor=east]  at (-9,4) {$a)$};

                    \node (of) at (0,1.35) {};

                    \draw[thick, dashed, rounded corners=10pt]  ($(-5,3)+(of)$) -- ($(-5,3)-(of)$)  -- ($(28.5,3)-(of)$) -- ($(28.5,3)+(of)$) -- cycle;
                    \node[anchor=center] (A) at (-3,3) {\corelabelsize $\node^{\outsymbol}$};

                    \draw[thick, dashed, rounded corners=10pt]  ($(-5,-3)+(of)$) -- ($(-5,-3)-(of)$)  -- ($(28.5,-3)-(of)$) -- ($(28.5,-3)+(of)$) -- cycle;
                    \node[anchor=center] (A) at (-3,-3) {\corelabelsize $\node^{\insymbol}$};

                    \node[circle, draw, thick, fill=\nodegrayscale, minimum size = \nodeminsize] (Y0) at (1.5,3) {};
                    \node[] (text) at (1.5,3) {\corelabelsize $\headvariableof{0}$};
                    \node[circle, draw, thick, fill=\nodegrayscale, minimum size = \nodeminsize] (X00) at (0,-3) {};
                    \node[] (text) at (0,-3) {\corelabelsize $\catvariableof{0}$};
                    \node[circle, draw, thick, fill=\nodegrayscale, minimum size = \nodeminsize] (X10) at (3,-3) {};
                    \node[] (text) at (3,-3) {\corelabelsize $\tildecatvariableof{0}$};
                    \node[circle, draw, thick, fill=\nodegrayscale, minimum size = \nodeminsize] (Z0) at (5,0) {};
                    \node[] (text) at (5,0) {\corelabelsize $\thirdcatvariableof{0}$};

                    \coordinate (m0) at (1.5,0);
                    \node[anchor=east]  at (1.5,0) {$\edgeof{0}$};
                    \draw[->-] (X00) -- (m0);
                    \draw[->-] (X10) -- (m0);
                    \draw[->-] (m0) -- (Y0);
                    \draw[->-] (m0) -- (Z0);

                    \node[circle, draw, thick, fill=\nodegrayscale, minimum size = \nodeminsize] (Y1) at (8.5,3) {};
                    \node[] (text) at (8.5,3) {\corelabelsize $\headvariableof{1}$};
                    \node[circle, draw, thick, fill=\nodegrayscale, minimum size = \nodeminsize] (X01) at (7,-3) {};
                    \node[] (text) at (7,-3) {\corelabelsize $\catvariableof{1}$};
                    \node[circle, draw, thick, fill=\nodegrayscale, minimum size = \nodeminsize] (X11) at (10,-3) {};
                    \node[] (text) at (10,-3) {\corelabelsize $\tildecatvariableof{1}$};
                    \node[circle, draw, thick, fill=\nodegrayscale, minimum size = \nodeminsize] (Z1) at (12,0) {};
                    \node[] (text) at (12,0) {\corelabelsize $\thirdcatvariableof{1}$};

                    \coordinate (m1) at (8.5,0);
                    \node[anchor=east]  at (8.5,0.75) {$\edgeof{1}$};
                    \draw[->-] (Z0) -- (m1);
                    \draw[->-] (X01) -- (m1);
                    \draw[->-] (X11) -- (m1);
                    \draw[->-] (m1) -- (Y1);
                    \draw[->-] (m1) -- (Z1);

                    \draw[->-] (Z1) -- (15,0) node[anchor=west]{$\cdots$};

                    \begin{scope}[shift={(15,0)}]
                        \node[circle, draw, thick, fill=\nodegrayscale, minimum size = \nodeminsize] (Z0) at (5,0) {};
                        \node[] (text) at (5,0) {\corelabelsize $\thirdcatvariableof{\catorder\shortminus2}$};

                        \node[circle, draw, thick, fill=\nodegrayscale, minimum size = \nodeminsize] (Y1) at (8.5,3) {};
                        \node[] (text) at (8.5,3) {\corelabelsize $\headvariableof{\catorder\shortminus 1}$};
                        \node[circle, draw, thick, fill=\nodegrayscale, minimum size = \nodeminsize] (X01) at (7,-3) {};
                        \node[] (text) at (7,-3) {\corelabelsize $\catvariableof{\catorder\shortminus1}$};
                        \node[circle, draw, thick, fill=\nodegrayscale, minimum size = \nodeminsize] (X11) at (10,-3) {};
                        \node[] (text) at (10,-3) {\corelabelsize $\tildecatvariableof{\catorder\shortminus1}$};
                        \node[circle, draw, thick, fill=\nodegrayscale, minimum size = \nodeminsize] (Z1) at (12,3) {};
                        \node[] (text) at (12,3) {\corelabelsize $\headvariableof{\catorder}$};

                        \coordinate (me) at (8.5,0);
                        \node[anchor=east]  at (8.5,1) {$\edgeof{\catorder\shortminus 1}$};
                        \draw[->-] (2,0)  -- (Z0);
                        \draw[->-] (Z0) -- (me);
                        \draw[->-] (X01) -- (me);
                        \draw[->-] (X11) -- (me);
                        \draw[->-] (me) -- (Y1);
                        \draw[->-] (me) -- (Z1);
                    \end{scope}

                \end{scope}

                \node[anchor=east]  at (-9,4) {$b)$};

                \begin{scope}[shift={(-8,0)}]
                    \draw (-2,-1) rectangle (4,1);
                    \draw[->-] (-1.25,1)--(-1.25,2.5) node[midway,left] {\colorlabelsize $\headvariableof{0}$};
                    \draw[->-] (3.25,1)--(3.25,2.5) node[midway,right] {\colorlabelsize $\headvariableof{\catorder\shortminus1}$};
                    \node[anchor=center] (A) at (1,2.5) {\corelabelsize $\cdots$};
                    \node[anchor=center] (A) at (1,0) {\corelabelsize $\bencodingof{\exfunctionof{\graph}}$};
                    \draw[-<-] (-1.5,-1)--(-1.5,-2.5) node[midway,left] {\colorlabelsize $\catvariableof{0}$};
                    \draw[-<-] (-1,-1)--(-1,-2.5) node[midway,right] {\colorlabelsize $\tildecatvariableof{0}$};
                    \draw[-<-] (3,-1)--(3,-2.5) node[midway,left] {\colorlabelsize $\catvariableof{\catorder\shortminus1}$};
                    \draw[-<-] (3.5,-1)--(3.5,-2.5) node[midway,right] {\colorlabelsize $\tildecatvariableof{\catorder\shortminus1}$};
                    \node[anchor=center] (A) at (1,-2.5) {\corelabelsize $\cdots$};
                \end{scope}

                \node[anchor=east]  at (-1.5,0) {$=$};

                \draw (-1,-1) rectangle (3,1);
                \node[anchor=center] (A) at (1,0) {\corelabelsize $\bencodingof{\modsumsymbol,0}$};
                \draw[->-] (1,1)--(1,3) node[midway,left] {\colorlabelsize $\headvariableof{0}$};
                \draw[-<-] (0,-1)--(0,-2.5) node[midway,left] {\colorlabelsize $\catvariableof{0}$};
                \draw[-<-] (2,-1)--(2,-2.5) node[midway,right] {\colorlabelsize $\tildecatvariableof{0}$};
                \draw[->-] (3,0)--(6,0) node[midway,above] {\colorlabelsize $\thirdcatvariableof{0}$};

                \begin{scope}[shift={(7,0)}]
                    \draw (-1,-1) rectangle (3,1);
                    \node[anchor=center] (A) at (1,0) {\corelabelsize $\bencodingof{\modsumsymbol,1}$};
                    \draw[->-] (1,1)--(1,3) node[midway,left] {\colorlabelsize $\headvariableof{1}$};
                    \draw[-<-] (0,-1)--(0,-2.5) node[midway,left] {\colorlabelsize $\catvariableof{1}$};
                    \draw[-<-] (2,-1)--(2,-2.5) node[midway,right] {\colorlabelsize $\tildecatvariableof{1}$};
                    \draw[->-] (3,0)--(6,0) node[midway,above] {\colorlabelsize $\thirdcatvariableof{1}$};
                \end{scope}

                \node[anchor=center] at (15.5,0) {$\cdots$};

                \begin{scope}[shift={(22,0)}]
                    \draw[->-] (-4,0)--(-1,0) node[midway,above] {\colorlabelsize $\thirdcatvariableof{\catorder\shortminus 2}$};
                    \draw (-1,-1) rectangle (3,1);
                    \node[anchor=center] (A) at (1,0) {\corelabelsize $\bencodingof{\modsumsymbol,\catorder-1}$};
                    \draw[->-] (1,1)--(1,3) node[midway,left] {\colorlabelsize $\headvariableof{\catorder\shortminus1}$};
                    \draw[-<-] (0,-1)--(0,-2.5) node[midway,left] {\colorlabelsize $\catvariableof{\catorder\shortminus1}$};
                    \draw[-<-] (2,-1)--(2,-2.5) node[midway,right] {\colorlabelsize $\tildecatvariableof{\catorder\shortminus1}$};
                    \draw (3,0)--(4.5,0) -- (4.5,1);
                    \draw[->-] (4.5,1)--(4.5,3) node[midway,left] {\colorlabelsize $\headvariableof{\catorder}$};
                \end{scope}


            \end{tikzpicture}
        \end{center}
        \caption{Example of a decomposition hypergraph to the sum of integers (see \exaref{exa:madicRepresentation}).
        a) Hypergraph of directed edges $\edgeof{\catenumerator}$ for $\catenumeratorin$, each decorated by an integer summation $+$ preparing an index $\headvariableof{\catenumerator}$ of the resulting sum.
        b) Corresponding tensor network decomposition of the basis encoded composition function, which is the sum of integers in $\catdim$-adic representation.}
        \label{fig:decompositionMarySum}
    \end{figure}


%
%
%    We parametrize numbers by bits in fixed point representations, which are understood as categorical variables in a factored system representation.
%
%
%%\section{Modular Calculus}
%
%    We have two basic functions calculating the mod
%    \begin{align*}
%        \exfunction : \facstates \rightarrow [2]
%        \andspace \exfunctionat{\shortcatindices} = \sum_{\atomenumeratorin} \catindexof{\atomenumerator} \,\, \mathrm{mod} \,\, 2
%    \end{align*}
%    and the integer division by two
%    \begin{align*}
%        \secexfunction : \facstates \rightarrow [2]
%        \andspace \secexfunctionat{\shortcatindices} = \left\lfloor \frac{\sum_{\atomenumeratorin} \catindexof{\atomenumerator}}{2}\right\rfloor
%    \end{align*}
%
%
%%\section{Sums}
%
%    Given the bit representations of summands, we want to calculate the bit representation of their sum.
%
%%\subsection{Binary Addition}
%
%    Basis calculus of binary additon is a TT architecture, where each core performs the addition of two bits and a carry bit, producing a sum bit and a carry bit.
%
%    Addition of two numbers with $d$ bits:
%    \begin{itemize}
%        \item Bit variables of the first number: $X_{[d]}$
%        \item Bit variables of the second number: $Z_{[d]}$
%        \item Output bit variables: $Y_{[d+1]}$
%        \item Carry bit variables: $C_{[d]}$, with $C_{1} = 0$
%    \end{itemize}
%
%    The sum of any two numbers is represented by the boolean tensor
%    \begin{align*}
%        &\hypercoreat{X_{[d]},Z_{[d]},Y_{[d+1]}}
%        \coloneqq \\
%        & \quad \contractionof{\{\onehotmapofat{0}{C_0},\identityat{C_{d-1},Y_{d}}\} \cup
%        \bigcup_{k\in[d]}\{
%            \bencodingofat{\exfunction}{Y_{k},X_{k},Z_{k},C_{k-1}},
%            \bencodingofat{\secexfunction}{C_{k},X_{k},Z_{k},C_{k-1}} \}
%        }{X_{[d]},Z_{[d]},Y_{[d+1]}} \, ,
%    \end{align*}
%    where $Y_k$ and $C_k$ are the head variables of the basis encodings to $\exfunction$ and $\secexfunction$.
%    If any only if for given indices $x_{[d]},z_{[d]},y_{[d+1]}$ we have $\hypercoreat{X_{[d]}=x_{[d]},Z_{[d]}=z_{[d]},Y_{[d+1]}=y_{[d+1]}}=1$, then the by the indices $y_{[d+1]}$ represented number is the sum of the by $x_{[d]},z_{[d]}$ represented numbers.
%
%%\subsection{Generic construction}
%
%    In general, when adding more than two variables, the carry bits need to be extended to a categorical variable with more than two states.
%    Let $X^i_{[d]}$ be the $d$ bits of the $i$th number, and let $X^{[n]}_{[d]}$ be all the bit variables (i.e. $n\cdot d$ many) of the $n$ numbers.
%    Then the same construction can be done as above, with cores
%    \begin{align*}
%        \bencodingofat{\exfunction}{Y_{k},X^{[n]}_{k},C_{k-1}},
%        \bencodingofat{\secexfunction}{C_{k},X^{[n]}_{k},Z_{k},C_{k-1}}
%    \end{align*}
%    Note that $C_k$ now takes values in $m_k$ where
%    \begin{align*}
%        m_k = \left\lfloor n \cdot \frac{m_{k-1}}{2}\right\rfloor \, .
%    \end{align*}
%    Further, the result might have more than $d+1$ bits, so we need further basis encoding cores to $\exfunction$ and $\secexfunction$.
%
%
%%\section{Products}
%
%    Products of numbers are decomposable into sums involving two bit variables of the factors, that is
%    \begin{align*}
%        \sum_{k,\tilde{k} \in [d]} 2^{k+\tilde{k}} \cdot (X_k \land Z_{\tilde{k}}) \, .
%    \end{align*}
%    Reordering the sum, we obtain
%    \begin{align*}
%        \sum_{r\in[2d-1]} 2^{r} \left(\sum_{k,\tilde{k} \in [d] \wcols k +\tilde{k}=r}  X_k \land Z_{\tilde{k}} \right) \, .
%    \end{align*}
%    From this, it is obvious that the calculation can be performed in basis calculus with basis encodings of $\land,\exfunction,\secexfunction$.
%    The head variables of the $\land$ encoding are used as the summand variabled in $\exfunction$ (output: bit of the product) and $\secexfunction$ (output: carry bit).
%
%
%%\section{Application}
%
%    Any of these tensor network schemes are considered batch schemes to perform arithmetic operations.
%    Contractions of the representing basis encodings calculate the number of true input-output relations, given e.g. a restriction onto specific outputs and inputs (by adding subset encodings of the numbers of interest).
%    One application is the countdown game, when in addition parametrizing the sum/negation operations with an additional selection variable.

%\end{document}
\end{example}

\subsection{Directed Message Passing}

We now present an efficient inference algorithm based on tensor network contractions.

\begin{algorithm}[hbt!]
    \caption{Directed Belief Propagation}\label{alg:directedBeliefPropagation}
    \begin{algorithmic}
        \Require Tensor network $\extnet$ on a directed hypergraph $\graph$
        \Ensure Messages $\{\messagewith\wcols(\secsedge,\sedge)\in\dirovedges\}$
        \iosepline
        \State Prepare directed message directions
        \begin{align*}
            \dirovedges = \left\{
                              \big((\innodes_{0},\outnodes_{0}),(\innodes_{1},\outnodes_{1})\big) \wcols
                              \innodes_{0} \cap (\innodes_{1},\outnodes_{1}) = \varnothing
                              \ncond
                              \outnodes_{1} \cap (\innodes_{0},\outnodes_{0}) = \varnothing
                              \ncond
                              \outnodes_{0} \cap \innodes_{1} \neq \varnothing
            \right\}
        \end{align*}
        \State Initialize a message queue $\scheduler=\{(\secsedge,\sedge) \wcols \secsedge \quad\text{has empty incoming nodes} \}]$
        \While{$\scheduler$ not empty}
            \State Pop a $(\sedge,\redge)$ pair from $\scheduler$
            \State Update the message
            \begin{align*}
                \messagewith
                = \contractionof{\{\hypercoreofat{\sedge}{\catvariableof{\sedge}}\}
                    \cup \{\mesfromtoat{\secsedge}{\sedge}{\catvariableof{\secsedge\cap\sedge}} \wcols (\secsedge,\sedge)\in\dirovedges \ncond \secsedge\neq \redge\}
                }{\catvariableof{\sedge\cap \redge}}
            \end{align*}
            \State Update $\scheduler$ by all messages $(\redge,\thirdsedge)$ which have not yet been sent, if all messages $(\secsedge,\redge)$ have been sent.
        \EndWhile
        \State \Return Messages $\{\messagewith\wcols(\secsedge,\sedge)\in\dirovedges\}$
    \end{algorithmic}
\end{algorithm}

% Application
Let us now apply the Directed Belief Propagation Algorithm on a decomposition hypergraph, where we add hyperedges to each leaf node and assign one-hot encodings of input states.
We then show that the messages are the one-hot encodings to the evaluations of the node functions.

\begin{theorem}
    \label{the:directedBeliefPropagationExactness}
    Let $\graph$ be a decomposition graph and let us add hyperedges containing single input nodes, which are decorated by one-hot encodings.
    Then the messages computed in \algoref{alg:directedBeliefPropagation} are characterized by
    \begin{align*}
        \mesfromtowith{\sedge}{\redge}
        = \bigotimes_{\node\in\sedge\cap\redge} \onehotmapofat{\exfunctionof{\node}(\shortcatindices)}{\catvariableof{\node}} \, .
    \end{align*}
\end{theorem}
\begin{proof}
    We show the theorem inductively over the messages computed in \algoref{alg:directedBeliefPropagation}.
    The first message is sent from an input edge $\{[\catenumerator]\}$ to an edge $\edge$ of the decomposition graph and is by assumption the one-hot encoding of an input state $\onehotmapofat{\catindexof{\catenumerator}}{\catvariableof{\catenumerator}}$.

    Let us now assume, that at an arbitrary stage of the algorithm all previous messages satisfy the claimed equation.
    The message computed in the while loop is then a contraction of one-hot encodings with basis encodings and
    \begin{align*}
        \mesfromtowith{\sedge}{\redge}
        &=\contractionof{\{\bencodingofat{\secexfunctionof{\sedge}}{\catvariableof{\outnodes},\catvariableof{\innodes}}\} \cup
            \{\mesfromtowith{\secsedge}{\sedge} \wcols (\secsedge,\sedge)\in\dirovedges\}
        }{\catvariableof{\sedge\cap\redge}} \\
        &=\contractionof{\{\bencodingofat{\secexfunctionof{\sedge}}{\catvariableof{\outnodes},\catvariableof{\innodes}}\} \cup
            \{\onehotmapofat{\exfunctionof{\node}(\shortcatindices)}{\catvariableof{\node}} \wcols \node\in\innodes\}
        }{\catvariableof{\sedge\cap\redge}} \\
        &= \bigotimes_{\node\in\sedge\cap\redge} \onehotmapofat{\exfunctionof{\node}(\shortcatindices)}{\catvariableof{\node}} \, .
    \end{align*}
    Thus also the new message is tensor product of the one-hot encodings of the evaluated node functions.
    By induction, the property is therefore true for all messages.
\end{proof}

% Arbitrary DAG Hypergraph
We notice, that we can interpret any directed acyclic hypergraph, for which each node appears exactly once as an outgoing node and which is decorated by boolean and directed tensors $\extnet$.
Edges with empty incoming sets are carrying one-hot encodings of input states, and all further edges carry functions.

\begin{example}[Continuation of \exaref{exa:madicRepresentation}]%[Computation of the sum of $\catdim$-adic integers]
    \label{exa:madicPropagation}
    We now show how \algoref{alg:directedBeliefPropagation} can be exploited to compute an efficient message passing algorithm for the digits of the $\catdim$-adic sum.
    Given two numbers in $\catdim$-adic representation by the tuples $\shortcatindices$ and $\tildecatindexof{[\catorder]}$, we add the hyperedges with empty incoming nodes and single outgoing node
    \begin{align*}
        \bigcup_{\catenumeratorin}\left\{(\varnothing,\{\catvariableof{\catenumerator}\}),(\varnothing,\{\tildecatvariableof{\catenumerator}\})\right\}
    \end{align*}
    to the hypergraph of \exaref{exa:madicRepresentation} and decorate them by the digit one-hot encodings $\onehotmapofat{\catindexof{\catenumerator}}{\catvariableof{\catenumerator}}$ and $\onehotmapofat{\tildecatindexof{\catenumerator}}{\tildecatvariableof{\catenumerator}}$ (see \figref{fig:propagationMary}).
    We then apply the Directed Belief Propagation \algoref{alg:directedBeliefPropagation}.
    The initial messages queue then consists of the messages from the digit encoding.
    As sketched in \figref{fig:propagationMary}, to each digit there are three messages (with the exception of the first being two), which can be scheduled in consecutive epochs $\messagesymbol^{(\catenumerator,[3])}$.
    %, namely
%    \begin{align*}
%        \scheduler =&
%        \left\{
%            \big((\varnothing,\{\catvariableof{0}\}),(\{\catvariableof{0},\tildecatvariableof{0}\},\{\headvariableof{0},\thirdcatvariableof{0}\})\big),
%            \big((\varnothing,\{\tildecatvariableof{0}\}),(\{\catvariableof{0},\tildecatvariableof{0}\},\{\headvariableof{0},\thirdcatvariableof{0}\})\big)
%        \right\} \cup \\
%&        \left(\bigcup_{\catenumerator\in\{1,\ldots,\catorder-2\}}\left\{
%                                                               \big((\varnothing,\{\catvariableof{\catenumerator}\}),(\{\catvariableof{\catenumerator},\tildecatvariableof{\catenumerator},\thirdcatvariableof{\catenumerator-1}\},\{\headvariableof{\catenumerator},\thirdcatvariableof{\catenumerator}\})\big),
%                                                               \big((\varnothing,\{\tildecatvariableof{\catenumerator}\}),(\{\catvariableof{\catenumerator},\tildecatvariableof{\catenumerator},\thirdcatvariableof{\catenumerator-1}\},\{\headvariableof{\catenumerator},\thirdcatvariableof{\catenumerator}\})\big)
%        \right\}\right) \\
%        & \cup  \Big\{
%            \big((\varnothing,\{\catvariableof{\catorder-1}\}),(\{\catvariableof{\catorder-1},\tildecatvariableof{\catorder-1},\thirdcatvariableof{\catenumerator-2}\},\{\headvariableof{\catorder-1},\thirdcatvariableof{\catorder-1}\})\big), \\
%        & \big((\varnothing,\{\tildecatvariableof{\catorder-1}\}),(\{\catvariableof{\catorder-1},\tildecatvariableof{\catorder-1},\thirdcatvariableof{\catenumerator-2}\},\{\headvariableof{\catorder-1},\thirdcatvariableof{\catorder-1}\})\big)
%        \Big\}
%    \end{align*}
    We then have by \theref{the:directedBeliefPropagationExactness} for $\catenumerator\in[\catorder-1]$ that
    \begin{align*}
        \contractionof{
            \bencodingofat{\modsumsymbol,\catenumerator}{\headvariableof{\catenumerator},\thirdcatvariableof{\catenumerator},\catvariableof{\catenumerator},\tildecatvariableof{\catenumerator},\thirdcatvariableof{\catenumerator-1}},
            \messagesymbol^{(\catenumerator-1,2)}[\thirdcatvariableof{\catenumerator-1}],
            \messagesymbol^{(\catenumerator,0)}[\catvariableof{\catenumerator}],
            \messagesymbol^{(\catenumerator,1)}[\tildecatvariableof{\catenumerator}]
        }{\thirdcatvariableof{\catenumerator}}
        =
        \onehotmapofat{\thirdcatindexof{\catenumerator}}{\thirdcatvariableof{\catenumerator}}\,,
    \end{align*}
    where $\thirdcatindexof{\catenumerator}$ is the value of the $\catenumerator$-th carry bit.
    The $\catenumerator$-th digit of the sum $\headindexof{\catenumerator}$ can further be read of by the contraction
    \begin{align*}
        \contractionof{
            \bencodingofat{\modsumsymbol,\catenumerator}{\headvariableof{\catenumerator},\thirdcatvariableof{\catenumerator},\catvariableof{\catenumerator},\tildecatvariableof{\catenumerator},\thirdcatvariableof{\catenumerator-1}},
            \messagesymbol^{(\catenumerator-1,2)}[\thirdcatvariableof{\catenumerator-1}],
            \messagesymbol^{(\catenumerator,0)}[\catvariableof{\catenumerator}],
            \messagesymbol^{(\catenumerator,1)}[\tildecatvariableof{\catenumerator}]
        }{\headvariableof{\catenumerator}}
        =
        \onehotmapofat{\headindexof{\catenumerator}}{\headvariableof{\catenumerator}} \, .
    \end{align*}
    % Tree belief also working
    Note that the hypergraph representing this instance is a tree and by \theref{the:treeBeliefPropagationExactness} also the message passing scheme of \algoref{alg:treeBeliefPropagation} is guaranteed to produce the exact contractions.
    We can exploit this fact for example in the efficient computation of averages of the summation digits, when we have an elementary distribution of input digits.
    We emphasize that the directed belief propagation \algoref{alg:treeBeliefPropagation} is exact even if the hypergraph fails to be a tree, provided that we have directed and boolean tensors..

    \begin{figure}[t]
        \begin{center}
            \begin{tikzpicture}[scale=0.5, thick]

                \draw (-1,-1) rectangle (3,1);
                \node[anchor=center] (A) at (1,0) {\corelabelsize $\bencodingof{\modsumsymbol,0}$};
                \draw[->-] (1,1)--(1,3) node[midway,left] {\colorlabelsize $\headvariableof{0}$};
                \draw[-<-] (0,-1)--(0,-2.5) node[midway,left] {\colorlabelsize $\catvariableof{0}$};
                \draw (-0.75,-2.5) rectangle (0.75,-4);
                \node[anchor=center] (A) at (0,-3.25) {\corelabelsize $\onehotmapof{\catindexof{0}}$};
                \draw[\newmessagecolor,dashed, ->] (-1.25,-3.25) to [bend right = -30] (-1.25,-1);
                \node[\newmessagecolor,anchor=center] (A) at (-1.75,-3.6) {\colorlabelsize $\messagesymbol^{(0,0)}$};
                \draw[-<-] (2,-1)--(2,-2.5) node[midway,right] {\colorlabelsize $\tildecatvariableof{0}$};
                \draw (1.25,-2.5) rectangle (2.75,-4);
                \node[anchor=center] (A) at (2,-3.25) {\corelabelsize $\onehotmapof{\tildecatindexof{0}}$};
                \draw[\newmessagecolor,dashed, ->] (3.25,-3.25) to [bend right = 30] (3.25,-1);
                \node[\newmessagecolor,anchor=center] (A) at (3.8,-3.6) {\colorlabelsize $\messagesymbol^{(0,1)}$};
                \draw[->-] (3,0)--(6,0) node[midway,above] {\colorlabelsize $\thirdcatvariableof{0}$};
                \draw[\newmessagecolor,dashed, ->] (3,1.25) to [bend right = -30] (6,1.25);
                \node[\newmessagecolor,anchor=center] (A) at (4.5,2.25) {\colorlabelsize $\messagesymbol^{(0,2)}$};

                \begin{scope}[shift={(7,0)}]
                    \draw (-1,-1) rectangle (3,1);
                    \node[anchor=center] (A) at (1,0) {\corelabelsize $\bencodingof{\modsumsymbol,1}$};
                    \draw[->-] (1,1)--(1,3) node[midway,left] {\colorlabelsize $\headvariableof{1}$};
                    \draw[-<-] (0,-1)--(0,-2.5) node[midway,left] {\colorlabelsize $\catvariableof{1}$};
                    \draw (-0.75,-2.5) rectangle (0.75,-4);
                    \node[anchor=center] (A) at (0,-3.25) {\corelabelsize $\onehotmapof{\catindexof{1}}$};
                    \draw[\newmessagecolor,dashed, ->] (-1.25,-3.25) to [bend right = -30] (-1.25,-1);
                    \node[\newmessagecolor,anchor=center] (A) at (-1.75,-3.6) {\colorlabelsize $\messagesymbol^{(1,0)}$};
                    \draw[-<-] (2,-1)--(2,-2.5) node[midway,right] {\colorlabelsize $\tildecatvariableof{1}$};
                    \draw (1.25,-2.5) rectangle (2.75,-4);
                    \node[anchor=center] (A) at (2,-3.25) {\corelabelsize $\onehotmapof{\tildecatindexof{1}}$};
                    \draw[\newmessagecolor,dashed, ->] (3.25,-3.25) to [bend right = 30] (3.25,-1);
                    \node[\newmessagecolor,anchor=center] (A) at (3.8,-3.6) {\colorlabelsize $\messagesymbol^{(1,1)}$};
                    \draw[->-] (3,0)--(6,0) node[midway,above] {\colorlabelsize $\thirdcatvariableof{1}$};
                    \draw[\newmessagecolor,dashed, ->] (3,1.25) to [bend right = -30] (6,1.25);
                    \node[\newmessagecolor,anchor=center] (A) at (4.5,2.25) {\colorlabelsize $\messagesymbol^{(1,2)}$};
                \end{scope}

                \node[anchor=center] at (15.5,0) {$\cdots$};

                \begin{scope}[shift={(22,0)}]
                    \draw[\newmessagecolor,dashed, ->] (-4,1.25) to [bend right = -30] (-1,1.25);
                    \node[\newmessagecolor,anchor=center] (A) at (-2.5,2.25) {\colorlabelsize $\messagesymbol^{(\catorder\shortminus2,2)}$};

                    \draw[->-] (-4,0)--(-1,0) node[midway,above] {\colorlabelsize $\thirdcatvariableof{\catorder\shortminus 2}$};
                    \draw (-1,-1) rectangle (3,1);
                    \node[anchor=center] (A) at (1,0) {\corelabelsize $\bencodingof{\modsumsymbol,\catorder\shortminus1}$};
                    \draw[->-] (1,1)--(1,3) node[midway,left] {\colorlabelsize $\headvariableof{\catorder\shortminus1}$};
                    \draw[-<-] (0,-1)--(0,-2.5) node[midway,left] {\colorlabelsize $\catvariableof{\catorder\shortminus1}$};
                    \draw (-0.75,-2.5) rectangle (0.75,-4);
                    \node[anchor=center] (A) at (0,-3.25) {\corelabelsize $\onehotmapof{\catindexof{\catorder\shortminus1}}$};
                    \draw[\newmessagecolor,dashed, ->] (-1.5,-3.25) to [bend right = -45] (-1.5,-0.75);
                    \node[\newmessagecolor,anchor=center] (A) at (-2.3,-3.8) {\colorlabelsize $\messagesymbol^{(\catorder\shortminus1,0)}$};
                    \draw[-<-] (2,-1)--(2,-2.5) node[midway,right] {\colorlabelsize $\tildecatvariableof{\catorder\shortminus1}$};
                    \draw (1.25,-2.5) rectangle (2.75,-4);
                    \draw[\newmessagecolor,dashed, ->] (3.5,-3.25) to [bend right = 45] (3.5,-0.75);
                    \node[\newmessagecolor,anchor=center] (A) at (4.3,-3.8) {\colorlabelsize $\messagesymbol^{(\catorder\shortminus1,1)}$};
                    \node[anchor=center] (A) at (2,-3.25) {\corelabelsize $\onehotmapof{\tildecatindexof{\catorder\shortminus1}}$};
                    \draw (3,0)--(4.5,0) -- (4.5,1);
                    \draw[->-] (4.5,1)--(4.5,3) node[midway,left] {\colorlabelsize $\headvariableof{\catorder}$};
                \end{scope}


            \end{tikzpicture}
        \end{center}
        \caption{Computation of the integer sum in $\catdim$-adic representation by the Directed Belief Propagation \algoref{alg:directedBeliefPropagation} (see \exaref{exa:madicPropagation}).
        The summands are represented by one-hot encodings of the digits $\shortcatindices$ and $\tildecatindexof{[\catorder]}$, from which the messages start.
        The $\catenumerator$-th digit (for $\catenumerator\in\{0,\ldots,\catorder-1\}$) of the sum is computed based on the first messages of the epoch labeled by $\messagesymbol^{(\catenumerator,[2])}$,
            The third message $\messagesymbol^{(\catenumerator,2)}$ in each epoch communicates the carry bit to the next digit summation core.
            In the last message epoch the digit $\catorder-1$ and $\catorder$ are computed based.
        }
        \label{fig:propagationMary}
    \end{figure}

\end{example}


%Let $\graph$ be a directed acyclic hypergraph, where each node appears exactly once as an outgoing node, and which is decorated by boolean and directed tensors $\extnet$.
%We then interpret the tensors to the nodes not appearing as incoming nodes as one-hot encodings $\onehotmapofat{\shortcatindices}{\shortcatvariables}$.
%The rest of the hypergraph is a decomposition graph with functions defined by the basis encodings $\hypercoreofat{\edge}{\catvariableof{\edge}}$.
%If all tensors are boolean and directed, then the messages in directed belief propagation give the one-hot encodings of the node formulas evaluated at the inverse $\shortcatindices$ one-hot encoding at the leafs.
%In particular, the messages are the correct contractions of the subgraph. % Needs the further above theorem?
%