\section{The Neural Paradigm}

%% TO DO: Formulas -> Functions in this section

\alex{We show in this section how functions decomposeable into neurons can be represented by tensor networks.
Use directed decomposition graphs, set enumerating functions for this in full generality.
Examples are propositional formulas with sparse syntactical descriptions.
}


\begin{lemma}
    \label{lem:formulaDecomp}
    Let $\formulaat{\shortcatvariables}$ be a composition of a $\seldim$-ary connective formula $\exconnective$ and propositional formulas $\formulaofat{\selindex}{\shortcatvariables}$, where $\selindexin$, i.e. for $\shortcatindices\in\atomstates$ we have
    \begin{align*}
        \formulaat{\indexedshortcatvariables}
        = \exconnective\left(\formulaofat{0}{\indexedshortcatvariables}, \dots, \formulaofat{\seldim-1}{\indexedshortcatvariables}\right) \, .
    \end{align*}
    Then we have
    \begin{align*}
        \bencodingofat{\formula}{\headvariableof{\formula},\shortcatvariables}
        = \contractionof{
            \{\bencodingofat{\exconnective}{\headvariableof{\formula},\headvariableof{[\seldim]}}\}
            \cup \{\bencodingofat{\formulaof{\selindex}}{\headvariableof{\selindex},\shortcatvariables} \wcols \selindexin\}
        }{\headvariableof{\formula},\shortcatvariables} \, .
    \end{align*}
\end{lemma}
\begin{proof}
    This can be shown on each index $\shortcatindices$.
\end{proof}

For the composition of two propositional formulas $\formulaat{\shortcatvariables}$ and $\secexformula\left[\shortcatvariables\right]$ the composition by some binary connective is pictured by:
\begin{center}
    \begin{tikzpicture}[scale=0.4, thick]
    \begin{scope}

        % \node[anchor=center] (text) at (-6,-4) {$b)$};

        \begin{scope}
        [shift={(-8,0)}]
            \draw[->-] (5.5,-9)--(5.5,-7) node[midway,right] {\colorlabelsize $\headvariableof{\exformula}$};
            \drawatomcore{3.5}{-8}{$\bencodingof{\exformula}$}
            \drawatomindices{3.5}{-12}
        \end{scope}

        \node[anchor=center] (text) at (1.5,-10) {${=}$};

        \draw[->-] (9.5,-5)--(9.5,-3) node[midway,right] {\colorlabelsize $\headvariableof{\exformula}$};

        \node[anchor=center] (text) at (9.5,-6) {$\bencodingof{\chainingfunction}$};
        \draw (4.5,-7) rectangle (14.5,-5);

        \draw[->-] (5.5,-9)--(5.5,-7) node[midway,right] {\colorlabelsize $\headvariableof{0}$};

        \node[anchor=center] at (9.5,-8) {$\cdots$};

        \drawatomcore{3.5}{-8}{$\bencodingof{\formulaof{0}}$}
        \drawatomindices{3.5}{-12}

        \begin{scope}
        [shift={(8,0)}]

            \draw[->-] (5.5,-9)--(5.5,-7) node[midway,right] {\colorlabelsize $\headvariableof{\seldim-1}$};

            \drawatomcore{3.5}{-8}{$\bencodingof{\formulaof{\seldim-1}}$}
            \drawatomindices{3.5}{-12}

        \end{scope}

        \draw[fill] (7.5,-15) circle (\dotsize);
        \draw[] (7.5,-15) to[bend left=25] (3.5,-13);
        \draw[] (7.5,-15) to[bend right=25] (11.5,-13);

        \draw[fill] (9,-15.25) circle (\dotsize);
        \draw[] (9,-15.25) to[bend left=25] (5,-13);
        \draw[] (9,-15.25) to[bend right=25] (13,-13);

        \draw[fill] (11.5,-15) circle (\dotsize);
        \draw[] (11.5,-15) to[bend left=25] (7.5,-13);
        \draw[] (11.5,-15) to[bend right=25] (15.5,-13);



        \draw[] (7.5,-15)--(7.5,-17) node[midway,left] {\colorlabelsize $\catvariableof{0}$};
        \draw[] (9,-15.25)--(9,-17) node[midway,left] {\colorlabelsize $\catvariableof{1}$};
        \node[anchor=center] (text) at (10.5,-16.5) {$\cdots$};
        \draw[] (11.5,-15)--(11.5,-17) node[midway,right] {\colorlabelsize $\catvariableof{\atomorder-1}$};

    \end{scope}
\end{tikzpicture}
\end{center}

Let us now define a more generic syntactical decomposition of propositional formulas.

\begin{definition}
    \label{def:formulaDecomposition}
    A syntactical hypergraph is a directed acyclic hypergraph $\graph=(\nodes,\edges)$ such that
    \begin{itemize}
        \item each hyperedge $\edge=(\incomingnodes,\outgoingnodes)$ has exactly one outgoing node, i.e. $\cardof{\outgoingnodes}=1$
        \item each node $\nodein$ carries a boolean variable $\headvariableof{\node}$ and appears at most once as the outgoing node of a hyperedge % well-definedness
        \item each hyperedge $(\incomingnodes,\{\node\})$ with $\incomingnodes\neq\varnothing$ is decorated by a propositional formula
        \begin{align*}
            \connectiveofat{\node}{\headvariableof{\incomingnodes}} \defcols \bigtimes_{\node\in\incomingnodes} [2] \rightarrow [2]
        \end{align*}
        \item the nodes not appearing as an outgoing node are labeled by $[\atomorder]$
    \end{itemize}
    We say that the syntactical hypergraph is single-rooted, if exactly one node $\secnode$ does not appear as an incoming node of a hyperedge.
    In this case this unique node is called the root node. % head node
    We assign atomic formulas to the nodes $[\atomorder]$ and recursively assign to each further node $\node$ a node formula % connective $\connectiveofat{\node}{\headvariableof{\incomingnodes}}$.
    \begin{align*}
        \formulaofat{\node}{\indexedshortcatvariables}
        = \connectiveofat{\node}{[\formulaofat{\thirdnode}{\indexedshortcatvariables}\wcols\thirdnode\in\incomingnodes]} \quad \forall\shortcatindicesin\, ,
    \end{align*}
    where $\incomingnodes$ are the incoming nodes in the unique hyperedge with outgoing nodes $\{\node\}$.
    We call the formula $\exformulaat{\shortcatvariables}\coloneqq\formulaofat{\secnode}{\shortcatvariables}$ to the root note $\secnode$ the syntactical composition of $\graph$ and $\graph$ is a syntactical decomposition of $\exformula$.
\end{definition}

\begin{theorem}
    \label{the:formulaDecompositionRep}
    For any syntactical hypergraph $\graph$ with composition $\exformula$ we have
    \begin{align*}
        \exformulaat{\shortcatvariables}
        = \breakablecontractionof{
            &\left\{
                 \bencodingofat{\connectiveof{\node}}{\headvariableof{\node},\headvariableof{\incomingnodes}} \wcols (\incomingnodes,\{\node\})\in\edges
            \right\} \cup \\
            & \{\identityat{\headvariableof{\atomenumerator},\catvariableof{\atomenumerator}} \wcols \atomenumeratorin\}
            \cup \{\tbasisat{\headvariableof{\secnode}}\}
        }{\shortcatvariables} \, .
    \end{align*}
\end{theorem}
\begin{proof}
    One can show this theorem by induction over the node formulas of the syntactical hypergraph, from the leafs to the root and iteratively applying \lemref{lem:formulaDecomp}.
\end{proof}

Thus we have a tensor network representation of any propositional formula based on its syntactical decomposition, where the hypergraph of the syntactical decomposition equals the hypergraph of the representing tensor network.
