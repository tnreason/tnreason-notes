\section{Implementation in the \python{} library \tnreason{}}\label{sec:code}

The concepts presented in this paper have been implemented in the \python{} library \tnreason{}\footnote{github.com/tnreason}, available on pypi.org\footnote{pypi.org/tnreason}
%\begin{figure}[t]
\begin{center}
    %! Author = alexgoessmann
%! Date = 09.03.25

\begin{tikzpicture}[scale=0.35]
    \draw[dashed] (-30,15) -- (12,15) -- (12,-3) -- (-30,-3) -- (-30,15);

    \draw (-10,10) rectangle (10,14);
    \node [anchor=center] at (0,12) {\spapplication{}};

    \node [anchor=center] at (-20,12) {\layerthreespec};
    \draw[dashed] (-30,9) -- (12,9);
    \node [anchor=center] at (-20,6) {\layertwospec};

    \draw[->-] (6,10) -- (6,8);
    \draw (2,4) rectangle (10,8);
    \node [anchor=center] at (6,6) {\spreasoning{}};
    \draw[->-] (6,4) -- (6,2);

    \draw[->-] (-6,10) -- (-6,8);
    \draw (-10,4) rectangle (-2,8);
    \node [anchor=center] at (-6,6) {\sprepresentation{}};
    \draw[->-] (-6,4) -- (-6,2);

    \draw[dashed] (-30,3) -- (12,3);
    \node [anchor=center] at (-20,0) {\layeronespec};

    \draw (-10,-2) rectangle (10,2);
    \node [anchor=center] at (0,0) {\spengine{}};
\end{tikzpicture}
\end{center}
%\end{figure}
While we in this section explain the basic design and functionality of this library, we provide in Appendix~\ref{sec:algExaImplementation} a detailed implementation of the algorithms and examples in this work.

\subsection{Architecture}

In the submodule \tnreason{}.\spengine{} we implement tensors, tensor networks, contractions and normalizations.
In the submodule \tnreason{}.\sprepresentation{} the basic tensor encoding schemes such as basis encodings are available.
In the submodule \tnreason{}.\spreasoning{} we implement reasoning algorithms, such as generalization of the message passing algorithms presented in \algoref{alg:beliefPropagation}, \algoref{alg:constraintPropagation} and \algoref{alg:AMM_HLN}.
In the submodule \tnreason{}.\spapplication{} one can construct tensor network encodings of propositional formulas and data sets. 

\subsection{Basic usage}

We demonstrate the basic usage of the \tnreason{} package on the Example~\ref{exa:propFormulaCoordinatewise}.
Dictionaries with string keys and valued by tensors store tensor networks.


\begin{minted}{python}
expressionsDict = {"f0" :  ["and", ["or", "a", "b"], ["not", "c"]]}
\end{minted}
After importing the package by
\begin{minted}{python}
from tnreason import representation, application, engine
\end{minted}
the CAN can then be build by defining all cores. Activation cores then have suffices \mintinline{python}{_aC} and computation cores are denoted with suffices \mintinline{python}{_cC} by
\begin{minted}{python}
cores = application.create_cores_to_expressionsDict(expressionsDict)
computationCores = {key: value for key, value in cores.items()
                                                         if key.endswith("_cC")}
\end{minted}
The generated \mintinline{python}{computationCores} is a dictionary with the following keys and tensors of given shapes as expected in example~\ref{exa:bencodingNegCon}.
\begin{minted}{python}
'f0_aC', [2]
'(and_(or_a_b)_(not_c))_cC', [2, 2, 2]
'(or_a_b)_cC', [2, 2, 2]
'(not_c)_cC', [2, 2]
\end{minted}
Based on this dictionary, the \ComputationActivationNetwork{} can be build by setting the activation network for the single output feature (the output of $f$) to a vector acting on the output of the basis encoding of $f$. Here a \mintinline{python}{SingleHybridFeature} is used, which allows for hard or soft activations. Then the value for the desired output of the feature is set to \mintinline{python}{True}.
\begin{minted}{python}
caNet = representation.ComputationActivationNetwork(
    computationCoreDict=computationCores,
    featureDict={"f0" : representation.SingleHybridFeature(
                                    featureColor="(and_(or_a_b)_(not_c))_cV")},
    canParamDict={"f0" : True}
    )
\end{minted}

As in example~\ref{exa:bencodingNegCon}, the CAN then has the following form.\alex{Should we use the engine.draw() for that?}
\begin{center}
\begin{tikzpicture}[scale=0.4, yscale=-1, thick] % , baseline = -3.5pt
        
            \begin{scope}
                [shift={(7,0)}]
        
                \draw[->-] (0,1)--(0,-1) node[midway,left] (Xa) {\colorlabelsize $\catvariableof{a}$};
                \draw[->-] (3,1)--(3,-1) node[midway,right] {\colorlabelsize $\catvariableof{b}$};
                \draw[->-] (6,1)--(6,-1) node[midway,right] {\colorlabelsize $\catvariableof{c}$};
        
                \draw (-2.5,-1) rectangle (4, -3);
                \node[anchor=center] (or) at (0.7,-2) {\corelabelsize \mintinline{python}{'(or_a_b)_cC'}};
        
                \draw[->-] (1.5,-3) --(1.5,-5) node[midway,right]{\colorlabelsize $\headvariableof{a \lor b}$};
        
                \draw (5,-1) rectangle (11, -3);
                \node[anchor=center] (not) at (8,-2) {\corelabelsize\mintinline{python}{'(not_c)_cC'}};
        
                \draw[->-] (6,-3) --(6,-5) node[midway,right]{\colorlabelsize $\headvariableof{\lnot c}$};
        
                \draw (-2.9,-5) rectangle (10,-7);
                \node[anchor=center] (and) at (3.5,-6) {\corelabelsize \mintinline{python}{'(and_(or_a_b)_(not_c))_cC'}};
        
                \draw[->-] (4,-7) -- (4,-8.5) node[right] (Yand) {\colorlabelsize $\headvariableof{(a \lor b) \land \lnot c}$};
                \drawvariabledot{4}{-8}
                \draw[] (4,-8) -- (4,-9);
                \draw (3,-9) rectangle (5,-11);
                \node[anchor=center] (f0) at (4,-10) {\mintinline{python}{"f0"}};

                  \node[
                    draw=teal,
                    line width=1.2pt,
                    ellipse,
                    fit=(or)(not)(and)(Xa),
                    inner xsep=8mm,   % horizontal padding
                    inner ysep=0mm     % vertical padding
                  ] (tealring) {};
                  \node[teal] at (20,0) {computation network};

                  
                  \node[
                    draw=purple!50,
                    line width=1.2pt,
                    ellipse,
                    fit=(f0)(Yand),
                    inner xsep=8mm,   % horizontal padding
                    inner ysep=0.2mm     % vertical padding
                  ] (tealring) {};
                  \node[purple!50] at (15.5,-8.5) {activation network};
            \end{scope}
\end{tikzpicture}
\end{center}

The other representation in example~\ref{exa:bencodingNegCon} can also be implemented.
Both architectures can be found the linked notebook\footnote{\url{https://colab.research.google.com/drive/14knFuMJHI683DAmUgJ-G10MQFueoXR6q\#scrollTo=vrOYBViZhX2S}}.
Note that more efficient representations of this network are possible and the one described here is mainly for pedagogic purposes.
Furthermore the right implementation of the formula can be checked by contractions.
\begin{minted}{python}
allCores = caNet.create_cores()
formula = engine.contract(coreDict = allCores,
                          openColors=["a_dV","b_dV", "c_dV"])
assert formula[{"a_dV": 1, "b_dV" : 1, "c_dV" : 0}] == 1
assert formula[{"a_dV": 1, "b_dV" : 1, "c_dV" : 1}] == 0
\end{minted}
The notebook also shows how the network can be normalized to represent a uniform probability distribution over all models of the formula.
\begin{minted}{python}
distribution = engine.normalize(coreDict = allCores,
                                outColors = ["a_dV","b_dV", "c_dV"],
                                inColors = [])
assert distribution[{"a_dV": 1, "b_dV" : 1, "c_dV" : 0}] == 1/3
assert distribution[{"a_dV": 1, "b_dV" : 1, "c_dV" : 1}] == 0
\end{minted}