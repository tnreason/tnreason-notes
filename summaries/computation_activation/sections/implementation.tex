\section{Implementation in the \python{} library \tnreason{}}\label{sec:code}

\setminted{
    fontsize=\fontsize{10}{10.5}\selectfont,
    %linenos,
    breaklines
}

The concepts presented in this paper have been implemented in the \python{} library \tnreason{}\footnote{\tnreason{} is available in version 2.0.0 at \url{pypi.org/tnreason} and maintained at  \url{github.com/tnreason/tnreason-py}.}.
While in this section, we explain the basic design and functionality of this library, we provide in Appendix~\ref{sec:algExaImplementation} a detailed implementation of the algorithms and examples in this work.

\subsection{Architecture}

The package consists of four subpackages and three layers of abstraction:
%\begin{figure}[t]
\begin{center}
    %! Author = alexgoessmann
%! Date = 09.03.25

\begin{tikzpicture}[scale=0.35]
    \draw[dashed] (-30,15) -- (12,15) -- (12,-3) -- (-30,-3) -- (-30,15);

    \draw (-10,10) rectangle (10,14);
    \node [anchor=center] at (0,12) {\spapplication{}};

    \node [anchor=center] at (-20,12) {\layerthreespec};
    \draw[dashed] (-30,9) -- (12,9);
    \node [anchor=center] at (-20,6) {\layertwospec};

    \draw[->-] (6,10) -- (6,8);
    \draw (2,4) rectangle (10,8);
    \node [anchor=center] at (6,6) {\spreasoning{}};
    \draw[->-] (6,4) -- (6,2);

    \draw[->-] (-6,10) -- (-6,8);
    \draw (-10,4) rectangle (-2,8);
    \node [anchor=center] at (-6,6) {\sprepresentation{}};
    \draw[->-] (-6,4) -- (-6,2);

    \draw[dashed] (-30,3) -- (12,3);
    \node [anchor=center] at (-20,0) {\layeronespec};

    \draw (-10,-2) rectangle (10,2);
    \node [anchor=center] at (0,0) {\spengine{}};
\end{tikzpicture}
\end{center}
%\end{figure}
In the subpackage \tnreason{}.\spengine{} we implement tensors, tensor networks, contractions and normalizations.
In the subpackage \tnreason{}.\sprepresentation{} the basic tensor encoding schemes such as basis encodings are available.
In the subpackage \tnreason{}.\spreasoning{} we implement reasoning algorithms, such as generalization of the message passing algorithms presented in \algoref{alg:beliefPropagation}, \algoref{alg:constraintPropagation} and \algoref{alg:AMM_HLN}.
In the subpackage \tnreason{}.\spapplication{} one can construct tensor network encodings of propositional formulas and data sets.

\subsection{Basic usage}

We demonstrate the basic usage of the \tnreason{} package with the implementation of Example~\ref{exa:propFormulaCoordinatewise}.
We first install the package (e.g. by \mintinline{shell}{pip install tnreason == 2.0.0}) and import it by
\begin{minted}{python}
from tnreason import engine, application
\end{minted}
%Tensors creation is a functionality of \tnreason{}.\spengine{}:
%\begin{minted}{python}
%createdTensor = engine.create_from_slice_iterator(shape, colors, sliceIterator)
%\end{minted}
Keeping \defref{def:tensor} in mind, the tensor instances in \mintinline{python}{shape} and \mintinline{python}{colors} arguments are \mintinline{python}{list} instances specifying the \mintinline{python}{int} dimension $\catdimof{\catenumerator}$ and a \mintinline{python}{str} identifier for $\catvariableof{\catenumerator}$.
The formula in \exaref{exa:propFormulaCoordinatewise} is a sum of the one-hot encodings of its three models (see \exaref{exa:propFormulaBasCP}) and is created by
\begin{minted}{python}
formula = engine.create_from_slice_iterator(shape=[2,2,2], colors=["X_0","X_1","X_2"], sliceIterator=[(1,{"X_0":0,"X_1":1,"X2":0}), (1,{"X_0":1,"X_1":0,"X2":0}), (1,{"X_0":1,"X_1":1,"X2":0})])
\end{minted}
The slice iterator is an iterator over tuples \mintinline{python}{(val,posDict)}, which specifies elementary tensors to be summed.
The \mintinline{python}{posDict} are \mintinline{python}{dict} instances with keys by the \mintinline{python}{str} tensor colors and values by \mintinline{python}{int}.
Each \mintinline{python}{posDict} specifies which leg vectors of the corresponding elementary tensor are not trivial (when the corresponding key is in \mintinline{python}{posDict}), in which case the leg vector is the basis vector enumerated by the corresponding \mintinline{python}{int} value.

Single coordinates of tensors can be retrieved by subscribing them with a \mintinline{python}{posDict}.
We can for example check whether the \mintinline{python}{{"X_0":0,"X_1":1,"X_2"}} is a model:
\begin{minted}{python}
assert formula[{"X_0":0,"X_1":1,"X_2"}] == 1
\end{minted}
By default the tensor is created as a \mintinline{python}{engine.NumpyCore} instance, where coordinates are stored as instances of \mintinline{python}{numpy.array}.
Further core types exploiting different sparsity principles can be chosen by the argument \mintinline{python}{coreType}, see \cite[Appendix~A]{goessmann_tensor-network_2025}.

Following \defref{def:tensorNetwork}, tensor networks are implemented as tensor valued \mintinline{python}{dict} instances with \mintinline{python}{str} keys.
For example a tensor network is created from the propositional syntax of the above formula (see \exaref{exa:propFormulaHeadSym}): % a tensor network decomposition by propositional syntax.
\begin{minted}{python}
fDecomp = application.create_cores_to_expressionsDict({"f0": ["and",["or","X_0","X_1"],["not","X_2"]]})
\end{minted}
Here we apply a nested list description of syntactical hypergraphs (see \defref{def:formulaDecomposition}) with a specification of the logical connectives at the first position of the list (by \mintinline{python}{"and","or","not"} we refer to the connectives $\land,\lor,\lnot$).
Equivalently, we can exploit the $\land$ symmetry and create it by multiple formulas:
\begin{minted}{python}
fDecomp = application.create_cores_to_expressionsDict({"f0": ["or","X_0","X_1"], "f1": ["not","X_2"]})
\end{minted}
A depiction of the underlying hypergraph as a factor graph, which highlights edges by blue blocks and nodes by red blocks is can be created by \mintinline{python}{engine.draw_factor_graph(fDecomp)} (see \figref{fig:factorGraphPlt}).
%The \mintinline{python}{str} keys to the created cores carry the suffices \mintinline{python}{"_cC"} and \mintinline{python}{"_aC"} indicating whether they are part of the computation or the activation network.
%Further, colors get the suffices \mintinline{python}{"_dV"} and \mintinline{python}{"_cV"} indicating whether they are distributed variables or computed and therefore auxiliary.
Single tensors can be retrieved by contractions depending on a tensor network and a specification of the open variables (for explanation of the suffices see \figref{fig:factorGraphPlt}), for example:
\begin{minted}{python}
contracted = engine.contract(fDecomp, openColors=["X_0_dV","X_1_dV","X_2_dV"])
\end{minted}
By default the contractions are performed using \mintinline{python}{numpy.einsum} and further execution schemes can be selected with the argument \mintinline{python}{contractionMethod}, see \cite[Appendix~A]{goessmann_tensor-network_2025}.

\begin{figure}
    \begin{center}
        \includegraphics[width=15cm]{../tikz_pics/factor_graph_plt.png}
    \end{center}
    \caption{
        Factor graph highlighting a tensor network decomposition of the syntactical decomposition of the propositional formula of \exaref{exa:propFormulaHeadSym}.
        Blue blocks highlight hyperedges carrying tensors and red blocks variables.
        The tensor label suffices \mintinline{python}{"_cC"} and \mintinline{python}{"_aC"} indicate whether the tensor is part of the computation network or the activation network.
        The variable label suffices \mintinline{python}{"_dV"} and \mintinline{python}{"_cV"} indicate whether the variable is distributed or computed and therefore auxiliary.
        This graph has been generated with the method \mintinline{python}{tnreason.engine.draw_factor_graph} of \tnreason{}.
    }\label{fig:factorGraphPlt}
\end{figure}


%
%\begin{minted}{python}
%expressionsDict = {"f0" :  ["and", ["or", "a", "b"], ["not", "c"]]}
%\end{minted}
%After importing the package by
%\begin{minted}{python}
%from tnreason import representation, application, engine
%\end{minted}
%the CAN can then be build by defining all cores.
%Activation cores then have suffices \mintinline{python}{_aC} and computation cores are denoted with suffices \mintinline{python}{_cC} by
%\begin{minted}{python}
%cores = application.create_cores_to_expressionsDict(expressionsDict)
%computationCores = {key: value for key, value in cores.items()
%                                                         if key.endswith("_cC")}
%\end{minted}
%The generated \mintinline{python}{computationCores} is a dictionary with the following keys and tensors of given shapes as expected in example~\ref{exa:bencodingNegCon}.
%\begin{minted}{python}
%'f0_aC', [2]
%'(and_(or_a_b)_(not_c))_cC', [2, 2, 2]
%'(or_a_b)_cC', [2, 2, 2]
%'(not_c)_cC', [2, 2]
%\end{minted}
%Based on this dictionary, the \ComputationActivationNetwork{} can be build by setting the activation network for the single output feature (the output of $f$) to a vector acting on the output of the basis encoding of $f$. Here a \mintinline{python}{SingleHybridFeature} is used, which allows for hard or soft activations. Then the value for the desired output of the feature is set to \mintinline{python}{True}.
%\begin{minted}{python}
%caNet = representation.ComputationActivationNetwork(
%    computationCoreDict=computationCores,
%    featureDict={"f0" : representation.SingleHybridFeature(
%                                    featureColor="(and_(or_a_b)_(not_c))_cV")},
%    canParamDict={"f0" : True}
%    )
%\end{minted}
%
%As in example~\ref{exa:bencodingNegCon}, the CAN then has the following form.\alex{Should we use the engine.draw() for that?}
%\begin{center}
\begin{tikzpicture}[scale=0.4, yscale=-1, thick] % , baseline = -3.5pt
        
            \begin{scope}
                [shift={(7,0)}]
        
                \draw[->-] (0,1)--(0,-1) node[midway,left] (Xa) {\colorlabelsize $\catvariableof{a}$};
                \draw[->-] (3,1)--(3,-1) node[midway,right] {\colorlabelsize $\catvariableof{b}$};
                \draw[->-] (6,1)--(6,-1) node[midway,right] {\colorlabelsize $\catvariableof{c}$};
        
                \draw (-2.5,-1) rectangle (4, -3);
                \node[anchor=center] (or) at (0.7,-2) {\corelabelsize \mintinline{python}{'(or_a_b)_cC'}};
        
                \draw[->-] (1.5,-3) --(1.5,-5) node[midway,right]{\colorlabelsize $\headvariableof{a \lor b}$};
        
                \draw (5,-1) rectangle (11, -3);
                \node[anchor=center] (not) at (8,-2) {\corelabelsize\mintinline{python}{'(not_c)_cC'}};
        
                \draw[->-] (6,-3) --(6,-5) node[midway,right]{\colorlabelsize $\headvariableof{\lnot c}$};
        
                \draw (-2.9,-5) rectangle (10,-7);
                \node[anchor=center] (and) at (3.5,-6) {\corelabelsize \mintinline{python}{'(and_(or_a_b)_(not_c))_cC'}};
        
                \draw[->-] (4,-7) -- (4,-8.5) node[right] (Yand) {\colorlabelsize $\headvariableof{(a \lor b) \land \lnot c}$};
                \drawvariabledot{4}{-8}
                \draw[] (4,-8) -- (4,-9);
                \draw (3,-9) rectangle (5,-11);
                \node[anchor=center] (f0) at (4,-10) {\mintinline{python}{"f0"}};

                  \node[
                    draw=teal,
                    line width=1.2pt,
                    ellipse,
                    fit=(or)(not)(and)(Xa),
                    inner xsep=8mm,   % horizontal padding
                    inner ysep=0mm     % vertical padding
                  ] (tealring) {};
                  \node[teal] at (20,0) {computation network};

                  
                  \node[
                    draw=purple!50,
                    line width=1.2pt,
                    ellipse,
                    fit=(f0)(Yand),
                    inner xsep=8mm,   % horizontal padding
                    inner ysep=0.2mm     % vertical padding
                  ] (tealring) {};
                  \node[purple!50] at (15.5,-8.5) {activation network};
            \end{scope}
\end{tikzpicture}
\end{center}
%
%The other representation in example~\ref{exa:bencodingNegCon} can also be implemented.
%Both architectures can be found the linked notebook\footnote{\url{https://colab.research.google.com/drive/14knFuMJHI683DAmUgJ-G10MQFueoXR6q\#scrollTo=vrOYBViZhX2S}}.
%Note that more efficient representations of this network are possible and the one described here is mainly for pedagogic purposes.
%Furthermore the right implementation of the formula can be checked by contractions.
%\begin{minted}{python}
%allCores = caNet.create_cores()
%formula = engine.contract(coreDict = allCores,
%                          openColors=["a_dV","b_dV", "c_dV"])
%assert formula[{"a_dV": 1, "b_dV" : 1, "c_dV" : 0}] == 1
%assert formula[{"a_dV": 1, "b_dV" : 1, "c_dV" : 1}] == 0
%\end{minted}
%The notebook also shows how the network can be normalized to represent a uniform probability distribution over all models of the formula.
%\begin{minted}{python}
%distribution = engine.normalize(coreDict = allCores,
%                                outColors = ["a_dV","b_dV", "c_dV"],
%                                inColors = [])
%assert distribution[{"a_dV": 1, "b_dV" : 1, "c_dV" : 0}] == 1/3
%assert distribution[{"a_dV": 1, "b_dV" : 1, "c_dV" : 1}] == 0
%\end{minted}
