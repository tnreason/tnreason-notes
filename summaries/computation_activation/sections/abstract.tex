The unification of the neural and the symbolic paradigms of artificial intelligence is a long-standing challenge.
We in this work introduce a tensor-network formalism, which captures sparsity principles originating in the different paradigms in tensor decompositions.
In particular, we describe a basis encoding scheme for functions and model neural decompositions by tensor decompositions.
This unified treatment identifies tensor-network contractions as a fundamental inference class and formulates efficiently scaling reasoning algorithms, originating from probability theory and propositional logic, as contraction message passing schemes.
The framework enables the definition and training of hybrid logical and probabilistic models, which we call \HybridLogicNetworks{}.
We further demonstrate the concepts in the accompanying python library \tnreason{}.



%We introduce \ComputationActivationNetworks{} (\CompActNets{}), a novel architecture for tensor networks, which is adapted to represent propositional formulas and exponential distributions.

%This work introduces the novel \ComputationActivationNetworks{} (\CompActNets{}), a tensor-network architecture that can represent probabilistic modeling and logical reasoning within a single mathematical framework.
%Propositional formulas and exponential-family distributions are unified in the resulting \HybridLogicNetworks{}.
%Computation or features, which are encoded by basis representations of sufficient statistics or logical formulas, are separated from activation, which assigns probabilistic or boolean values to these features.
%This separation yields sparse and interpretable representations that subsume graphical models, exponential-family factorizations, and logical inference as special cases.
%Probabilistic and logical queries reduce to tensor contractions, providing a common operational substrate for both reasoning paradigms.
%Altogether, \CompActNets{} offer an expressive and intrinsically explainable foundation for neuro-symbolic AI, enabling exact reasoning alongside efficient implementation for learning tasks.


OR

The unification of neural and symbolic approaches to artificial intelligence remains a central open challenge. In this work, we introduce a tensor-network architecture that provides a unified mathematical framework for both paradigms. 
Furthermore, the proposed formalism can be applied to represent logical formulas and probability distributions as structured tensor decompositions. A common treatment of sparsity is enabled through compositionality.

At the core of the framework is a basis-encoding scheme representing functions as tensors, which is applied to sufficient statistics and propositional formulas.
The framework enables the definition and training of hybrid logical and probabilistic models, which we call \HybridLogicNetworks{}, combining hard logical constraints with soft probabilistic activations in a single, interpretable model. The framework subsumes graphical models, exponential families, and symbolic knowledge bases.

% Inference in probabilistic models, 
Function evaluation and entailment decisions in propositional logic are expressed as tensor contractions. They can be tackled by message-passing algorithms based on local contractions for large networks.

The theoretical concepts are accompanied by the Python library \tnreason{}, which enables the implementation and practical use of the proposed architectures.