\begin{example}
    For the formula $\formulaat{X_{[1]} = x_1} = \lnot x_1$, that is $\formulaat{X_{[1]}} = (1, 0)^T$, we have with \eqref{eq:basisencboolean}
    \begin{center}
        \begin{tikzpicture}
            \node at (0,0) {$
            \bencodingofat{\formula}{\formulavar,X_{[1]}}
            = \begin{pmatrix}
                  0 &1\\
                  1 & 0
            \end{pmatrix}$};
            \draw[<-] (2,0.2)--(2.5,0.2);
            \node at (3.1,0.2) {$\formulavar=0$};
            \draw[<-] (2,-0.2)--(2.5,-0.2);
            \node at (3.1,-0.2) {$\formulavar=1$};
            \draw[<-] (1,-0.5)--(1,-1);
            \node at (0.5,-1.2) {$X_1=0$};
            \draw[<-] (1.4,-0.5)--(1.4,-1);
            \node at (1.9,-1.2) {$X_1=1$};
        \end{tikzpicture}
    \end{center}
    with~\eqref{eq:basisencnegsum} we have
    \begin{align*}
        \bencodingofat{\formula}{\formulavar,X_{[1]}} = \begin{pmatrix}
                                                            1\\0
        \end{pmatrix} \otimes \begin{pmatrix}
                                  0\\1
        \end{pmatrix} + \begin{pmatrix}
                            0\\1
        \end{pmatrix} \otimes \begin{pmatrix}
                                  1\\0
        \end{pmatrix}
        = \begin{pmatrix}
              0 &1\\0 &0
        \end{pmatrix} + \begin{pmatrix}
                            0 & 0\\1 &0
        \end{pmatrix}
        = \begin{pmatrix}
              0 &1\\
              1 & 0
        \end{pmatrix}
    \end{align*}
    and in graphical notation we get
    \begin{center}
        \begin{tikzpicture}[scale=0.35, thick] % , baseline = -3.5pt

            \draw[->-] (1.75,-1)--(1.75,1) node[midway,right] {\colorlabelsize $\formulavar$};
            \draw (1,-1) rectangle (3,-3);
            \node[anchor=center] (text) at (2,-2) {\corelabelsize $\bencodingof{\exformula}$};
            \draw[-<-] (1.75,-3)--(1.75,-5) node[midway,left] {\colorlabelsize $\catvariableof{0}$};

            \node[anchor=center] (text) at (5.5,-2) {${=}$};
            \begin{scope}
            [shift={(10,-0.5)}]

                \draw (-2,1) rectangle (4,-1);
                \node[anchor=center] (text) at (1,0) {\corelabelsize $\onehotmapof{1}$};
                \draw[->-] (1,1)--(1,3) node[midway,right] {\colorlabelsize $\formulavar$};

                \draw (0,-2) rectangle (2,-4);
                \node[anchor=center] (text) at (1,-3) {\corelabelsize $\onehotmapof{0}$};
                \draw[->-] (1,-4)--(1,-6) node[midway,right] {\colorlabelsize $\catvariableof{0}$};

            \end{scope}
            \node[anchor=center] (text) at (16,-2) {${+}$};
            \begin{scope}
            [shift={(21,-0.5)}]

                \draw (-2,1) rectangle (4,-1);
                \node[anchor=center] (text) at (1,0) {\corelabelsize $\onehotmapof{0}$};
                \draw[->-] (1,1)--(1,3) node[midway,right] {\colorlabelsize $\formulavar$};

                \draw (0,-2) rectangle (2,-4);
                \node[anchor=center] (text) at (1,-3) {\corelabelsize $\onehotmapof{1}$};
                \draw[->-] (1,-4)--(1,-6) node[midway,right] {\colorlabelsize $\catvariableof{0}$};

            \end{scope}

        \end{tikzpicture}
    \end{center}
\end{example}