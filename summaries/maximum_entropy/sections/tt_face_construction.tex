\section{TT Face Activation}

\subsection{Generic Construction}

When the vertices are characterized by the sum of local terms, we can construct the face activation tensor in $\ttformat$ format, by summation cores.
This is for example the case for the examples constructed based on Hamming distances.

\subsection{Relation with Facet Complexity}

Facet complexity \cite[Definition 6.2.2]{grotschel_geometric_1993} bounds the encoding length of the inequalities describing a polytope.

For any face $\facesymbol$ we find a normal $\normalvec\in\parspace$ and $\normalbound\in\rr$ such that $x$
\begin{align*}
    \genfaceset =
    \{\meanparamwith\in\meanset \wcols \contraction{\normalvecwith,\meanparamwith}=\normalbound \}\, .
\end{align*}
We can thus represent a face activating tensor by computing the contraction $\contraction{\normalvecwith,\headindex[\selvariable]}$.

To this end, let $\arbset\subset\rr$ be a set with index interpretation function
\begin{align*}
    \indexinterpretation : [r] \rightarrow \arbset \, .
\end{align*}
We demand $0,\normalbound\in\arbset$.
We can for example choose $\arbset$ as the $\catorder$-bit binary integer representation.

We then define for $\selindexin$ the summation operations
\begin{align*}
    +^{\selindex,\normalvecat{\indexedselvariable}} \defcols \arbset \times [\headdim] \rightarrow \arbset
\end{align*}
which satisfies for any $\arbelementin$ and $\headindexof{\selindex}$ for which $\arbelement + \normalvecat{\indexedselvariable} \cdot \headindexof{\selindex}\in\arbset$ the equation
\begin{align*}
    +^{\selindex,\normalvecat{\indexedselvariable}}(\arbelement,\headindexof{\selindex})
    = \arbelement + \normalvecat{\indexedselvariable} \cdot \headindexof{\selindex} \, .
\end{align*}
We further assume, that for any tuple $\headindexof{[\seldim]}\in\headstates$ and any $\secselindex\in[\seldim]$
\begin{align*}
    \sum_{\selindex\leq\secselindex} \normalvecat{\indexedselvariable} \cdot \headindexof{\secselindex} \in \arbset \, .
\end{align*}

%Note that we might also use an approximation scheme of the sum on $\arbset$.


Then we have
\begin{align*}
    \sum_{\meanparam\in\genfaceimset} \onehotmapofat{\meanparam}{\headvariables}
    = \contractionof{\{
        \bencodingofat{+^{\selindex,\normalvecat{\indexedselvariable}}}{
            \headvariableof{+^{\selindex,\normalvecat{\indexedselvariable}}},\headvariableof{\selindex}
        }
        \wcols\selindexin\}
        \cup \{\onehotmapofat{\invindexinterpretationat{0}}{\headvariableof{+^{-1,\normalvecat{\selvariable=-1}}}},
        \onehotmapofat{\invindexinterpretationat{\normalbound}}{\headvariableof{+^{\seldim-1,\normalvecat{\selvariable=\seldim-1}}}}\}
    }{\headvariables} \, .
\end{align*}
The variable $\headvariableof{+^{-1,\normalvecat{\selvariable=-1}}}$ is auxiliary to initialize the sum with $0$.

We can weaken the assumptions on $\arbset$:
When $\normalvecwith$ is positive, then any partial sum which exceeds $\normalbound$ can be represented by any member of $\arbset$ which is larger than $\normalbound$.
This brings further sparsity, since no values larger than $\normalbound$ are needed to represent the face (add for example the $\infty$ element to $\arbset$ and represent any partial sum exceeding $\normalbound$ by it).


\begin{definition}
    We say that a face $\facesymbol$ has \emph{TT face complexity} $r$ if there exists a normal $\normalvec\in\parspace$ and $\normalbound\in\rr$ with
    \begin{align*}
        \genfaceset =
        \{\meanparamwith\in\meanset \wcols \contraction{\normalvecwith,\meanparamwith}=\normalbound \}\, .
    \end{align*}
    and there is a subset $\arbset$ of cardinality at most $r$, for which $0,\normalbound\in\arbset$ and for any tuple $\headindexof{[\seldim]}\in\headstates$ and any $\secselindex\in[\seldim]$
    \begin{align*}
        \sum_{\selindex\leq\secselindex} \normalvecat{\indexedselvariable} \cdot \headindexof{\secselindex} \in \arbset \, .
    \end{align*}
\end{definition}

Comparison with facet complexity \cite[Definition 6.2.2]{grotschel_geometric_1993}:
\begin{itemize}
    \item We define complexity here without the log (which would get the string length), to make a direct connection with TT ranks.
    \item Facet complexity considers only the representation of the equation, not the demand of computing it.
    \item Facet complexity considers just facets, not general faces, since it makes a statement about encoding the halfspace representation of the whole polytope.
\end{itemize}

\begin{theorem}
    Let $\facesymbol$ be a face with TT face complexity $r$.
    Then the face activation tensor has a $\ttformat$ rank of at most $r$.
\end{theorem}
\begin{proof}
    See above construction.
\end{proof}

