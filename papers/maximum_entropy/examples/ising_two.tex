\begin{example}[Ising model on $2$ nodes]
    % See Example~3.6 in \cite{wainwright_graphical_2008}
    Consider two boolean variables $\catvariableof{0},\catvariableof{1}$ and the Ising statistic by three propositional formulas
    \begin{align*}
        \formulaofat{0}{\catvariableof{[2]}} = \catvariableof{0} \quad , \quad
        \formulaofat{1}{\catvariableof{[2]}} = \catvariableof{1} \andspace
        \formulaofat{2}{\catvariableof{[2]}} = \catvariableof{0}\land\catvariableof{1} \, .
    \end{align*}
    In the Ising interpretation, the boolean variables represent interacting spins at two locations.
    Their value is then measured by the first two formulas and their interaction by the third.

    The vertices of the mean polytope to this statistic are
    \begin{align*}
        \hlnstatat{\catindexof{0},\catindexof{1}} = [\formulaofat{0}{\indexedcatvariableof{[2]}},\formulaofat{1}{\indexedcatvariableof{[2]}},\formulaofat{2}{\indexedcatvariableof{[2]}}]^T =
        \begin{cases}
        [0,0,0]
            ^T & \ifspace (\catindexof{0},\catindexof{1}) = (0,0) \\
            [0,1,0]^T & \ifspace (\catindexof{0},\catindexof{1}) = (0,1) \\
            [1,0,0]^T & \ifspace (\catindexof{0},\catindexof{1}) = (1,0) \\
            [1,1,1]^T & \ifspace (\catindexof{0},\catindexof{1}) = (1,1)
        \end{cases}
    \end{align*}
    and the mean polytope is the convex hull of these, sketched as:
    \begin{center}
        \tdplotsetmaincoords{55}{20} % Set the viewpoint: (theta, phi)

        \begin{tikzpicture}[tdplot_main_coords, scale=2,
            mainline/.style={thick},
            invisibleline/.style={dashed, gray}
        ]

            % Define the 4 vertices
            \coordinate (A) at (0, 0, 0);
            \coordinate (B) at (0, 1, 0);
            \coordinate (C) at (1, 0, 0);
            \coordinate (D) at (1, 1, 1);

            % Draw the faces with some opacity (optional, requires more complex sorting for correct rendering)
            \fill[cyan!20, opacity=0.6] (A) -- (B) -- (C) -- cycle;
            % \fill[cyan!20, opacity=0.6] (A) -- (B) -- (D) -- cycle;
            % \fill[cyan!20, opacity=0.6] (A) -- (C) -- (D) -- cycle;
            % \fill[cyan!20, opacity=0.6] (B) -- (C) -- (D) -- cycle;

            % Draw visible edges (adjust based on your chosen viewpoint)
            \draw[mainline] (B) -- (C) -- (D) -- cycle; % Front face BCD
            \draw[mainline] (A) -- (B);
            \draw[mainline] (A) -- (C);
            \draw[mainline] (A) -- (D);

            % Draw the hidden edge (adjust based on your chosen viewpoint)
            % In this view, edge BC is visible, but the edge connecting the back vertex to D is also potentially visible.
            % The edge BC is visible, the edge connecting A to C and B is also visible.
            % A is behind for this viewpoint, so the edges connected to A are back edges
            \draw[invisibleline] (A) -- (B);
            \draw[invisibleline] (A) -- (C);
            \draw[invisibleline] (A) -- (D);


            % Draw vertices as nodes and label them
            \foreach \point/\label/\pos in {A/000/below left, B/010/below right, C/100/above left, D/111/above} {
                \draw[fill=black] (\point) circle (1pt) node[\pos] {$\label$};
            }

            % Add coordinate system axes for context
            \draw[->, gray] (0,0,0) -- (1.2,0,0) node[right] {$\meanparamat{\selvariable=0}$};
            \draw[->, gray] (0,0,0) -- (0,1.2,0) node[above] {$\meanparamat{\selvariable=1}$};
            \draw[->, gray] (0,0,0) -- (0,0,1.2) node[above] {$\meanparamat{\selvariable=2}$};

        \end{tikzpicture}
    \end{center}

    We notice, that the convex hull of any three out of the four vertices builds a facet.
    The only cube-like facet of these four is the convex hull of $\{[0,0,0]^T,[1,0,0]^T,[0,1,0]\}$ (sketched in cyan in the above plot) to which we have the parametrization by $(\hardlegset,\headindexof{\hardlegset})=(\{2\},0)$.

    As a consequence, there are maximum entropy distributions of mean parameters in the Ising statistics, which do not have a representation by an elementary \CompActNet{} in the Ising statistic.
\end{example}