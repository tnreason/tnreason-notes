\begin{example}[Atomic formulas]
    \label{exa:atomicFormulasHypercube}
    %The assumption of \theref{the:sufficientHLNExpressivity} is satisfied in
    Let us consider the case of atomic formulas. % where the formulas $\formulaof{\formulaset,\canparam}$ are the atoms.
    The mean polytope in this case is the $\catorder$-dimensional hypercube
    \begin{align*}
        \meansetof{\atomformulaset,\ones} = \fullparcube
    \end{align*}
    which is called a simple polytope, since each vertex is contained in the minimal number of $\catorder$ facets.
    %Since the cube $\fullparcube$ is a face of itself, \theref{the:sufficientHLNExpressivity} implies $\hlnmeanset|_{\hlnsetof{\atomformulaset}}=\hlnmeanset$.

    The faces of a hypercube are enumerated in the following way.
    Each face is characterized by the projections onto each variable, which is either $\{0\}$, $\{1\}$ or $[0,1]$.
    The projections are represented by the tuple $\hardparam$ defined in the following way:
    \begin{itemize}
        \item We define the set $\hardlegset\subset[\atomorder]$ of variables, such that the projection onto the variable is $\{0\}$ or $\{1\}$
        \item We define to each $\selindex\in\hardlegset$ an index $\headindexof{\selindex}=0$ if the projection is $\{0\}$ and $\headindexof{\selindex}=1$ if the projection is $\{1\}$.
    \end{itemize}

%    There are thus $2^{\atomorder}$ different sets $\hardlegset$, each with $2^{\hardlegset}$ faces by a choice of $\headindexof{\selindex}$.

    Trivially, each face of the hypercube is a cube face and $\elmeansetof{\atomformulaset}=\meansetof{\atomformulaset}$.

    \red{More general: If and only if no combination of possibly negated formulas is unsatisfiable, then the mean polytope is a hypercube.}
\end{example}