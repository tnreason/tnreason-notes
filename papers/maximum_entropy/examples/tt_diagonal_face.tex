\begin{example}[$\ttformat$ representation of diagonal faces]
    \label{exa:tt_diagonal_face}
    Consider the vertex set
    \begin{align*}
        \imset \coloneqq \{0,1\}^{\seldim} / \imelementat{\selvariable}
    \end{align*}
    convex polytope.
    We are interested in the face $\facesymbolof{\triangleleft,1}$ with the normal $\frac{1}{2} (2\imelementwith - \onesat{\selvariable})$.
    Its vertices are the $\seldim$ elements of $\{0,1\}^{\seldim}$, which differ from $\imelementwith$ in exactly one coordinate (i.e. those with Hamming distance of $1$ from $\imelementat{\selvariable}$).
    We denote these vertices by $\imelementof{\selindex}$ for $\selindexin$, which coordinates to $\secselindex\in[\seldim]$ are
    \begin{align*}
        \imelementofat{\selindex}{\selvariable=\secselindex}
        = \begin{cases}
              \imelementat{\selvariable=\secselindex} & \ifspace \secselindex\neq\selindex \\
              1-\imelementat{\selvariable=\secselindex} & \ifspace \secselindex=\selindex
        \end{cases} \, .
    \end{align*}

    We now represent their sum in an $\ttformat$ with hidden ranks $r_0=\ldots=r_{\seldim-2}=2$.
    The boolean hidden variables are denoted be $\decvariableof{[\seldim-1]}$ and can be interpreted as indicators, whether the coordinate flip has happened in $[\selindex]$ coordinates.
    We now construct a $\ttformat$ cores for $\selindex=0$ by
    \begin{align*}
        \hypercoreofat{0}{\headvariableof{0},\decvariableof{0}}
        = \onehotmapofat{\imelementat{\selvariable=0}}{\headvariableof{0}} \otimes \fbasisat{\decvariableof{0}}
        + \onehotmapofat{1-\imelementat{\selvariable=0}}{\headvariableof{0}} \otimes \tbasisat{\decvariableof{0}}
    \end{align*}
    further for $\selindex\notin\{0,\seldim-1\}$ the cores
    \begin{align*}
        \hypercoreofat{\selindex}{\decvariableof{\selindex-1},\headvariableof{\selindex},\decvariableof{\selindex}}
        = \, & \fbasisat{\decvariableof{\selindex-1}} \otimes \onehotmapofat{1-\imelementat{\indexedselvariable}}{\headvariableof{\selindex}} \otimes \tbasisat{\decvariableof{\selindex}}
        + \fbasisat{\decvariableof{\selindex-1}} \otimes \onehotmapofat{\imelementat{\indexedselvariable}}{\headvariableof{\selindex}} \otimes \fbasisat{\decvariableof{\selindex}} \\
        &+ \tbasisat{\decvariableof{\selindex-1}} \otimes \onehotmapofat{\imelementat{\indexedselvariable}}{\headvariableof{\selindex}} \otimes \tbasisat{\decvariableof{\selindex}}
    \end{align*}
    and for $\selindex=\seldim-1$
    \begin{align*}
        \hypercoreofat{\seldim-1}{\decvariableof{\seldim-1},\headvariableof{\seldim-1}}
        = \fbasisat{\decvariableof{\seldim-1}} \otimes \onehotmapofat{1-\imelementat{\selvariable=\seldim-1}}{\headvariableof{\seldim-1}}
        + \tbasisat{\decvariableof{\seldim-1}} \otimes \onehotmapofat{\imelementat{\selvariable=\seldim-1}}{\headvariableof{\seldim-1}} \, .
    \end{align*}
    For this tensor network in the $\ttformat$ format we have
    \begin{align*}
        \sum_{\selindexin} \onehotmapofat{\imelementof{\selindex}}{\headvariables}
        = \contractionof{\{\hypercoreofat{0}{\headvariableof{0},\decvariableof{0}},\hypercoreofat{\seldim-1}{\decvariableof{\seldim-1},\headvariableof{\seldim-1}}\}\cup\{\hypercoreofat{\selindex}{\decvariableof{\selindex-1},\headvariableof{\selindex},\decvariableof{\selindex}}\wcols\selindex\notin\{0,\seldim-1\}\}}{\headvariables} \, .
    \end{align*}
    The $\ttformat$ multirank of $2$ is furthermore minimal, since each matrification based on a partition of $[\seldim]$ into non-empty sets has a matrix rank of $2$.
    With respect to such partitions also the tensor $\onehotmapofat{\imelement}{\headvariables}+\left(\sum_{\selindexin}\onehotmapofat{\imelementof{\selindex}}{\headvariables}\right)$ has matrix ranks of $2$.
    Since these are the only two activation tensors for the face $\facesymbolof{\triangleleft,1}$.
\end{example}

\begin{example}[Generalization of \exaref{exa:tt_diagonal_face} to larger $\ttformat$ ranks]
    \label{exa:gen_tt_diagonal_face}
    %% Extension to larger Hadamard distances
    We now generalize the construction of \exaref{exa:tt_diagonal_face} by using the Hadamard distance $d(\cdot,\cdot)$ in $\{0,1\}^{\seldim}$, which counts the number of coordinates two vertices differ in.
    For $s\in\{1,\ldots,\seldim-1\}$ we define for a fixed $\imelementwith\in\{0,1\}^{\seldim}$ a polytope as the convex hull
    \begin{align*}
        \{\tilde{\imelement}[\selvariable] \wcols d(\tilde{\imelement},\imelement)\geq s\}  \, .
    \end{align*}
    The face to the normal $\frac{1}{2} (2\imelementwith - \onesat{\selvariable})$ is the convex hull
    \begin{align*}
        \facesymbolof{\triangleleft,s}\coloneqq \convhullof{
            \{\tilde{\imelement}[\selvariable] \wcols d(\tilde{\imelement},\imelement) = s\}
        }
    \end{align*}
    containing $\binom{\seldim}{s}$ vertices.
    We label these vertices by subsets $\variableset\subset[\seldim]$ of cardinality $s$ and define for $\selindexin$
    \begin{align*}
        \imelementofat{\variableset}{\indexedselvariable} =
        \begin{cases}
            \imelementat{\selindexin} & \ifspace \selindex\notin\variableset \\
            1-\imelementat{\selindexin} & \ifspace \selindex\notin\variableset
        \end{cases}
    \end{align*}
    We now construct a $\ttformat$ with hidden variables $\decvariableof{\selindex}$ and ranks $r_{\selindex} = \min (\selindex+1,\seldim-\selindex+1, s+1)$ to represent the sum of their one-hot encodings.
    To this end, let there be $\ttformat$ cores for $\selindex=0$ by
    \begin{align*}
        \hypercoreofat{0}{\headvariableof{0},\decvariableof{0}}
        = \onehotmapofat{\imelementat{\selvariable=0}}{\headvariableof{0}} \otimes \fbasisat{\decvariableof{0}}
        + \onehotmapofat{1-\imelementat{\selvariable=0}}{\headvariableof{0}} \otimes \tbasisat{\decvariableof{0}}
    \end{align*}
    further for $\selindex\notin\{0,\seldim-1\}$ the cores
    \begin{align*}
        \hypercoreofat{\selindex}{\decvariableof{\selindex-1},\headvariableof{\selindex},\decvariableof{\selindex}}
        = \sum_{\selindex\in\{\max(s-\seldim+\selindex,0),\ldots,\max(\selindex,s)\}}
        \onehotmapofat{\selindex}{\decvariableof{\selindex}}
        \otimes \left(\onehotmapofat{\imelementat{\indexedselvariable}}{\headvariableof{\selindex}} \otimes \onehotmapofat{\selindex}{\decvariableof{\selindex+1}}
                    + \onehotmapofat{1-\imelementat{\indexedselvariable}}{\headvariableof{\selindex}} \otimes \onehotmapofat{\selindex+1}{\decvariableof{\selindex+1}}
        \right)
    \end{align*}
    and for $\selindex=\seldim-1$
    \begin{align*}
        \hypercoreofat{\seldim-1}{\decvariableof{\seldim-1},\headvariableof{\seldim-1}}
        = \fbasisat{\decvariableof{\seldim-1}} \otimes \onehotmapofat{1-\imelementat{\selvariable=\seldim-1}}{\headvariableof{\seldim-1}}
        + \tbasisat{\decvariableof{\seldim-1}} \otimes \onehotmapofat{\imelementat{\selvariable=\seldim-1}}{\headvariableof{\seldim-1}} \, .
    \end{align*}
    Based on the interpretation, that the hidden variables $\decvariableof{\selindex}$ count the Hamming distance of the vectors $\restrictionofto{\imelementwith}{\rr^{\selindex}\times 0_{\seldim-\selindex}}$ and the respective $\restrictionofto{\imelementofat{\variableset}{\selvariable}}{\rr^{\selindex}\times 0_{\seldim-\selindex}}$ one can show that
    \begin{align*}
        \sum_{\variableset\subset[\seldim] \wcols \cardof{\variableset}=s} \onehotmapofat{\imelementof{\variableset}}{\headvariables}
        = \contractionof{\{\hypercoreofat{0}{\headvariableof{0},\decvariableof{0}},\hypercoreofat{\seldim-1}{\decvariableof{\seldim-1},\headvariableof{\seldim-1}}\}\cup\{\hypercoreofat{\selindex}{\decvariableof{\selindex-1},\headvariableof{\selindex},\decvariableof{\selindex}}\wcols\selindex\notin\{0,\seldim-1\}\}}{\headvariables} \, .
    \end{align*}
\end{example}

\begin{example}[Generalization of \exaref{exa:gen_tt_diagonal_face} to arbitrary $\htformat$ formats]
    Instead of aligning the Hamming count variables linearly, one can find a representation of the activation tensor
    \begin{align*}
        \sum_{\variableset\subset[\seldim] \wcols \cardof{\variableset}=s} \onehotmapofat{\imelementof{\variableset}}{\headvariables}
    \end{align*}
    in an arbitrary directed acyclic tree hypergraph format $\graph$, which hidden ranks are bounded by the number of leafs in the subtree and $s+1$.

    At any leaf of the tree we define a $2x2$ matrix
    \begin{align*}
        \hypercoreofat{\selindex}{\headvariableof{\selindex},\decvariableof{\selindex}}
        = \onehotmapofat{\imelementat{\indexedselvariable}}{\headvariableof{\selindex}} \otimes \fbasisat{\decvariableof{\selindex}}
        + \onehotmapofat{1-\imelementat{\indexedselvariable}}{\headvariableof{\selindex}} \otimes \tbasisat{\decvariableof{\selindex}} \, .
    \end{align*}
    At each intermediate non-root hyperedge we choose an outgoing counting variable $\decvariableof{\outgoingnodes}$ with dimension by $\decdimof{\outgoingnodes}=\min(\sum_{\node\in\incomingnodes}\decdimof{\incomingnodes},s+1)$ and define a tensor with the slices
    \begin{align*}
        \hypercoreofat{\edge}{\decvariableof{\outgoingnodes},\decvariableof{\incomingnodes}}
        = \bencodingofat{+}{\headvariableof{+}=\decindexof{\outgoingnodes},\decvariableof{\incomingnodes}} \, .
    \end{align*}
    We further build at the root hyperedge $\edge$ a tensor
    \begin{align*}
        \hypercoreofat{\edge}{\decvariableof{\edge}}
        = \sum_{\decindex \wcols \contraction{\decindex}=s}
        \onehotmapofat{\decindexof{\edge}}{\decvariableof{\edge}} \, .
    \end{align*}
    Using the counting variable interpretation one can now show that
    \begin{align*}
        \sum_{\variableset\subset[\seldim] \wcols \cardof{\variableset}=s} \onehotmapofat{\imelementof{\variableset}}{\headvariables}
        =\contractionof{
        \{\hypercoreofat{\edge}{\decvariableof{\edge}} \wcols \edgein \} \cup \{\hypercoreofat{\selindex}{\headvariableof{\selindex},\decvariableof{\selindex}} \wcols \nodein\}
        }{\headvariables} \, .
    \end{align*}
\end{example}