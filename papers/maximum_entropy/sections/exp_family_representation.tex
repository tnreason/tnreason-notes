\section{Tensor Network Representation of Exponential Families}

We now introduce an architecture of tensor networks, namely \ComputationActivationNetworks{}, and show that exponential families can be represented by them.

\subsection{\ComputationActivationNetworks{}}

Given a statistic $\sstat:\facstates\rightarrow\selstates$ we build its basis encoding tensor
\begin{align*}
    \sstatccwith = \sum_{\shortcatindicesin} \onehotmapofat{\sstat(\shortcatindices)}{\headvariables} \otimes \onehotmapofat{\shortcatindices}{\shortcatvariables} \, .
\end{align*}
A computation network is any representation of $\sstatccwith$ as a tensor network.
These can be constructed in the case statistics being a composition of connective functions.

An activation tensor is $\hypercoreat{\headvariables}$ and the \ComputationActivationNetwork{} of $\sstat$ and $\hypercore$ the tensor
\begin{align*}
    \probwith = \normalizationof{\sstatccwith,\hypercoreat{\headvariables}}{\shortcatvariables} \, .
\end{align*}

We are interested in decomposition formats of $\hypercoreat{\headvariables}$, where we use sets of tensor networks $\tnsetof{\graph}$ on a hypergraph $\graph$.

\begin{definition}
    The family of by $\sstat$ and a $\graph$ computable distributions are
    \begin{align*}
        \realizabledistsof{\sstat,\graph,\basemeasure}
        = \left\{
              \frac{\contractionof{\hypercoreat{\headvariables},\bencsstatwith}{\shortcatvariables}}{\contraction{\hypercoreat{\headvariables},\bencsstatwith,\basemeasurewith}}
              \wcols \hypercoreat{\headvariables} \in \tnsetof{\graph} \right\} \, .
    \end{align*}
\end{definition}

%\subsection{CP decompositions}
%We here introduce the CP decomposition of tensors and the restriction to bas+.
%This will be used to represent face measures as computation activation networks.

\subsection{Exponential Families}

\begin{definition}[Exponential Family]
    \label{def:expFamily}
    Given a statistic function
    \begin{align*}
        \sstat \defcols \facstates \rightarrow \parspace
    \end{align*}
    and a base measure
    \begin{align*}
        \basemeasure \defcols \facstates \rightarrow \rr^+
    \end{align*}
    with $\contraction{\basemeasure}\neq0$, the set $\expfamily=\{\expdist\wcols\canparamwithin\}\subset\bmrealprobof{\basemeasure}$ of probability distributions
    \begin{align*}
        \expdistat{\shortcatvariables}
        = \frac{
            \expof{\contractionof{\sencsstatat{\shortcatvariables,\selvariable},\canparamwith}{\shortcatvariables}}
        }{
            \contraction{\expof{\contractionof{\sencsstatat{\shortcatvariables,\selvariable},\canparamwith}{\shortcatvariables}},\basemeasurewith}
        }
    \end{align*}
    is called the exponential family to $\sstat$.
\end{definition}

To present a tensor network representation, we introduce image interpretation maps
\begin{align*}
    \indexinterpretationof{\selindex} \defcols
    [\cardof{\imageof{\sstatcoordinateof{\selindex}}}] \rightarrow \imageof{\sstatcoordinateof{\selindex}} \, ,
\end{align*}
which enumerate the possible values of each feature.
We treat these maps as tensors with in a variable $\headvariableof{\selindex}$ with values in $[\cardof{\imageof{\sstatcoordinateof{\selindex}}}]$.

\begin{theorem}[Exponential Families are in \ComputationActivationNetworks{}]
    \label{the:expFamilyTensorRep}
    Given any base measure $\basemeasure$ and a sufficient statistic $\sstat$ we enumerate for each coordinate $\selindexin$ the image $\imageof{\sstatcoordinateof{\selindex}}$ by a variable $\headvariableof{\selindex}$ taking values in $[\cardof{\imageof{\sstatcoordinateof{\selindex}}}]$, given an interpretation map
    \begin{align*}
        \indexinterpretationof{\selindex} \defcols
        [\cardof{\imageof{\sstatcoordinateof{\selindex}}}] \rightarrow \imageof{\sstatcoordinateof{\selindex}} \, .
    \end{align*}
    For any canonical parameter vector $\canparamwithin$ we build the activation cores $\softactlegwith$ for each coordinate $\headindexof{\selindex}\in[\cardof{\imageof{\sstatcoordinateof{\selindex}}}]$ by
    \begin{align*}
        \softactleg\left[\indexedheadvariableof{\selindex}\right]
        = \expof{\canparamat{\indexedselvariable} \cdot \indexinterpretationofat{\selindex}{\headindexof{\selindex}} } \,
    \end{align*}
    and have (see \figref{fig:expdistUnaryRealizable})
    \begin{align*}
        \expdistat{\shortcatvariables}
        = \frac{
            \contractionof{\{\bencodingofat{\sstatcoordinateof{\selindex}}{\headvariableof{\selindex},\shortcatvariables} \wcols \selindexin\}\cup\{\softactlegwith \wcols \selindexin\}}{\shortcatvariables}
        }{
            \contraction{\{\bencodingofat{\sstatcoordinateof{\selindex}}{\headvariableof{\selindex},\shortcatvariables} \wcols \selindexin\}\cup\{\softactlegwith \wcols \selindexin\}\cup\{\basemeasurewith\}}
        } \, .
    \end{align*}
\end{theorem}
\begin{proof} %% SHORTENED THE PROOF IN THE REPORT
    For each $\shortcatindicesin$ we have
    \begin{align*}
        &\contractionof{\{\bencodingofat{\sstatcoordinateof{\selindex}}{\headvariableof{\selindex},\shortcatvariables} \wcols \selindexin\}\cup\{\softactlegwith \wcols \selindexin\}}{\indexedshortcatvariables} \\
        &\quad = \prod_{\selindexin} \contraction{\bencodingofat{\sstatcoordinateof{\selindex}}{\headvariableof{\selindex},\indexedshortcatvariables},\softactlegwith} \\
        &\quad = \prod_{\selindexin} \expof{\canparamat{\indexedselvariable}\cdot\indexinterpretationofat{\selindex}{\sstatcoordinateofat{\selindex}{\shortcatindices}}} \\
        &\quad = \expof{\sum_{\selindexin}\canparamat{\indexedselvariable}\cdot\indexinterpretationofat{\selindex}{\sstatcoordinateofat{\selindex}{\shortcatindices}}} \\
        &\quad = \expof{\contractionof{\sencsstatat{\shortcatvariables,\selvariable},\canparamwith}{\indexedshortcatvariables}} \, .
    \end{align*}
    Therefore we have
    \begin{align*}
        \contractionof{\{\bencodingofat{\sstatcoordinateof{\selindex}}{\headvariableof{\selindex},\shortcatvariables} \wcols \selindexin\}\cup\{\softactlegwith \wcols \selindexin\}}{\shortcatvariables}
        = \expof{\contractionof{\sencsstatat{\shortcatvariables,\selvariable},\canparamwith}{\shortcatvariables}} \, .
    \end{align*}
    The claim follows, since this implies that also the contraction of both sides with $\basemeasurewith$ is equivalent.
\end{proof}

\begin{figure}[t]
    \begin{center}
        \begin{tikzpicture}[scale=0.35,thick,xscale=1] % , baseline = -3.5pt

    \begin{scope}
        [shift={(-11,0)}]
        \draw (-2,-1) rectangle (6,-3);
        \node[anchor=center] (text) at (2,-2) {\corelabelsize $\partitionfunctionof{\sstat,\canparam,\basemeasure} \cdot \expdist$};
        \draw[-<-] (0,-3)--(0,-5) node[midway,left] {\colorlabelsize $\catvariableof{0}$};
        \draw[-<-] (1.5,-3)--(1.5,-5) node[midway,left] {\colorlabelsize $\catvariableof{1}$};
        \node[anchor=center] (text) at (3,-4) {$\cdots$};
        \draw[-<-] (4,-3)--(4,-5) node[midway,right] {\colorlabelsize $\catvariableof{\atomorder\shortminus1}$};

        \node[anchor=center] (text) at (8,-2) {${=}$};
    \end{scope}

    \draw (-1.25,1) rectangle (1.25,3);
    \node[anchor=center] (text) at (0,2) {\corelabelsize $\softactsymbolof{0,\canparam}$};

    \draw (2.75,1) rectangle (5.25,3);
    \node[anchor=center] (text) at (4,2) {\corelabelsize $\softactsymbolof{\seldim\shortminus 1,\canparam}$};

    \draw[->-] (0,-1)--(0,0);
    \node[left] (text) at (0,0) {\colorlabelsize $\headvariableof{0}$};
    \draw[] (0,0)--(0,1);
    \drawvariabledot{0}{0}
    \node[anchor=center] (text) at (2,0) {$\cdots$};

    \draw[->-] (4,-1)--(4,0);
    \node[right] (text) at (4,0) {\colorlabelsize $\headvariableof{\seccatorder\shortminus1}$};
    \draw[] (4,0)--(4,1);
    \drawvariabledot{4}{0}

    \draw (-1,-1) rectangle (5,-3);
    \node[anchor=center] (text) at (2,-2) {\corelabelsize $\bencodingof{\sstat}$};
    \draw[-<-] (0,-3)--(0,-5) node[midway,left] {\colorlabelsize $\catvariableof{0}$};
    \draw[-<-] (1.5,-3)--(1.5,-5) node[midway,left] {\colorlabelsize $\catvariableof{1}$};
    \node[anchor=center] (text) at (3,-4) {$\cdots$};
    \draw[-<-] (4,-3)--(4,-5) node[midway,right] {\colorlabelsize $\catvariableof{\atomorder\shortminus1}$};


    \begin{scope}
        [shift={(0,-4)}]
        \draw[] (0,1)--(0,-6);
        \node[below] (text) at (0,-6) {\colorlabelsize $\catvariableof{0}$};
        \drawvariabledot{0}{-5}
        \draw[] (1.5,1)--(1.5,-6);
        \node[below] (text) at (1.5,-6) {\colorlabelsize $\catvariableof{1}$};
        \drawvariabledot{1.5}{-3}
        \node[anchor=center] (text) at (3,0) {$\cdots$};
        \node[anchor=center] (text) at (3,-5.5) {$\cdots$};
        \draw[] (4,1)--(4,-6);
        \node[below] (text) at (4,-6) {\colorlabelsize $\catvariableof{\atomorder\shortminus1}$};
        \drawvariabledot{4}{-2}

        \draw[] (0,-5) -- (6,-5);
        \draw[] (1.5,-3) -- (6,-3);
        \node[anchor=center] (text) at (5,-3.75) {$\vdots$};
        \draw[] (4,-2) -- (6,-2);
        \draw (6,-1) rectangle (9, -6);
        \node[anchor=center] (text) at (7.5,-3.5) {\corelabelsize $\basemeasure$};

        \node[anchor=center] (text) at (10,-4.5) {$.$};

    \end{scope}

\end{tikzpicture}
    \end{center}
    \caption{Representation of a member in an exponential family by a \ComputationActivationNetwork{} with elementary activation.
    Since the right hand side is not normalized both sides are equal up to a constant.}\label{fig:expdistUnaryRealizable}
\end{figure}

We will use the following well known property (see e.g. \cite{brown_fundamentals_1987}), that there is a one-to-one map between the canonical parameters and the mean parameters.

\begin{lemma}
    \label{lem:interiorRepExpFamily}
    The set of mean parameters of the members of an exponential family is the relative interior of the mean polytope.
    For $\canparam,\seccanparam\in\parspace$ with
    \begin{align*}
        \uniquantwrtof{\selindexin}{
            \contraction{\expdistofat{\sstat,\canparam,\basemeasure}{\shortcatvariables},\sstatcoordinateofat{\selindex}{\shortcatvariables},\basemeasurewith}
            = \contraction{\expdistofat{\sstat,\seccanparam,\basemeasure}{\shortcatvariables},\sstatcoordinateofat{\selindex}{\shortcatvariables},\basemeasurewith}
        }
    \end{align*}
    we furthermore have $\expdistofat{\sstat,\canparam,\basemeasure}{\shortcatvariables}=\expdistofat{\sstat,\seccanparam,\basemeasure}{\shortcatvariables}$.
\end{lemma}
\begin{proof}
    See The~3.3 in \cite{wainwright_graphical_2008}.
\end{proof}

Based on this property we define the forward and backward mappings to an exponential family.

\begin{definition}
    The forward map of an exponential family is the map
    \begin{align*}
        \forwardmap  \defcols \parspace \rightarrow \sbinteriorof{\genmeanset}
    \end{align*}
    defined as
    \begin{align*}
        \uniquantwrtof{\selindexin}{
            \forwardmapofat{\canparam}{\indexedselvariable}
            = \contraction{\expdistofat{\sstat,\canparam,\basemeasure}{\shortcatvariables},\sstatcoordinateofat{\selindex}{\shortcatvariables},\basemeasurewith}
        }
    \end{align*}
    Any map $\backwardmap: \sbinteriorof{\genmeanset}\rightarrow \parspace$ with $\backwardmap\circ\forwardmap=\mathrm{Id}$ is called a backward map.
\end{definition}