\section{Characterization for boolean statistics}

\red{We here study the face CP ranks in case of boolean statistics.
We further show that any elementary \ComputationActivationNetwork{} to boolean statistics is a maximum entropy distribution.}

For boolean statistics $\hlnstat:\facstates\rightarrow\bigtimes_{\selindexin}[2]$ the mean polytope is a subset of the cube $\fullparcube$.
In this case, any boolean vector in $\meansetof{\hlnstat,\basemeasure}$ is a vertex.
It follows, that any distribution reproducing a mean parameter $\meanparamwith$ on the relative interior of $\meansetof{\hlnstat,\basemeasure}$ is positive with respect to $\basemeasure$.

We apply the exponential distribution characterization of the maximum entropy distribution and get that the maximum entropy distribution is in $\elrealizabledistsof{\sstat}$, if and only if the face measure is in $\elrealizabledistsof{\sstat}$.
This is exactly the case, when the face is an intersection of the mean polytope with a face of the cupe $\fullparcube$.

\subsection{Characterization of elementary representable faces}

We first show in the following example, that all faces of a hypercube are representable by elementary activation tensors.

\begin{example}[Hypercube]
    \label{exa:hypercubeFaces}
    In cases where $\shortcatvariables$ are boolean and we have $\selindexin$ features
    \begin{align*}
        \formulaofat{\selindex}{\indexedshortcatvariables} = \catindexof{\selindex}
    \end{align*}
    the mean polytope is the hypercube
    \begin{align*}
        \meansetof{\{\formulaof{\selindex}\wcols\selindexin\},\trivbm} = \fullparcube \, .
    \end{align*}
    The non-empty faces in the face lattice $\facelatticeof{\fullparcube}$ can be enumerated by subsets $\variableset\subset[\seldim]$ and indices $\headindexof{\variableset}\in\bigtimes_{\selindex\in\variableset}[2]$ and represented by the cartesian products
    \begin{align*}
        % \facesetofspec{(\variableset,\headindexof{\variableset})}{} %{\{\formulaof{\selindex}\wcols\selindexin\},\trivbm}
        \facesymbolof{(\variableset,\headindexof{\variableset})}
        = \bigtimes_{\selindexin} \mathcal{I}^{l,(\variableset,\headindexof{\variableset})}
    \end{align*}
    where
    \begin{align*}
        \mathcal{I}^{l,(\variableset,\headindexof{\variableset})}
        = \begin{cases}
        [0,1]
              & \ifspace \selindex\notin\variableset \\
              \{\headindexof{\selindex}\}& \ifspace \selindex\in\variableset
        \end{cases} \, .
    \end{align*}
    Each of these faces can be represented with respect to the elementary graph $\elgraph$, namely by the tensor product of leg vectors
    \begin{align*}
        %\actcoreofat{\selindex}{\headvariableof{\selindex}}
        \hardactlegwith
        = \begin{cases}
              \onesat{\headvariableof{\selindex}} & \ifspace \selindex\notin\variableset \\
              \onehotmapofat{\headindexof{\selindex}}{\headvariableof{\selindex}}& \ifspace \selindex\in\variableset
        \end{cases} \, .
    \end{align*}
    %\red{This will later be interpreted by propositional logics as the example of atomic formulas.}
\end{example}

Orienting on \exaref{exa:hypercubeFaces} we define cube-likeness of faces and polytopes.

\begin{definition}
    \label{def:cubeLike}
    We say that a face $\facesymbol$ of $\hlnmeanset$ is cube-like, if it is empty or there is $\variableset\subset[\seldim]$ and $\headindexof{\variableset}\in\bigtimes_{\selindex\in\variableset}[2]$ such that
    \begin{align*}
        \facesymbol = \meansetof{\hlnstat,\basemeasure} \cap \facesymbolof{(\variableset,\headindexof{\variableset})} \, .
    \end{align*}
    We further say that a polytope $\hlnmeanset$ is cube-like, if all faces $\facesymbol\in\facelatticeof{\hlnmeanset}$ are cube-like.
\end{definition}

We now show that a face is cube-like if and only if it is representable by an elementary tensor.

\begin{theorem}
    \label{the:faceMeasureHardLogicNetworks}
    Let $\hlnstat$ be a boolean statistic, $\basemeasure$ a base measure and $\facesymbol$ be a face of $\hlnmeanset$.
    Then the following are equivalent:
    \begin{itemize}
        \item[(i)] $\facesymbol$ is cube-like (see \defref{def:cubeLike}).
        \item[(ii)] $\facesymbol$ is representable by an elementary tensor (see \defref{def:faceRepresentability}).
   \end{itemize}
\end{theorem}
\begin{proof}
    If the face is empty, i.e. $\facesymbol=\varnothing$, it is by definition cube-like and has $\zerosat{\headvariables}$ as an elementary activation tensor.
    We therefore assume in the following $\facesymbol\neq\varnothing$.

    (i)$\Rightarrow$(ii):
    Let us assume that $\facesymbol$ is cube-like, that is there is $\variableset\subset[\seldim]$ and $\headindexof{\variableset}\in\bigtimes_{\selindex\in\variableset}[2]$ such that $\facesymbol = \meansetof{\hlnstat,\basemeasure} \cap \facesymbolof{(\variableset,\headindexof{\variableset})}$.
    We use that $\facesymbol,\meansetof{\hlnstat,\basemeasure}$ and $\facesymbolof{(\variableset,\headindexof{\variableset})}$ are the convex hulls of the cube vertex sets $\imsetof{\hlnstat,\meanset}{\facesymbol}$, $\imsetof{\hlnstat,\meanset}{}$ and
    \begin{align}\label{eq:cubeFaceVertices}
        \imsetof{}{(\variableset,\headindexof{\variableset})} = \{\imelementwith \wcols \forall \selindex\in\variableset \imelementat{\indexedselvariable}=\headindexof{\variableset}\} \, ,
    \end{align}
    which implies that
    \begin{align*}
        \imsetof{\hlnstat,\meanset}{\facesymbol} = \imsetof{}{(\variableset,\headindexof{\variableset})} \cap \imsetof{\hlnstat,\meanset}{} \, .
    \end{align*}
    Thus, we can choose $\arbset=\imsetof{\hlnstat,\meanset}{\facesymbol}$ for the representation of the face $\facesymbol$ (see \defref{def:faceRepresentability}) and further have that
    \begin{align*}
        \sum_{\imelementwith\in\imsetof{}{(\variableset,\headindexof{\variableset})}}
        = \bigotimes_{\selindexin} \hardactlegwith \, .
    \end{align*}
    We have thus found an elementary activation tensor for the face $\facesymbol$.

    (ii)$\rightarrow$(i)
    Conversely, let $\acttensorwith$ be an elementary activation tensor of the face $\facesymbol$ in $\hlnmeanset$.
    Since $\acttensorwith$ is the sum of different one-hot encodings it is boolean and we find an elementary decomposition $\acttensorwith=\bigotimes_{\selindexin}\acttensorlegwith$ such that the leg vectors $\acttensorwith$ are boolean.
    Since $\facesymbol\neq\varnothing$ we further have for $\selindexin$ that $\acttensorlegwith\neq\zerosat{\headvariableof{\selindex}}$, and thus $\acttensorlegwith\in \{\fbasisat{\headvariableof{\selindex}},\tbasisat{\headvariableof{\selindex}},\onesat{\headvariableof{\selindex}}\}$
    We construct a set $\variableset\subset[\seldim]$
    \begin{align*}
        \variableset = \left\{\selindex \wcols {\acttensorlegwith}\neq\onesat{\headvariableof{\selindex}}\right\}
    \end{align*}
    and a tuple
    \begin{align*}
        \headindexof{\selindex}
        = \begin{cases}
              1 & \ifspace {\acttensorlegwith}=\onehotmapofat{1}{\headvariableof{\selindex}} \\
              0 & \ifspace {\acttensorlegwith}=\onehotmapofat{0}{\headvariableof{\selindex}}
        \end{cases} \, .
    \end{align*}
    By construction ${\acttensorlegwith} = \hardactlegwith$ (see \exaref{exa:hypercubeFaces}) follows for all $\selindexin$ and therefore $\acttensorwith=\hardacttensorwith$.
    We therefore have for the set \eqref{eq:cubeFaceVertices}
    \begin{align*}
        \acttensorwith = \sum_{\imelement\in\imsetof{}{\variableset,\headindexof{\variableset}}} \onehotmapofat{\imelement}{\headvariables}
    \end{align*}
    and since $\acttensorwith$ is an activation tensor for $\facesymbol$, that 
    \begin{align*}
        \imsetof{\hlnstat,\meanset}{\facesymbol}
        = \imsetof{}{\variableset,\headindexof{\variableset}}\cap\imsetof{\hlnstat,\meanset}{} \, .
    \end{align*}
    Taking convex hulls on both sides, this is equivalent to
    \begin{align*}
        \facesymbol = \facesymbolof{(\variableset,\headindexof{\variableset})} \cap \meansetof{\hlnstat,\basemeasure} \, .
    \end{align*}
    We conclude that $\facesymbol$ is cube-like.
\end{proof}

Let us now give with the standard simplices a class of convex polytopes, which are not hypercubes, but cube-like.

\begin{example}[Simplices are cube-like]
    The $(\seldim-1)$-dimensional standard simplex $\meansetof{\triangle,\seldim-1}$ in $[0,1]^{\seldim}$ is the convex hull of the vertex sets
    \begin{align*}
        \imset = \{\onehotmapofat{\selindex}{\selvariable} \wcols\selindexin\} \, . % selvariable
    \end{align*}
    %where $\selvariable$ takes values in $[\seldim-1]$.
    The simplex is the mean polytope of the $\seldim$-dimensional statistic with
    \begin{align*}
        \sstatcoordinateof{\selindex} \defcols [\seldim] \rightarrow [2] \quad, \quad
        \sstatcoordinateofat{\selindex}{\secselindex}
        = \begin{cases}
              1 & \ifspace \secselindex = \selindex \\
              0 & \ifspace \secselindex \neq \selindex
        \end{cases} \, .
    \end{align*}
    This is sometimes referred to as the probability simplex, since it represents all distributions of a discrete variable with $\seldim$ states.

    The face lattice of the simplex consists is enumerated by subsets of $[\seldim]$ as
    \begin{align*}
        \facelatticeof{\meansetof{\triangle,\seldim-1}}
        = \{\facesymbolof{\variableset}\wcols\variableset\subset[\seldim]\}
    \end{align*}
    where the faces are
    \begin{align*}
        \facesymbolof{\variableset} = \convhullof{\onehotmapofat{\selindex}\wcols\selindex\in\variableset}
    \end{align*}
    and therefore itself $(\cardof{\variableset}-1)$-dimensional standard simplices.
    The partial order of the faces coincides with the inclusion order of subsets $\variableset$.
    Furthermore, each face $\facesymbolof{\variableset}$ is cube-like since
    \begin{align*}
        \facesymbolof{\variableset} = \meansetof{\triangle,\seldim-1} \cap \facesymbolof{[\seldim]/\variableset,0_{[\seldim]/\variableset}}
    \end{align*}
    where $\facesymbolof{[\seldim]/\variableset,0_{[\seldim]/\variableset}}$ are faces of the hypercube following the notation of \exaref{exa:hypercubeFaces}.
\end{example}

\subsection{Set of maximum entropy distributions}

\begin{theorem}
    Any distribution in $\realizabledistsof{\hlnstat,\elgraph,\basemeasure}$ is a maximum entropy distribution with respect to $(\hlnstat,\meanparam,\basemeasure)$ where $\meanparam$ is its mean parameter.
    Any maximum entropy distribution is realized by $\realizabledistsof{\hlnstat,\elgraph,\basemeasure}$ if and only if the mean parameter is in the relative interior of a cube-like face.
\end{theorem}
\begin{proof}
    First claim by decomposing any elementary tensor into exponential and hard activation core.
    Second claim by characterization of elementary faces by cube-likeness.
\end{proof}

We can now use the same notation as applied for hypercubes to classify the faces of a cube-like polytope.

\subsection{Interpretation by propositional formulas}

We can understand each feature as a propositional formula and the variables $\shortcatvariables$ as atoms (possibly after a binarization).

Each vertex of the cube, which is not a vertex of the polytope corresponds with the unsatisfiability of a formula
\begin{align*}
    \bigwedge_{\selindexin} \lnot^{1-\meanparamat{\indexedselvariable}} \formulaofat{\selindex}{\shortcatvariables}
\end{align*}
which is equal with any of the entailment statements for $\variableset\subset[\seldim]$ % Can extend to any partition!
\begin{align*}
    \left(\bigwedge_{\selindex\in\variableset} \lnot^{1-\meanparamat{\indexedselvariable}} \formulaofat{\selindex}{\shortcatvariables} \right)
    \models \left(\bigwedge_{\selindex\in\variableset}\lnot^{\meanparamat{\indexedselvariable}} \formulaofat{\selindex}{\shortcatvariables} \right) \, .
\end{align*}

Along this interpretation we can easily construct examples of statistics, which polytopes are not cube-like.

\begin{example}[Maximum entropy distribution with non-elementary activation cores]\label{exa:nonelHlnstat}

    Consider two atomic variables $\catvariableof{0}$ and $\catvariableof{1}$ and a statistic $\formulaset$ consisting in the formulas
    \begin{align*}
        \formulaof{0} = \left( \catvariableof{0} \land \catvariableof{1} \right) \quad, \quad \formulaof{1} = \left( \catvariableof{0} \Rightarrow \catvariableof{1} \right)
    \end{align*}
    with the coordinatewise expressions
    \begin{align*}
        \formulaof{0} =
        \begin{bmatrix}
            0 & 0 \\
            0 & 1
        \end{bmatrix}
        \quad, \quad
        \formulaof{1} =
        \begin{bmatrix}
            1 & 1 \\
            0 & 1
        \end{bmatrix} \, .
    \end{align*}
    % Interpretation in accounting
    We can think of $\catvariableof{0}$ as a feature on an invoice, and $\catvariableof{1}$ as a feature on the accounting proposal.

    From this we have
    \begin{align*}
        &\bencodingofat{(\formulaof{0},\formulaof{1})}{\headvariableof{0}=0,\headvariableof{1}=0,\catvariableof{0},\catvariableof{1}} =
        \begin{bmatrix}
            0 & 0 \\
            1 & 0
        \end{bmatrix} \quad, \quad
        \bencodingofat{(\formulaof{0},\formulaof{1})}{\headvariableof{0}=0,\headvariableof{1}=1,\catvariableof{0},\catvariableof{1}} =
        \begin{bmatrix}
            1 & 1 \\
            0 & 0
        \end{bmatrix} \quad, \quad \\
        &\bencodingofat{(\formulaof{0},\formulaof{1})}{\headvariableof{0}=1,\headvariableof{1}=0,\catvariableof{0},\catvariableof{1}} =
        \begin{bmatrix}
            0 & 0 \\
            0 & 0
        \end{bmatrix} \quad \text{and} \quad
        \bencodingofat{(\formulaof{0},\formulaof{1})}{\headvariableof{0}=1,\headvariableof{1}=1,\catvariableof{0},\catvariableof{1}} =
        \begin{bmatrix}
            0 & 0 \\
            0 & 1
        \end{bmatrix} \, .
    \end{align*}

    Since the only vanishing slice of $\bencodingof{\formulaset}$ with respect to the head variables is that to $\headindexof{0,1} = (1,0)$, the vertices of the mean polytope are the vectors to the other head indices.
    The mean polytope is the convex hull of these vertices
    %\begin{align*}
    %    \formulaof{0} \models \formulaof{1}
    %\end{align*}
    %and therefore $\lnot\formulaof{0}\land\formulaof{1}$ is unsatisfiable.
    %The other combinations $\lnot\formulaof{0}\land\lnot\formulaof{1}, \, \formulaof{0}\land\lnot\formulaof{1}$ and $\formulaof{0}\land\formulaof{1}$ are all satisfiable.
    %The mean polytope is thus the convex hull
    \begin{align*}
        \meansetof{(\formulaof{0},\formulaof{1})} =
        \convhullof{\begin{bmatrix}
                        0 \\ 0
        \end{bmatrix},
            \begin{bmatrix}
                0 \\ 1
            \end{bmatrix},
            \begin{bmatrix}
                1 \\ 1
            \end{bmatrix}} \, .
    \end{align*}

    This polytope has a non cube-like face (sketched blue in \figref{fig:nonelHlnstatMaxent}), which is the convex hull of the vertices $[0 \, 0]^T, \, [1 \, 1]^T$.
    This face is parametrized by the ($\cpformat$-rank 2) hard activation core
    \begin{align*}
        \kcoreofat{(0,0),(1,1)}{\headvariableof{0},\headvariableof{1}} =
        \onehotmapofat{(0,0)}{\headvariableof{0},\headvariableof{1}} + \onehotmapofat{(1,1)}{\headvariableof{0},\headvariableof{1}} =
        \begin{bmatrix}
            1 & 0 \\
            0 & 1
        \end{bmatrix}
    \end{align*}
    and has the face measure
    \begin{align*}
        \contractionof{\kcoreofat{(0,0),(1,1)}{\headvariableof{0},\headvariableof{1}}
            ,\bencodingofat{\formulaset}{\headvariableof{0},\headvariableof{1},\catvariableof{0},\catvariableof{1}}}{\catvariableof{0},\catvariableof{1}}
        =   \begin{bmatrix}
                0 & 0 \\
                1 & 1
        \end{bmatrix} \, .
    \end{align*}
    Any mean parameter $\meanparam$ on the interior of that face can be parametrized by a scalar $\lambda\in(0,1)$
    \begin{align*}
        \meanparamofat{\lambda}{\selvariable} = \begin{bmatrix}
                                                    \lambda & \lambda
        \end{bmatrix}^T \, .
    \end{align*}
    With the canonical parameters $\canparamat{\selvariable}\in\rr^2$ of the maximum entropy distributions on this face by
    \begin{align*}
        \probat{\catvariableof{0},\catvariableof{1}} =
        \frac{1}{1+\expof{\canparamat{\selvariable=0}+\canparamat{\selvariable=1}}}
        \begin{bmatrix}
            0 & 0                                                               \\
            1 & \expof{\canparamat{\selvariable=0}+\canparamat{\selvariable=1}} \\
        \end{bmatrix} \,
    \end{align*}
    we get the correspondence by the sigmoid
    \begin{align*}
        \lambda = \frac{1}{1+\expof{-(\canparamat{\selvariable=0}+\canparamat{\selvariable=1})}} \, .
    \end{align*}

    Note, that the hard activation core $\kcoreofat{(0,0),(1,1)}{\headvariableof{0},\headvariableof{1}}$ to the blue face is the only non-elementary activation core.
    While the vertices have always elementary cores, the further non-vertex faces have elementary activation cores
    \begin{align*}
        &\kcoreofat{(0,0),(1,0),(1,1)}{\headvariableof{0},\headvariableof{1}}
        = \begin{bmatrix}
              1 & 1 \\
              1 & 1
        \end{bmatrix}
        = \onesat{\headvariableof{0}} \otimes \onesat{\headvariableof{1}}
        \quad, \quad
        \kcoreofat{(0,0),(1,0)}{\headvariableof{0},\headvariableof{1}}
        = \begin{bmatrix}
              1 & 0 \\
              1 & 0
        \end{bmatrix}
        = \onesat{\headvariableof{0}} \otimes \onehotmapofat{0}{\headvariableof{1}}
        \quad, \quad \\
        &\kcoreofat{(1,0),(1,1)}{\headvariableof{0},\headvariableof{1}}
        = \begin{bmatrix}
              1 & 1 \\
              0 & 0
        \end{bmatrix}
        = \onehotmapofat{0}{\headvariableof{0}} \otimes \onesat{\headvariableof{1}}   \, .
    \end{align*}
    The maximum entropy distributions to mean parameters on the interior of all other faces than the blue face are represented by \ComputationActivationNetwork{}s with only elementary activation cores.

    \begin{figure}
        \begin{center}
            \begin{tikzpicture}[scale=0.35]

    \node[anchor=center] at (-4,6) {$a)$};

    \node[anchor=east] at (0,0) {$\begin{bmatrix}
                                      0 \\ 0
    \end{bmatrix}$};
    \node[anchor=west] at (5,0) {$\begin{bmatrix}
                                      1 \\ 0
    \end{bmatrix}$};
    \node[anchor=west] at (5,5) {$\begin{bmatrix}
                                      1 \\ 1
    \end{bmatrix}$};
    \node[anchor=east] at (0,5) {$\begin{bmatrix}
                                      0 \\ 1
    \end{bmatrix}$};

    \drawvectormark{0}{0}
    \drawvectormark{0}{5}
    \drawvectormark{5}{0}
    \drawvectormark{5}{5}

    \draw[thick] (5,5) -- (5,0) -- (0,0);
    \draw[dashed] (0,0) -- (0,5) -- (5,5);
    \draw[\concolor, thick] (0,0) -- (5,5);

    \drawvectormark{3}{3}
    \node[anchor=east] at (3,3) {$\meanparam^{\lambda}$};

    \begin{scope}
        [shift={(20,2)}]

        \node[anchor=center] at (-4,4) {$b)$};

        %\draw[\concolor] (4,3) to[bend right=20] (2,5);
        %\draw[\concolor] (0,3) to[bend left=20] (2,5);
        %\draw[fill,\concolor] (2,5) circle (\dotsize);

        %\draw[\concolor] (-1,1) rectangle (1,3);
        %\node[anchor=center,\concolor] (text) at (0,2) {\corelabelsize $\hardactsymbolof{0}$};

        %\draw[\concolor] (3,1) rectangle (5,3);
        %\node[anchor=center,\concolor] (text) at (4,2) {\corelabelsize $\hardactsymbolof{1}$};

        \draw[\concolor] (-1,1) rectangle (5,4);
        \node[anchor=center,\concolor] (A) at (2,2.5) {\corelabelsize $\begin{bmatrix}
                                        1 & 0 \\
                                        0 & 1
        \end{bmatrix}$};


        \draw[->-] (0,-1)--(0,0);
        \node[right] (text) at (0,0) {\colorlabelsize $\headvariableof{0}$};
        \draw[\concolor] (0,0)--(0,1);
        \drawvariabledot{0}{0}

        \draw[\probcolor] (0,0) -- (-1.5,0);
        \draw[\probcolor] (-1.5,1.5) rectangle (-8.5,-1.5);
        \node[anchor=center,\probcolor] (text) at (-5,0) {\corelabelsize $\begin{bmatrix}
                                                                             1 \\
                                                                             \expof{\canparamat{\selvariable=0}}
        \end{bmatrix}$};
%       \node[anchor=center,\probcolor] (text) at (-3,0) {\corelabelsize $\softactsymbolof{0,\canparam}$};

        \draw[->-] (4,-1)--(4,0);
        \node[left] (text) at (4,0) {\colorlabelsize $\headvariableof{1}$};
        \draw[\concolor] (4,0)--(4,1);
        \drawvariabledot{4}{0}

        \draw[\probcolor] (4,0) -- (5.5,0);
        \draw[\probcolor] (5.5,1.5) rectangle (12.5,-1.5);
        \node[anchor=center,\probcolor] (text) at (9,0) {\corelabelsize $\begin{bmatrix}
                                                                             1 \\
                                                                             \expof{\canparamat{\selvariable=1}}
        \end{bmatrix}$};

%        \node[anchor=center,\probcolor] (text) at (7,0) {\corelabelsize $\softactsymbolof{1,\canparam}$};

        \draw (-1,-1) rectangle (5,-3);
        \node[anchor=center] (text) at (2,-2) {\corelabelsize $\bencodingof{(\formulaof{0},\formulaof{1})}$};
        \draw[-<-] (0,-3)--(0,-5) node[midway,left] {\colorlabelsize $\catvariableof{0}$};

        \draw[-<-] (4,-3)--(4,-5) node[midway,right] {\colorlabelsize $\catvariableof{1}$};

    \end{scope}

\end{tikzpicture}
        \end{center}
        \caption{
            a) Mean polytope of the statistic $\formulaset=(\catvariableof{0} \land \catvariableof{1}, \catvariableof{0} \Rightarrow \catvariableof{1})$ (thick), as a subset of the cube $[0,1]^2$ (dashed).
            The blue line is the face of the polytope, which is not cube like, that is not an intersection of the polytope with the faces of the polytope.
            We further define for $\lambda\in(0,1)$ a mean parameter $\meanparamofat{\lambda}{\selvariable} = [\lambda \,  \lambda]^T$ which is on the interior of the blue face.
            b) Corresponding \ComputationActivationNetwork{} being the maximum entropy distribution reproducing $\meanparamofat{\lambda}{\selvariable}$, when $\lambda$ is the sigmoid of $\canparamat{\selvariable=0}+\canparamat{\selvariable=1}$.
        }\label{fig:nonelHlnstatMaxent}
    \end{figure}


        \begin{figure}
        \begin{center}
            \begin{tikzpicture}[scale=0.35]

    \node[anchor=center] at (0,9) {$\facesymbolof{\varnothing}$};
    \draw[->] (-1,8.5) -- (-6,6.5);
    \draw[->] (0,8) -- (0,7);
    \draw[->] (1,8.5) -- (6,6.5);

    \node[anchor=center] at (-8,6) {$\facesymbolof{(0,0)}$};
    \draw[->] (-8,5) -- (-8,4);

    \node[anchor=center] at (0,6) {$\facesymbolof{(1,0)}$};
    \draw[->] (-1,5.25) -- (-6,3.5);
    \draw[->] (0,5) -- (0,4);

    \node[anchor=center] at (8,6) {$\facesymbolof{(1,1)}$};

    \draw[->] (7,5.25) -- (2,3.5);

    \node[anchor=center] at (-8,3) {$\facesymbolof{(0,0),(1,0)}$};
    \draw[->] (-7.5,2) -- (-1.5,1);
    \node[anchor=center] at (0,3) {$\facesymbolof{(1,0),(1,1)}$};
    \draw[->] (0,2) -- (0,1);

    \node[] (shift) at (5,0) {};
    \draw[->] (8,5) -- ($(7,4)+(shift)$);
    \draw[->] (-7.5,5) -- ($(6,3.5)+(shift)$);
    \node[anchor=center] at ($(8,3)+(shift)$) {\textcolor{\concolor}{$\facesymbolof{(0,0),(1,1)}$}};
    \draw[->] ($(7,2)+(shift)$) -- (1.5,1);

    \draw[dashed] (9.5,-1) -- (9.5,10);
    \node[anchor=west] at (9.75,9.5) {\corelabelsize $\cprankof{\facesymbol}=2$};
    \node[anchor=east] at (9.25,9.5) {\corelabelsize $\cprankof{\facesymbol}=1$};

    \node[anchor=center] at (0,0) {$\facesymbolof{(0,0),(1,0),(1,1)}$};

\end{tikzpicture}
        \end{center}
        \caption{
           Face lattice $\facelatticeof{(\catvariableof{0} \land \catvariableof{1},\catvariableof{0} \Rightarrow \catvariableof{1})}$ to \exaref{exa:nonelHlnstat}.
            The directed arrows represent inclusion of the faces, which is a partial order of the faces.
        The face \textcolor{\concolor}{$\facesymbolof{(0,0),(1,1)}$} is the only face, which is not representable by an elementary hypergraph.
            It is representable in a $\cpformat$ with hidden rank $2$
        }\label{fig:latticeNonelHlnstat}
    \end{figure}

\end{example}

\begin{example}[Atomic formulas]
    \label{exa:atomicFormulasHypercube}
    %The assumption of \theref{the:sufficientHLNExpressivity} is satisfied in
    Let us consider the case of atomic formulas. % where the formulas $\formulaof{\formulaset,\canparam}$ are the atoms.
    The mean polytope in this case is the $\catorder$-dimensional hypercube
    \begin{align*}
        \meansetof{\atomformulaset,\ones} = \fullparcube
    \end{align*}
    which is called a simple polytope, since each vertex is contained in the minimal number of $\catorder$ facets.
    %Since the cube $\fullparcube$ is a face of itself, \theref{the:sufficientHLNExpressivity} implies $\hlnmeanset|_{\hlnsetof{\atomformulaset}}=\hlnmeanset$.

    The faces of a hypercube are enumerated in the following way.
    Each face is characterized by the projections onto each variable, which is either $\{0\}$, $\{1\}$ or $[0,1]$.
    The projections are represented by the tuple $\hardparam$ defined in the following way:
    \begin{itemize}
        \item We define the set $\hardlegset\subset[\atomorder]$ of variables, such that the projection onto the variable is $\{0\}$ or $\{1\}$
        \item We define to each $\selindex\in\hardlegset$ an index $\headindexof{\selindex}=0$ if the projection is $\{0\}$ and $\headindexof{\selindex}=1$ if the projection is $\{1\}$.
    \end{itemize}

%    There are thus $2^{\atomorder}$ different sets $\hardlegset$, each with $2^{\hardlegset}$ faces by a choice of $\headindexof{\selindex}$.

    Trivially, each face of the hypercube is a cube face and $\elmeansetof{\atomformulaset}=\meansetof{\atomformulaset}$.

    \red{More general: If and only if no combination of possibly negated formulas is unsatisfiable, then the mean polytope is a hypercube.}
\end{example}

\begin{example}[Minterm formulas]
    \label{exa:mintermHLNSet}
    The set of minterm formulas is indexed by $\shortcatindicesin$ and given by
    \begin{align*}
        \formulaofat{\shortcatindices}{\shortcatvariables} = \bigwedge_{\selindex\in[\catorder]} \lnot^{1-\catindexof{\selindex}} \formulaofat{\selindex}{\shortcatvariables}
        = \onehotmapofat{\shortcatindices}{\shortcatvariables}
    \end{align*}
    where by $\formulaofat{\selindex}{\shortcatvariables}$ we denote the $\selindex$-th atomic formula (see \exaref{exa:atomicFormulasHypercube}).
    The mean polytope is in the case of the minterm statistic (also referred to as universal statistic) and a boolean base measure $\basemeasure$ the standard simplex of dimension
    \begin{align*}
        \cardof{\shortcatindicesin \wcols \basemeasureat{\indexedshortcatvariables}\neq 0}-1 \, ,
    \end{align*}
    that is the set
    \begin{align*}
        \elmeansetof{\mintermformulaset,\basemeasure}
        = \convhullof{\onehotmapofat{\shortcatindices}{\headvariables}\wcols\shortcatindicesin\ncond\basemeasureat{\indexedshortcatvariables}\neq 0} \, .
    \end{align*}
    In this case, $\hlnsetof{\mintermformulaset}$ contains any distribution and therefore trivially realizes any mean parameter in $\meansetof{\mintermformulaset,\ones}$.

    The faces of the standard simplex are itself standard simplices to base measures $\secbasemeasure$ with $\secbasemeasure\prec\basemeasure$.
    We store them by the tuple $\hardparam$, where $\hardlegset$ is the support of $\secbasemeasure$ and $\headindexof{\selindex}=0$ for $\selindex\in\hardlegset$.
    Each face is a cube face, since it is the intersection of $\hlnsetof{\mintermformulaset}$ with the cube face $\hardparam$.
    In particular, we have $\elmeansetof{\mintermformulaset}=\meansetof{\mintermformulaset}$.

    \red{More general the mean polytope is a standard simplex, if and only if each formula contradicts all others.}
\end{example}

\begin{example}[$\ttformat$ representation of diagonal faces]
    \label{exa:tt_diagonal_face}
    Consider the vertex set
    \begin{align*}
        \imset \coloneqq \{0,1\}^{\seldim} / \imelementat{\selvariable}
    \end{align*}
    convex polytope.
    We are interested in the face $\facesymbolof{\triangleleft,1}$ with the normal $\frac{1}{2} (2\imelementwith - \onesat{\selvariable})$.
    Its vertices are the $\seldim$ elements of $\{0,1\}^{\seldim}$, which differ from $\imelementwith$ in exactly one coordinate (i.e. those with Hamming distance of $1$ from $\imelementat{\selvariable}$).
    We denote these vertices by $\imelementof{\selindex}$ for $\selindexin$, which coordinates to $\secselindex\in[\seldim]$ are
    \begin{align*}
        \imelementofat{\selindex}{\selvariable=\secselindex}
        = \begin{cases}
              \imelementat{\selvariable=\secselindex} & \ifspace \secselindex\neq\selindex \\
              1-\imelementat{\selvariable=\secselindex} & \ifspace \secselindex=\selindex
        \end{cases} \, .
    \end{align*}

    We now represent their sum in an $\ttformat$ with hidden ranks $r_0=\ldots=r_{\seldim-2}=2$.
    The boolean hidden variables are denoted be $\decvariableof{[\seldim-1]}$ and can be interpreted as indicators, whether the coordinate flip has happened in $[\selindex]$ coordinates.
    We now construct a $\ttformat$ cores for $\selindex=0$ by
    \begin{align*}
        \hypercoreofat{0}{\headvariableof{0},\decvariableof{0}}
        = \onehotmapofat{\imelementat{\selvariable=0}}{\headvariableof{0}} \otimes \fbasisat{\decvariableof{0}}
        + \onehotmapofat{1-\imelementat{\selvariable=0}}{\headvariableof{0}} \otimes \tbasisat{\decvariableof{0}}
    \end{align*}
    further for $\selindex\notin\{0,\seldim-1\}$ the cores
    \begin{align*}
        \hypercoreofat{\selindex}{\decvariableof{\selindex-1},\headvariableof{\selindex},\decvariableof{\selindex}}
        = \, & \fbasisat{\decvariableof{\selindex-1}} \otimes \onehotmapofat{1-\imelementat{\indexedselvariable}}{\headvariableof{\selindex}} \otimes \tbasisat{\decvariableof{\selindex}}
        + \fbasisat{\decvariableof{\selindex-1}} \otimes \onehotmapofat{\imelementat{\indexedselvariable}}{\headvariableof{\selindex}} \otimes \fbasisat{\decvariableof{\selindex}} \\
        &+ \tbasisat{\decvariableof{\selindex-1}} \otimes \onehotmapofat{\imelementat{\indexedselvariable}}{\headvariableof{\selindex}} \otimes \tbasisat{\decvariableof{\selindex}}
    \end{align*}
    and for $\selindex=\seldim-1$
    \begin{align*}
        \hypercoreofat{\seldim-1}{\decvariableof{\seldim-1},\headvariableof{\seldim-1}}
        = \fbasisat{\decvariableof{\seldim-1}} \otimes \onehotmapofat{1-\imelementat{\selvariable=\seldim-1}}{\headvariableof{\seldim-1}}
        + \tbasisat{\decvariableof{\seldim-1}} \otimes \onehotmapofat{\imelementat{\selvariable=\seldim-1}}{\headvariableof{\seldim-1}} \, .
    \end{align*}
    For this tensor network in the $\ttformat$ format we have
    \begin{align*}
        \sum_{\selindexin} \onehotmapofat{\imelementof{\selindex}}{\headvariables}
        = \contractionof{\{\hypercoreofat{0}{\headvariableof{0},\decvariableof{0}},\hypercoreofat{\seldim-1}{\decvariableof{\seldim-1},\headvariableof{\seldim-1}}\}\cup\{\hypercoreofat{\selindex}{\decvariableof{\selindex-1},\headvariableof{\selindex},\decvariableof{\selindex}}\wcols\selindex\notin\{0,\seldim-1\}\}}{\headvariables} \, .
    \end{align*}
    The $\ttformat$ multirank of $2$ is furthermore minimal, since each matrification based on a partition of $[\seldim]$ into non-empty sets has a matrix rank of $2$.
    With respect to such partitions also the tensor $\onehotmapofat{\imelement}{\headvariables}+\left(\sum_{\selindexin}\onehotmapofat{\imelementof{\selindex}}{\headvariables}\right)$ has matrix ranks of $2$.
    Since these are the only two activation tensors for the face $\facesymbolof{\triangleleft,1}$.
\end{example}

\begin{example}[Generalization of \exaref{exa:tt_diagonal_face} to larger $\ttformat$ ranks]
    \label{exa:gen_tt_diagonal_face}
    %% Extension to larger Hadamard distances
    We now generalize the construction of \exaref{exa:tt_diagonal_face} by using the Hadamard distance $d(\cdot,\cdot)$ in $\{0,1\}^{\seldim}$, which counts the number of coordinates two vertices differ in.
    For $s\in\{1,\ldots,\seldim-1\}$ we define for a fixed $\imelementwith\in\{0,1\}^{\seldim}$ a polytope as the convex hull
    \begin{align*}
        \{\tilde{\imelement}[\selvariable] \wcols d(\tilde{\imelement},\imelement)\geq s\}  \, .
    \end{align*}
    The face to the normal $\frac{1}{2} (2\imelementwith - \onesat{\selvariable})$ is the convex hull
    \begin{align*}
        \facesymbolof{\triangleleft,s}\coloneqq \convhullof{
            \{\tilde{\imelement}[\selvariable] \wcols d(\tilde{\imelement},\imelement) = s\}
        }
    \end{align*}
    containing $\binom{\seldim}{s}$ vertices.
    We label these vertices by subsets $\variableset\subset[\seldim]$ of cardinality $s$ and define for $\selindexin$
    \begin{align*}
        \imelementofat{\variableset}{\indexedselvariable} =
        \begin{cases}
            \imelementat{\selindexin} & \ifspace \selindex\notin\variableset \\
            1-\imelementat{\selindexin} & \ifspace \selindex\notin\variableset
        \end{cases}
    \end{align*}
    We now construct a $\ttformat$ with hidden variables $\decvariableof{\selindex}$ and ranks $r_{\selindex} = \min (\selindex+1,\seldim-\selindex+1, s+1)$ to represent the sum of their one-hot encodings.
    To this end, let there be $\ttformat$ cores for $\selindex=0$ by
    \begin{align*}
        \hypercoreofat{0}{\headvariableof{0},\decvariableof{0}}
        = \onehotmapofat{\imelementat{\selvariable=0}}{\headvariableof{0}} \otimes \fbasisat{\decvariableof{0}}
        + \onehotmapofat{1-\imelementat{\selvariable=0}}{\headvariableof{0}} \otimes \tbasisat{\decvariableof{0}}
    \end{align*}
    further for $\selindex\notin\{0,\seldim-1\}$ the cores
    \begin{align*}
        \hypercoreofat{\selindex}{\decvariableof{\selindex-1},\headvariableof{\selindex},\decvariableof{\selindex}}
        = \sum_{\selindex\in\{\max(s-\seldim+\selindex,0),\ldots,\max(\selindex,s)\}}
        \onehotmapofat{\selindex}{\decvariableof{\selindex}}
        \otimes \left(\onehotmapofat{\imelementat{\indexedselvariable}}{\headvariableof{\selindex}} \otimes \onehotmapofat{\selindex}{\decvariableof{\selindex+1}}
                    + \onehotmapofat{1-\imelementat{\indexedselvariable}}{\headvariableof{\selindex}} \otimes \onehotmapofat{\selindex+1}{\decvariableof{\selindex+1}}
        \right)
    \end{align*}
    and for $\selindex=\seldim-1$
    \begin{align*}
        \hypercoreofat{\seldim-1}{\decvariableof{\seldim-1},\headvariableof{\seldim-1}}
        = \fbasisat{\decvariableof{\seldim-1}} \otimes \onehotmapofat{1-\imelementat{\selvariable=\seldim-1}}{\headvariableof{\seldim-1}}
        + \tbasisat{\decvariableof{\seldim-1}} \otimes \onehotmapofat{\imelementat{\selvariable=\seldim-1}}{\headvariableof{\seldim-1}} \, .
    \end{align*}
    Based on the interpretation, that the hidden variables $\decvariableof{\selindex}$ count the Hamming distance of the vectors $\restrictionofto{\imelementwith}{\rr^{\selindex}\times 0_{\seldim-\selindex}}$ and the respective $\restrictionofto{\imelementofat{\variableset}{\selvariable}}{\rr^{\selindex}\times 0_{\seldim-\selindex}}$ one can show that
    \begin{align*}
        \sum_{\variableset\subset[\seldim] \wcols \cardof{\variableset}=s} \onehotmapofat{\imelementof{\variableset}}{\headvariables}
        = \contractionof{\{\hypercoreofat{0}{\headvariableof{0},\decvariableof{0}},\hypercoreofat{\seldim-1}{\decvariableof{\seldim-1},\headvariableof{\seldim-1}}\}\cup\{\hypercoreofat{\selindex}{\decvariableof{\selindex-1},\headvariableof{\selindex},\decvariableof{\selindex}}\wcols\selindex\notin\{0,\seldim-1\}\}}{\headvariables} \, .
    \end{align*}
\end{example}

\begin{example}[Generalization of \exaref{exa:gen_tt_diagonal_face} to arbitrary $\htformat$ formats]
    Instead of aligning the Hamming count variables linearly, one can find a representation of the activation tensor
    \begin{align*}
        \sum_{\variableset\subset[\seldim] \wcols \cardof{\variableset}=s} \onehotmapofat{\imelementof{\variableset}}{\headvariables}
    \end{align*}
    in an arbitrary directed acyclic tree hypergraph format $\graph$, which hidden ranks are bounded by the number of leafs in the subtree and $s+1$.

    At any leaf of the tree we define a $2x2$ matrix
    \begin{align*}
        \hypercoreofat{\selindex}{\headvariableof{\selindex},\decvariableof{\selindex}}
        = \onehotmapofat{\imelementat{\indexedselvariable}}{\headvariableof{\selindex}} \otimes \fbasisat{\decvariableof{\selindex}}
        + \onehotmapofat{1-\imelementat{\indexedselvariable}}{\headvariableof{\selindex}} \otimes \tbasisat{\decvariableof{\selindex}} \, .
    \end{align*}
    At each intermediate non-root hyperedge we choose an outgoing counting variable $\decvariableof{\outgoingnodes}$ with dimension by $\decdimof{\outgoingnodes}=\min(\sum_{\node\in\incomingnodes}\decdimof{\incomingnodes},s+1)$ and define a tensor with the slices
    \begin{align*}
        \hypercoreofat{\edge}{\decvariableof{\outgoingnodes},\decvariableof{\incomingnodes}}
        = \bencodingofat{+}{\headvariableof{+}=\decindexof{\outgoingnodes},\decvariableof{\incomingnodes}} \, .
    \end{align*}
    We further build at the root hyperedge $\edge$ a tensor
    \begin{align*}
        \hypercoreofat{\edge}{\decvariableof{\edge}}
        = \sum_{\decindex \wcols \contraction{\decindex}=s}
        \onehotmapofat{\decindexof{\edge}}{\decvariableof{\edge}} \, .
    \end{align*}
    Using the counting variable interpretation one can now show that
    \begin{align*}
        \sum_{\variableset\subset[\seldim] \wcols \cardof{\variableset}=s} \onehotmapofat{\imelementof{\variableset}}{\headvariables}
        =\contractionof{
        \{\hypercoreofat{\edge}{\decvariableof{\edge}} \wcols \edgein \} \cup \{\hypercoreofat{\selindex}{\headvariableof{\selindex},\decvariableof{\selindex}} \wcols \nodein\}
        }{\headvariables} \, .
    \end{align*}
\end{example}

\subsection{Construction of \HybridLogicNetworks{}}

We now constructively show, that any convex polytope with boolean vertices in $\parspace$ (a so called 0-1 polytope, see \cite{ziegler_lectures_2000}) is the mean polytope of a family of \HybridLogicNetworks{}.

\begin{theorem}
    Let $\meanset$ an arbitrary polytope with boolean vertices in $\parspace$.
    Then we construct propositional formulas on atoms $\catvariableof{[\seldim]}$ by
    \begin{align*}
        \formulaofat{0}{\catvariableof{0}} =
        \begin{cases}
            \top & \ifspace \restrictionofto{\meanset}{\rr^1\times 0_{\seldim-1}} = \{1\} \\
            \bot & \ifspace \restrictionofto{\meanset}{\rr^1 \times 0_{\seldim-1}} = \{0\} \\
            \catvariableof{0} & \ifspace \restrictionofto{\meanset}{\rr^1 \times 0_{\seldim-1}} = [0,1]
        \end{cases}
    \end{align*}
    and iteratively for $\selindexin$ with $\selindex\geq1$ by
    \begin{align*}
        \formulaofat{\selindex}{\catvariableof{[\selindex+1]}} =
        \bigwedge_{v[\selvariable] \in \restrictionofto{\meanset}{\rr^\selindex\times 0_{\seldim-\selindex}}\cap \{0,1\}^{\seldim}}
        \left(\left(\bigwedge_{\secselindex\in[\selindex+1]} \lnot^{1-v[\indexedselvariable]} \formulaofat{\selindex}{\catvariableof{[\selindex]}}\right) \Rightarrow
            \begin{cases}
                \top & \ifspace \restrictionofto{\meanset}{v\times\rr^{1} \times 0_{\seldim-\selindex-1}} = \{1\} \\
                \bot & \ifspace \restrictionofto{\meanset}{v\times\rr^{1} \times 0_{\seldim-\selindex-1}} = \{0\} \\
                \catvariableof{\selindex} & \ifspace \restrictionofto{\meanset}{v\times\rr^{1} \times 0_{\seldim-\selindex-1}} = \{0,1\} \\
            \end{cases}
        \right).
    \end{align*}
    Here we denote by $\restrictionofto{\meanset}{V}$ the projections of the vertices in $\meanset$ onto the subspaces $V$, and by $0_{\seldim}$ the zero vector in $\rr^{\seldim}$.
\end{theorem}
\begin{proof}
    We show per induction, that for any $\selindexin$ the family of \HybridLogicNetworks{} with the statistic $\formulaof{[\selindex+1]}$ by the first $\selindex+1$ formulas has the mean polytope
    \begin{align}
        \label{eq:HLNindConTBS}
        \meansetof{\formulaof{[\selindex+1]},\trivbm} = \restrictionofto{\meanset}{\rr^\selindex\times 0_{\seldim-\selindex}} \, .
    \end{align}

    $\selindex=0$: The polytope $\restrictionofto{\meanset}{\rr^\selindex\times 0_{\seldim-\selindex}}=\{1\}$ (respectively $\restrictionofto{\meanset}{\rr^\selindex\times 0_{\seldim-\selindex}}=\{0\}$) is reproduced by $\formulaof{0}$ being a tautology (respectively a contradiction).
    In the case $\restrictionofto{\meanset}{\rr^\selindex\times 0_{\seldim-\selindex}}=[0,1]$ the polytope is reproduced by the any formula, which is neither a tautology nor a contradiction, and the atomic formula $\catvariableof{0}$ is an example of such an contingency.
    Since the projection of a 0-1 polytope onto the first coordinates is itself a 0-1 polytope, these are the only possible cases and we conclude that in all
    \begin{align*}
        \meansetof{\formulaof{[\selindex+1]},\trivbm} = \restrictionofto{\meanset}{\rr^\selindex\times 0_{\seldim-\selindex}} \, .
    \end{align*}

    $\selindex\rightarrow\selindex+1$: Let us assume, that \eqref{eq:HLNindConTBS} holds for a $\selindexin$.
    Then for any $\shortcatindicesin$ there is exactly one $v[\selvariable] \in \restrictionofto{\meanset}{\rr^\selindex\times 0_{\seldim-\selindex}}$ such that $\shortcatindices$ is a model of
    \begin{align*}
        \formulaof{\selindex,v}\coloneqq
        \bigwedge_{\secselindex\in[\selindex+1]} \lnot^{1-v[\indexedselvariable]} \formulaofat{\selindex}{\catvariableof{[\selindex]}} \, .
    \end{align*}
    This holds, since by the mutual contradiction of these formulas at most one can have $\shortcatindices$ as a model.
    Further, if none would have $\shortcatindices$ as a model, then the to $\shortcatindices$ corresponding vector of satisfactions $(\formulaofat{\selindex}{\indexedshortcatvariables})_{\selindex\in[\selindex]}$ is not in $\restrictionofto{\meanset}{\rr^\selindex\times 0_{\seldim-\selindex}}$, which can not be the case.
    Now, the implications to all $\tilde{v}$ except for $v$ are for the models $\shortcatindices$ of $\formulaof{\selindex,v}$ satisfied, and the satisfaction of $\formulaof{\selindex}$ thus only depends on the head of the implication to $v$.
    It follows, that the vertices of $\meansetof{\formulaof{[\selindex+2]},\trivbm}$ sharing the first $\selindex$ coordinates with $v$ are determined by the head of the implication at $v$.
    With the same arguments as in the case $\selindex=0$ we now notice, that in the three cases we construct vertex sets $v\times\{1\},\, v\times\{1\}$ or $v\times\{0,1\}$ if and only if they appear in the polytope $\restrictionofto{\meanset}{\rr^{\selindex+1}\times 0_{\seldim-\selindex-1}}$.
    This establishes for each vertex $v$ of $\meansetof{\formulaof{[\selindex+1]},\trivbm}$ that
    \begin{align*}
        \restrictionofto{\left(\meansetof{\formulaof{[\selindex+2]},\trivbm}\right)}{v\times\rr\times0_{\seldim-\selindex-1}}
        = \restrictionofto{\left(\restrictionofto{\meanset}{\rr^\selindex\times 0_{\seldim-\selindex}}\right)}{v\times\rr\times0_{\seldim-\selindex-1}} \, .
    \end{align*}
    Since for any vertex in $\meanset$ we find a unique vertex in $\meansetof{\formulaof{[\selindex+1]},\trivbm}$ sharing the first $\selindex$ coordinates, we have that \eqref{eq:HLNindConTBS} holds for $\selindex+1$.

    By induction the equation \eqref{eq:HLNindConTBS} holds for arbitrary $\selindexin$.
    For $\selindex=\seldim-1$ the equation is the claim.
\end{proof}

\subsection{Further Examples of non-cubelike polytopes}

\begin{example}[Ising model on $2$ nodes]
    % See Example~3.6 in \cite{wainwright_graphical_2008}
    Consider two boolean variables $\catvariableof{0},\catvariableof{1}$ and the Ising statistic by three propositional formulas
    \begin{align*}
        \formulaofat{0}{\catvariableof{[2]}} = \catvariableof{0} \quad , \quad
        \formulaofat{1}{\catvariableof{[2]}} = \catvariableof{1} \andspace
        \formulaofat{2}{\catvariableof{[2]}} = \catvariableof{0}\land\catvariableof{1} \, .
    \end{align*}
    In the Ising interpretation, the boolean variables represent interacting spins at two locations.
    Their value is then measured by the first two formulas and their interaction by the third.

    The vertices of the mean polytope to this statistic are
    \begin{align*}
        \hlnstatat{\catindexof{0},\catindexof{1}} = [\formulaofat{0}{\indexedcatvariableof{[2]}},\formulaofat{1}{\indexedcatvariableof{[2]}},\formulaofat{2}{\indexedcatvariableof{[2]}}]^T =
        \begin{cases}
        [0,0,0]
            ^T & \ifspace (\catindexof{0},\catindexof{1}) = (0,0) \\
            [0,1,0]^T & \ifspace (\catindexof{0},\catindexof{1}) = (0,1) \\
            [1,0,0]^T & \ifspace (\catindexof{0},\catindexof{1}) = (1,0) \\
            [1,1,1]^T & \ifspace (\catindexof{0},\catindexof{1}) = (1,1)
        \end{cases}
    \end{align*}
    and the mean polytope is the convex hull of these, sketched as:
    \begin{center}
        \tdplotsetmaincoords{55}{20} % Set the viewpoint: (theta, phi)

        \begin{tikzpicture}[tdplot_main_coords, scale=2,
            mainline/.style={thick},
            invisibleline/.style={dashed, gray}
        ]

            % Define the 4 vertices
            \coordinate (A) at (0, 0, 0);
            \coordinate (B) at (0, 1, 0);
            \coordinate (C) at (1, 0, 0);
            \coordinate (D) at (1, 1, 1);

            % Draw the faces with some opacity (optional, requires more complex sorting for correct rendering)
            \fill[cyan!20, opacity=0.6] (A) -- (B) -- (C) -- cycle;
            % \fill[cyan!20, opacity=0.6] (A) -- (B) -- (D) -- cycle;
            % \fill[cyan!20, opacity=0.6] (A) -- (C) -- (D) -- cycle;
            % \fill[cyan!20, opacity=0.6] (B) -- (C) -- (D) -- cycle;

            % Draw visible edges (adjust based on your chosen viewpoint)
            \draw[mainline] (B) -- (C) -- (D) -- cycle; % Front face BCD
            \draw[mainline] (A) -- (B);
            \draw[mainline] (A) -- (C);
            \draw[mainline] (A) -- (D);

            % Draw the hidden edge (adjust based on your chosen viewpoint)
            % In this view, edge BC is visible, but the edge connecting the back vertex to D is also potentially visible.
            % The edge BC is visible, the edge connecting A to C and B is also visible.
            % A is behind for this viewpoint, so the edges connected to A are back edges
            \draw[invisibleline] (A) -- (B);
            \draw[invisibleline] (A) -- (C);
            \draw[invisibleline] (A) -- (D);


            % Draw vertices as nodes and label them
            \foreach \point/\label/\pos in {A/000/below left, B/010/below right, C/100/above left, D/111/above} {
                \draw[fill=black] (\point) circle (1pt) node[\pos] {$\label$};
            }

            % Add coordinate system axes for context
            \draw[->, gray] (0,0,0) -- (1.2,0,0) node[right] {$\meanparamat{\selvariable=0}$};
            \draw[->, gray] (0,0,0) -- (0,1.2,0) node[above] {$\meanparamat{\selvariable=1}$};
            \draw[->, gray] (0,0,0) -- (0,0,1.2) node[above] {$\meanparamat{\selvariable=2}$};

        \end{tikzpicture}
    \end{center}

    We notice, that the convex hull of any three out of the four vertices builds a facet.
    The only cube-like facet of these four is the convex hull of $\{[0,0,0]^T,[1,0,0]^T,[0,1,0]\}$ (sketched in cyan in the above plot) to which we have the parametrization by $(\hardlegset,\headindexof{\hardlegset})=(\{2\},0)$.

    As a consequence, there are maximum entropy distributions of mean parameters in the Ising statistics, which do not have a representation by an elementary \CompActNet{} in the Ising statistic.
\end{example}

\begin{example}[Crossword Polytopes]
    % See Example~3.9 in \cite{wainwright_graphical_2008}
    Consider the crossword polytope with the vertices
    \begin{align*}
        \{\headindexof{[3]} \wcols \sum_{\catenumerator\in[3]} \headindexof{\catenumerator}=0 \}
    \end{align*}
    We interpret the vertices by the strings of length $3$, which parity vanishes.
    Such constrained are common in the construction of codewords using parity-check codes (see \cite{gallager_low-density_1963}).
    \begin{center}
        \tdplotsetmaincoords{55}{30} % Set the viewpoint: (theta, phi)

        \begin{tikzpicture}[tdplot_main_coords, scale=2,
            mainline/.style={thick},
            invisibleline/.style={dashed, gray}
        ]

            % Define the 4 vertices
            \coordinate (A) at (0, 0, 0);
            \coordinate (B) at (0, 1, 1);
            \coordinate (C) at (1, 0, 1);
            \coordinate (D) at (1, 1, 0);

            % Draw the faces with some opacity (optional, requires more complex sorting for correct rendering)
            %\fill[cyan!20, opacity=0.6] (A) -- (B) -- (C) -- cycle;
            % \fill[cyan!20, opacity=0.6] (A) -- (B) -- (D) -- cycle;
            % \fill[cyan!20, opacity=0.6] (A) -- (C) -- (D) -- cycle;
            % \fill[cyan!20, opacity=0.6] (B) -- (C) -- (D) -- cycle;

            % Draw visible edges (adjust based on your chosen viewpoint)
            \draw[mainline] (B) -- (C) -- (D) -- cycle; % Front face BCD
            \draw[mainline] (A) -- (B);
            \draw[mainline] (A) -- (C);
            \draw[mainline] (A) -- (D);

            % Draw the hidden edge (adjust based on your chosen viewpoint)
            % In this view, edge BC is visible, but the edge connecting the back vertex to D is also potentially visible.
            % The edge BC is visible, the edge connecting A to C and B is also visible.
            % A is behind for this viewpoint, so the edges connected to A are back edges
            \draw[invisibleline] (A) -- (B);
            \draw[invisibleline] (A) -- (C);
            \draw[invisibleline] (A) -- (D);


            % Draw vertices as nodes and label them
            \foreach \point/\label/\pos in {A/000/below left, B/011/below right, C/101/above left, D/110/above} {
                \draw[fill=black] (\point) circle (1pt) node[\pos] {$\label$};
            }

            % Add coordinate system axes for context
            \draw[->, gray] (0,0,0) -- (1.2,0,0) node[right] {$\meanparamat{\selvariable=0}$};
            \draw[->, gray] (0,0,0) -- (0,1.2,0) node[above] {$\meanparamat{\selvariable=1}$};
            \draw[->, gray] (0,0,0) -- (0,0,1.2) node[above] {$\meanparamat{\selvariable=2}$};

        \end{tikzpicture}
    \end{center}

    All facets are non-cube-like, whereas all other faces are cube-like (the edges are labeled by the $6$ facets of $[0,1]^3$).

    %% Hamming distance characterization and TT characterization
    Each facet is characterized by the Hamming distance of exactly $1$ to a single of the $4$ cube vertices, which are not codewords (see Example~3.9 in \cite{wainwright_graphical_2008}).
    We can construct $\ttformat$ activation tensors with rank $2$, which are calculating in the hidden ranks the Hamming distance to these vertex.
    Each core tensor in that format adds the contribution of a mode to the Hamming distance.

    %% LDPC in more generality: Cycle Polytopes
    In more generality, Low-Density-Parity-Checking Codes (see \cite{gallager_low-density_1963}) construct codewords by a collection of parity constraints on subsets of variables.
    Such polytopes are also known as cycle polytopes, and studied e.g. in \cite{grotschel_geometric_1993}.

\end{example}