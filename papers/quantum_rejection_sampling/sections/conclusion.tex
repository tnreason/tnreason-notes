\section{Conclusion and Outlook}

We have characterized an intuitive class of Quantum Circuits, namely graph-controlled circuits.
we have shown that their prepared distributions are exactly the Bayesian Networks on the corresponding hypergraph.
Along this class the demonstrated rejection sampling schemes by \ComputationActivationCircuits{} achieve the best acceptance probability.
Using amplitude amplification, we get a square root improvement of the quantum rejection sampling scheme compared with classical rejection sampling.

To improve on this quantum speedup, different circuit designs are necessary.
We could allow for hypergraphs with cycles, where e.g. the distributed qubits are rotated conditioned on the statistic qubits.
Such schemes shall be investigated in future research.
However, an improvement beyond scaling constants is not expected, since the amplitude amplification procedure already achieves the Heisenberg limit \cite{brassard_quantum_2002,zwierz_general_2010}.

We have seen that preparing a classically hard to compute contraction can be done efficiently by a quantum circuit.
We showed in particular that contracted basis encodings can be prepared by circuits, which are classically hard to contract in the presence of undirected loops in the decomposition hypergraph.
However, when assessing the prepared contraction state by computational basis measurements, we effectively draw samples from the corresponding distribution, which can also be achieved classically.
A quantum advantage could arise from different measurement schemes studied in the quantum state tomography community (see \cite{gross_quantum_2010,cramer_efficient_2010,roth_semi-device-dependent_2023}).
To this end, future research could show, how hard to achieve expectation queries can efficiently be prepared by customized measurement schemes.

%% Quantum Moment Matching
One might further investigate the relation between amplitude amplification and change of the canonical parameters in \HybridLogicNetwork{}.
This will enable a more integrated alternative to the Parameter Estimation routines suggested here.
Furthermore it will enable the preparation of \ComputationActivationNetworks{} without the need of ancilla augmentation.
While we expect limits due to the discrete degrees of freedom of amplitude amplification, we could refine prepared circuits based on ancilla augmentation.
To this end, we would understand such preparation as a coarse representation of \ComputationActivationNetworks{}, which serve as initial states for ancilla augmentation.
The advantage then lies in the improvement of the $\acceptanceprob$, when the $\infty$ Renyi divergence with respect to the coarse distribution is less than the $\infty$ Renyi divergence with respect to the uniform distribution.
%Such construction schemes are also of interest for the preparation of more efficient ancilla augmentation schemes.