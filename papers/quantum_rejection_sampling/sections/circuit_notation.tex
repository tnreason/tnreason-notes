\section{Notation}

We present the basic notation of quantum circuits using the tensor network notation used in the \tnreason{} formalism.

\subsection{State Encoding Schemes}

Basis Encoding in Quantum Computation refers to the representation of classical $n$ bit strings by $n$ qubit basis states, and is called one-hot encoding in \tnreason{}.
The Basis Encoding scheme in \tnreason{} goes beyond this scheme and also encodes subsets by sums of one-hot encodings to the members of the set.
In this way, relations and functions are represented by boolean tensors and contraction of them is refered as \BasisCalculus{}.

Amplitude Encoding in Quantum Computation refers to the storage of complex numbers in the amplitudes of quantum states.
The pendant in \tnreason{} is the Coordinate Encoding scheme, where real numbers are stored in the coordinates of real-valued tensors.
Compared to Amplitude Encoding, Coordinate Encoding does not have the normalization constraint of quantum states.
The Amplitude Encoding of the square root of a probability distribution is sometimes called q-sample.

\subsection{Controlled Single Qubit Gates}

We define the rotation gate around the Y-axis by an angle $\alpha$ as
\begin{align*}
    \yrotationofat{\alpha}{\avariableof{\insymbol},\avariableof{\outsymbol}} \coloneqq
    \begin{bmatrix}
        \cosof{\frac{\alpha}{2}} & -\sinof{\frac{\alpha}{2}} \\
        \sinof{\frac{\alpha}{2}} & \cosof{\frac{\alpha}{2}}
    \end{bmatrix}
\end{align*}

Further we define the Pauli-X:
\begin{align*}
    \paulixat{\avariableof{\insymbol},\avariableof{\outsymbol}} \coloneqq
    \begin{bmatrix}
        0 & 1 \\
        1 & 0
    \end{bmatrix}
\end{align*}

%% Control notation
Controlled single qubit gates are defined using control qubits, where the gate is applied to the target qubit if the control qubits are in a specific state and the identity is applied otherwise.
In the tensor network diagrams, we do not distinguish between incoming and outgoing control qubit variables, since the control acts as a Dirac tensor.
Thus, controlled unitary with target qubit $\catvariableof{t}$ and control qubits $\catvariableof{c}$ are represented by tensors
\begin{align*}
    \contunitaryat{\catvariableof{t,\insymbol},\catvariableof{t,\outsymbol},\catvariableof{c}}
\end{align*}
where for each state $\catindexof{c}$ to the control variables we have that
\begin{align*}
    \contunitaryat{\catvariableof{t,\insymbol},\catvariableof{t,\outsymbol},\indexedcatvariableof{c}}
\end{align*}
is a unitary matrix acting on the leg space of the target variable.

\subsection{Measurements and Contractions}

\subsubsection{Direct measurement}

The computational basis measurement of the qubits $\catvariableof{\variableset}$ of a Quantum State $\qstatewith$ is equal to drawing samples from a distribution
\begin{align*}
    \probat{\catvariableof{\variableset}} = \contractionof{
        \qstatewith, \comconqstatewith
    }{\catvariableof{\variableset}} \, .
\end{align*}
Here $\comconqstatewith$ is the complex conjugate of $\qstatewith$.
When $\qstate$ is prepared by a quantum circuit acting on a initial state, the complex conjugate is the hermitean conjugate of the circuit acting on the complex conjgate of the initial state.

%% Add drawing!
We abbreviate these contractions by extending the contraction diagrams with measurement symbols (see \figref{fig:measurementSketch}).
\begin{figure}
    \begin{center}
        \input{tikz_pics/directed_measurement_sketch.tikz}
    \end{center}
    \caption{Computational Basis Measurement of a quantum state $\qstate$.
    The measurement symbols on the left side indicate the measured qubits and the first equation is understood as a definition.
    In the second equation we sketch, that the measurement distribution is equal to the contraction of the square absolute transform of $\qstate$ to the measured variables.
    }\label{fig:measurementSketch}
\end{figure}

%% Phase-Absolut decomposition
Each complex-valued tensor $\qstatewith$ has a decomposition into a phase tensor $\phasecorewith$ and an absolute tensor $\absof{\qstate}[\shortcatvariables]$ defined by
\begin{align*}
    \qstatewith = \contractionof{\expof{i\cdot\phasecorewith}, \absof{\qstatewith}}{\shortcatvariables}\, .
\end{align*}

The measurement distribution is depends only on $\absof{\phi}$, that is
\begin{align*}
    \probwith = \absof{\qstatewith}^2 \, .
\end{align*}
Note, that this measurement distribution is not sensitive to the phases of the amplitudes.

When only a subset of variables is measured, the distribution is the contraction of the absolute square transform (these operations do not commute)
\begin{align*}
    \probat{\catvariableof{\secnodes}}
    = \contractionof{\absof{\qstatewith}^2}{\catvariableof{\secnodes}} \, .
\end{align*}

%% Phasecore vanishing as gauging
When we are interested in the preparation of quantum states with a specific computational basis measurement distribution, we can restrict to states with vanishing phase cores, that is
\begin{align*}
    \qstatewith
    = \contractionof{\expof{i\cdot\zerosat{\shortcatvariables}}, \absof{\qstate}[\shortcatvariables]}{\shortcatvariables}
    = \absof{\qstate}[\shortcatvariables]  \, .
\end{align*}

\subsubsection{Phase sensitive measurement}

When applying a Walsh-Hadamard transform on the qubits $\nodes\backslash\secnodes$ to be closed (see \figref{fig:phaseContractionMeasurement}), the probability of measuring the closed qubits in the ground state is %we perform the contraction
\begin{align*}
    \probat{\catvariableof{\nodes\backslash\secnodes}=0_{\nodes\backslash\secnodes},\catvariableof{\secnodes}}
    = \frac{1}{2^{\cardof{\secnodes}}} \cdot \absof{\contractionof{\qstatewith}{\catvariableof{\secnodes}}}^2 \, .
\end{align*}
We thus sample from the contracted quantum state.
Note that in this way, the relative phases of the amplitudes are contracted before the absolutes are taken.
In this way, the measurement distribution is sensitive to the relative phases.
This property is exploited in the Deutsch-Jozsa algorithm, which utilizes signs of the amplitudes.

\begin{figure}
    \begin{center}
        \begin{tikzpicture}[scale=0.35,thick]


    \draw (0,0) rectangle (2,10);
    \node[anchor=center] (text) at (1,5) {$\qstate$};

    \draw (2,1) -- (4,1);

    \draw (2,9) -- (4,9) node[midway,above] {\colorlabelsize $\tilde{\catvariable}_{0}$};
    \drawqcmeasuresymbol{5}{9}
    \draw (6,9) -- (8,9) node[midway,above] {\colorlabelsize $\catvariableof{0}$};
    \node[anchor=center] (text) at (3,8.25) {$\vdots$};
    \draw (2,6.5) -- (4,6.5) node[midway,above] {\colorlabelsize $\tilde{\catvariable}_{\atomenumerator}$};
    \drawqcmeasuresymbol{5}{6.5}
    \draw (6,6.5) -- (8,6.5) node[midway,above] {\colorlabelsize $\catvariableof{\atomenumerator}$};


    \draw (2,4) -- (4,4) node[midway,above] {\colorlabelsize $\catvariableof{\atomenumerator+1}$};
    \draw (4,3) rectangle (6,5);
    \node[anchor=center] at (5,4) {$\hgate$};
    \draw (6,4) -- (7,4);
    \drawqcmeasuresymbol{8}{4}
    \draw (9,4) -- (10,4);
    \draw (10,3) rectangle (12,5);
    \node[anchor=center] at (11,4) {$\fbasis$};

    \node[anchor=center] (text) at (3,3.25) {$\vdots$};
    \draw (2,1) -- (4,1) node[midway,above] {\colorlabelsize $\catvariableof{\atomorder-1}$};
    \draw (4,0) rectangle (6,2);
    \node[anchor=center] at (5,1) {$\hgate$};
    \draw (6,1) -- (7,1);
    \drawqcmeasuresymbol{8}{1}
    \draw (9,1) -- (10,1);
    \draw (10,0) rectangle (12,2);
    \node[anchor=center] at (11,1) {$\fbasis$};

    \node[anchor=center] (text) at (14.5,5) {${=}\,\, \frac{1}{2^{\catorder-\catenumerator}}$};


    \begin{scope}
    [shift={(17,0)}]
        \draw (0,0) rectangle (2,10);
        \node[anchor=center] (text) at (1,5) {$\qstate$};

        \draw (2,1) -- (4,1);

        \draw (2,9) -- (4,9) node[midway,above] {\colorlabelsize $\catvariableof{0}$};
        \node[anchor=center] (text) at (3,8.25) {$\vdots$};
        \draw (2,6) -- (4,6) node[midway,above] {\colorlabelsize $\catvariableof{\atomenumerator-1}$};

        \drawvariabledot{4}{9}
        \draw (4,9) -- (4,11);

        \drawvariabledot{6}{6}
        \draw (6,6) -- (6,11);

        \node[anchor=center] (text) at (5,10.25) {$\cdots$};

        \draw (4,9) -- (6,9);
        \draw (4,6) -- (6,6);
        %\draw (4,4) -- (6,4);
        \drawvariabledot{4}{4}
        \drawvariabledot{4}{1}

        \draw (2,4) -- (4,4) node[midway,above] {\colorlabelsize $\catvariableof{\atomenumerator}$};
        \node[anchor=center] (text) at (3,3.25) {$\vdots$};
        \draw (2,1) -- (4,1) node[midway,above] {\colorlabelsize $\catvariableof{\atomorder-1}$};


        \begin{scope}
        [xscale = -1, shift={(-10,0)}]
            \draw (0,0) rectangle (2,10);
            \node[anchor=center] (text) at (1,5) {$\comconqstate$};

            \draw (2,1) -- (4,1);

            \draw (2,9) -- (4,9);% node[midway,above] {\colorlabelsize $\catvariableof{0}$};

            %\node[anchor=center] (text) at (3,8.25) {$\vdots$};
            \draw (2,6) -- (4,6);% node[midway,above] {\colorlabelsize $\catvariableof{\atomenumerator}$};

            \drawvariabledot{4}{4}
            \draw (2,4) -- (4,4) node[midway,above] {\colorlabelsize $\tilde{\catvariable}_{\atomenumerator}$};
            %\node[anchor=center] (text) at (3,3.25) {$\vdots$};
            \drawvariabledot{4}{1}
            \draw (2,1) -- (4,1) node[midway,above] {\colorlabelsize $\tilde{\catvariable}_{\atomorder-1}$};
        \end{scope}

    \end{scope}

    \node[anchor=center] (text) at (30,5) {${=\,\, \frac{1}{2^{\catorder-\catenumerator}}}$};

    \begin{scope}
    [shift={(34,0)}]
        \draw (2,5) ellipse (4cm and 6cm);
        \node[anchor=center] at (5,10.75) {$\absof{\cdot}^2$};

        \draw (0,0) rectangle (2,10);
        \node[anchor=center] (text) at (1,5) {$\qstate$};

        \draw (2,1) -- (4,1);

        \draw (2,9) -- (4,9) node[midway,above] {\colorlabelsize $\catvariableof{0}$};
        \draw (4,9) -- (7,9);
        \node[anchor=center] (text) at (3,8.25) {$\vdots$};
        \draw (2,6) -- (4,6) node[midway,above] {\colorlabelsize $\catvariableof{\atomenumerator-1}$};
        \draw (4,6) -- (7,6);

        \draw (2,4) -- (4,4) node[midway,above] {\colorlabelsize $\catvariableof{\atomenumerator}$};
        \drawvariabledot{4}{4}
%        \draw (4,3) rectangle (6,5);
%        \node[anchor=center] (text) at (5,4) {$\ones$};
        \node[anchor=center] (text) at (3,3.25) {$\vdots$};
        \draw (2,1) -- (4,1) node[midway,above] {\colorlabelsize $\catvariableof{\atomorder-1}$};
        \drawvariabledot{4}{1}
%        \draw (4,0) rectangle (6,2);
%        \node[anchor=center] (text) at (5,1) {$\ones$};
    \end{scope}

\end{tikzpicture}
    \end{center}
    \caption{Computational Basis Measurement of a quantum state $\qstate$, after the Walsh-Hadamard Transform on the closed qubits.
    By the ellipse we indicate a coordinatewise transform to the square of the absolute, after the contraction.}
    \label{fig:phaseContractionMeasurement}
\end{figure}
