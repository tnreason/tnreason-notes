\begin{example}[Encoding of discretized distributions]
    We here review the preparation of a discretized distribution from \cite{grover_creating_2002} as an application of \activationCircuits{}.
    Let $\mathbb{Q}[\catvariable]$ be a continuous probability distribution of a variable $\catvariable$ taking values in $[0,1]$.
    For any $\catorder\in\nn$ we define its discretization of order $\catorder$ as a discrete probability distribution $\probwith$ with coordinates
    \begin{align*}
        \probofat{\catorder}{\indexedshortcatvariables}
        = \mathbb{Q}\left[\catvariable\in[
            \sum_{\catenumeratorin}2^{-(\catenumerator+1)}\catindexof{\catenumerator},
            2^{-\catorder}+\sum_{\catenumeratorin}2^{-(\catenumerator+1)}\catindexof{\catenumerator}]
            \right] \, .
    \end{align*}
    Then we have for any $\seccatorder\in\nn$ with $\seccatorder\leq\catorder$
    \begin{align*}
        \probofat{\seccatorder}{\catvariableof{[\seccatorder]}}
        = \contractionof{\probwith}{\catvariableof{[\seccatorder]}} \, .
    \end{align*}
    The discretization order is thus increased by the contraction
    \begin{align*}
        \probofat{\catorder+1}{\catvariableof{[\catorder+1]}}
        = \contractionof{
            \probofat{\catorder}{\shortcatvariables},
            \probofat{\catorder+1}{\catvariableof{\catorder}|\shortcatvariables}
        }{\catvariableof{[\catorder+1]}} \, .
    \end{align*}
    Having an \activationCircuit{} preparing $\probof{\catorder}$ we can therefore prepare $\probof{\catorder+1}$ by adding the activation circuit for the function
    \begin{align*}
        \shortcatindices \rightarrow \probofat{\catorder+1}{\catvariableof{\catorder}=1|\indexedshortcatvariables} \, .
    \end{align*}
    Note that the function evaluation amounts to the integration of $\mathbb{Q}$ as
    \begin{align*}
        \probofat{\catorder+1}{\catvariableof{\catorder}=1|\indexedshortcatvariables}
        = \frac{
            \mathbb{Q}\left[\catvariable\in[
                2^{-(\catorder+1)}+\sum_{\catenumeratorin}2^{-(\catenumerator+1)}\catindexof{\catenumerator},
                2^{-\catorder}+\sum_{\catenumeratorin} 2^{-(\catenumerator+1)}\catindexof{\catenumerator}]
                \right]
        }{
            \mathbb{Q}\left[\catvariable\in[
                \sum_{\catenumeratorin}2^{-(\catenumerator+1)}\catindexof{\catenumerator},
                2^{-\catorder}+\sum_{\catenumeratorin}2^{-(\catenumerator+1)}\catindexof{\catenumerator}]
                \right]
        } \, .
    \end{align*}
\end{example}