%\section{Decompositions based on Propositional Syntax}
\section{The logical paradigm}\label{sec:logPar}

A tensor-based representation of propositional logic is developed by defining formulas as boolean valued tensors, and showing how logical connectives and normal forms can be expressed as tensor contractions.

\subsection{Propositional semantics by boolean tensors}

Starting with the introduction of propositional formulas as boolean tensors their decomposition is discussed with respect to a basis encoding.
%The basis encoding representation enables dealing with the negation operation of propositional formulas still in tensor format.
%Furthermore, the propositional formula then fits into the concept of \CompActNets{}.

\begin{definition}
    \label{def:formulas}
    A \emph{propositional formula} $\formulaat{\shortcatvariables}$ depending on $\atomorder$ boolean variables $\catvariableof{\atomenumerator}$ is a tensor
    \begin{align*}
        \formulaat{\shortcatvariables} \in \bigotimes_{\atomenumeratorin} \rr^2
    \end{align*}
    with coordinates in $\ozset$.
    We call a state $\shortcatindices \in \atomstates$ a \emph{model} of a propositional formula $\formula$, if
    \begin{align*}
        \formulaat{\indexedshortcatvariables}=1 \, ,
    \end{align*}
    where we understand $1$ as a representation of $\mathrm{True}$ and $0$ of $\mathrm{False}$.
    If there is a model of a propositional formula, we say the formula is \emph{satisfiable}.
\end{definition}

\begin{example}
    \label{exa:propFormulaCoordinatewise}
    Let there be $\catorder=3$ boolean variables $\catvariableof{[3]}$ and a propositional formula
    \begin{align*}
        \formulaat{\catvariableof{[3]}} = (\catvariableof{0} \lor \catvariableof{1}) \land \lnot \catvariableof{2} \, .
    \end{align*}
    In a graphical depiction and in the coordinatewise representation this formula can be represented as
    \begin{center}
        \begin{tikzpicture}[scale=1]

            \begin{scope}[shift={(-4,-0.2)}]
                \node[anchor=east] (A) at (-0.25,0.2) {$\formulaat{\catvariableof{[3]}}\,=$};
                \draw (0,0) rectangle (1.6,0.8);
                \node[anchor=center] (A) at (0.8,0.4) {$\formula$};
                \draw (0.2,0) -- (0.2,-0.6) node[midway,left] {\tiny $\catvariableof{0}$};
                \draw (0.8,0) -- (0.8,-0.6) node[midway,left] {\tiny $\catvariableof{1}$};
                \draw (1.4,0) -- (1.4,-0.6) node[midway,left] {\tiny $\catvariableof{2}$};
            \end{scope}

            \node[anchor=east] (A) at (-1.5,0) {$=$};
            \node (A) at (0,0) {
                $\begin{bmatrix}
                     0 & 1 \\
                     1 & 1
                \end{bmatrix}$
            };
            \node (A) at (1.25,0.3) {
                $\begin{bmatrix}
                     0 & 0 \\
                     0 & 0
                \end{bmatrix}$
            };
            \draw[<-,dashed] (-0.9,-0.275) node[right] {\tiny $1$} -- (-0.9,0.275) node [midway, left] {\tiny $\catvariableof{0}$} node[right] {\tiny $0$};
            \draw[->,dashed] (-0.3,0.85) node[below] {\tiny $0$} -- (0.3,0.85) node [midway, above] {\tiny $\catvariableof{1}$} node[below] {\tiny $1$};
            \draw[->,dashed] (0,-0.85) node[above] {\tiny $0$} -- (1.25,-0.55) node [midway, below] {\tiny $\catvariableof{2}$} node[above] {\tiny $1$};

            \node[anchor=east] (A) at (2.25,-0.8) {$\cdot$};
        \end{tikzpicture}
    \end{center}
    In the state set $\atomstates = \{0,1\}\times \{0,1\} \times \{0,1\}$ we have three models of the formula by the positions of the non-zero entries in the tensor, i.e. $\formulaat{\indexedcatvariableof{[3]}}=1$ if and only if
    \begin{align*}
        \catindexof{[3]}\in\{(1,0,0),(0,1,0),(1,1,0)\} \, .
    \end{align*}
    The formula $\formula$ is therefore satisfiable.
\end{example}

\paragraph{Model counts by contraction}
Each coordinate of a propositional formula is either $1$ or $0$, indicating whether the indexed state is a model of the formula or not.
In this way, the contraction $\contraction{\formula}$ counts the number of models of the propositional formula $\formula$.
One can therefore decide the satisfiability of a formula by testing for $\contraction{\formula}>0$.

\paragraph{CP decomposition}
We can decompose a formula into the sum of the one-hot encodings of its models:
%Since the tensor $\formulaat{\shortcatvariables}$ is equal to one at index $x_{[d]}$ if and only if $x_{[d]}$ is a model of $\formula$, a propositional formula can be written as the sum over the one-hot encodings of its models.
\begin{center}
    \begin{tikzpicture}[scale=0.35, thick]

    \draw (-1,-1) rectangle (5,-3);
    \node[anchor=center] (text) at (2,-2) {\corelabelsize ${\exformula}$};
    \draw[] (0,-3)--(0,-5) node[midway,left] {\colorlabelsize $\catvariableof{0}$};
    \draw[] (1.5,-3)--(1.5,-5) node[midway,left] {\colorlabelsize $\catvariableof{1}$};
    \node[anchor=center] (text) at (3,-4) {$\cdots$};
    \draw[] (4,-3)--(4,-5) node[midway,right] {\colorlabelsize $\catvariableof{\atomorder\shortminus1}$};


    \node[anchor=center] (text) at (7,-2) {${=}$};

    \node[anchor=center] (text) at (12,-2.5) {${\sum\limits_{\atomindices\in\atomstates}}$};
    \node[anchor=center] (text) at (12,-4) {\colorlabelsize $\exformula(\atomindices)=1$};

    \begin{scope}
        [shift={(19.5,1)}]

        \draw (-3,-2) rectangle (-1,-4);
        \node[anchor=center] (text) at (-2,-3) {\corelabelsize $\onehotmapof{\atomlegindexof{0}}$};
        \draw[->-] (-2,-4)--(-2,-6) node[midway,right] {\colorlabelsize $\catvariableof{0}$};

        \node[anchor=center] (text) at (1,-3) {\corelabelsize $\cdots$};

        \draw (3,-2) rectangle (5,-4);
        \node[anchor=center] (text) at (4,-3) {\corelabelsize $\onehotmapof{\atomlegindexof{\atomorder\shortminus1}}$};
        \draw[->-] (4,-4)--(4,-6) node[midway,right] {\colorlabelsize $\catvariableof{\atomorder\shortminus1}$};

    \end{scope}

\end{tikzpicture}
\end{center}
As already depicted, one can exploit this summation to find a $\cpformat$ decomposition of the formula.
To this end, we enumerate the models $\shortcatindices^{\decindex}$ of $\formula$ by a decomposition variable $\decvariable$ with values $\decindex\in[\contraction{\formula}]$ and define, for $\catenumeratorin$, cores with slices
\begin{align*}
    \hypercoreofat{\catenumerator}{\catvariableof{\catenumerator},\indexeddecvariable}
    = \onehotmapofat{\catindexof{\catenumerator}^{\decindex}}{\catvariableof{\catenumerator}} \, .
\end{align*}

\begin{example}\label{exa:propFormulaBasCP}
    For the formula described in \exaref{exa:propFormulaCoordinatewise}, we have
    \begin{align*}
        \formulaat{\catvariableof{[3]}}
        &= \left(\tbasisat{\catvariableof{0}} \otimes \fbasisat{\catvariableof{1}} \otimes \fbasisat{\catvariableof{2}}\right)
        + (\fbasisat{\catvariableof{0}} \otimes \tbasisat{\catvariableof{1}} \otimes \fbasisat{\catvariableof{2}}) \\
        &\quad+ (\tbasisat{\catvariableof{0}} \otimes \tbasisat{\catvariableof{1}} \otimes \fbasisat{\catvariableof{2}}) \, .
    \end{align*}
    Note that we have $\contraction{\formulawith}=3$ and we can interpret this sum as a $\cpformat$ decomposition of $\formula$ with rank $3$.
%    where we denote the vectors $\tbasisat{Y} = [0,1]^T$ and $\fbasisat{Y} = [1,0]^T$.
    We use the decomposition to evaluate the formula $\formula$ at $\catindexof{[3]} = (1,1,0)$ and get
    \begin{align*}
        \formulaat{\indexedcatvariableof{[3]}}
        &= \left(\tbasisat{\catvariableof{0}=1} \otimes \fbasisat{\catvariableof{1}=1} \otimes \fbasisat{\catvariableof{2}=0}\right) \\
       &\quad + (\fbasisat{\catvariableof{0}=1} \otimes \tbasisat{\catvariableof{1}=1} \otimes \fbasisat{\catvariableof{2}=0}) \\
       &\quad + (\tbasisat{\catvariableof{0}=1} \otimes \tbasisat{\catvariableof{1}=1} \otimes \fbasisat{\catvariableof{2}=0}) \\
        &=  1\cdot 0 \cdot 1 + 0\cdot 1 \cdot 1 + 1 \cdot 1 \cdot 1 = 1 \, ,
    \end{align*}
    which verifies that $\catindexof{[3]} = (1,1,0)$ is a model of the formula $\formula$.
\end{example}

\paragraph{Basis encoding}
Representing booleans by elements in $\{0,1\}$ leads to the problem that the negation is an affine transformation and cannot be represented by multilinear tensors. %~\cite[Section 4.1.1]{goessmann_tensor-network_2025}.
%Therefore, instead of using this \emph{coordinate calculus} scheme an approach based on \emph{basis calculus} is employed, which is explained in this section.
To be able to express different kinds of connectives by contractions, booleans are encoded by one-hot encodings as defined in \defref{def:onehotenc}.
Propositional formulas $\formula$ can then be expressed by their basis encodings
\begin{align*}
    \bencodingofat{\formula}{\indexedheadvariableof{\formula},\indexedshortcatvariables}
    = \begin{cases}
          1 & \ifspace \formulaat{\indexedshortcatvariables} = \headindexof{\formula}\\
          0 & \text{else}
    \end{cases} \, .
\end{align*}
This basis encoding $\bencodingofat{\formula}{\headvariableof{\formula},\shortcatvariables} \in \{0,1\}^{2\times 2^d}$ encodes the formula itself and its negation in its slices, since
\begin{align*}
    \bencodingofat{\formula}{\headvariableof{\formula},\shortcatvariables}
    = \tbasisat{\headvariableof{\formula}} \otimes \formulaat{\shortcatvariables}
    + \fbasisat{\headvariableof{\formula}} \otimes \lnot\formulaat{\shortcatvariables} \, .
\end{align*}
In our graphical notation this property is visualized by
\begin{center}
    \begin{tikzpicture}[scale=0.35, thick] % , baseline = -3.5pt

    \draw[->-] (2,-1)--(2,1) node[midway,right] {\colorlabelsize $\formulavar$};
    \draw (-1,-1) rectangle (5,-3);
    \node[anchor=center] (text) at (2,-2) {\corelabelsize $\bencodingof{\exformula}$};
    \draw[-<-] (0,-3)--(0,-5) node[midway,left] {\colorlabelsize $\catvariableof{0}$};
    \draw[-<-] (1.5,-3)--(1.5,-5) node[midway,left] {\colorlabelsize $\catvariableof{1}$};
    \node[anchor=center] (text) at (3,-4) {$\cdots$};
    \draw[-<-] (4,-3)--(4,-5) node[midway,right] {\colorlabelsize $\catvariableof{\atomorder\shortminus1}$};


    \node[anchor=center] (text) at (7,-2) {${=}$};

    \node[anchor=center] (text) at (10,-2.5) {${\sum\limits_{\shortcatindices\in\atomstates}}$};

    \begin{scope}
    [shift={(15.5,-0.5)}]

        \draw (-2,1) rectangle (4,-1);
        \node[anchor=center] (text) at (1,0) {\corelabelsize $\onehotmapof{\exformulaat{\indexedshortcatvariables}}$};
        \draw[->-] (1,1)--(1,2.5) node[midway,right] {\colorlabelsize $\formulavar$};

        \draw (-2,-2) rectangle (4,-4);
        \node[anchor=center] (text) at (1,-3) {\corelabelsize $\onehotmapof{\shortcatindices}$};

        \draw[->-] (-1.5,-4)--(-1.5,-5.5) node[midway,left] {\colorlabelsize $\catvariableof{0}$};
        \draw[->-] (0.5,-4)--(0.5,-5.5) node[midway,left] {\colorlabelsize $\catvariableof{1}$};
        \node[anchor=center] (text) at (2,-5) {$\cdots$};
        \draw[->-] (3.5,-4)--(3.5,-5.5) node[midway,right] {\colorlabelsize $\catvariableof{\atomorder\shortminus1}$};

    \end{scope}

    \node[anchor=center] (text) at (21.25,-2) {${=}$};

    \begin{scope} [shift={(25.5,-0.5)}]

        \draw (-2,1) rectangle (4,-1);
        \node[anchor=center] (text) at (1,0) {\corelabelsize $\onehotmapof{0}$};
        \draw[->-] (1,1)--(1,2.5) node[midway,right] {\colorlabelsize $\formulavar$};

        \draw (-2,-2) rectangle (4,-4);
        \node[anchor=center] (text) at (1,-3) {\corelabelsize $\lnot\formula$};%$\sum_{\shortcatindices \wcols \formula(\shortcatindices)=0}\onehotmapof{\shortcatindices}$};

        \draw[] (-1.5,-4)--(-1.5,-5.5) node[midway,left] {\colorlabelsize $\catvariableof{0}$};
        \draw[] (0.5,-4)--(0.5,-5.5) node[midway,left] {\colorlabelsize $\catvariableof{1}$};
        \node[anchor=center] (text) at (2,-5) {$\cdots$};
        \draw[] (3.5,-4)--(3.5,-5.5) node[midway,right] {\colorlabelsize $\catvariableof{\atomorder\shortminus1}$};

    \end{scope}


    \node[anchor=center] (text) at (31.25,-2) {${+}$};

    \begin{scope} [shift={(35.5,-0.5)}]

        \draw (-2,1) rectangle (4,-1);
        \node[anchor=center] (text) at (1,0) {\corelabelsize $\onehotmapof{1}$};
        \draw[->-] (1,1)--(1,2.5) node[midway,right] {\colorlabelsize $\formulavar$};

        \draw (-2,-2) rectangle (4,-4);
        \node[anchor=center] (text) at (1,-3) {\corelabelsize $\formula$};%$\sum_{\shortcatindices \wcols \formula(\shortcatindices)=1}\onehotmapof{\shortcatindices}$};

        \draw[] (-1.5,-4)--(-1.5,-5.5) node[midway,left] {\colorlabelsize $\catvariableof{0}$};
        \draw[] (0.5,-4)--(0.5,-5.5) node[midway,left] {\colorlabelsize $\catvariableof{1}$};
        \node[anchor=center] (text) at (2,-5) {$\cdots$};
        \draw[] (3.5,-4)--(3.5,-5.5) node[midway,right] {\colorlabelsize $\catvariableof{\atomorder\shortminus1}$};

    \end{scope}

\end{tikzpicture}
\end{center}
We further provide a more detailed example in coordinate sensitive notation in the following.
\begin{example}[Logical Negation and Conjunction]
    \label{exa:bencodingNegCon} %\cite[Example 4.9]{goessmann_tensor-network_2025}
    The basis encodings of the negation $\notucon: [2]\rightarrow [2]$ is the matrix
    \begin{center}
        \begin{tikzpicture}[scale=1]
            \node (A) at (-2.5,0) {$\bencodingofat{\lnot}{\headvariableof{\lnot},\catvariable}$=};
            \node (A) at (0,0) {
                $\begin{bmatrix}
                     0 & 1 \\
                     1 & 0
                \end{bmatrix}$
            };
            \draw[<-,dashed] (-0.9,-0.275) node[right] {\tiny $1$} -- (-0.9,0.275) node [midway, left] {\tiny $\catvariableof{0}$} node[right] {\tiny $0$};
            \draw[->,dashed] (-0.3,0.85) node[below] {\tiny $0$} -- (0.3,0.85) node [midway, above] {\tiny $\headvariableof{\lnot}$} node[below] {\tiny $1$};
        \end{tikzpicture}
    \end{center}
    The $2$-ary conjunctions $\land:  [2]\times[2] \rightarrow[2]$ is encoded by the order-$3$ tensor
    \begin{center}
        \begin{tikzpicture}[scale=1]
            \node (A) at (-4.5,0) {$\bencodingofat{\land}{\headvariableof{\land},\catvariableof{0},\catvariableof{1}}$=};

            \begin{scope}[shift={(0,0)}]
                \node(B) at (-2,0){
                    $\begin{bmatrix}
                         1 \\
                         0
                    \end{bmatrix}$
                };
                \draw[<-,dashed] (-2.5,-0.275) node[right] {\tiny $1$} -- (-2.5,0.275) node [midway, left] {\tiny $\headvariableof{\land}$} node[right] {\tiny $0$};
                \node (A) at (-1.6,0) {$\otimes$};
                \node (A) at (0,0) {
                    $\begin{bmatrix}
                         1 & 1 \\
                         1 & 0
                    \end{bmatrix}$
                };
                \draw[<-,dashed] (-0.9,-0.275) node[right] {\tiny $1$} -- (-0.9,0.275) node [midway, left] {\tiny $\catvariableof{0}$} node[right] {\tiny $0$};
                \draw[->,dashed] (-0.3,0.85) node[below] {\tiny $0$} -- (0.3,0.85) node [midway, above] {\tiny $\catvariableof{1}$} node[below] {\tiny $1$};
            \end{scope}

            \begin{scope}[shift={(4.25,0)}]

                \node[anchor=center] (A) at (-3.25,0) {$+$};

                \node(B) at (-2,0){
                    $\begin{bmatrix}
                         0 \\
                         1
                    \end{bmatrix}$
                };
                \draw[<-,dashed] (-2.5,-0.275) node[right] {\tiny $1$} -- (-2.5,0.275) node [midway, left] {\tiny $\headvariableof{\land}$} node[right] {\tiny $0$};
                \node (A) at (-1.6,0) {$\otimes$};
                \node (A) at (0,0) {
                    $\begin{bmatrix}
                         0 & 0 \\
                         0 & 1
                    \end{bmatrix}$
                };
                \draw[<-,dashed] (-0.9,-0.275) node[right] {\tiny $1$} -- (-0.9,0.275) node [midway, left] {\tiny $\catvariableof{0}$} node[right] {\tiny $0$};
                \draw[->,dashed] (-0.3,0.85) node[below] {\tiny $0$} -- (0.3,0.85) node [midway, above] {\tiny $\catvariableof{1}$} node[below] {\tiny $1$};
            \end{scope}

            \begin{scope}[shift={(7,0)}]
                \node[anchor=center] (A) at (-1.75,0) {$=$};

                \node (A) at (0,0) {
                    $\begin{bmatrix}
                         1 & 1 \\
                         1 & 0
                    \end{bmatrix}$
                };
                \node (A) at (1.25,0.3) {
                    $\begin{bmatrix}
                         0 & 0 \\
                         0 & 1
                    \end{bmatrix}$
                };
                \draw[<-,dashed] (-0.9,-0.275) node[right] {\tiny $1$} -- (-0.9,0.275) node [midway, left] {\tiny $\catvariableof{0}$} node[right] {\tiny $0$};
                \draw[->,dashed] (-0.3,0.85) node[below] {\tiny $0$} -- (0.3,0.85) node [midway, above] {\tiny $\catvariableof{1}$} node[below] {\tiny $1$};
                \draw[->,dashed] (0,-0.85) node[above] {\tiny $0$} -- (1.25,-0.55) node [midway, below] {\tiny $\headvariableof{\land}$} node[above] {\tiny $1$};
            \end{scope}
        \end{tikzpicture}
    \end{center}
    Further, the $2$-ary disjunction $\lor:  [2]\times[2] \rightarrow[2]$ is encoded by the order-$3$ tensor
    \begin{center}
        \begin{tikzpicture}[scale=1]
            \node (A) at (-3,0) {$\bencodingofat{\lor}{\headvariableof{\lor},\catvariableof{0},\catvariableof{1}}$=};
            \node (A) at (0,0) {
                $\begin{bmatrix}
                     1 & 0 \\
                     0 & 0
                \end{bmatrix}$
            };
            \node (A) at (1.25,0.3) {
                $\begin{bmatrix}
                     0 & 1 \\
                     1 & 1
                \end{bmatrix}$
            };
            \draw[<-,dashed] (-0.9,-0.275) node[right] {\tiny $1$} -- (-0.9,0.275) node [midway, left] {\tiny $\catvariableof{0}$} node[right] {\tiny $0$};
            \draw[->,dashed] (-0.3,0.85) node[below] {\tiny $0$} -- (0.3,0.85) node [midway, above] {\tiny $\catvariableof{1}$} node[below] {\tiny $1$};
            \draw[->,dashed] (0,-0.85) node[above] {\tiny $0$} -- (1.25,-0.55) node [midway, below] {\tiny $\headvariableof{\lor}$} node[above] {\tiny $1$};
        \end{tikzpicture}
    \end{center}
\end{example}

\paragraph{Interpretation as \CompActNets{}}
The propositional formula and its negation can be represented by this tensor via
\begin{align*}
    \formulaat{\shortcatvariables}
    = \contractionof{\tbasisat{\headvariableof{\formula}},\bencodingofat{\formula}{\headvariableof{\formula},\shortcatvariables}}{\shortcatvariables}
    \andspace
    \lnot\formulaat{\shortcatvariables}
    = \contractionof{\fbasisat{\headvariableof{\formula}},\bencodingofat{\formula}{\headvariableof{\formula},\shortcatvariables}}{\shortcatvariables} \, .
\end{align*}
Both $\formula$ and $\lnot\formula$ are thus \ComputationActivationNetworks{} to the statistic $\{\formula\}$ and the hard activation tensor $\tbasisat{\headvariableof{\formula}}$, respectively $\fbasisat{\headvariableof{\formula}}$.

\subsection{Syntactic decomposition of propositional formulas}

Propositional formulas of concern often have a syntactic specification as composed functions.
We can therefore apply the neural paradigm to find efficient representations of them.

\begin{definition}[Syntactic decompositions]
    \label{def:syntacticalDecomposition}
    A syntactic decomposition of a propositional formula $\exformula$ is a decomposition hypergraph (see \defref{def:decompositionHypergraph}) such that all nodes are decorated with the dimension $\catdimof{\node}=2$ and the composition function $\exformula$.
\end{definition}

We thus have a tensor network representation of any propositional formula based on its syntactic decomposition, where the hypergraph of the syntactic decomposition equals the hypergraph of the representing tensor network.

\input{../examples/logical_paradigm/prop_formula_syntax}

\subsection{Entailment decision by contractions}

We have already seen that the contraction of a propositional formula counts its models.
This allows to define entailment between two propositional formulas as defined in the following.
To generalize the treatment, we no longer demand that the variables of a formula are of dimension $2$.
We further use $\lnot\formulawith=\oneswith-\formulawith$.

\begin{definition}[Entailment of propositional formulas]
    \label{def:logicalEntailment}
    Given two propositional formulas $\kb$ and $\formula$, we say that $\kb$ entails $\formula$, denoted by $\kb\models\formula$, if any model of $\kb$ is also a model of $\formula$, that is
    \begin{align*}
        \contraction{\kbwith,\lnot\formulawith}=0 \, .
    \end{align*}
    If $\kb\models\lnot\formula$ holds (i.e. $\contraction{\kbwith,\formulawith}$=0), we say that $\kb$ contradicts $\formula$.
\end{definition}

% Relation to classical definition of entailment
Classically (see e.g. \cite{russell_artificial_2021}) entailment in propositional logics is defined as the unsatisfiability of $\kb\land\lnot\exformula$.
This is equivalent to \defref{def:logicalEntailment} due to the equivalence of $\contraction{\kbwith,\lnot\formulawith}=0$ and $\contraction{(\kb \land (\lnot\exformula))[\shortcatvariables]}=0$, which is the unsatisfiability of $\kb\land\lnot\exformula$.

%Entailment is the central operation of "logical inference", i.e. deduce true statements from known statements.
%In the tensor network representation, these entailments can be decided by contracting the whole representing tensor with the statement, that needs to be checked.

\begin{example}[$\sudokunum^2\,\times \,\sudokunum^2$ Sudoku]
    \label{exa:sudokuEntailment}%{\alex{Attempt to match the above Sudoku example with our notation of boolean variables and the entailment formalism}}
    We index the rows and the columns by tuples $(r0,r1)$ and $(co,c1)$, where $r0,r1,c0,c1\in[\sudokunum]$. The first index indicates the block and the second counts the row or column inside that block.
    For each $r0,r1,c0,c1\in[\sudokunum]$ and $i\in[\sudokunum^2]$ we then define an atomic variable $\catvariableof{r0,r1,c0,c1,i}\in\{0,1\}$ indicating whether in the row $(r0,r1)$ and column $(co,c1)$ the number $i$ is written.
    The Sudoku rules then amount to the formula
    \begin{align*}
        \sudokukbof{\sudokunum}  \coloneqq
        &\left( \bigwedge_{r0,r1,c0,c1\in[\sudokunum]} \left( \woneoplus_{i\in[\sudokunum^2]} \catvariableof{r0,r1,c0,c1,i} \right) \right) \land
        \left( \bigwedge_{r0,r1\in[\sudokunum], i\in[\sudokunum^2]} \left( \woneoplus_{c0,c1\in[\sudokunum]} \catvariableof{r0,r1,c0,c1,i} \right) \right) \land \\
        &\left( \bigwedge_{c0,c1\in[\sudokunum], i\in[\sudokunum^2]} \left( \woneoplus_{c0,c1\in[\sudokunum]} \catvariableof{r0,r1,c0,c1,i} \right) \right) \land
        \left( \bigwedge_{r0,c0\in[\sudokunum], i\in[\sudokunum^2]} \left( \woneoplus_{r1,c1\in[\sudokunum]} \catvariableof{r0,r1,c0,c1,i} \right) \right) \, ,
    \end{align*}
    where $\woneoplus$ is the $\sudokunum^2$-ary exclusive or connective (that is $1$ if and only if exactly one of the arguments is $1$).
    The four outer brackets in $\kb$ mark the constraints that at each position exactly one number is assigned, further that in each row each number is assigned once, and similar for the columns and the squares of the board.
    When solving a specific Sudoku instance, one typically knows from an initial board assignment $\sudokustartevidence$ a collection of atomic variables, which hold, and needs to find further atomic variables, which are entailed.
    This means, we need to decide for each $(r_0,r_1,c_0,c_1,i)\notin \sudokustartevidence$ whether the Sudoku rules and the initial board imply that the atomic variable $\catvariableof{r0,r1,c0,c1,i}$ (i.e. assignment to the board) is true
    \begin{align*}
        \sudokukbof{\sudokunum} \land \left(\bigwedge_{(r_0,r_1,c_0,c_1,i)\in \sudokustartevidence} \catvariableof{r0,r1,c0,c1,i} \right) \models \catvariableof{r0,r1,c0,c1,i}
    \end{align*}
    or false
    \begin{align*}
%        \label{eq:sudokukb}
        \kb \land \left(\bigwedge_{(r_0,r_1,c_0,c_1,i)\in \sudokustartevidence} \catvariableof{r0,r1,c0,c1,i} \right) \models \lnot\catvariableof{r0,r1,c0,c1,i} \, .
    \end{align*}
%    In other words, for each assignment to the board, that fulfills the Sudoku rules and the initial board, do we write the number $n$ in row $(r0,r1)$ and column $(c0,c1)$?
    If and only if the Sudoku has a unique solution given the initial board assignment $\sudokustartevidence$, exactly one of these entailment statements holds for each $(r_0,r_1,c_0,c_1,i)\notin \sudokustartevidence$.
    Deciding which is equivalent to solving the Sudoku.
%    We model Sudoku as a hypergraph of
%    \begin{itemize}
%        \item Nodes labeled by $\sudokunum^6$ tuples $(r0,r1,c0,c1,i)$:
%        \begin{align*}
%            \nodes = \{(r0,r1,c0,c1,i) \wcols r0,r1,c0,c1 \in [\sudokunum]\ncond i \in [\sudokunum^2]\}
%        \end{align*}
%        \item Hyperedges by the $4\cdot \sudokunum^4$ constraints (implementing the position, row, columns and square contraints):
%        \begin{align*}
%            \edges =& \big\{\{(r0,r1,c0,c1,i) \wcols r0,r1,c0,c1\in[\sudokunum]\} \wcols r0,r1,c0,c1\in[\sudokunum]\big\} \cup \\
%            &\big\{\{(r0,r1,c0,c1,i) \wcols r0,r1,c0,c1\in[\sudokunum]\} \wcols r0,r1\in[\sudokunum], i\in[\sudokunum^2]\big\} \cup \\
%            &\big\{\{(r0,r1,c0,c1,i) \wcols r0,r1,c0,c1\in[\sudokunum]\} \wcols c0,c1\in[\sudokunum], i\in[\sudokunum^2]\big\} \cup \\
%            &\big\{\{(r0,r1,c0,c1,i) \wcols r0,r1,c0,c1\in[\sudokunum]\} \wcols r0,c0\in[\sudokunum], i\in[\sudokunum^2]\big\} \cup
%        \end{align*}
%        Each hypercore is carried by a tensor representing the logical formula $\woneoplus$ on $\sudokunum^2$ boolean variables.
%    \end{itemize}

    For a more concrete example, let $n=2$ and
    \begin{align*}
        \sudokustartevidence = \{&(0,0,0,0,0),(0,0,1,0,2),(0,0,1,1,1), %first row
        (0,1,0,1,1), \\ %second row
        &(1,0,1,0,3), %third row
        (1,1,0,0,3),(1,1,0,1,2) %fourth row
        \} \, .
    \end{align*}
    We visualize this evidence by writing $i+1$ in a grid cell $(r0,r1,c0,c1)$ to indicate that $(r0,r1,c0,c1,i)\in \sudokustartevidence$:
    \begin{center}
        \begin{sudoku4x4}
            \matrix[sudokumatrix] (M) at (0,0) {
                1 & \ & 3 & 2 \\
                \ & 2 & \  & \  \\
                \ & \ & 4 & \ \\
                4 & 3 &  \ & \  \\
            };
            \draw[thick]([yshift=9.5pt,xshift=-0.6pt]M-1-2.east) -- ([yshift=-9.5pt,xshift=-0.6pt]M-4-2.east);
            \draw[thick]([xshift=-9.5pt,yshift=0.6pt]M-2-1.south) -- ([xshift=9.5pt,yshift=0.6pt]M-2-4.south);
        \end{sudoku4x4}
    \end{center}
    After deriving a sparse tensor network representations in \exaref{exa:sudokuDecomposition}, we demonstrate a solution algorithm to solve this instance in \exaref{exa:sudokuEntailment}.
\end{example}

\subsection{Efficient representation of knowledge bases}

We now investigate the representation of propositional knowledge bases $\kb=\{\formulaof{\selindex}\wcols\selindexin\}$, which are sets of propositional formulas $\formulaof{\selindex}$.
The conjunction of these formulas is the knowledge base formula
\begin{align*}
    \kbwith
    = \bigwedge_{\selindexin} \formulaofat{\selindex}{\shortcatvariables} \, .
\end{align*}
To show efficient representations, we use the following identities.

\begin{lemma}[Computation Network Symmetries]
    \label{lem:comNetSymmetries}
    For the $\catorder$-ary $\land$-connective (where $\catorder\in\nn$) and the unary $\lnot$-connective it holds that
    \begin{align*}
        \contractionof{\tbasisat{\headvariable},\bencodingofat{\land}{\headvariable,\shortcatvariables}}{\shortcatvariables}
        = \bigotimes_{\catenumeratorin} \tbasisat{\catvariableof{\catenumerator}}
        \andspace
        \contractionof{\tbasisat{\headvariable},\bencodingofat{\lnot}{\headvariable,\catvariable}}{\catvariable}
        = \fbasisat{\catvariable} \, .
    \end{align*}
\end{lemma}
\begin{proof}
    Follows directly from the definitions of the basis encodings and the connectives.
\end{proof}

\begin{example}[Computation Network Symmetries]
    %see the notebook: \url{https://colab.research.google.com/drive/1p2wp61fFMu0otnfFhKoNsLiCNfWpuEsn?usp=sharing}
    For the propositional formula $\formulaat{\catvariableof{[3]}}={(\catvariableof{0} \lor \catvariableof{1}) \land \lnot \catvariableof{2}}$ (see \exaref{exa:propFormulaCoordinatewise}), we can write the formula in terms of a \ComputationActivationNetwork{} with activation tensor $\tbasis$ and computation network decomposed by the basis encodings. First, it is written with one activation vector. Second, we see that it can also be interpreted with multiple features.
    \begin{center}
        \begin{tikzpicture}[scale=0.4, yscale=-1, thick] % , baseline = -3.5pt

            \draw[] (-2,1)--(-2,-1) node[midway,left] {\colorlabelsize $\catvariableof{0}$};
            \draw[] (0.5,1)--(0.5,-1) node[midway,right] {\colorlabelsize $\catvariableof{1}$};
            \draw[] (3,1)--(3,-1) node[midway,right] {\colorlabelsize $\catvariableof{2}$};
            \draw (-3,-1) rectangle (4, -3);
            \node[anchor=center] (text) at (0.5,-2) {\corelabelsize $(\catvariableof{0} \lor \catvariableof{1}) \land \lnot \catvariableof{2}$};
            %\draw[->-] (1.5,-3)--(1.5,-5) node[midway,right] {\colorlabelsize $\headvariableof{a \land b \land \lnot c}$};

            \node[anchor=center] (text) at (5,-2) {${=}$};


            \begin{scope}
            [shift={(7,0)}]

                \draw[->-] (0,1)--(0,-1) node[midway,left] {\colorlabelsize $\catvariableof{0}$};
                \draw[->-] (3,1)--(3,-1) node[midway,right] {\colorlabelsize $\catvariableof{1}$};
                \draw[->-] (6,1)--(6,-1) node[midway,right] {\colorlabelsize $\catvariableof{2}$};

                \draw (-1,-1) rectangle (4, -3);
                \node[anchor=center] (text) at (1.5,-2) {\corelabelsize $\bencodingof{\lor}$};

                \draw[->-] (1.5,-3) --(1.5,-5) node[midway,right]{\colorlabelsize $\headvariableof{0 \lor 1}$};

                \draw (5,-1) rectangle (7, -3);
                \node[anchor=center] (text) at (6,-2) {\corelabelsize $\bencodingof{\lnot}$};

                \draw[->-] (6,-3) --(6,-5) node[midway,right]{\colorlabelsize $\headvariableof{\lnot 2}$};

                \draw (0.5,-5) rectangle (6.5,-7);
                \node[anchor=center] (text) at (3.5,-6) {\corelabelsize $\bencodingof{\land}$};

                \draw[->-] (4,-7) -- (4,-8.5) node[right] {\colorlabelsize $\headvariableof{(0 \lor 1) \land \lnot 2}$};
                \drawvariabledot{4}{-8}
                \draw[] (4,-8) -- (4,-9);
                \draw (3,-9) rectangle (5,-11);
                \node[anchor=center] (text) at (4,-10) {$\tbasis$};

            \end{scope}

            \node[anchor=center] (text) at (15,-2) {${=}$};

            \begin{scope}
            [shift={(17,0)}]

                \draw[->-] (0,1)--(0,-1) node[midway,left] {\colorlabelsize $\catvariableof{0}$};
                \draw[->-] (3,1)--(3,-1) node[midway,right] {\colorlabelsize $\catvariableof{1}$};
                \draw[] (7,1)--(7,-1) node[midway,right] {\colorlabelsize $\catvariableof{2}$};

                \draw (-1,-1) rectangle (4, -3);
                \node[anchor=center] (text) at (1.5,-2) {\corelabelsize $\bencodingof{\lor}$};

                \draw (1.5,-4.5) -- (1.5,-5);
                \draw[->-] (1.5,-3) --(1.5,-4.5) node[midway,right]{\colorlabelsize $\headvariableof{0 \lor 1}$};

                \drawvariabledot{1.5}{-4}
                \draw (0.5,-5) rectangle (2.5,-7);
                \node[anchor=center] (text) at (1.5,-6) {\corelabelsize $\tbasis$};

                \node[anchor=center] (text) at (5,-2) {$\otimes$};


                \draw (6,-1) rectangle (8, -3);
                \node[anchor=center] (text) at (7,-2) {\corelabelsize $\fbasis$};

                %\draw[->-] (6,-3) --(6,-5) node[midway,right]{\colorlabelsize $\headvariableof{\lnot c}$};


                %\draw (0.5,-5) rectangle (6.5,-7);
                %\node[anchor=center] (text) at (3.5,-6) {\corelabelsize $\bencodingof{\land}$};

                %\draw[->-] (4,-7) -- (4,-8.5) node[right] {\colorlabelsize $\headvariableof{(a \lor b) \land \lnot c}$};
                %\drawvariabledot{4}{-8}
                %\draw[] (4,-8) -- (4,-9);
                %\draw (3,-9) rectangle (5,-11);
                %\node[anchor=center] (text) at (4,-10) {$\tbasis$};

            \end{scope}

        \end{tikzpicture}
    \end{center}
\end{example}


We use this to decompose knowledge bases into their individual formulas as follows.

\begin{theorem}
    \label{the:kbDecomposition}
    For any knowledge base $\kbwith = \bigwedge_{\selindexin} \formulaofat{\selindex}{\shortcatvariables}$ it holds that
    \begin{align*}
        \kbwith
        = \contractionof{\{\formulaofat{\selindex}{\shortcatvariables} \wcols \selindexin\}}{\shortcatvariables} \, .
    \end{align*}
\end{theorem}
\begin{proof}
    With \lemref{lem:comNetSymmetries} we have
    \begin{align*}
        \kbwith
        &= \contractionof{\{\tbasisat{\headvariableof{\land}},\bencodingofat{\land}{\headvariableof{\land},\headvariables}\}
            \cup \{\bencodingofat{\formulaof{\selindex}}{\headvariableof{\selindex},\shortcatvariables}\wcols \selindexin\}}{\shortcatvariables} \\
        &= \contractionof{
            \bigcup_{\selindexin} \{\tbasisat{\headvariableof{\selindex}},\bencodingofat{\formulaof{\selindex}}{\headvariableof{\selindex},\shortcatvariables}\wcols \selindexin\}}{\shortcatvariables} \\
        &= \contractionof{\{\formulaofat{\selindex}{\shortcatvariables} \wcols \selindexin\}}{\shortcatvariables} \, .
    \end{align*}
\end{proof}

\begin{example}[Sparse representation of Sudoku rule knowledge base]
    \label{exa:sudokuDecomposition}%{\alex{Attempt to match the above Sudoku example with our notation of boolean variables and the entailment formalism}}
    We exploit \theref{the:kbDecomposition} to find an efficient tensor network representation of the Sudoku knowledge base from \exaref{exa:sudokuEntailment}.
    We directly get that the knowledge base $\sudokukbof{\sudokunum}$ of Sudoku rules is a tensor network of the $4\cdot \sudokunum^4$ constraint formulas using the $\sudokunum^2$-ary connective $\woneoplus$, and the evidence $\sudokustartevidence$ can be encoded by vectors $\tbasisat{\catvariableof{(r_0,r_1,c_0,c_1,i)}}$.
    To get a representation by matrices instead of tensors of order $\sudokunum^2$, we introduce a hidden variable $\decvariable$ taking values in $[\sudokunum^2]$ for each of the constraints, one can further increase the sparsity of the representation.
    Using the matrices
    \begin{align*}
        \hypercoreofat{\catenumerator}{\catvariableof{\catenumerator},\decvariable}
        = \fbasisat{\catvariableof{\catenumerator}} \otimes \onesat{\decvariable} + (\tbasisat{\catvariableof{\catenumerator}}-\fbasisat{\catvariableof{\catenumerator}}) \otimes \onehotmapofat{\catenumerator}{\decvariable},
    \end{align*}
    we have the decomposition
    \begin{align*}
        \woneoplus[\catvariableof{[\sudokunum^2]}]
        = \contractionof{\{\hypercoreofat{\catenumerator}{\catvariableof{\catenumerator},\decvariable} \wcols \catenumerator\in[\sudokunum^2]\}}{\catvariableof{[\sudokunum^2]}} \, ,
    \end{align*}
    which is a $\cpformat$ decomposition (see \exaref{exa:cpFormat}) depicted in \figref{fig:sudokuDecomposition} a).

    %% Alternative TT decomposition
    Alternatively, there is a $\ttformat$ decomposition (see \exaref{exa:ttFormat}) of the constraint $\woneoplus$, which we depict in \figref{fig:sudokuDecomposition} b).
    We introduce for $\catenumerator\in[\catorder-1]$ hidden variables $\decvariable^{\catenumerator}$ of dimension $2$, which are interpreted as the indicator whether one of the variables $\catvariableof{[\catenumerator]}$ is true.
    Following this interpretation we introduce $\ttformat$ cores
    \begin{align*}
        \sechypercoreofat{0}{\catvariableof{0},\secdecvariable^{0}}
        &= \tbasisat{\catvariableof{0}}\otimes\tbasisat{\secdecvariable^{0}}
        + \fbasisat{\catvariableof{0}}\otimes\fbasisat{\secdecvariable^{0}}\,, \\
        \sechypercoreofat{\catorder-1}{\secdecvariable^{\catorder-2},\catvariableof{\catorder-1}}
        &= \fbasisat{\secdecvariable^{\catorder-2}} \otimes \tbasisat{\catvariableof{\catorder-1}}
        + \tbasisat{\secdecvariable^{\catorder-2}} \otimes \fbasisat{\catvariableof{\catorder-1}}
    \end{align*}
    and for $\catenumerator\in\{1,\ldots,\catorder-2\}$
    \begin{align*}
        \sechypercoreofat{\catenumerator}{\secdecvariable^{\catenumerator-1},\catvariableof{\catenumerator},\secdecvariable^{\catenumerator}}
        = \identityat{\secdecvariable^{\catenumerator-1},\secdecvariable^{\catenumerator}}  \otimes \fbasisat{\catvariableof{\catenumerator}}
        + \fbasisat{\secdecvariable^{\catenumerator-1}} \otimes \tbasisat{\catvariableof{\catenumerator}} \otimes \tbasisat{\secdecvariable^{\catenumerator-1}} \, .
    \end{align*}
    Note that the $\ttformat$ decomposition of the constraint $\woneoplus$ introduces $\catorder-1$ many hidden variables of dimension $2$ whereas the $\cpformat$ decomposition introduces a single hidden variable of dimension $\catorder$.
    In the following, we further apply the $\cpformat$ decomposition. % Reason: fewer variables to handle in message passing

    \begin{figure}[t]

        \begin{center}
            \begin{tikzpicture}[scale=0.35,thick]

                \draw (-6,-1) rectangle (-12,1);
                \node[anchor=center] (A) at (-9,0) {\corelabelsize $\woneoplus$};
                \draw (-11.5,-1)--(-11.5,-2.5) node[midway,left] {\colorlabelsize $\catvariableof{0,0,0,0,0}$};
                \draw (-11,-1)--(-11,-2.5) node[midway,right] {\colorlabelsize $\catvariableof{0,0,0,0,1}$};
                \node[anchor=center] (A) at (-9,-2.5) {$\cdots$};
                \draw (-7,-1)--(-7,-2.5) node[midway,right] {\colorlabelsize $\catvariableof{0,0,0,0,\sudokunum^2\shortminus 1}$};

                \node[anchor=center] (A) at (-3.5,1) {$a)$};
                \node[anchor=center] (A) at (-3.5,0) {$=$};

                \draw (-1,-1) rectangle (1,1);
                \node[anchor=center] (A) at (0,0) {\corelabelsize $\hypercoreof{0}$};
                \draw (0,-1)--(0,-2.5) node[midway,right] {\colorlabelsize $\catvariableof{0,0,0,0,0}$};

                \draw (3,-1) rectangle (5,1);
                \node[anchor=center] (A) at (4,0) {\corelabelsize $\hypercoreof{1}$};
                \draw (4,-1)--(4,-2.5) node[midway,right] {\colorlabelsize $\catvariableof{0,0,0,0,1}$};

                \node[anchor=center] (text) at (8,0) {$\hdots$};

                \draw (10.75,-1) rectangle (13.25,1);
                \node[anchor=center] (A) at (12,0) {\corelabelsize $\hypercoreof{\sudokunum^2\shortminus1}$};
                \draw (12,-1)--(12,-2.5) node[midway,right] {\colorlabelsize $\catvariableof{0,0,0,0,\sudokunum^2\shortminus1}$};

                \drawvariabledot{6}{4}
                \node[anchor=south] (text) at (6,4) {\colorlabelsize $\decvariableof{0,0,0,0,:}$};

                \draw (6,4) to[bend right= 20] (0,1);
                \draw (6,4) to[bend right= 10] (4,1);
                \draw (6,4) to[bend right= -20] (12,1);

                \begin{scope}[shift={(0,-7)}]
                    \node[anchor=center] (A) at (-3.5,1) {$b)$};
                    \node[anchor=center] (A) at (-3.5,0) {$=$};

                    \draw (-2,-1) rectangle (0,1);
                    \node[anchor=center] (A) at (-1,0) {\corelabelsize $\sechypercoreof{0}$};
                    \draw (-1,-1)--(-1,-2.5) node[midway,right] {\colorlabelsize $\catvariableof{0,0,0,0,0}$};

                    \draw (0,0) -- (3,0) node [midway,above] {\colorlabelsize $\secdecvariableof{0,0,0,0,:}^0$};

                    \draw (3,-1) rectangle (5,1);
                    \node[anchor=center] (A) at (4,0) {\corelabelsize $\sechypercoreof{1}$};
                    \draw (4,-1)--(4,-2.5) node[midway,right] {\colorlabelsize $\catvariableof{0,0,0,0,1}$};

                    \node[anchor=center] (text) at (6.5,1) {\colorlabelsize $\secdecvariableof{0,0,0,0,:}^{1}$};
                    \draw (5,0) -- (7,0); % node [midway,above] {\colorlabelsize $\secdecvariableof{0,0,0,0,:}^1$};

                    \node[anchor=center] (text) at (8,0) {$\hdots$};

                    \node[anchor=center] (text) at (10.25,1) {\colorlabelsize $\secdecvariableof{0,0,0,0,:}^{\sudokunum^2\shortminus2}$};
                    \draw (9.75,0) -- (11.75,0);

                    \draw (11.75,-1) rectangle (14.25,1);
                    \node[anchor=center] (A) at (13,0) {\corelabelsize $\sechypercoreof{\sudokunum^2\shortminus1}$};
                    \draw (13,-1)--(13,-2.5) node[midway,right] {\colorlabelsize $\catvariableof{0,0,0,0,\sudokunum^2\shortminus1}$};


                \end{scope}

            \end{tikzpicture}
        \end{center}
        \caption{Decomposition of the position constraint $\woneoplus$ at position $(r0,r1,c0,c1)=(0,0,0,0)$ into a) a $\cpformat$ decomposition with hidden variable $\decvariableof{0,0,0,0,:}$ and b) a $\ttformat$ decomposition with $d-1$ hidden variables $\decvariableof{0,0,0,0,:}^k, k\in[d-1]$.}
        \label{fig:sudokuDecomposition}
    \end{figure}

    Given evidence $\sudokustartevidence$ we denote the Sudoku Knowledge Base $\sudokukbof{\sudokunum,\sudokustartevidence}$.
    It is modelled as a tensor network on a hypergraph $\graphof{\mathrm{Sudoku},n}$ consisting of
    \begin{itemize}
        \item $\sudokunum^6+4\cdot \sudokunum^4$ nodes by $\sudokunum^6$ categorical variables $\catvariableof{(r0,r1,c0,c1,i)}$ and by $4\cdot \sudokunum^4$ decomposition variables to the constraints
        \item $5\cdot \sudokunum^6$ edges
        \begin{align*}
            \edges=
            \bigcup_{r0,r1,c0,c1\in[\sudokunum]}
            \big\{
            &\{\catvariableof{(r0,r1,c0,c1,i)}\},\{\catvariableof{(r0,r1,c0,c1,i)},\decvariableof{r0,r1,c0,c1,:}\},\{\catvariableof{(r0,r1,c0,c1,i)},\decvariableof{r0,r1,:,:,i}\},\\
            &\{\catvariableof{(r0,r1,c0,c1,i)},\decvariableof{:,:,c0,c1,i}\},
            \{\catvariableof{(r0,r1,c0,c1,i)},\decvariableof{r0,:,c0,:,i}\}\big\}
        \end{align*}
        We denote the decomposition variables to the position, row, column and square constraints by $\decvariableof{r0,r1,c0,c1,:},\decvariableof{r0,r1,:,:,i},\decvariableof{:,:,c0,c1,i}$ and $\decvariableof{r0,:,c0,:,i}$.
    \end{itemize}
    Each edge containing a decomposition variable is decorated by a matrix $\hypercoreofat{\catenumerator}{\catvariable,\decvariable}$ corresponding to a core in the $\cpformat$ decomposition of a constraint.
    Here, $\catenumerator$ is determined by the tuple $(r0,r1,c0,c1,i)$ and the type of the constraint (for example, for the variable $\catvariableof{(0,1,1,2,1)}$ and the row constraint $\decvariableof{(0,1,:,:,1)}$ we have $\catenumerator=1\cdot n + 2$.
    We further assign to each edge containing a single variable $\{\catvariableof{(r0,r1,c0,c1,i)}\}$ either the vector $\tbasisat{\catvariableof{(r0,r1,c0,c1,i)}}$ if $(r0,r1,c0,c1,i)\in \sudokustartevidence$ or the trivial vector $\onesat{\catvariableof{(r0,r1,c0,c1,i)}}$.
\end{example}

\subsection{Entailment decision by message passing}

% Infeasible contractions
Since contracting the whole tensor network is often infeasible, local contractions can be considered to decide entailment in some cases.
Here, a local contraction describes the calculation of contractions along few closely connected tensors in the network.
Before presenting the resulting Constraint Propagation algorithm, we first show two important properties of local entailment motivating the procedure.

\begin{theorem}[Monotonicity of propositional logics]
    \label{the:monotonicityPL}
    If $\seckb\subset\kb$ and $\seckb\models\formula$ then also $\kb\models\formula$.
\end{theorem}
\begin{proof}
    Since $\seckb\models\formula$ it holds that $\contraction{\seckb[\shortcatvariables],\lnot\formula[\shortcatvariables]}=0$ and thus
    \begin{align*}
        \contractionof{\seckb[\shortcatvariables],\lnot\formula[\shortcatvariables]}{\shortcatvariables}=\zerosat{\shortcatvariables} \, .
    \end{align*}
    Denoting by $\kb/\seckb$ the conjunctions of formulas in $\kb$ not in $\seckb$, we have
    \begin{align*}
        \contraction{\kbwith,\lnot\formulawith}
        &= \contraction{(\kb/\seckb)[\shortcatvariables],\seckb[\shortcatvariables],\lnot\formulawith} \\
        &= \contraction{(\kb/\seckb)[\shortcatvariables],\contractionof{\seckb[\shortcatvariables],\lnot\formulawith}{\shortcatvariables}} \\
        &= \contraction{(\kb/\seckb)[\shortcatvariables],\zerosat{\shortcatvariables}} \\
        &= 0 \, .
    \end{align*}
\end{proof}
To decide entailment, we can therefore investigate entailment on smaller parts of the knowledge base.
This is sound by the above theorem but not complete since it can happen that no smaller part of the knowledge base entails the formula while the whole knowledge base does.
We can furthermore add entailed formulas to the knowledge base without changing it as is shown next.

\begin{theorem}[Invariance of adding entailed formulas]
    \label{the:addingEntailed}
    If and only if $\kb\models\formula$ we have that
    \begin{align*}
        \kbwith
        = \contractionof{\kbwith,\formulawith}{\shortcatvariables} \, .
    \end{align*}
\end{theorem}
\begin{proof}
    We use that $\formulawith+\lnot\formulawith=\onesat{\shortcatvariables}$ and thus
    \begin{align*}
        \kbwith
        &= \contractionof{\kbwith,(\formulawith+\lnot\formulawith)}{\shortcatvariables} \\
        &= \contractionof{\kbwith,\formulawith}{\shortcatvariables}  + \contractionof{\kbwith,\lnot\formulawith}{\shortcatvariables} \,. \\
        %&= \contractionof{\kbwith,\formulawith}{\shortcatvariables} \, .
    \end{align*}
    Since $\contractionof{\kbwith,\lnot\formulawith}{\shortcatvariables}$ is boolean, we have that
    \begin{align*}
        \kbwith=\contractionof{\kbwith,\formulawith}{\shortcatvariables}
    \end{align*}
    if and only if $\contraction{\kbwith,\lnot\formulawith}=0$, that is $\kb\models\formula$.
\end{proof}

% Interpreting entailment
The mechanism of \theref{the:addingEntailed} provides us with a means to store entailment information in small-order auxiliary tensors.
%Adding deduced statements to a knowledge base does not change the knowledge base as a tensor, but one can exploit it in smaller contractions.
% Constraint Propagation
One way to exploit this accessibility of local entailment information are message passing schemes similar to \algoref{alg:treeBeliefPropagation} propagating the information.
This approach decides local entailment by iteratively adding entailed formulas to the knowledge base and checking further entailment on neighboring tensors of the knowledge base.
Since for entailment decisions the support of the contractions is sufficient, we can apply non-zero indicators before sending contraction messages.
We then schedule new messages in the direction $(\sedge,\redge)$ once the support of a message received at $\sedge$ has been changed.
Note that such a scheduling system is guaranteed to converge since there can only be a finite number of message changes.
We further directly reduce the computation of messages to their support and call the resulting Constraint Propagation (\algoref{alg:constraintPropagation}).

\begin{algorithm}[hbt!]
    \caption{Constraint Propagation}\label{alg:constraintPropagation}
    \begin{algorithmic}
        \Require Tensor network $\extnet$ on a hypergraph $\graph$
        \Ensure Messages $\{\messagewith\wcols(\sedge,\redge)\in\dirovedges\}$ containing entailment statements
        \iosepline
        \State Initialize a queue $\scheduler = \dirovedges$ of message directions
        \State Initialize messages $\messagewith = \onesat{\catvariableof{\sedge\cap\redge}}$ for $(\sedge,\redge)\in\dirovedges$
        \While{$\scheduler$ not empty}
            \State Pop a $(\sedge,\redge)$ pair from $\scheduler$
            \State Update the message
            \begin{align*}
                \messagewith
                = \nonzeroof{\contractionof{\{\hypercoreofat{\sedge}{\catvariableof{\sedge}}\}
                    \cup \{\mesfromtoat{\secsedge}{\sedge}{\catvariableof{\secsedge\cap\sedge}} \wcols (\secsedge,\sedge)\in\dirovedges \ncond \secsedge\neq \redge\}
                }{\catvariableof{\sedge\cap \redge}}}
            \end{align*}
            \If{$\hypercoreat{\catvariableof{\sedge\cap \redge}}\neq\messagewith$}
                \State Update the message: $\messagewith\coloneqq\hypercoreat{\catvariableof{\sedge\cap \redge}}$
                \State Add $\scheduler = \scheduler \cup \{(\redge,\secsedge) \wcols (\redge,\secsedge)\in\dirovedges\}$ % Clear?
            \EndIf
        \EndWhile
        \State \Return Messages $\{\messagewith\wcols(\sedge,\redge)\in\dirovedges\}$
    \end{algorithmic}
\end{algorithm}

\begin{theorem}
    \label{the:constraintPropagationSoundness}
    All messages during constraint propagation are sound, meaning that for all $(\sedge,\redge)\in\dirovedges$ it holds that
    \begin{align*}
        \nonzeroof{\contractionof{\extnet}{\catvariableof{\sedge\cap\redge}}} \prec \messagewith \, .
    \end{align*}
\end{theorem}
\begin{proof}
    We show this theorem by induction over the \whileSymbol{} loop of \algoref{alg:constraintPropagation}.
    At the first iteration, we have for all messages $\messagewith=\onesat{\catvariableof{\sedge\cap\redge}}$ and thus
    \begin{align}
        \label{eq:nzMessageAddingEquivalence}
        \extnet = \contractionof{\{\extnet\}\cup\{\messagewith\wcols(\sedge,\redge)\in\dirovedges\}}{\nodevariables} \, .
    \end{align}
    By \theref{the:monotonicityPL} we then have for the first message send along the pair $(\sedge,\redge)$ that
    \begin{align*}
        \nonzeroof{\contractionof{\extnet}{\catvariableof{\sedge\cap\redge}}} \prec
        &\nonzeroof{\contractionof{\{\hypercoreofat{\sedge}{\catvariableof{\sedge}}\}
            \cup \{\mesfromtoat{\secsedge}{\sedge}{\catvariableof{\secsedge\cap\sedge}} \wcols (\secsedge,\sedge)\in\dirovedges \ncond \secsedge\neq \redge\}
        }{\catvariableof{\sedge\cap \redge}}} \\
        &= \messagewith \, .
    \end{align*}

    We now assume that at an arbitrary state of the algorithm the inequality holds for all previously sent messages.
    By \theref{the:addingEntailed} we can contract the messages with the tensor network without changing it and \eqref{eq:nzMessageAddingEquivalence} thus still holds.
    We then conclude with \theref{the:monotonicityPL} that the claimed property also holds for the new message.
\end{proof}

\begin{example}[Constraint Propagation for the Sudoku of \exaref{exa:sudokuEntailment}]
    \label{exa:sudokuMessagePassing}
    We iteratively solve a Sudoku puzzle by determining a possible value based on neighboring cells, rows and squares (using \theref{the:monotonicityPL}) and adding to our knowledge (using \theref{the:addingEntailed}).
    For example, consider the following $\sudokunum=2$ Sudoku puzzle, where a first entailment step uses only the knowledge of the rules and the \textcolor{\concolor}{blue} cells to determine the value $3$ in the first square:
    \begin{center}
        \begin{sudoku4x4}
            \matrix[sudokumatrix] (M) at (0,0) {
                1 & \ & \textcolor{\concolor}{3} & 2 \\
                \ & \textcolor{\concolor}{2} & \  & \  \\
                \ & \ & 4 & \ \\
                4 & 3 &  \ & \  \\
            };
            \draw[thick]([yshift=9.5pt,xshift=-0.6pt]M-1-2.east) -- ([yshift=-9.5pt,xshift=-0.6pt]M-4-2.east);
            \draw[thick]([xshift=-9.5pt,yshift=0.6pt]M-2-1.south) -- ([xshift=9.5pt,yshift=0.6pt]M-2-4.south);

            \node[anchor=center] (ist) at (1.75,0) {$=$};

            \matrix[sudokumatrix] (M) at (3.5,0) {
                1 & \ & 3 & 2 \\
                \textcolor{\probcolor}{3} & 2 & \  & \  \\
                \ & \ & 4 & \ \\
                4 & 3 &  \ & \  \\
            };
            \draw[thick]([yshift=9.5pt,xshift=-0.6pt]M-1-2.east) -- ([yshift=-9.5pt,xshift=-0.6pt]M-4-2.east);
            \draw[thick]([xshift=-9.5pt,yshift=0.6pt]M-2-1.south) -- ([xshift=9.5pt,yshift=0.6pt]M-2-4.south);

            \node[anchor=center] (ist) at (6.25,0) {$= \quad \ldots \quad =$};

            \matrix[sudokumatrix] (M) at (9,0) {
                1 & \textcolor{\probcolor}{4} & 3 & 2 \\
                \textcolor{\probcolor}{3} & 2 & \textcolor{\probcolor}{1} & \textcolor{\probcolor}{4}  \\
                \textcolor{\probcolor}{2} & \textcolor{\probcolor}{1} & 4 & \textcolor{\probcolor}{3} \\
                4 & 3 & \textcolor{\probcolor}{2} & \textcolor{\probcolor}{1}  \\
            };
            \draw[thick]([yshift=9.5pt,xshift=-0.6pt]M-1-2.east) -- ([yshift=-9.5pt,xshift=-0.6pt]M-4-2.east);
            \draw[thick]([xshift=-9.5pt,yshift=0.6pt]M-2-1.south) -- ([xshift=9.5pt,yshift=0.6pt]M-2-4.south);
        \end{sudoku4x4}
    \end{center}

    To illustrate the first reasoning step of assigning $\textcolor{\probcolor}{\catvariableof{0,1,0,0,2}}$, we make the following entailment steps applying \theref{the:monotonicityPL}.
    We also depict in \figref{fig:contractionPropagationSudoku} the corresponding messages in the Constraint Propagation Algorithm on the hypergraph $\graph^{\mathrm{Sudoku},n}$.
    \begin{itemize}
        \item From $\textcolor{\concolor}{\catvariableof{0,1,0,1,1}}$ (i.e. the $2$ in the cell $(0,1,0,1)$) and the Sudoku rule that at the cell $(0,1,0,1)$ exactly one number is assigned, we get
        \begin{align*}
            \left( \woneoplus_{i\in[\sudokunum^2]} \catvariableof{0,1,0,1,i} \right) \land \textcolor{\concolor}{\catvariableof{0,1,0,1,1}} \models \lnot\catvariableof{0,1,0,1,2} \, ,
        \end{align*}
        That is, that the number $3$ is not in the cell $(0,1,0,1)$.
        This entailment step is performed by three consecutive messages (see $\messagesymbol^{(0,[3])}$ in \figref{fig:contractionPropagationSudoku}) along the directions %involving decomposition cores of the position constraint in the position $(r_0,r_1,c_0,c_1)=(0,0,0,0)$.
        \begin{align*}
        (\sedge,\redge)
            \in \big[&(\{\catvariableof{0,1,0,1,1}\},\{\catvariableof{0,1,0,1,1},\decvariableof{0,1,0,1,:}\}),
                (\{\catvariableof{0,1,0,1,1},\decvariableof{0,1,0,1,:}\},\{\catvariableof{0,1,0,1,2},\decvariableof{0,1,0,1,:}\}), \\
                &(\{\catvariableof{0,1,0,1,2},\decvariableof{0,1,0,1,:}\},\{\catvariableof{0,1,0,1,2},\decvariableof{0,:,0,:,2}\})\big] \, .
        \end{align*}
        Intuitively, the messages commmunicate to the square constraint $\decvariableof{0,:,0,:,2}$, that by the position constraint $\decvariableof{0,1,0,1,:}$ the variable $3$ cannot be assigned at $(0,1,0,1)$.
        %This entailment step is performed by two consecutive messages along the directions $(\catvariableof{0,1,0,1,1},\decvariableof{0,1,0,1,:})$ and $(\decvariableof{0,1,0,1,:},\catvariableof{0,1,0,1,2})$.
        \item From $\textcolor{\concolor}{\catvariableof{0,0,1,0,2}}$ (i.e. the $3$ in the cell $(0,0,1,0)$) and the Sudoku rule that at the row $(0,0)$ exactly one number is assigned, we get
        \begin{align*}
            \left( \woneoplus_{c0,c1\in[\sudokunum]} \catvariableof{0,0,c0,c1,2} \right) \land \textcolor{\concolor}{\catvariableof{0,0,1,0,2}} \models \lnot\catvariableof{0,0,0,0,2}\land \lnot\catvariableof{0,0,0,1,2} \, ,
        \end{align*}
        meaning that the number $3$ is neither in the cell $(0,0,0,0)$ nor in $(0,0,0,1)$.
        This entailment step is performed by five consecutive messages (see $\messagesymbol^{(1,[5])}$ in \figref{fig:contractionPropagationSudoku}) along the directions
        %This entailment step is performed by three consecutive messages along the directions $(\catvariableof{0,0,1,0,2},\decvariableof{0,0,:,:,2})$, $(\decvariableof{0,0,:,:,2},\catvariableof{0,0,0,0,2})$ and $(\decvariableof{0,0,:,:,2},\catvariableof{0,0,0,1,2})$.
        \begin{align*}
        (\sedge,\redge)
            \in \big[
            &(\{\catvariableof{0,0,1,0,2}\},\{\catvariableof{0,0,1,0,2},\decvariableof{0,0,:,:,2}\}),
                (\{\catvariableof{0,0,1,0,2},\decvariableof{0,0,:,:,2}\},\{\catvariableof{0,0,0,0,2},\decvariableof{0,0,:,:,2}\}), \\
                &(\{\catvariableof{0,0,1,0,2},\decvariableof{0,0,:,:,2}\},\{\catvariableof{0,0,0,1,2},\decvariableof{0,0,:,:,2}\}),
                (\{\catvariableof{0,0,0,0,2},\decvariableof{0,0,:,:,2}\},\{\catvariableof{0,0,0,0,2},\decvariableof{0,:,0,:,2}\}) \\
                &(\{\catvariableof{0,0,0,1,2},\decvariableof{0,0,:,:,2}\},\{\catvariableof{0,0,0,1,2},\decvariableof{0,:,0,:,2}\})
                \big] \, .
        \end{align*}
        The messages communicate that based on the decomposition cores of the constraint to the number $i=3$ in the first row $(r_0,r_1)=(0,0)$, the number $3$ cannot be assigned at $(0,0,0,0)$ and $(0,0,0,1)$.
    \end{itemize}
    We add these formulas to our knowledge base (justified by \theref{the:addingEntailed}) and use the rule that $3$ appears exactly once in the first square
    \begin{align*}
        &\left( \woneoplus_{r1,c1\in[\sudokunum]} \catvariableof{0,r1,0,c1,2} \right)
        \land (\lnot\catvariableof{0,1,0,1,2})
        \land (\lnot\catvariableof{0,0,0,0,2}\land \lnot\catvariableof{0,0,0,1,2})
        \models \textcolor{\probcolor}{\catvariableof{0,1,0,0,2}} \, .
    \end{align*}
    We hence conclude that the number $3$ must be in the cell $(0,1,0,0)$.
    This information is also included in the updated knowledge base for further reasoning steps.
    This last entailment step is performed by four consecutive messages (see $\messagesymbol^{(2,[4])}$ in \figref{fig:contractionPropagationSudoku}) along the directions
    \begin{align*}
    (\sedge,\redge)
        \in \big[
            &(\{\catvariableof{0,1,0,1,2},\decvariableof{0,:,0,:,2}\},\{\catvariableof{0,1,0,0,2},\decvariableof{0,:,0,:,2}\}),
            (\{\catvariableof{0,0,0,1,2},\decvariableof{0,:,0,:,2}\},\{\catvariableof{0,1,0,0,2},\decvariableof{0,:,0,:,2}\}), \\
            &(\{\catvariableof{0,0,1,0,2},\decvariableof{0,:,0,:,2}\},\{\catvariableof{0,1,0,0,2},\decvariableof{0,:,0,:,2}\}),
            (\{\catvariableof{0,1,0,0,2},\decvariableof{0,:,0,:,2}\},\{\catvariableof{0,1,0,0,2}\})\big]
    \end{align*}
    The first three messages communicate that the $3$ is not possible in positions $(0,1,0,1),(0,0,0,1)$ and $(0,0,1,0)$ and the fourth message concludes that the $3$ then has to be at position $(0,1,0,0)$.

    %% Further reasoning steps
    We now iteratively apply similar reasoning steps and store the entailed variables in \textcolor{\probcolor}{$E^{\mathrm{entailed}}$} until we arrive at the right side of the above sketch.
    \begin{align*}
        \sudokukbof{2} \land \left(\bigwedge_{(r_0,r_1,c_0,c_1,i)\in \sudokustartevidence} \catvariableof{r0,r1,c0,c1,i} \right)
        \models \textcolor{\probcolor}{\left(\bigwedge_{(r_0,r_1,c_0,c_1,i)\in E^{\mathrm{entailed}}} \catvariableof{r0,r1,c0,c1,i} \right)} \, .
    \end{align*}
    Since all Sudoku rules are satisfied in the final assignment and to each cell $(r_0,r_1,c_0,c_1)$ we found exactly one $i\in[\sudokunum^2]$ such that $(r_0,r_1,c_0,c_1,i)\in \sudokustartevidence\cup\textcolor{\probcolor}{E^{\mathrm{entailed}}}$, there is a unique solution of the puzzle and we conclude that
    \begin{align*}
        &\sudokukbof{2} \land \left(\bigwedge_{(r_0,r_1,c_0,c_1,i)\in \sudokustartevidence} \catvariableof{r0,r1,c0,c1,i} \right) \\
        &\quad= \left(\bigwedge_{(r_0,r_1,c_0,c_1,i)\in \sudokustartevidence} \catvariableof{r0,r1,c0,c1,i} \right)
        \land \textcolor{\probcolor}{\left(\bigwedge_{(r_0,r_1,c_0,c_1,i)\in E^{\mathrm{entailed}}} \catvariableof{r0,r1,c0,c1,i} \right)} \, .
    \end{align*}
\end{example}

\begin{figure}[t]
    \begin{center}
        \begin{tikzpicture}[scale=0.35,thick]

            %% I 0:0:2 constraint (3 is in the 00 square)
            \draw (-1,-1) rectangle (1,1);
            \node[anchor=center] (A) at (0,0) {\corelabelsize $\hypercoreof{0}$};
            \draw (0,-1)--(0,-3) node[midway,right] {\colorlabelsize $\catvariableof{0,0,0,0,2}$};

            \draw (3,-1) rectangle (5,1);
            \node[anchor=center] (A) at (4,0) {\corelabelsize $\hypercoreof{1}$};
            \draw (4,-1)--(4,-3) node[midway,right] {\colorlabelsize $\catvariableof{0,0,0,1,2}$};

            \draw (7,-1) rectangle (9,1);
            \node[anchor=center] (A) at (8,0) {\corelabelsize $\hypercoreof{2}$};
            \draw (8,-1)--(8,-3); % node[midway,right] {\colorlabelsize $\catvariableof{0,1,0,0,2}$};

            \drawvariabledot{8}{-3}
            \draw (8,-3) -- (7.25,-3);
            \draw[\probcolor] (7,-5) rectangle (9,-7);
            \node[anchor=center, \probcolor] (A) at (8,-6) {\corelabelsize $\tbasis$};
            \draw[\probcolor] (8,-5)--(8,-3) node[midway,right] {\colorlabelsize $\catvariableof{0,1,0,0,2}$};

            \draw[\newmessagecolor,dashed, ->] (6.5,-1) to [bend right = 30] (6.5,-5);
            \node[\newmessagecolor,anchor=center] (A) at (5.25,-4) {\colorlabelsize $\messagesymbol^{(2,3)}$};

            \draw[\newmessagecolor,dashed, ->] (11,1.25) to [bend right = 30] (9,1.25);
            \node[\newmessagecolor,anchor=center] (A) at (9.75,2) {\colorlabelsize $\messagesymbol^{(2,0)}$};

            \draw[\newmessagecolor,dashed, ->] (5,1.5) to [bend right = -40] (6.8,1.1);
            \node[\newmessagecolor,anchor=center] (A) at (6,0.75) {\colorlabelsize $\messagesymbol^{(2,1)}$};

            \draw[\newmessagecolor,dashed, ->] (1,1.5) to [bend right = -40] (7,1.75);
            \node[\newmessagecolor,anchor=center] (A) at (3.5,2) {\colorlabelsize $\messagesymbol^{(2,2)}$};


            \draw (11,-1) rectangle (13,1);
            \node[anchor=center] (A) at (12,0) {\corelabelsize $\hypercoreof{3}$};
            \draw (12,-1)--(12,-2.5) node[midway,right] {\colorlabelsize $\catvariableof{0,1,0,1,2}$};

            \drawvariabledot{6}{4}
            \node[anchor=south] (text) at (6,4) {\colorlabelsize $\decvariableof{0,:,0,:,2}$};

            \draw (6,4) to[bend right= 20] (0,1);
            \draw (6,4) to[bend right= 10] (4,1);
            \draw (6,4) to[bend right= -10] (8,1);
            \draw (6,4) to[bend right= -20] (12,1);


            %% I 00::2 constraint (3 in the first row)
            \draw (3,-7) rectangle (5,-5);
            \node[anchor=center] (A) at (4,-6) {\corelabelsize $\hypercoreof{1}$};
            \drawvariabledot{4}{-3}
            \draw (4,-3) -- (3.25,-3);
            \draw (4,-5)--(4,-1);

            \draw (-1,-7) rectangle (1,-5);
            \node[anchor=center] (A) at (0,-6) {\corelabelsize $\hypercoreof{0}$};
            \drawvariabledot{0}{-3}
            \draw (0,-3) -- (-0.75,-3);
            \draw (0,-5)--(0,-1);

            \draw (-5,-7) rectangle (-3,-5);
            \node[anchor=center] (A) at (-4,-6) {\corelabelsize $\hypercoreof{2}$};
            \draw (-4,-5)--(-4,-3);

            \drawvariabledot{-4}{-3}
            \draw (-4,-3) -- (-4.75,-3);

            \draw[\concolor] (-5,-1) rectangle (-3,1);
            \node[anchor=center, \concolor] (A) at (-4,0) {\corelabelsize $\tbasis$};
            \draw[\concolor] (-4,-1)--(-4,-3) node[midway,right] {\colorlabelsize $\catvariableof{0,0,1,0,2}$};

            %% Messages from 3 in row 0,0
            \draw[\newmessagecolor,dashed, ->] (-5.5,-1) to [bend right = 30] (-5.5,-5);
            \node[\newmessagecolor,anchor=center] (A) at (-6.75,-2) {\colorlabelsize $\messagesymbol^{(1,0)}$};

            \draw[\newmessagecolor,dashed, ->] (-3,-7.25) to [bend right = 30] (-1,-7.25);
            \node[\newmessagecolor,anchor=center] (A) at (-2,-6.75) {\colorlabelsize $\messagesymbol^{(1,1)}$};

            \draw[\newmessagecolor,dashed, ->] (-3,-7.5) to [bend right = 40] (3,-7.25);
            \node[\newmessagecolor,anchor=center] (A) at (0,-9) {\colorlabelsize $\messagesymbol^{(1,2)}$};

            \draw[\newmessagecolor,dashed, <-] (-1.5,-1) to [bend right = 30] (-1.5,-5);
            \node[\newmessagecolor,anchor=center] (A) at (-2.75,-4) {\colorlabelsize $\messagesymbol^{(1,3)}$};

            \draw[\newmessagecolor,dashed, <-] (2.5,-1) to [bend right = 30] (2.5,-5);
            \node[\newmessagecolor,anchor=center] (A) at (1.25,-4) {\colorlabelsize $\messagesymbol^{(1,4)}$};

            \draw (-9,-7) rectangle (-7,-5);
            \node[anchor=center] (A) at (-8,-6) {\corelabelsize $\hypercoreof{3}$};
            \draw (-8,-5)--(-8,-3) node[midway,left] {\colorlabelsize $\catvariableof{0,0,1,1,2}$};

            \drawvariabledot{-2}{-10}
            \node[anchor=north] (text) at (-2,-10) {\colorlabelsize $\decvariableof{0,0,:,:,2}$};

            \draw (-2,-10) to[bend right= -20] (-8,-7);
            \draw (-2,-10) to[bend right= -10] (-4,-7);
            \draw (-2,-10) to[bend right= 10] (0,-7);
            \draw (-2,-10) to[bend right= 20] (4,-7);

            % I 0001: (position 0001)
            \draw (11,-7) rectangle (13,-5);
            \node[anchor=center] (A) at (12,-6) {\corelabelsize $\hypercoreof{2}$};
            \draw (12,-5)--(12,-2.5);

            \drawvariabledot{12}{-3}
            \draw (12,-3) -- (12.5,-3);

            \draw (15,-7) rectangle (17,-5);
            \node[anchor=center] (A) at (16,-6) {\corelabelsize $\hypercoreof{0}$};
            \draw (16,-5)--(16,-3) node[midway,right] {\colorlabelsize $\catvariableof{0,1,0,1,0}$};

            \draw (19,-7) rectangle (21,-5);
            \node[anchor=center] (A) at (20,-6) {\corelabelsize $\hypercoreof{1}$};
            \draw (20,-5)--(20,-3) node[midway,right] {\colorlabelsize $\catvariableof{0,1,0,1,1}$};

            \draw[\concolor] (19,-1) rectangle (21,1);
            \node[anchor=center, \concolor] (A) at (20,0) {\corelabelsize $\tbasis$};
            \draw[\concolor] (20,-1)--(20,-3) node[midway,right] {\colorlabelsize $\catvariableof{0,1,0,1,1}$};

            \drawvariabledot{20}{-3}
            \draw (20,-3) -- (20.5,-3);

            %% Messages from 2 at position 2,2,2,2
            \draw[\newmessagecolor,dashed, ->] (18.5,-1) to [bend right = 30] (18.5,-5);
            \node[\newmessagecolor,anchor=center] (A) at (17.25,-2) {\colorlabelsize $\messagesymbol^{(0,0)}$};

            \draw[\newmessagecolor,dashed, ->] (19,-7.5) to [bend right = -40] (13,-7.25);
            \node[\newmessagecolor,anchor=center] (A) at (16,-9) {\colorlabelsize $\messagesymbol^{(0,1)}$};

            \draw[\newmessagecolor,dashed, <-] (13.5,-1) to [bend right = -30] (13.5,-5);
            \node[\newmessagecolor,anchor=center] (A) at (14.9,-4) {\colorlabelsize $\messagesymbol^{(0,2)}$};

            \draw (23,-7) rectangle (25,-5);
            \node[anchor=center] (A) at (24,-6) {\corelabelsize $\hypercoreof{3}$};
            \draw (24,-5)--(24,-3) node[midway,right] {\colorlabelsize $\catvariableof{0,1,0,1,3}$};

            \drawvariabledot{18}{-10}
            \node[anchor=north] (text) at (18,-10) {\colorlabelsize $\decvariableof{0,1,0,1,:}$};

            \draw (18,-10) to[bend right= -20] (12,-7);
            \draw (18,-10) to[bend right= -10] (16,-7);
            \draw (18,-10) to[bend right= 10] (20,-7);
            \draw (18,-10) to[bend right= 20] (24,-7);

        \end{tikzpicture}
    \end{center}
    \caption{
        The tensor network decomposition of $3$ out of $4\cdot2^2=64$ rules in the $2^2\times2^2$ Sudoku knowledge base (see \exaref{exa:sudokuDecomposition}),  namely to the number $3$ appearing once in the $(0,0)$-square (top), the number $3$ appearing once in the $(0,0)$-row (bottom left) and a unique number appearing at the $(0,1,0,1)$-position (bottom right).
        The evidence of the number $3$ already being assigned at the position $(0,0,1,0)$ is sketched by a basis vector $\textcolor{\concolor}{\tbasis}$ on the left side, and the number $2$ assigned at position $(0,1,0,1)$ analogously on the right side.
        During Constraint Propagation \algoref{alg:constraintPropagation} on the hypergraph of Sudoku rules and evidence (see \exaref{exa:sudokuMessagePassing}), this evidence is in three epochs of messages propagated to the constraints by partial entailment steps and imply that $\textcolor{\probcolor}{\catvariableof{0,1,0,0,2}}$ is true, i.e. that at the position $(0,1,0,0)$ the number $3$ needs to be assigned.
        We depict the messages between the cores by dashed lines labeled by $\messagesymbol^{(0,[3])},\messagesymbol^{(1,[5])}$ and $\messagesymbol^{(2,[4])}$ and provide further interpretation in \exaref{exa:sudokuMessagePassing}.
    }\label{fig:contractionPropagationSudoku}
\end{figure}

