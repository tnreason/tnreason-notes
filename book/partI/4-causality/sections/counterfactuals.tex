\section{Rung 3: Counterfactuals}

Following the formalism of Chapter~21 in \cite{koller_probabilistic_2009}, counterfactual queries are modeled as intervention queries on counterfactual twin networks.
Counterfactual twin networks are two copies of the original causal model, one representing the factual world and one representing the counterfactual world.
They further share response variables, which capture the functional relationships between variables in the causal model.
These response variables are modeled as selection variables in the \tnreason{} formalism.

\begin{example}[Patient counterfactual query]
    Let us assume, that we know whether a patient did not get assigned to the treatment group, and we know whether he did improve.
    We want to ask, whether he would have improved, if he had been assigned to the treatment group.
    This is not an intervention query, but a counterfactual query, since we reason about a different outcome, although we already know that the patient has not been treated.

    The causally augmented Bayesian Network is not rich enough to answer such queries.
    Let us consider a situation where $\catvariableof{T}$ is chosen uniformly from $[2]$.
    There might be situations, where the probabilistic (and intervention) queries are the same, but the counterfactual queries differ:
    \begin{itemize}
        \item[(i)] The outcome is independent of the treatment.
        \item[(ii)] With probability 0.5, the treatment always leads to improvement and missing the treatment not.
        And also with probability 0.5, the treatment never leads to improvement, but missing the treatment does.
    \end{itemize}
    Intuitively, in the  situation $(i)$, the counterfactual query should return probability $1/2$ for improvement under treatment.
    In the situation $(ii)$, the counterfactual query returns the same value with probability $1$ as observed in the real world.
    Both cases are however indistinguishable by causal models itself, since in both cases the causally augmented conditional probabilities are
    \begin{align*}
        \condprobwrtof{\causalsymbol}{\catvariableof{T}}{\dovariableof{T}}
        =\frac{1}{2}\onesat{\catvariableof{T}} \otimes \onehotmapofat{2}{\dovariableof{T}}
        + \left(\sum_{\doindexof{T}\in[2]} \onehotmapofat{\doindexof{T}}{\dovariableof{T}}
              \otimes \onehotmapofat{\doindexof{T}}{\catvariableof{T}} \right)
    \end{align*}
    and
    \begin{align*}
        \condprobwrtof{\causalsymbol}{\catvariableof{O}}{\dovariableof{O},\catvariableof{T}}
        =\frac{1}{2}\onesat{\catvariableof{O},\catvariableof{T}} \otimes \onehotmapofat{2}{\dovariableof{O}}
        + \left(\sum_{\doindexof{O}\in[2]} \onehotmapofat{\doindexof{O}}{\dovariableof{O}}
              \otimes \onehotmapofat{\doindexof{O}}{\catvariableof{O}} \otimes \onesat{\catvariableof{T}} \right) \, .
    \end{align*}
    To answer counterfactual queries, we therefore have to also reason about the mechanisms behind the causal relationships, which we do by introducing response variables in the following.

\end{example}

\subsection{Response Variables}

We introduce response variables $\selvariableof{\catenumerator}$, such that $\catvariableof{\catenumerator}$ has a deterministic dependence on $\selvariableof{\catenumerator}$ and the parents of a variable to the variable itself.
The dependence is captured by a function
\begin{align*}
    \exfunctionof{\catenumerator} \defcols [\seldimof{\catenumerator}] \times \left(\bigtimes_{\seccatenumerator\in\parentsof{\catenumerator}}[\catdimof{\seccatenumerator}]\right) \rightarrow [\catdimof{\catenumerator}] \, .
\end{align*}
The response augmented conditional probability core is defined as (see \figref{fig:responseVariable}):
\begin{align*}
    \condprobwrtof{\responsesymbol}{\catvariableof{\atomenumerator}}{\selvariableof{\catenumerator},\catvariableof{\parentsof{\catenumerator}}}
    = \bencodingofat{\exfunctionof{\catenumerator}}{
        \catvariableof{\catenumerator},
        \selvariableof{\catenumerator},
        \catvariableof{\parentsof{\catenumerator}}
    }
\end{align*}

\begin{figure}[t]
    \begin{center}
        \begin{tikzpicture}[scale=0.35, thick]

    \node[anchor=center] at (-9,5) {$a)$};

    \node[circle, draw, thick, fill=\nodegrayscale, minimum size = \nodeminsize] (XK) at (-5,0) {};
    \node[anchor=center] (A) at (-5,0) {\corelabelsize $\catvariableof{\catenumerator}$};

    \node[circle, draw, thick, fill=\nodegrayscale, minimum size = \nodeminsize] (LK) at (0,3) {};
    \node[anchor=center] (A) at (0,3) {\corelabelsize $\selvariableof{\catenumerator}$};

    \node[circle, draw, thick, fill=\nodegrayscale, minimum size = 0.5*\nodeminsize] (PK0) at (-7,4) {};
    \node[circle, draw, thick, fill=\nodegrayscale, minimum size = 0.5*\nodeminsize] (PK1) at (-3,4) {};


    \draw[->] (LK) -- (XK);

    \draw[<-] (XK) -- (PK0);
    \draw[<-] (XK) -- (PK1);
    \node[anchor=center] at (-5,4) {\corelabelsize $\hdots$};
    \node[anchor=center] at (-5,5) {\corelabelsize $\parentsof{\catvariableof{\catenumerator}}$};

    \node[anchor=center] at (9,5) {$b)$};

    \begin{scope}[shift={(10,0)}]

        \draw (0,1) rectangle (6,3);
        \node[anchor=center] at (3,2) {\corelabelsize $\bencodingof{\exfunctionof{\atomenumerator}}$};
        \draw[-<-] (6,2) -- (8,2) node[above,midway] {\colorlabelsize $\selvariableof{\catenumerator}$};

        \draw[-<-] (1,3) -- (1,5);
        \node[anchor=center] at (3,4) {\colorlabelsize $\parentsof{\catvariableof{\catenumerator}}$};
        \draw[-<-] (5,3) -- (5,5);

        \draw[->-] (3,1) -- (3,-1) node[midway,right] {\colorlabelsize $\catvariableof{\catenumerator}$};
    \end{scope}

\end{tikzpicture}
    \end{center}
    \caption{Introduction of response variables: a) as a Bayesian network, b) decoration by the basis encoding of the corresponding function.}
    \label{fig:responseVariable}
\end{figure}

\subsection{Counterfactual Twinned Network}

To answer counterfactual queries, we now construct the counterfactual twinned network (see \figref{fig:counterfactualTwinnedNetwork}).

\begin{definition}
    Let there be a response augmented Bayesian Network on a directed acyclic hypergraph $\graph=([\catorder],\edges)$ with response variables $\selvariableof{[\catorder]}$.
    Further another hypergraph $\secgraph$ and a Bayesian Network on $\secgraph$ modelling the joint distribution of the response variables $\selvariableof{[\catorder]}$.
    Be build counterfactual copies $\tildecatvariableof{\catenumerator}$ of each variable $\catvariableof{\catenumerator}$, which are controlled by the same response variables $\selvariableof{\catenumerator}$ as the factual variables $\catvariableof{\catenumerator}$.
    The counterfactual twinned network is then the tensor network consists of the causal and response augmented conditional probability cores to $\shortcatvariables$, the conditional probability cores of the response variables $\selvariableof{[\catorder]}$ and the response augmented conditional probability cores to the counterfactual variables $\tildecatvariableof{[\catorder]}$.
\end{definition}


\begin{figure}[t]
    \begin{center}
        \begin{tikzpicture}[scale=0.35, thick]
    \node[anchor=center] at (-10,7) {$a)$};

    \node[circle, draw, thick, fill=\nodegrayscale, minimum size = \nodeminsize] (DK) at (-10,0) {};
    \node[anchor=center] (A) at (-10,0) {\corelabelsize $\dovariableof{\catenumerator}$};

    \node[circle, draw, thick, fill=\nodegrayscale, minimum size = \nodeminsize] (XK) at (-5,0) {};
    \node[anchor=center] (A) at (-5,0) {\corelabelsize $\catvariableof{\catenumerator}$};

    \node[circle, draw, thick, fill=\nodegrayscale, minimum size = \nodeminsize] (LK) at (0,3) {};
    \node[anchor=center] (A) at (0,3) {\corelabelsize $\selvariableof{\catenumerator}$};

    \node[circle, draw, thick, fill=\nodegrayscale, minimum size = \nodeminsize] (TXK) at (5,0) {};
    \node[anchor=center] (A) at (5,0) {\corelabelsize $\tildecatvariableof{\catenumerator}$};

    \draw[->] (DK) -- (XK);
    \draw[->] (LK) -- (XK);
    \draw[->] (LK) -- (TXK);

    \draw[<-] (XK) -- ++(-2,4) node[circle, draw, thick, fill=\nodegrayscale, minimum size = 0.5*\nodeminsize] {};
    %\node[anchor=center] at (-5,6.5) {\corelabelsize $\vdots$};
    \node[anchor=center] at (-5,5) {\corelabelsize $\parentsof{\catvariableof{\catenumerator}}$};
    \node[anchor=center] at (-5,4) {\corelabelsize $\hdots$};
    \draw[<-] (XK) -- ++(2,4) node[circle, draw, thick, fill=\nodegrayscale, minimum size = 0.5*\nodeminsize] {};

    \draw[<-] (TXK) -- ++(-2,4) node[circle, draw, thick, fill=\nodegrayscale, minimum size = 0.5*\nodeminsize] {};
    %\node[anchor=center] at (5,6.5) {\corelabelsize $\vdots$};
    \node[anchor=center] at (5,5) {\corelabelsize $\parentsof{\tildecatvariableof{\catenumerator}}$};
    \node[anchor=center] at (5,4) {\corelabelsize $\hdots$};
    \draw[<-] (TXK) -- ++(2,4) node[circle, draw, thick, fill=\nodegrayscale, minimum size = 0.5*\nodeminsize] {};

    \draw[<-] (LK) -- ++(-2,4) node[circle, draw, thick, fill=\nodegrayscale, minimum size = 0.5*\nodeminsize] {};
    %\node[anchor=center] at (0,9.5) {\corelabelsize $\vdots$};
    \node[anchor=center] at (0,8) {\corelabelsize $\parentsof{\selvariableof{\catenumerator}}$};
    \node[anchor=center] at (0,7) {\corelabelsize $\hdots$};
    \draw[<-] (LK) -- ++(2,4) node[circle, draw, thick, fill=\nodegrayscale, minimum size = 0.5*\nodeminsize] {};

    \node[anchor=center] at (14,7) {$b)$};

    \begin{scope}[shift={(15,0)}]

        \draw (0,1) rectangle (6,3);
        \node[anchor=center] at (3,2) {\corelabelsize $\bencodingof{\causalsymbol,\exfunctionof{\atomenumerator}}$};
        \draw[->-] (3,1) -- (3,-1) node[midway,right] {\colorlabelsize $\catvariableof{\catenumerator}$};
        \draw[-<-] (1,3) -- (1,5);
        \node[anchor=center] at (3,4) {\colorlabelsize $\parentsof{\catvariableof{\catenumerator}}$};
        \draw[-<-] (5,3) -- (5,5);

        \draw[-<-] (0,2) -- (-2,2) node[midway,above] {\colorlabelsize $\dovariableof{\catenumerator}$};;

        \draw[-<-] (6,2) -- (8,2);
        \drawvariabledot{8}{2}
        \draw[-<-] (10,2) -- (8,2);

        \draw[->-] (8,6) -- (8,2) node[midway,right] {\colorlabelsize $\selvariableof{\catenumerator}$};
        \draw (5,6) rectangle (11,8);
        \node[anchor=center] at (8,7) {\corelabelsize $\condprobof{\selvariableof{\catenumerator}}{\parentsof{\selvariableof{\catenumerator}}}$};
        \draw[-<-] (6,8) -- (6,10);
        \draw[-<-] (10,8) -- (10,10);
        \node[anchor=center] at (8,9) {\colorlabelsize $\parentsof{\selvariableof{\catenumerator}}$};

        \begin{scope}[shift={(10,0)}]
            \draw (0,1) rectangle (6,3);
            \node[anchor=center] at (3,2) {\corelabelsize $\bencodingof{\exfunctionof{\atomenumerator}}$};
            \draw[->-] (3,1) -- (3,-1) node[midway,right] {\colorlabelsize $\tildecatvariableof{\catenumerator}$};
            \draw[-<-] (1,3) -- (1,5);
            \node[anchor=center] at (3,4) {\colorlabelsize $\parentsof{\tildecatvariableof{\catenumerator}}$};
            \draw[-<-] (5,3) -- (5,5);
        \end{scope}


    \end{scope}

\end{tikzpicture}
    \end{center}
    \caption{Counterfactual twinned network: a) as a Bayesian network.
    b) Decorations by causal augmented basis encodings by a do-variable for $\catvariableof{\catenumerator}$, a conditional probability distribution for $\selvariableof{\catenumerator}$ and a basis encoding for the counterfactual variable $\tildecatvariableof{\catenumerator}$.}
    \label{fig:counterfactualTwinnedNetwork}
\end{figure}

