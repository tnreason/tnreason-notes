\begin{example}[Patient counterfactual query]
    Let us assume, that we know whether a patient did not get assigned to the treatment group, and we know whether he did improve.
    We want to ask, whether he would have improved, if he had been assigned to the treatment group.
    This is not an intervention query, but a counterfactual query, since we reason about a different outcome, although we already know that the patient has not been treated.

    The causally augmented Bayesian Network is not rich enough to answer such queries.
    Let us consider a situation where $\catvariableof{T}$ is chosen uniformly from $[2]$.
    There might be situations, where the probabilistic (and intervention) queries are the same, but the counterfactual queries differ:
    \begin{itemize}
        \item[(i)] The outcome is independent of the treatment.
        \item[(ii)] With probability 0.5, the treatment always leads to improvement and missing the treatment not.
        And also with probability 0.5, the treatment never leads to improvement, but missing the treatment does.
    \end{itemize}
    Intuitively, in the  situation $(i)$, the counterfactual query should return probability $1/2$ for improvement under treatment.
    In the situation $(ii)$, the counterfactual query returns the same value with probability $1$ as observed in the real world.
    Both cases are however indistinguishable by causal models itself, since in both cases the causally augmented conditional probabilities are
    \begin{align*}
        \condprobwrtof{\causalsymbol}{\catvariableof{T}}{\dovariableof{T}}
        =\frac{1}{2}\onesat{\catvariableof{T}} \otimes \onehotmapofat{2}{\dovariableof{T}}
        + \left(\sum_{\doindexof{T}\in[2]} \onehotmapofat{\doindexof{T}}{\dovariableof{T}}
              \otimes \onehotmapofat{\doindexof{T}}{\catvariableof{T}} \right)
    \end{align*}
    and
    \begin{align*}
        \condprobwrtof{\causalsymbol}{\catvariableof{O}}{\dovariableof{O},\catvariableof{T}}
        =\frac{1}{2}\onesat{\catvariableof{O},\catvariableof{T}} \otimes \onehotmapofat{2}{\dovariableof{O}}
        + \left(\sum_{\doindexof{O}\in[2]} \onehotmapofat{\doindexof{O}}{\dovariableof{O}}
              \otimes \onehotmapofat{\doindexof{O}}{\catvariableof{O}} \otimes \onesat{\catvariableof{T}} \right) \, .
    \end{align*}
    To answer counterfactual queries, we therefore have to also reason about the mechanisms behind the causal relationships, which we do by introducing response variables in the following.

\end{example}