\begin{example}[Continuation of \exaref{exa:studentHC}]
    \label{exa:studentBP}
    We exemplify the Belief Propagation \algoref{alg:treeBeliefPropagation} on the Markov network in the student example (see \exaref{exa:studentHC}).
    The directions of the messages result from the hyperedge overlaps (see \figref{fig:studentMessagePassingDirections} a) and the resulting directions $\dirovedges$ are sketched in \figref{fig:studentMessagePassingDirections} b).
    The messages to $\{(\edgeof{2},\edgeof{0}),(\edgeof{0},\edgeof{2})\}$ are vectors of $\catvariableof{G}$ and the messages $\{(\edgeof{0},\edgeof{1}),(\edgeof{1},\edgeof{0})\}$ are vectors of $\catvariableof{I}$.

    Since the hyperedges are minimally connected, we can implement \algoref{alg:treeBeliefPropagation} by a tree scheduler $\scheduler$:
    \begin{itemize}
        \item The scheduler is initialized with messages from leafs, in our example $\{(\edgeof{2},\edgeof{0}),(\edgeof{1},\edgeof{0})\}$.
        \item Each message is placed exactly once on $\scheduler$, when at a hyperedge all but the reverse message have been received.
        In our example, after execution of $(\edgeof{2},\edgeof{0})$ the message $(\edgeof{0},\edgeof{1})$ is placed on $\scheduler$ and after execution of $(\edgeof{1},\edgeof{0})$ the message $(\edgeof{0},\edgeof{2})$.
    \end{itemize}
    In this implementation, \algoref{alg:treeBeliefPropagation} terminates after $\cardof{\dirovedges}=4$ iterations of the \whileSymbol{} loop.
    The exact marginals of the edge variables are then
    \begin{align*}
        \probat{\catvariableof{L},\catvariableof{G}}
        &= \normalizationof{
            \hypercoreofat{\edgeof{2}}{\catvariableof{L},\catvariableof{G}},
            \mesfromto{\edgeof{0}}{\edgeof{2}}\left[\catvariableof{G}\right]
        }{\catvariableof{L},\catvariableof{G}}, \\
        \probat{\catvariableof{G},\catvariableof{D},\catvariableof{I}}
        &= \normalizationof{\hypercoreofat{\edgeof{0}}{\catvariableof{G},\catvariableof{D},\catvariableof{I}},
            \mesfromto{\edgeof{2}}{\edgeof{0}}\left[\catvariableof{G}\right],
            \mesfromto{\edgeof{1}}{\edgeof{0}}\left[\catvariableof{I}\right]
        }{\catvariableof{G},\catvariableof{D},\catvariableof{I}}, \\
        \probat{\catvariableof{I},\catvariableof{S}}
        &= \normalizationof{
            \hypercoreofat{\edgeof{1}}{\catvariableof{I},\catvariableof{S}},
            \mesfromto{\edgeof{0}}{\edgeof{1}}\left[\catvariableof{I}\right]
        }{\catvariableof{I},\catvariableof{S}} \, .
    \end{align*}

\end{example}


\begin{figure}[t]
    \begin{center}
        \begin{tikzpicture}
            \node[anchor=center] (A) at (-0.5,3) {\corelabelsize $a)$};

            \node[circle, draw, thick, fill=\nodegrayscale, minimum size = \nodeminsize] (A) at (0,2) {};
            \node[anchor=center] (A) at (0,2) {\corelabelsize $\catvariableof{L}$};
            \node[circle, draw, thick, fill=\nodegrayscale, minimum size = \nodeminsize] (A) at (1,2) {};
            \node[anchor=center] (A) at (1,2) {\corelabelsize $\catvariableof{G}$};
            \node[circle, draw, thick, fill=\nodegrayscale, minimum size = \nodeminsize] (A) at (2,2) {};
            \node[anchor=center] (A) at (2,2) {\corelabelsize $\catvariableof{D}$};
            \node[circle, draw, thick, fill=\nodegrayscale, minimum size = \nodeminsize] (A) at (3,2) {};
            \node[anchor=center] (A) at (3,2) {\corelabelsize $\catvariableof{I}$};
            \node[circle, draw, thick, fill=\nodegrayscale, minimum size = \nodeminsize] (A) at (4,2) {};
            \node[anchor=center] (A) at (4,2) {\corelabelsize $\catvariableof{S}$};

            \node (of) at (0,0.5) {};
            \draw[thick, dashed, rounded corners=10pt]  ($(1.5,2)+(of)$) -- ($(1.5,2)-(of)$)  -- ($(-0.5,2)-(of)$) -- ($(-0.5,2)+(of)$) -- cycle;
            \node[anchor=center] (A) at (-1,2) {\corelabelsize $\edgeof{2}$};

            \draw[thick, dashed, rounded corners=10pt]  ($(2.5,2)+(of)$) -- ($(2.5,2)-(of)$)  -- ($(4.5,2)-(of)$) -- ($(4.5,2)+(of)$) -- cycle;
            \node[anchor=center] (A) at (5,2) {\corelabelsize $\edgeof{1}$};
            \node (of) at (0,0.75) {};
            \draw[thick, dashed, rounded corners=10pt]  ($(0.5,2)+(of)$) -- ($(0.5,2)-(of)$)  -- ($(3.5,2)-(of)$) -- ($(3.5,2)+(of)$) -- cycle;
            \node[anchor=center] (A) at (2,1) {\corelabelsize $\edgeof{0}$};

            \begin{scope}[shift={(7,0)}]
                \node[anchor=center] (A) at (-0.5,3) {\corelabelsize $b)$};

                \node[circle, draw, thick, minimum size = \nodeminsize] (A) at (0,2) {};
                \node[anchor=center] (E0) at (0,2) {\corelabelsize $\edgeof{2}$};

                \node[circle, draw, thick, minimum size = \nodeminsize] (A) at (2,2) {};
                \node[anchor=center] (E1) at (2,2) {\corelabelsize $\edgeof{0}$};

                \node[circle, draw, thick, fill=white, minimum size = \nodeminsize] (A) at (4,2) {};
                \node[anchor=center] (E2) at (4,2) {\corelabelsize $\edgeof{1}$};

                \draw[->-] (E0) to[bend right = 40] (E1);
                \node[anchor=center] at (1,2.75) {\corelabelsize $(\edgeof{2},\edgeof{0})$};
                \draw[->-] (E1) to[bend right = 40] (E0);
                \node[anchor=center] at (1,1.25) {\corelabelsize $(\edgeof{2},\edgeof{0})$};
                \draw[->-] (E1) to[bend right = 40] (E2);
                \node[anchor=center] at (3,2.75) {\corelabelsize $(\edgeof{0},\edgeof{1})$};
                \draw[->-] (E2) to[bend right = 40] (E1);
                \node[anchor=center] at (3,1.25) {\corelabelsize $(\edgeof{1},\edgeof{0})$};
            \end{scope}
        \end{tikzpicture}
    \end{center}
    \caption{a) Sketch of the overlap of the edges, resulting in the message directions b) $\dirovedges=\{(\edgeof{2},\edgeof{0}),(\edgeof{0},\edgeof{2}),(\edgeof{0},\edgeof{1}),(\edgeof{1},\edgeof{0})\}$.}\label{fig:studentMessagePassingDirections}
\end{figure}